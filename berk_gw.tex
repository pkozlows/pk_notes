\section{Formulation for the dTDA}
\subsection{Comparing to Booth's ED}
For simplicity, we will just work with a single candle, the lesser one. In the dTDA case, Booth's formulation for the upfolded Hamiltonian is
\begin{equation}
    \bm{H} = \begin{pmatrix} \bm{F} & \bm{W} \\ \bm{W}^{\dagger} & \bm{d} \end{pmatrix}
\label{eq:booth_hamiltonian}
\end{equation}
where we have the definitions
\begin{equation}
    W_{pkv} = \sum_{ia} (pk|ia) X_{ia}^{v} \quad \text{and} \quad d_{kv,lv'} = \left(\epsilon_k - \Omega_v\right) \delta_{k,l} \delta_{v,v'}
\label{eq:booth_definitions}
\end{equation}
Now, Tim's version of the Hamiltonian is given by
\begin{equation}
    \bm{H} = \begin{pmatrix} \bm{F} & \bm{V}^{2 \mathrm{h1p}}\\ \left(\bm{V}^{2 \mathrm{~h} 1 \mathrm{p}}\right)^{\dagger} & \bm{C}^{2 \mathrm{hlp}} \end{pmatrix}
\label{eq:tim_hamiltonian}
\end{equation}
where the definitions of the matrix elements are
\begin{align}
    V_{p, k[l c]}^{2 \mathrm{~h} 1 \mathrm{p}} &= \langle p c | k l \rangle \equiv (pk|lc) \\
    C_{i[j a], k[l c]}^{2 \mathrm{~h} 1 \mathrm{p}} &= \left[\left(\epsilon_i+\epsilon_j-\epsilon_a\right) \delta_{j l} \delta_{a c}-\langle j c | a l \rangle\right] \delta_{i k}
\end{align}
and in particular, we have a definition 
\begin{equation}
    \bm{C}^{2 \mathrm{hlp}} = \epsilon^{1 \mathrm{~h}} \oplus (-\bm{A}) = \epsilon^{1 \mathrm{~h}} \otimes \bm{1} + \bm{1} \otimes (-\bm{A})
\label{eq:tim_c_hlp}
\end{equation}
Let us define a unitary rotation $\bm{U} = \bm{1} \oplus_{\text{diag}} \bm{X}$. Application of this unitary to the Hamiltonian will not change the spectrum and actually transforms the problem into
\begin{equation}
    \bm{H}' = \bm{U}^\dag \bm{H} \bm{U} = \begin{pmatrix} \bm{1} & \bm{0}\\ \bm{0}&\bm{X }^\dag \end{pmatrix} \begin{pmatrix} \bm{F} & \bm{V}^{2 \mathrm{h1p}}\\ \left(\bm{V}^{2 \mathrm{~h} 1 \mathrm{p}}\right)^{\dagger} & \bm{C}^{2 \mathrm{hlp}} \end{pmatrix} \begin{pmatrix} \bm{1} & \bm{0} \\ \bm{0}&\bm{X } \end{pmatrix} = \begin{pmatrix} \bm{F} & \bm{V}^{2 \mathrm{h1p}}\bm{X}\\ \bm{X}^\dag \left(\bm{V}^{2 \mathrm{~h} 1 \mathrm{p}}\right)^{\dagger} & \bm{X }^\dagger \bm{C}^{2 \mathrm{hlp}} \bm{X } \end{pmatrix}.
\label{eq:booth_upfolded_hamiltonian}
\end{equation}
Now let us evaluate $\bm{X}^\dagger \bm{C}^{2 \mathrm{hlp}} \bm{X }$ in the dTDA case. We have
\begin{align}
    \bm{X}^\dagger \bm{C}^{2 \mathrm{hlp}} \bm{X } &= \bm{X}^\dagger \left(\epsilon^{1 \mathrm{~h}} \otimes \bm{1} + \bm{1} \otimes (-\bm{A})\right) \bm{X } \\
&= \bm{X}^\dagger \left(\epsilon^{1 \mathrm{~h}} \otimes \bm{1}\right) \bm{X } + \bm{X}^\dagger \left(\bm{1} \otimes (-\bm{A})\right) \bm{X } \\
&= \left(\epsilon^{1 \mathrm{~h}} \underbrace{\bm{X}^\dag \bm{X}}_{\bm{I}} \right) \otimes \left(\underbrace{\bm{X}^\dag \bm{X}}_{\bm{I}}\right) + \left(\underbrace{\bm{X}^\dag \bm{X}}_{\bm{I}}\right) \otimes (-\underbrace{\bm{X}^\dag \bm{A} \bm{X}}_{\bm{\Omega }}) \\
&= \epsilon^{1 \mathrm{~h}} \oplus (-\bm{\Omega }) \equiv \bm{d}^{<}
\end{align}
where we have used the fact that $\bm{X}^\dag \bm{X} = \bm{1}$, since $\bm{X}$ is unitary. Similarly, we can evaluate the other term
\begin{align}
    \bm{V}^{2 \mathrm{~h} 1 \mathrm{p}} \bm{X } &= \bm{W}^< \equiv \sum_{lc} (pk|lc) X_{lc}^{v} \\
\end{align}
where we have used the definition of the $W$ matrix in Booth's formulation, as given in \eqref{eq:booth_definitions}.
\subsection{Deriving the matrix vector products}
Now, we can define a vector $\bm{R} = ( r_i,\; r_a,\; r_{i[jb]},\; r_{[jb]a} )$. Application of the Hamiltonian to this vector gives us the matrix-vector product $\bm{H} \bm{R} = \bm{\sigma }$, where $\bm{\sigma} = ( \sigma_i,\; \sigma_a,\; \sigma_{i[jb]},\; \sigma_{[jb]a} )$. Consider
\begin{align}
\bm{H} \bm{R} &= \begin{pmatrix} \bm{F} & \bm{V}^{2 \mathrm{h1p}} & \bm{V}^{2 \mathrm{p} 1 \mathrm{~h}} \\ \left(\bm{V}^{2 \mathrm{~h} 1 \mathrm{p}}\right)^{\dagger} & \bm{C}^{2 \mathrm{hlp}} & \bm{0} \\ \left(\bm{V}^{2 \mathrm{plh}}\right)^{\dagger} & \bm{0} & \bm{C}^{2 \mathrm{plh}} \end{pmatrix} \begin{pmatrix} r_i \\ r_a \\ r_{i[jb]} \\ r_{[jb]a} \end{pmatrix} = \begin{pmatrix} \sigma_i \\ \sigma_a \\ \sigma_{i[jb]} \\ \sigma_{[jb]a} \end{pmatrix} \\[6pt]
\end{align}
Let us enumerate now what we actually will get:
\begin{align}
\sigma_i &= 
  \sum_{j} f_{i j}\,r_j
  + \sum_{b} f_{i b}\,r_b
  + \sum_{k l c} \bigl\langle i\,c \mid k\,l \bigr\rangle\,r_{k[l c]}
  + \sum_{k c d} \bigl\langle i\,k \mid d\,c \bigr\rangle\,r_{[k c]d}, \\[6pt]
\sigma_a &=
    \sum_{j} f_{a j}\,r_j
    + \sum_{b} f_{a b}\,r_b
    + \sum_{k l c} \bigl\langle a\,c \mid k\,l \bigr\rangle\,r_{k[l c]}
    + \sum_{k c d} \bigl\langle a\,k \mid d\,c \bigr\rangle\,r_{[k c]d}, \\[6pt]
\sigma_{i[ja]} &
    \sum_{k} \bigl\langle k\,a \mid i\,j \bigr\rangle\,r_k
    + \sum_{b} \bigl\langle b\,a \mid i\,j \bigr\rangle\,r_b
    + \left(\epsilon _{i} + \epsilon _{j} - \epsilon _{a}\right) r_{i[j a]}
    - \sum_{lc} \bigl\langle j\,c \mid a\,l \bigr\rangle\,r_{i[l c]} \\[6pt]
\sigma_{[ia]b} &= 
    \sum_{j} \bigl\langle j\,i \mid b\,a \bigr\rangle\,r_j
    + \sum_{c} \bigl\langle c\,i \mid b\,a \bigr\rangle\,r_c
    + \left(\epsilon _{a} + \epsilon _{b} - \epsilon _{i}\right) r_{[i a] b}
    + \sum_{k c} \bigl\langle a\,k \mid i\,c \bigr\rangle\,r_{[k c] b}.
\end{align}
\section{Formulation for the dRPA}
\subsection{Deriving the dRPA approach as they propose}
So we start with this generalized eigenvalue equation
$$
\mathbf{M}\bm{Z}=\mathbf{N}\bm{Z}\left(\begin{array}{cc}
\boldsymbol{\Omega}_{+} & 0 \\
0 & -\boldsymbol{\Omega}_{+}
\end{array}\right)
$$
where
$$
\mathbf{M} =\left(\begin{array}{ll}
\mathbf{A} & \mathbf{B} \\
\mathbf{B} & \mathbf{A}
\end{array}\right) \quad 
\mathbf{N} =\left(\begin{array}{cc}
\mathbf{1} & \mathbf{0} \\
\mathbf{0} & -\mathbf{1}
\end{array}\right) \quad
\bm{Z} =\left(\begin{array}{ll}
\bm{X} & \bm{Y} \\
\bm{Y} & \bm{X}
\end{array}\right)
$$
and $\boldsymbol{\Omega}_{+}$is a diagonal matrix of positive excitation energies. Left multiplying both sides by $\bm{N}$ and right multiplying by $\bm{Z}^{-1}$ gives us
\begin{equation}
    \bm{N}\bm{M}\underbrace{\bm{Z} \bm{Z}^{-1}}_{\bm{1}} = \underbrace{\bm{N}\bm{N}}_{\bm{1}}\bm{Z}\left(\begin{array}{cc}
\boldsymbol{\Omega}_{+} & 0 \\
0 & -\boldsymbol{\Omega}_{+}
\end{array}\right) \bm{Z^{-1}} \implies - \bm{N}\bm{M} = -\bm{Z}\left(\begin{array}{cc}\boldsymbol{\Omega}_{+} & 0 \\
0 & -\boldsymbol{\Omega}_{+}
\end{array}\right)\bm{Z}^{-1}
\end{equation}
Now we can use the fact that the action of a scalar function $f$, such as the step function, on a diagonalizable matrix $\bm{X} \equiv \bm{Y} \bm{\Lambda} \bm{Y}^{-1}$ can be expressed as
$$
f(\bm{X}) = \bm{Y} f(\bm{\Lambda}) \bm{Y}^{-1}
$$
so we can write
\begin{equation}
    \Theta(-\bm{N}\bm{M}) = \bm{Z} \left(\begin{array}{cc}\Theta(-\boldsymbol{\Omega}_{+}) & 0 \\ 0 & \Theta(\boldsymbol{\Omega}_{+})\end{array}\right)\bm{Z}^{-1} = \bm{Z} \left(\begin{array}{cc}\bm{0} & 0 \\ 0 & \bm{1}\end{array}\right)\bm{Z}^{-1}
\end{equation}
and so it becomes clear that if we define $\tilde{\bm{M}} = \bm{M} + \eta \bm{N}\Theta(-\bm{N}\bm{M})$, we can write
\begin{equation}
    \tilde{\bm{M}} \bm{Z} = \bm{M} \bm{Z} + \eta \bm{N}\Theta(-\bm{N}\bm{M}) \bm{Z} = \mathbf{N}\bm{Z}\left(\begin{array}{cc}
\boldsymbol{\Omega}_{+} & 0 \\
0 & -\boldsymbol{\Omega}_{+}
\end{array}\right) + \bm{N}\bm{Z} \left(\begin{array}{cc}\bm{0} & 0 \\ 0 & \bm{\eta}\end{array}\right) \underbrace{\bm{Z}^{-1} \bm{Z}}_{ \bm{1}} = \bm{N}\bm{Z}\left(\begin{array}{cc}\boldsymbol{\Omega}_{+} & 0 \\ 0 & -\boldsymbol{\Omega}_{+} + \bm{\eta}\end{array}\right).
\end{equation}
Now they define a super-matrix,
\begin{equation}
\bm{H} =
\begin{pmatrix}
\bm{F} & \bm{V}^{2\mathrm{h1p}} & \bm{V}^{2\mathrm{h1p}} & \bm{V}^{2\mathrm{plh}} & \bm{V}^{2\mathrm{plh}} \\
\left(\bm{V}^{2\mathrm{h1p}}\right)^{\dagger} &  &  & & \\
\left(\bm{V}^{2\mathrm{h1p}}\right)^{\dagger} &  & \bm{C}^{2\mathrm{hlp}} & & \bm{0} \\
\left(\bm{V}^{2\mathrm{plh}}\right)^{\dagger} & & & & \\
\left(\bm{V}^{2\mathrm{plh}}\right)^{\dagger} &  & \bm{0} & & \bm{C}^{2\mathrm{plh}}
\end{pmatrix}
\end{equation}
where $\mathbf{C}^{2 \mathrm{hlp}}=\varepsilon^{1 \mathrm{~h}} \oplus(-\tilde{\mathbf{M}})$ and $\mathbf{C}^{2 \mathrm{plh}}=\varepsilon^{1 \mathrm{p}} \oplus \tilde{\mathbf{M}}$, which for which we use the super-metric
$$
\bm{\mathcal{N}}=\left(\begin{array}{ccc}
\bm{1} & 0 & 0 \\
0 & \bm{1} \oplus \bm{N} & 0 \\
0 & 0 & \bm{1} \oplus \bm{N}
\end{array}\right)
$$
and the excitation vector is $\bm{R} = (\bm{R}^{1\mathrm{h}+1\mathrm{p}}, \bm{R}^{2\mathrm{h}1\mathrm{p}}, \bm{R}^{2\mathrm{p}1\mathrm{h}})$ with $\bm{R}^{2\mathrm{h}1\mathrm{p}}=(\bm{r}^{2\mathrm{h}1\mathrm{p}}, \bm{\bar{r}}^{2\mathrm{h}1\mathrm{p}})$ and $\bm{R}^{2\mathrm{p}1\mathrm{h}}=(\bm{r}^{2\mathrm{p}1\mathrm{h}}, \bm{\bar{r}}^{2\mathrm{p}1\mathrm{h}})$. So 
$\bm{\mathcal{N}} \bm{H} \bm{R} = \bm{R} \bm{E}$ leads to:
\begin{equation}
\begin{pmatrix}
\bm{1} & 0 & 0 \\
0 & \bm{1} \oplus \bm{N} & 0 \\
0 & 0 & \bm{1} \oplus \bm{N}
\end{pmatrix}
\begin{pmatrix}
\bm{F} & \bm{V}^{2\mathrm{h1p}} & \bm{V}^{2\mathrm{h1p}} & \bm{V}^{2\mathrm{plh}} & \bm{V}^{2\mathrm{plh}} \\
\left(\bm{V}^{2\mathrm{h1p}}\right)^{\dagger} &  &  & & \\
\left(\bm{V}^{2\mathrm{h1p}}\right)^{\dagger} &  & \bm{C}^{2\mathrm{hlp}} & & \bm{0} \\
\left(\bm{V}^{2\mathrm{plh}}\right)^{\dagger} & & & & \\
\left(\bm{V}^{2\mathrm{plh}}\right)^{\dagger} &  & \bm{0} & & \bm{C}^{2\mathrm{plh}}
\end{pmatrix}
\begin{pmatrix}
\bm{r}^{1\mathrm{h}+1\mathrm{p}} \\
\bm{r}^{2\mathrm{h}1\mathrm{p}} \\
\bm{\bar{r}}^{2\mathrm{h}1\mathrm{p}} \\
\bm{r}^{2\mathrm{p}1\mathrm{h}} \\
\bm{\bar{r}}^{2\mathrm{p}1\mathrm{h}}
\end{pmatrix} = E
\begin{pmatrix}
\bm{r}^{1\mathrm{h}+1\mathrm{p}} \\
\bm{r}^{2\mathrm{h}1\mathrm{p}} \\
\bm{\bar{r}}^{2\mathrm{h}1\mathrm{p}} \\
\bm{r}^{2\mathrm{p}1\mathrm{h}} \\
\bm{\bar{r}}^{2\mathrm{p}1\mathrm{h}}
\end{pmatrix}
\end{equation}
The 5 coupled equations can be written as
\begin{align}
\bm{F} \bm{r}^{1\mathrm{h}+1\mathrm{p}} + \bm{V}^{2\mathrm{h}1\mathrm{p}} \left(\bm{r}^{2\mathrm{h}1\mathrm{p}} + \bm{\bar{r}}^{2\mathrm{h}1\mathrm{p}}\right) + \bm{V}^{2\mathrm{plh}} \left(\bm{r}^{2\mathrm{p}1\mathrm{h}} + \bm{\bar{r}}^{2\mathrm{p}1\mathrm{h}}\right) &= E \bm{r}^{1\mathrm{h}+1\mathrm{p}} \\
\left(\bm{V}^{2\mathrm{h}1\mathrm{p}}\right)^\dagger \bm{r}^{1\mathrm{h}+1\mathrm{p}} + [\bm{\tilde{C}}^{2\mathrm{hlp}}]^{xx}\bm{r}^{2\mathrm{h}1\mathrm{p}}+ [\bm{\tilde{C}}^{2\mathrm{hlp}}]^{xd}\bm{\bar{r}}^{2\mathrm{h}1\mathrm{p}} &= E \bm{r}^{2\mathrm{h}1\mathrm{p}} \\
\left(\bm{V}^{2\mathrm{h}1\mathrm{p}}\right)^\dagger \bm{r}^{1\mathrm{h}+1\mathrm{p}} + [\bm{\tilde{C}}^{2\mathrm{hlp}}]^{dx}\bm{r}^{2\mathrm{h}1\mathrm{p}}+ [\bm{\tilde{C}}^{2\mathrm{hlp}}]^{dd}\bm{\bar{r}}^{2\mathrm{h}1\mathrm{p}} &= E \bm{\bar{r}}^{2\mathrm{h}1\mathrm{p}} \\
\left(\bm{V}^{2\mathrm{plh}}\right)^\dagger \bm{r}^{1\mathrm{h}+1\mathrm{p}} + [\bm{\tilde{C}}^{2\mathrm{plh}}]^{xx}\bm{r}^{2\mathrm{p}1\mathrm{h}}+ [\bm{\tilde{C}}^{2\mathrm{plh}}]^{xd}\bm{\bar{r}}^{2\mathrm{p}1\mathrm{h}} &= E \bm{r}^{2\mathrm{p}1\mathrm{h}} \\
\left(\bm{V}^{2\mathrm{plh}}\right)^\dagger \bm{r}^{1\mathrm{h}+1\mathrm{p}} + [\bm{\tilde{C}}^{2\mathrm{plh}}]^{dx}\bm{r}^{2\mathrm{p}1\mathrm{h}}+ [\bm{\tilde{C}}^{2\mathrm{plh}}]^{dd}\bm{\bar{r}}^{2\mathrm{p}1\mathrm{h}} &= E \bm{\bar{r}}^{2\mathrm{p}1\mathrm{h}}
\end{align}
% We can combine into 3 by adding the second to the third and the fourth to the fifth, which gives us
% \begin{align}
% \bm{F} \bm{r}^{1\mathrm{h}+1\mathrm{p}} + \bm{V}^{2\mathrm{h}1\mathrm{p}} \left(\bm{r}^{2\mathrm{h}1\mathrm{p}} + \bm{\bar{r}}^{2\mathrm{h}1\mathrm{p}}\right) + \bm{V}^{2\mathrm{plh}} \left(\bm{r}^{2\mathrm{p}1\mathrm{h}} + \bm{\bar{r}}^{2\mathrm{p}1\mathrm{h}}\right) &= E \bm{r}^{1\mathrm{h}+1\mathrm{p}} \\[6pt]
% 2\left(\bm{V}^{2\mathrm{h}1\mathrm{p}}\right)^\dagger \bm{r}^{1\mathrm{h}+1\mathrm{p}} + \left([\bm{\tilde{C}}^{2\mathrm{hlp}}]^{xx}+[\bm{\tilde{C}}^{2\mathrm{hlp}}]^{xd}\right)\bm{r}^{2\mathrm{h}1\mathrm{p}}+ \left([\bm{\tilde{C}}^{2\mathrm{hlp}}]^{dx}+[\bm{\tilde{C}}^{2\mathrm{hlp}}]^{dd}\right)\bm{\bar{r}}^{2\mathrm{h}1\mathrm{p}} &= E \bm{r}^{2\mathrm{h}1\mathrm{p}} \\[6pt]
% \end{align}
where now we have $\bm{\tilde{C}}^{2\mathrm{hlp}} = \varepsilon^{1\mathrm{~h}} \oplus (-\bm{N}\bm{\tilde{M}})$ and $\bm{\tilde{C}}^{2\mathrm{plh}} = \varepsilon^{1\mathrm{p}} \oplus \bm{N}\bm{\tilde{M}}$.
Solving the last two equations gives:
\begin{align}
\bm{r}^{2\mathrm{h}1\mathrm{p}} + \bm{\bar{r}}^{2\mathrm{h}1\mathrm{p}} &= \left(E\bm{1} - \bm{\tilde{C}}^{2\mathrm{hlp}}\right)^{-1} \left(\left(\bm{V}^{2\mathrm{h}1\mathrm{p}}\right)^\dagger \bm{r}^{1\mathrm{h}+1\mathrm{p}} - \bm{\tilde{C}}^{2\mathrm{plh}} \left(\bm{r}^{2\mathrm{p}1\mathrm{h}} + \bm{\bar{r}}^{2\mathrm{p}1\mathrm{h}}\right)\right) \\[6pt]
\bm{r}^{2\mathrm{p}1\mathrm{h}} + \bm{\bar{r}}^{2\mathrm{p}1\mathrm{h}} &= \left(\bm{\tilde{C}}^{2\mathrm{plh}} - E \bm{1}\right)^{-1} \left(\left(\bm{V}^{2\mathrm{plh}}\right)^\dagger \bm{r}^{1\mathrm{h}+1\mathrm{p}} + \bm{\tilde{C}}^{2\mathrm{hlp}} \left(\bm{r}^{2\mathrm{h}1\mathrm{p}} + \bm{\bar{r}}^{2\mathrm{h}1\mathrm{p}}\right)\right)
\end{align}
 The eigenvalue equation is then given by
what they claim gives $\bm{H} \bm{R} = \bm{\mathcal{N}} \bm{R} \bm{E}$ and then leading to the non-Hermitian eigenvalue equation $\mathcal{\bm{N}} \bm{H} \bm{R} = \bm{R} \bm{E}$.
\subsubsection{Downfolding equivalence}
Let us carry this matrix multiplication out, just for the 2h1p sector. Collect the relevant components of the excitation vector into a block vector:
\[
\bm{R} = 
\begin{pmatrix}
\bm{r}_i &\bm{r}_a &
\bm{r}_{i[jb]} &
\bm{\bar{r}}_{i[jb]}
\end{pmatrix}^\dag
\]
We will define the total dimension of this excitation vector as $T=P+A_O$, where $P=O+V$ and $A_O=2O^2V$.
The super-Hamiltonian decomposes as:
\[
\bm{H} =
\begin{pmatrix}
\begin{pmatrix}
    \bm{F}_{oo} & \bm{F}_{ov} \\
    \bm{F}_{vo} & \bm{F}_{vv}
\end{pmatrix}_{PP} &
\begin{pmatrix}
    \bm{V}^{2 \mathrm{h1p}}_{p,i[jb]} & \bm{V}^{2 \mathrm{h1p}}_{p,i[\bar{jb}]} \\
\end{pmatrix}_{PA_O} \\ 
\begin{pmatrix}
    \bm{V}^{2 \mathrm{h1p}}_{p,k[lc]} & \bm{V}^{2 \mathrm{h1p}}_{p,k[\bar{lc}]}
\end{pmatrix}^\dagger_{A_OP} &
\left(\left(\varepsilon^{1h}\right) \oplus_{\text{kron}} -\bm{\tilde{M}}_{jb,jb;lc,lc}\right)_{A_OA_O}
\end{pmatrix}
\] 
\subsubsection{Equivalence to Booth form}
We would need to perform a rotation by
The matrix-vector product becomes:
\[
\bm{\sigma}
= 
\bm{\mathcal{N}}
\begin{pmatrix}
 \begin{pmatrix}
\bm{F}_{oo} & \bm{F}_{ov} \\ \bm{F}_{vo} & \bm{F}_{vv}
\end{pmatrix}_{PP} \begin{pmatrix}
\bm{r}_i \\ \bm{r}_a
\end{pmatrix}_{PT} + \begin{pmatrix}
    \bm{V}^{2 \mathrm{h1p}}_{p,i[jb]} & \bm{V}^{2 \mathrm{h1p}}_{p,i[\bar{jb}]} \\
\end{pmatrix}_{PA_O} \begin{pmatrix}
    \bm{r}_{i[jb]} \\
\bm{\bar{r}}_{i[\bar{jb}]}
\end{pmatrix}_{A_OT} \\[6pt]
 \begin{pmatrix}
    \bm{V}^{2 \mathrm{h1p}}_{p,i[jb]} & \bm{V}^{2 \mathrm{h1p}}_{p,i[\bar{jb}]} \\
\end{pmatrix}^\dag_{A_OP} \begin{pmatrix}
\bm{r}_i \\ \bm{r}_a
\end{pmatrix}_{PT} +
\left(\left(\varepsilon^{1h}\right) \oplus_{\text{kron}} -\bm{\tilde{M}}_{jb,jb;lc,lc}\right)_{A_OA_O} \begin{pmatrix}
    \bm{r}_{i[jb]} \\
\bm{\bar{r}}_{i[\bar{jb}]}
\end{pmatrix}_{A_OT}
\end{pmatrix}
\]
Now we are left to decide what the super-metric $\bm{\mathcal{N}}$ is. We can define it as
\[\bm{\mathcal{N}} =
\begin{pmatrix}
\bm{I_{PP}} & 0 \\[3pt]
0 & \bm{N_{A_OA_O}}
\end{pmatrix}
\]
 with $\bm{N}_{A_OA_O} \equiv \bm{1}_{O} \oplus_{\text{kron}} \bm{N}_{\text{2OV}}=\bm{1}_O \otimes \bm{1}_{2OV} + \bm{1}_{O} \otimes \bm{N}_{\text{2OV}}$. But if we take this definiton, then $\bm{\mathcal{N}}\bm{\mathcal{N}} \neq \bm{1}$, so we can't do $\bm{H} \bm{R} = \bm{\mathcal{N}} \bm{R} \bm{E} \implies \bm{\mathcal{N}} \bm{H} \bm{R} = \bm{R} \bm{E}$, as they claim. Next, assume they made a typo, so do $\bm{N}_{A_OA_O} = \begin{pmatrix}
 \bm{1}_{O^2V} & 0 \\
0 & \bm{-1}_{O^2V}
 \end{pmatrix}$. Now, we have  $\bm{N}_{A_OA_O}\bm{N}_{A_OA_O} = \bm{1}_{A_OA_O}$ and
\begin{align}
&    \bm{\mathcal{N}} \bm{H} \bm{R}
= 
\begin{pmatrix}
\bm{1}_{PP}
 \begin{pmatrix}
\bm{F}_{oo} & \bm{F}_{ov} \\ \bm{F}_{vo} & \bm{F}_{vv}
\end{pmatrix}_{PP} \begin{pmatrix}
\bm{r}_i \\ \bm{r}_a
\end{pmatrix}_{PT} + \bm{1}_{PP}\begin{pmatrix}
    \bm{V}^{2 \mathrm{h1p}}_{p,i[jb]} & \bm{V}^{2 \mathrm{h1p}}_{p,i[{jb}]} \\
\end{pmatrix}_{PA_O} \begin{pmatrix}
    \bm{r}_{i[jb]} \\
\bm{\bar{r}}_{i[\bar{jb}]}
\end{pmatrix}_{A_OT} \\[6pt]
 \bm{N}_{A_OA_O}\begin{pmatrix}
    \bm{V}^{2 \mathrm{h1p}}_{p,i[jb]} & \bm{V}^{2 \mathrm{h1p}}_{p,i[{jb}]} \\
\end{pmatrix}^\dag_{A_OP} \begin{pmatrix}
\bm{r}_i \\ \bm{r}_a
\end{pmatrix}_{PT} +
\bm{N}_{A_OA_O}\left(\left(\varepsilon^{1h}\right) \oplus_{\text{kron}} -\bm{\tilde{M}}_{jb,jb;lc,lc}\right)_{A_OA_O} \begin{pmatrix}
    \bm{r}_{i[jb]} \\
\bm{\bar{r}}_{i[\bar{jb}]}
\end{pmatrix}_{A_OT}
\end{pmatrix} \\
&= \begin{pmatrix}
 \begin{pmatrix}
\bm{F}_{oo}\bm{r}_i + \bm{F}_{ov}\bm{r}_a \\ \bm{F}_{vo}\bm{r}_i + \bm{F}_{vv}\bm{r}_a
\end{pmatrix}_{PT} + \begin{pmatrix}
    \bm{V}^{2 \mathrm{h1p}}_{k,j[lc]}\bm{r}_{i[jb]} + \bm{V}^{2 \mathrm{h1p}}_{k,i[{jb}]}\bm{\bar{r}}_{i[\bar{jb}]}\\ \bm{V}^{2 \mathrm{h1p}}_{c,a[jb]}\bm{r}_{a[jb]} + \bm{V}^{2 \mathrm{h1p}}_{c,a[{jb}]}\bm{\bar{r}}_{a[{jb}]}
\end{pmatrix}_{PT} \\[6pt]
 \bm{N}_{A_OA_O}\begin{pmatrix}
    \bm{V}^{2 \mathrm{h1p}}_{k,i[jb]}\bm{r}_k+ \bm{V}^{2 \mathrm{h1p}}_{a,i[jb]}\bm{r}_a \\
\end{pmatrix}^\dag_{A_OT} +
\bm{N}_{A_OA_O}\left(\left(\varepsilon^{1h}\right) \oplus_{\text{kron}} -\begin{pmatrix}
\bm{\tilde{M}}^{xx}_{jb,lc} & \bm{\tilde{M}}^{xd}_{jb,lc} \\
\bm{\tilde{M}}^{dx}_{jb,lc} & \bm{\tilde{M}}^{dd}_{jb,lc}
\end{pmatrix}\right)_{A_OA_O} \begin{pmatrix}
    \bm{r}_{i[jb]} \\
\bm{\bar{r}}_{i[\bar{jb}]}
\end{pmatrix}_{A_OT}
\end{pmatrix} \\
&= \begin{pmatrix}
 \begin{pmatrix}
\sum_j f_{ij}r_j+ \sum_b f_{ib}r_b + \sum_{jlc} \langle i c | jl \rangle r_{j[lc]} + \sum_{jlc} \langle i c | jl \rangle \bar{r}_{j[lc]} \\ \sum_j f_{aj}r_j+ \sum_b f_{ab}r_b + \sum_{jlc} \langle a c | jl \rangle r_{j[lc]} + \sum_{jlc} \langle a c | jl \rangle \bar{r}_{j[lc]}
\end{pmatrix}_{PT} \\[6pt]
\begin{pmatrix}
    \sum_k \langle ka | ij  \rangle {r}_k + \sum_b \langle ba | ij  \rangle {r}_b \\
-\sum_k \langle ka | ij  \rangle {r}_k - \sum_b \langle ba | ij  \rangle {r}_b
\end{pmatrix}_{A_OT} +
\bm{N}_{A_OA_O}\left(\left(\varepsilon^{1h}\right) \oplus_{\text{kron}} -\begin{pmatrix}
\bm{\tilde{M}}^{xx}_{jb,lc} & \bm{\tilde{M}}^{xd}_{jb,lc} \\
\bm{\tilde{M}}^{dx}_{jb,lc} & \bm{\tilde{M}}^{dd}_{jb,lc}
\end{pmatrix}\right)_{A_OA_O} \begin{pmatrix}
    \bm{r}_{i[jb]} \\
\bm{\bar{r}}_{i[\bar{jb}]}
\end{pmatrix}_{A_OT}
\end{pmatrix} \\
\end{align}
% \subsubsection{Proving equivalence to Booth's ED}
% Again, we will just work with a single channel, the lesser one. In the dRPA case, Booth's formulation for the upfolded Hamiltonian is
% \begin{equation}
%     \bm{H} = \begin{pmatrix} \bm{F} & \bm{W} \\ \bm{W}^{\dagger} & \bm{d} \end{pmatrix}
% \label{eq:booth_upfolded_hamiltonian}
% \end{equation}
% where we have the definitions
% \begin{equation}
%     W_{pkv} = \sum_{ia} (pk|ia) \left( X_{ia}^{v} + Y_{ia}^{v} \right) \quad \text{and} \quad d_{kv,lv'} = \left(\epsilon_k - \Omega_v\right) \delta_{k,l} \delta_{v,v'}
% \end{equation}
% Now Tim's version of the Hamiltonian for the lesser channel is given by
% \begin{equation}
%     \bm{H} = \begin{pmatrix} \bm{F} & \bm{V}^{2 \mathrm{h1p}} & \bm{V}^{2 \mathrm{h1p}} \\ \left(\bm{V}^{2 \mathrm{~h} 1 \mathrm{p}}\right)^{\dagger} & &\\ \left(\bm{V}^{2 \mathrm{~h} 1 \mathrm{p}}\right)^{\dagger} && \bm{C}^{2 \mathrm{hlp}} \end{pmatrix}
% \label{eq:tim_upfolded_hamiltonian}
% \end{equation}
% where the formal defintions are $\bm{C}^{2 \mathrm{hlp}} = \epsilon^{1 \mathrm{~h}} \oplus (-\tilde{\bm{M}})$ and $\bm{V}^{2 \mathrm{~h} 1 \mathrm{p}} = \langle p c | k l \rangle $. Note that now the auxiliary block contain two seperate paricle-hole excitations, which is why the coupling blocks are duplicated. Let us define a unitary rotation $\bm{U} = \bm{1} \otimes \bm{Z}$. Application of this unitary to the Hamiltonian will not change the spectrum and actually transforms the problem into
% \begin{equation}
%     \bm{H}' = \bm{U}^\dag \bm{H} \bm{U} = \begin{pmatrix} \bm{1} & \bm{0}\\ \bm{0}&\bm{Z }^\dag \end{pmatrix} \begin{pmatrix} \bm{F} & \bm{V}^{2 \mathrm{h1p}} & \bm{V}^{2 \mathrm{h1p}} \\ \left(\bm{V}^{2 \mathrm{~h} 1 \mathrm{p}}\right)^{\dagger} & &\\ \left(\bm{V}^{2 \mathrm{~h} 1 \mathrm{p}}\right)^{\dagger} && \bm{C}^{2 \mathrm{hlp}} \end{pmatrix} \begin{pmatrix} \bm{1} & \bm{0} \\ \bm{0}&\bm{Z } \end{pmatrix}
% = \begin{pmatrix} \bm{F} & \bm{V}^{2 \mathrm{h1p}}\bm{Z} & \bm{V}^{2 \mathrm{h1p}}\bm{Z} \\ \bm{Z}^\dag \left(\bm{V}^{2 \mathrm{~h} 1 \mathrm{p}}\right)^{\dagger} & &\\ \bm{Z}^\dag \left(\bm{V}^{2 \mathrm{~h} 1 \mathrm{p}}\right)^{\dagger} && \bm{Z }^\dagger \bm{C}^{2 \mathrm{hlp}} \bm{Z } \end{pmatrix}.
% \label{eq:tim_upfolded_hamiltonian}
% \end{equation}
\subsection{Proving ED equivalence to Booth for the dRPA using Z-vector trick}
For dRPA, we will choose 
\begin{equation}
    \bm{C}^{2 \mathrm{hlp}} = \epsilon^{1 \mathrm{~h}} \oplus (-\bm{M}^{1/2})
\implies C_{i[j a], k[l c]}^{2 \mathrm{~h} 1 \mathrm{p}} = \left[\epsilon _i - \bm{M^{1/2}}_{ja,lc}\right] \delta_{i k}
\end{equation}
 where $\bm{M}$ is the matrix defined in \eqref{eq:berk_gw_M}.
If we choose our unitary transformation to be $\bm{U'} = \bm{1} \otimes \bm{T}$, where $\bm{T}$ is the matrix of eigenvectors of the symmetric formulation as 
\begin{equation}
    \bm{M}\bm{T}= \bm{\lambda } \bm{T}
\label{eq:berk_gw_eigenproblem}
\end{equation}
with the matrix $\bm{M}$ defined as
\begin{equation}
    \bm{M}=(\bm{A}-\bm{B})^{1 / 2}(\bm{A}+\bm{B})(\bm{A}-\bm{B})^{1 / 2}
\label{eq:berk_gw_M}
\end{equation}
with the identities $\bm{\lambda } = \bm{\Omega }^2$ and $\bm{T} = \bm{\Omega }^{1/2} \left[\bm{A}-\bm{B}\right]^{-1/2}\left(\bm{X} + \bm{Y}\right)$, where $\bm{X}$ and $\bm{Y}$ are the excitation and de-excitation vectors, respectively. 

Then, we can apply the unitary transformation to the Hamiltonian in Tim's formulation as
\begin{equation}
    \bm{H'} = \bm{U'}^\dag \bm{H} \bm{U'} = \begin{pmatrix} \bm{1} & \bm{0}\\ \bm{0}&\bm{T }^\dag \end{pmatrix} \begin{pmatrix} \bm{F} & \bm{V}^{2 \mathrm{h1p}}\\ \left(\bm{V}^{2 \mathrm{~h} 1 \mathrm{p}}\right)^{\dagger} & \bm{C}^{2 \mathrm{hlp}} \end{pmatrix} \begin{pmatrix} \bm{1} & \bm{0} \\ \bm{0}&\bm{T } \end{pmatrix} = \begin{pmatrix} \bm{F} & \bm{V}^{2 \mathrm{h1p}}\bm{T}\\ \bm{T}^\dag \left(\bm{V}^{2 \mathrm{~h} 1 \mathrm{p}}\right)^{\dagger} & \bm{T }^\dagger \bm{C}^{2 \mathrm{hlp}} \bm{T } \end{pmatrix}.
\label{eq:tim_upfolded_hamiltonian}
\end{equation}
But now let us try evaluating the term $\bm{T}^\dagger \bm{C}^{2 \mathrm{hlp}} \bm{T }$ in the dRPA case. We have
\begin{align}
    \bm{T}^\dagger \bm{C}^{2 \mathrm{hlp}} \bm{T } &= \bm{T}^\dagger \left(\epsilon^{1 \mathrm{~h}} \otimes \bm{1} + \bm{1} \otimes (-\bm{M}^{1/2})\right) \bm{T } \\
&= \bm{T}^\dagger \left(\epsilon^{1 \mathrm{~h}} \otimes \bm{1}\right) \bm{T } + \bm{T}^\dagger \left(\bm{1} \otimes (-\bm{M}^{1/2})\right) \bm{T } \\
&= \left(\epsilon^{1 \mathrm{~h}} \underbrace{\bm{T}^\dag \bm{T}}_{\bm{\bm{I}}} \right) \otimes \left(\underbrace{\bm{T}^\dag \bm{T}}_{\bm{\bm{I}}}\right) + \left(\underbrace{\bm{T}^\dag \bm{T}}_{\bm{\bm{I}}}\right) \otimes (-\underbrace{\bm{T}^\dag \bm{M}^{1/2} \bm{T}}_{\bm{\Omega }}) \\
\end{align}
Next, let us consider the product $\bm{V}^{2 \mathrm{~h} 1 \mathrm{p}} \bm{T }$. We have
\begin{align}
    \bm{V}^{2 \mathrm{~h} 1 \mathrm{p}} \bm{T } &= \sum_{ia} (pk|ia) \left( X_{ia}^{v} + Y_{ia}^{v} \right) \Omega_{ia;v}^{1/2} \underbrace{\sum_{jb} \left(A_{ia,jb} - B_{ia,jb}\right)^{-1/2}}_{\Delta _{ia;v}^{-1/2}} \\
\end{align}
Now we can approximate $\frac{\Omega _{ia;v}}{\Delta _{ia;v}} \approx \sqrt{1+2\frac{K_{ia,jb; v}}{\Delta _{ia;v}}} \implies \left(\frac{\Omega _{ia;v}}{\Delta _{ia;v}}\right)^{1/2} \approx \sqrt[4]{1 + 2\frac{K_{ia,jb;v}}{\Delta _{ia;v}}}$. For weak to moderately correlated systems, we can assume that the ratio of Coulomb coupling to orbital energy gaps $\frac{K_{ia,jb;v}}{\Delta _{ia;v}}$
is negligible, so we can just assume that $\left(\frac{\Omega _{ia;v}}{\Delta _{ia;v}}\right)^{1/2} \approx 1$. 
\subsection{Evaluating the matrix-vector products}
Now, we can define a vector $\bm{R} = ( r_i,\; r_a,\; r_{i[jb]},\; r_{[jb]a} )$. Application of the Hamiltonian to this vector gives us the matrix-vector product $\bm{H'} \bm{R} = \bm{\sigma }$, where $\bm{\sigma} = ( \sigma_i,\; \sigma_a,\; \sigma_{i[jb]},\; \sigma_{[jb]a} )$. So the one body matrix vector products are the same as in the dTDA case
\begin{align}
\sigma_i &= 
  \sum_{j} f_{i j}\,r_j
  + \sum_{b} f_{i b}\,r_b
  + \sum_{k l c} \bigl\langle i\,c \mid k\,l \bigr\rangle\,r_{k[l c]}
  + \sum_{k c d} \bigl\langle i\,k \mid d\,c \bigr\rangle\,r_{[k c]d}, \\[6pt]
\sigma_a &=
    \sum_{j} f_{a j}\,r_j
    + \sum_{b} f_{a b}\,r_b
    + \sum_{k l c} \bigl\langle a\,c \mid k\,l \bigr\rangle\,r_{k[l c]}
    + \sum_{k c d} \bigl\langle a\,k \mid d\,c \bigr\rangle\,r_{[k c]d}, \\[6pt]
\end{align}
For the two-body matrix vector products, we have
\begin{align}
\sigma_{i[ja]} &=
    \sum_{k} \bigl\langle k\,a \mid i\,j \bigr\rangle\,r_k
    + \sum_{b} \bigl\langle b\,a \mid i\,j \bigr\rangle\,r_b
    + \sum_{klc} C_{i[j a], k[l c]}^{2 \mathrm{~h} 1 \mathrm{p}} r_{i[l c]}
\end{align}
Now, consider just
\begin{equation}
    \sum_{klc} C_{i[j a], k[l c]}^{2 \mathrm{~h} 1 \mathrm{p}} r_{i[l c]}= \sum_{klc} \left[\left(\epsilon_i-\bm{M}^{1/2}_{ja,lc}\right) \delta_{ik} r_{i[l c]} \right] = \sum_{l c} \left(\epsilon_i - \bm{M}^{1/2}_{ja,lc}\right) r_{i[l c]}
\end{equation}
Consider now just the term 
\begin{equation}
    \sum_{l c} \bm{M}^{1/2}_{ja,lc} r_{i[l c]} =  \sum_{l c} \left[\underbrace{\Delta _{ja}^{1/2} \left(\Delta_{ja} + 2(ja|lc)\right)_{ja,lc} \Delta_{ja}^{1/2}}_{\bm{D}}\right]^{1/2} r_{i[l c]}.
\end{equation}
Let us consider instead the similarity transformed $\tilde{\bm{D}} = \Delta^{-1/2} \bm{D} \Delta^{-1/2} = \left(\Delta + 2K\right)$. This is useful because we have $D^{1/2} = \Delta^{1/2}\tilde{D}^{1/2}\Delta^{1/2}=\left(\Delta+2K\right)^{1/2}$ for our SPD D.Now make the change of variables $u = \Delta^{1/2}r \implies r = \Delta^{-1/2}u$. Then we have $\left(\Delta + 2K\right) \Delta^{-1/2}u = \Delta^{1/2} \left[ {I} + \underbrace{2\Delta^{-1/2}K\Delta^{-1/2}}_{L}\right] u$. Again, because we are working with weak to moderately correlated systems, we can assume that L is small, and neglect it for now. So we have
\begin{equation}
    \sum_{l c} \bm{M}^{1/2}_{ja,lc} r_{i[l c]} \approx \sum_{l c} \Delta^{1/2} I \Delta^{1/2} r_{i[l c]} \approx \sum_{l c} \Delta r_{i[l c]}
\end{equation}

We can rewrite this as
So really we have reduced the scaling to $O(O^3V^2)$. If we do a similar thing for the other term, we have
\begin{align}
\sigma_{[ia]b} &= 
    \sum_{j} \bigl\langle j\,i \mid b\,a \bigr\rangle\,r_j
    + \sum_{c} \bigl\langle c\,i \mid b\,a \bigr\rangle\,r_c
    + \sum_{kc} \left[\epsilon_b + \bm{M}^{1/2}_{ia,kc}\right] r_{[kc] b}
\end{align}
which gives the scaling of $O(O^2V^3)$.
\section{Extra}
Recall that we are trying to recover the form 
\begin{equation}
    \bm{H} = \begin{pmatrix} \bm{F} & \bm{W}^<\\ \bm{W}^{<,\dagger} & \bm{d}^< \end{pmatrix}
\label{eq:booth_upfolded_hamiltonian2}
\end{equation}
where now we have the definitions
\begin{align}
    W_{pkv}^{<} &= \sum_{ia} (pk|ia) \left( X_{ia}^{v} + Y_{ia}^{v} \right) \\
    d_{kv,lv'}^{<} &= \left(\epsilon_k - \Omega_v\right) \delta_{k,l} \delta_{v,v'}
\label{eq:booth_definitions2}
\end{align}
where now $\bm{\Omega }$ is taken fromm the eigenproblem $\bm{M}\bm{T}= \bm{\Omega}^2 \bm{T}$, which is the symmetric formulation of the RPA.
Application of downfolding on this upfolded form then gave us the expression
\begin{equation}
    \mathbf{\Sigma }(\omega )=\mathbf{W}^<(\omega \mathbf{I}-{\mathbf{d}^<})^{-1} {\mathbf{W}}^{<,\dagger} + \mathbf{W}^>(\omega \mathbf{I}-{\mathbf{d}^>})^{-1} {\mathbf{W}}^{>,\dagger}
\label{eq:booth_self_energy}
\end{equation}
Meanwhile, downfolding Tim's formulation gives us the expression for the correlation self energy
\begin{equation}
    \Sigma(\omega)= \mathbf{V}^{2 \mathrm{hlp}}\left[\omega \mathbf{1}-\mathbf{C}^{2 \mathrm{hlp}}\right]^{-1}\left[\mathbf{V}^{2 \mathrm{hlp}}\right]^{\dagger} +\mathbf{V}^{2 \mathrm{plh}}\left[\omega \mathbf{1}-\mathbf{C}^{2 \mathrm{plh}}\right]^{-1}\left[\mathbf{V}^{2 \mathrm{plh}}\right]^{\dagger}
\end{equation}
where in the dTDA case the supermatrix is
\begin{equation}
    \mathbf{H}=\left(\begin{array}{ccc}
\mathbf{f} & \mathbf{V}^{2 \mathrm{hlp}} & \mathbf{V}^{2 \mathrm{p} 1 \mathrm{~h}} \\
\left(\mathbf{V}^{2 \mathrm{~h} 1 \mathrm{p}}\right)^{\dagger} & \mathbf{C}^{2 \mathrm{hlp}} & \mathbf{0} \\
\left(\mathbf{V}^{2 \mathrm{plh}}\right)^{\dagger} & \mathbf{0} & \mathbf{C}^{2 \mathrm{plh}}
\end{array}\right)
\end{equation}
with the definitions
\begin{align}
    V_{p, k[l c]}^{2 \mathrm{~h} 1 \mathrm{p}} &= \langle p c | k l \rangle \\
    V_{p,[k c] d}^{2 \mathrm{p} 1 \mathrm{~h}} &= \langle p k | d c \rangle \\
    C_{i[j a], k[l c]}^{2 \mathrm{~h} 1 \mathrm{p}} &= \left[\left(\epsilon_i+\epsilon_j-\epsilon_a\right) \delta_{j l} \delta_{a c}-\langle j c | a l \rangle\right] \delta_{i k} \\
    C_{[i a] b,[k c] d}^{2 \mathrm{p} 1 \mathrm{~h}} &= \left[\left(\epsilon_a+\epsilon_b-\epsilon_i\right) \delta_{i k} \delta_{a c}+\langle a k | i c\rangle\right] \delta_{b d}
\end{align}
Presumably they should be equivalent but I am a little bet confused. In both kisses we have physical and auxiliary spaces. However, Tim uses a bare Coulomb interaction for the coupling matrix, while Booth uses the screened interaction. Then also the definitions of the auxiliary spaces $\bm{d}$ versus $\bm{C}$ are different, where the former makes use of the $\Omega $, assuming that the RPA calculation has already been performed, while the latter is just defined by the A matrix in the dTDA case.  Please show how to get from Tim's formulation to Booth's formulation, or vice versa, and how the two are equivalent. let's just focus on a single channel, so just the lesser channel. Let us start by proving for the dTDA case, where we have the definitions for Booth's formulation
\begin{align}
    W_{pkv}^{<} &= \sum_{ia} (pk|ia) \left( X_{ia}^{v} \right) \\
    d_{kv,lv'}^{<} &= \left(\epsilon_k - \Omega_v\right) \delta_{k,l} \delta_{v,v'}
\end{align}
where in the dtda case, we are working with the eigenproblem $\bm{A}\bm{X} = \bm{\Omega }\bm{X}$.
while in Tim's formulation we have
\begin{align}
    \bm{C}^{2 \mathrm{hlp}} &= \epsilon^{1 \mathrm{~h}} \oplus (-\bm{A}) \\
\end{align}
with the metrics aliments defined as
\begin{align}
    V_{p, k[l c]}^{2 \mathrm{~h} 1 \mathrm{p}} &= \langle p c | k l \rangle \\
    C_{i[j a], k[l c]}^{2 \mathrm{~h} 1 \mathrm{p}} &= \left[\left(\epsilon_i+\epsilon_j-\epsilon_a\right) \delta_{j l} \delta_{a c}-\langle j c | a l \rangle\right] \delta_{i k}
\end{align}
\section{Derivation with RPA screening using Furche's symmetric formulation}
\label{sec:berk_gw}
In the symmetric formulation, we have the hermitian eigenproblem:
\begin{equation}
    \bm{M}\bm{T}= \bm{\lambda } \bm{T}
\label{eq:berk_gw_eigenproblem}
\end{equation}
where we define
\begin{equation}
    \bm{M}=(\bm{A}-\bm{B})^{1 / 2}(\bm{A}+\bm{B})(\bm{A}-\bm{B})^{1 / 2}
\end{equation}
with the $\bm{A}$ and $\bm{B}$ matrices defined as
\begin{align}
    A_{i a, j b} &= \left(\epsilon_a-\epsilon_i\right) \delta_{i j} \delta_{a b}+\mathcal{K}_{i a, b j} \\
    B_{i a, j b} &= \mathcal{K}_{i a, j b}.
\end{align}
From this eigenproblem, we can get the quantities that we need for GW as
\begin{equation}
    \begin{aligned}
& \Omega_n=\sqrt{\lambda_n} \\
& (X+Y)_n=\frac{1}{\sqrt{\Omega_n}}(A-B)^{1 / 2} T_n \\
\end{aligned}
\end{equation}
In direct RPA, one defines a frequency-independent "super-matrix" H,
$$
\mathbf{H}=\left(\begin{array}{ccc}
\mathbf{f} & \mathbf{V}^{2 \mathrm{hlp}} & \mathbf{V}^{2 \mathrm{p} 1 \mathrm{~h}} \\
\left(\mathbf{V}^{2 \mathrm{~h} 1 \mathrm{p}}\right)^{\dagger} & \mathbf{C}^{2 \mathrm{hlp}} & \mathbf{0} \\
\left(\mathbf{V}^{2 \mathrm{plh}}\right)^{\dagger} & \mathbf{0} & \mathbf{C}^{2 \mathrm{plh}}
\end{array}\right)
$$
If the Tamm-Dancoff approximation is used, we have $\mathbf{C}^{2 \text { hlp }}=\varepsilon^{1 \mathrm{~h}} \oplus(-\mathbf{A})$ and $\mathbf{C}^{2 \mathrm{plh}}=\varepsilon^{1 \mathrm{p}} \oplus \mathbf{A}$. 

with matrix elements
$$
\begin{gathered}
V_{p, k[l c]}^{2 \mathrm{~h} 1 \mathrm{p}}=\langle p c \mid k l\rangle \\
V_{p,[k c] d}^{2 \mathrm{p} 1 \mathrm{~h}}=\langle p k \mid d c\rangle \\
C_{i[j a], k[l c]}^{2 \mathrm{~h} 1 \mathrm{p}}=\left[\left(\varepsilon_i+\varepsilon_j-\varepsilon_a\right) \delta_{j l} \delta_{a c}-\langle j c \mid a l\rangle\right] \delta_{i k} \\
C_{[i a] b,[k c] d}^{2 \mathrm{p} 1 \mathrm{~h}}=\left[\left(\varepsilon_a+\varepsilon_b-\varepsilon_i\right) \delta_{i k} \delta_{a c}+\langle a k \mid i c\rangle\right] \delta_{b d}
\end{gathered}
$$
But now, to incorporate the RPA screening, we can instead define $\bm{C}^{2 \text{h1p}} \equiv \varepsilon^{1 \mathrm{~h}} \oplus(-\bm{M})$ and $\bm{C}^{2 \mathrm{p} 1 \mathrm{~h}} \equiv \varepsilon^{1 \mathrm{p}} \oplus \bm{M}$, where $\bm{M}$ is the matrix defined above. We can get the matrix elements of $\bm{M}$ as
\begin{align}
    M_{i a, j b} &= \left(\bm{A}-\bm{B}\right)^{1 / 2}_{i a, j b} \left(\bm{A}+\bm{B}\right)_{i a, j b} \left(\bm{A}-\bm{B}\right)^{1 / 2}_{i a, j b} \\
\\
&= \sqrt{\Delta _{ia}}\left[\left(\epsilon_a-\epsilon_i\right) \delta_{i j} \delta_{a b}+2 \mathcal{K}_{i a, b j}
\right] \sqrt{\Delta _{jb}}. \\
\end{align}
In the second line we have used the fact that in the direct formulation $\bm{A}-\bm{B}$ is purely diagonal, so in fact $\left(\bm{A}-\bm{B}\right)^{1/2}$ is just the square root of the diagonal elements, which we denote as $\sqrt{\Delta_{ia}}$. Furthermore, we have that the interaction kernels $\mathcal{K}_{ia,bj} \equiv \mathcal{K}_{i a, j b}$ for the HF reference, where in the physicist's notation we have $\mathcal{K}_{i a, j b} = \langle i j | a b \rangle$. So the expressions become
\begin{align}
    \bm{C}^{2 \text{h1p}} &= \epsilon^{1 \mathrm{~h}} \oplus (-\bm{M}) \\
& = \begin{pmatrix}
    \epsilon ^{1h} & 0 \\
    0 & -\bm{M}
\end{pmatrix}
\end{align}
and
\begin{align}
    \bm{C}^{2 \mathrm{p} 1 \mathrm{~h}} &= \epsilon^{1 \mathrm{p}} \oplus \bm{M} \\
& = \begin{pmatrix}
    \epsilon ^{1p} & 0 \\
    0 & \bm{M}
\end{pmatrix}
\end{align}
where $\epsilon^{1 \mathrm{~h}}$ and $\epsilon^{1 \mathrm{p}}$ are the diagonal matrices of the one-electron energies for the hole and particle states, respectively. 

% Now, we actually want to evaluate the metrics vector products, so we define a excitation vector $\boldsymbol{R} = ( r_i,\; r_a,\; r_{i[jb]},\; r_{[jb]a} )$ and the matrix-vector product $\bm{H} \boldsymbol{R} = \bm{\sigma }$, where $\bm{\sigma} = ( \sigma_i,\; \sigma_a,\; \sigma_{i[jb]},\; \sigma_{[jb]a} )$. Let us enumerate now what we actually will get:
% \begin{align}
% \sigma_i &= 
%   \sum_{j} f_{i j}\,r_j
%   + \sum_{b} f_{i b}\,r_b
%   + \sum_{k l c} \bigl\langle i\,c \mid k\,l \bigr\rangle\,r_{k[l c]}
%   + \sum_{k c d} \bigl\langle i\,k \mid d\,c \bigr\rangle\,r_{[k c]d}, \\[6pt]
% \sigma_a &= 
%   \sum_{j} f_{a j}\,r_j
%   + \sum_{b} f_{a b}\,r_b
%   + \sum_{j l c} \bigl\langle a\,c \mid j\,l \bigr\rangle\,r_{j[l c]}
%   + \sum_{k c d} \bigl\langle a\,k \mid d\,c \bigr\rangle\,r_{[k c]d}, \\[6pt]
% \sigma_{i[jb]} &
%   \sum_{k} \bigl\langle k\,b \mid i\,j \bigr\rangle\,r_k
%   + \sum_{a} \bigl\langle a\,j \mid b\,i \bigr\rangle\,r_a
%   + M_{i a,\,j b}\,r_{i[jb]}, \\[6pt]
% \sigma_{[jb]a} &= 
%   \sum_{k} \bigl\langle k\,j \mid a\,b \bigr\rangle\,r_k
%   + \sum_{c} \bigl\langle c\,b \mid j\,a \bigr\rangle\,r_c
%   + M_{i a,\,j b}\,r_{[jb]a}.
% \end{align}
% The interesting part will be to figure out how to evaluate elements like $M_{i a, j b}r_{i[jb]}$. We can write this as
% \begin{align}
%     M_{i a, j b} r_{i[jb]} &= \sqrt{\Delta_{ia}} \left[\left(\epsilon_a-\epsilon_i\right) \delta_{i j} \delta_{a b}+2 \mathcal{K}_{i a, b j}\right] \sqrt{\Delta_{jb}} r_{i[jb]} \\
% &= \sqrt{\Delta_{ia}} \left[\left(\epsilon_a-\epsilon_i\right) \delta_{i j} \delta_{a b}+2 \mathcal{K}_{i a, b j}\right] r_{i[jb]} \sqrt{\Delta_{jb}}.
% \end{align}