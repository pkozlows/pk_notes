\documentclass[12pt]{article}
\usepackage{amsmath}
\usepackage{amssymb}
\usepackage{graphicx}
\usepackage{physics}
\usepackage{hyperref}
\usepackage{tikz}
\usepackage{geometry}

\geometry{
    margin=1in,
    textwidth=6.5in,
    textheight=9in
}

\title{$O(N^6)$ GW at the Gamma Point}
\author{Patryk Kozlowski}
\date{\today}

\begin{document}

\maketitle
\section{Molecular implementation}
We follow the Subotnik paper. The first thing to do is to solve the Casida equation for the polarizability in the direct formulation of the RPA:
\begin{equation}\label{eq:kptsCasida}
	\begin{pmatrix}
        \mathbf{A}  & \mathbf{B} \\
        \mathbf{B}^{*} & \mathbf{A}^{*}
    \end{pmatrix}
    \begin{pmatrix}
        \mathbf{X} \\
        \mathbf{Y}
    \end{pmatrix}
    =
    \begin{pmatrix}
        \Omega & 0 \\
        0 & -\Omega
    \end{pmatrix}
    \begin{pmatrix}
        \mathbf{X} \\
        \mathbf{Y}
    \end{pmatrix}
\end{equation}
with $\mathbf{A}$ and $\mathbf{B}$ given by
    \begin{align}\nonumber
    \mathbf{A}_{ia, jb}^{\sigma \sigma ^{\prime}} &= \delta_{ij}\delta_{ab}\delta_{\sigma \sigma ^{\prime}}(\varepsilon_a - \varepsilon_i) + (i_{\sigma}a_{\sigma}|b_{\sigma ^{\prime}}j_{\sigma ^{\prime}}) \\
    \mathbf{B}_{ia, jb}^{\sigma \sigma ^{\prime}} &= (i_{\sigma}a_{\sigma}|j_{\sigma ^{\prime}}b_{\sigma ^{\prime}})
\end{align}
Therefore, with the different spins we form a super matrix:
\begin{equation}
    \begin{pmatrix}
\begin{pmatrix}
    \mathbf{A}_{\alpha \alpha } & \mathbf{A}_{\alpha \beta } \\
    \mathbf{A}_{\beta \alpha } & \mathbf{A}_{\beta \beta }
\end{pmatrix}
&
\begin{pmatrix}
    \mathbf{B}_{\alpha \alpha } & \mathbf{B}_{\alpha \beta } \\
    \mathbf{B}_{\beta \alpha } & \mathbf{B}_{\beta \beta }
\end{pmatrix}
\\
\begin{pmatrix}
    \mathbf{B}_{\alpha \alpha }^{*} & \mathbf{B}_{\alpha \beta }^{*} \\
    \mathbf{B}_{\beta \alpha }^{*} & \mathbf{B}_{\beta \beta }^{*}
\end{pmatrix}
&
\begin{pmatrix}
    \mathbf{A}_{\alpha \alpha }^{*} & \mathbf{A}_{\alpha \beta }^{*} \\
    \mathbf{A}_{\beta \alpha }^{*} & \mathbf{A}_{\beta \beta }^{*}
\end{pmatrix}
\end{pmatrix}
    \begin{pmatrix}
        \mathbf{X}_{\alpha\alpha} & \mathbf{X}_{\alpha\beta}\\
        \mathbf{X}_{\beta\alpha} & \mathbf{X}_{\beta\beta}\\
        \mathbf{Y}_{\alpha\alpha} & \mathbf{Y}_{\alpha\beta}\\
        \mathbf{Y}_{\beta\alpha} & \mathbf{Y}_{\beta\beta}
    \end{pmatrix}
    =
    \begin{pmatrix}
        \Omega & 0 & 0 & 0\\
        0 & -\Omega & 0 & 0\\
        0 & 0 & \Omega & 0\\
        0 & 0 & 0 & -\Omega
    \end{pmatrix}
    \begin{pmatrix}
        \mathbf{X}_{\alpha\alpha} & \mathbf{X}_{\alpha\beta}\\
        \mathbf{X}_{\beta\alpha} & \mathbf{X}_{\beta\beta}\\
        \mathbf{Y}_{\alpha\alpha} & \mathbf{Y}_{\alpha\beta}\\
        \mathbf{Y}_{\beta\alpha} & \mathbf{Y}_{\beta\beta}
    \end{pmatrix}
\end{equation}
This implies a way to get the excitation energies $\Omega^\mu$ and the eigenvectors $\mathbf{X}^\mu$ and $\mathbf{Y}^\mu$ for a certain spin channel $\sigma $. Next, for each spin channel, we need to formulate the matrix $\mathbf{M}^{\mu }$, which is used to form the transition densities.
\begin{equation}
M_{i a j b}^{\mu }=X_{i a}^{\mu } X_{j b}^{\mu }+X_{i a}^{\mu } Y_{j b}^{\mu }+Y_{i a}^{\mu } X_{j b}^{\mu }+Y_{i a}^{\mu } Y_{j b}^{\mu }
\end{equation}
With these quantities, we can then form the self energy for the given spin channel:
\begin{equation}
\Sigma_{p q}^c\left(\omega \right)= \sum_{j b k c} \sum_{\mu }\left(\sum_i \frac{(i p \mid j b)(i q \mid k c)}{\omega -\Omega_{\mu }-\varepsilon_i^{\mathrm{MF}}-\mathrm{i} \eta}\right.
\left.+\sum_a \frac{(a p \mid j b)(a q \mid k c)}{\omega +\Omega_{\mu }-\varepsilon_a^{\mathrm{MF}}+\mathrm{i} \eta}\right) M_{j b k c}^{\mu }
\end{equation}
But in solving the case partial equation, we are just interested in the real, diagonal part of the self energy, so this reduces to:
\begin{equation}
\Sigma_{p p}^c\left(\omega \right)= \sum_{j b k c} \sum_{\mu }\left(\sum_i \frac{(i p \mid j b)(i p \mid k c)}{\omega -\Omega_{\mu }-\varepsilon_i^{\mathrm{MF}}} + \sum_a \frac{(a p \mid j b)(a p \mid k c)}{\omega +\Omega_{\mu }-\varepsilon_a^{\mathrm{MF}}}\right) M_{j b k c}^{\mu }
\end{equation}

\end{document}