\documentclass{beamer}
\usetheme{Madrid}
\usepackage{amsmath}
\usepackage{amssymb}
\usepackage{graphicx}
\usepackage{physics}
\usepackage{tikz}
\usepackage{simpler-wick}
\usepackage{cancel}
\usepackage{geometry}
\usepackage{algorithm}
\usepackage{algorithmic}
\usepackage[numbers]{natbib}
\bibliographystyle{plainnat}  % or another style like 'apsrev4-2'
\usepackage{hyperref}
\usepackage{bm}
\usepackage{xcolor}
\usepackage{listings}
\usepackage{tikz}
\graphicspath{{figs/}}
% \bibfile{../src/citations.bib}
\title{Electrochemical methods for water treatment}
\author{Patryk Kozlowski}
\date{\today}

\hypersetup{
    colorlinks=true,    % false: boxed links; true: colored links
    linkcolor=blue,     % color of internal links
    urlcolor=cyan       % color of external links
}
\setbeamertemplate{section in toc}[sections numbered]

% % Automatically create section title slides
% \AtBeginSection[]{
%   \begin{frame}
%   \frametitle{Outline}
%   \tableofcontents[currentsection]
%   \end{frame}
% }

% Alternative: Simple section title slide
% \AtBeginSection[]{
%   \begin{frame}
%   \centering
%   \LARGE \insertsectionhead
%   \end{frame}
% }

\begin{document}

\begin{frame}
    \titlepage
\end{frame}
\begin{frame}
  \frametitle{Outline}
  \tableofcontents
\end{frame}
\section{Motivation}
\begin{frame}
    \frametitle{Why the status quo is not enough}
    \begin{figure}
        \centering
        % Try the image in the figs/ directory (graphicspath set above); if it's missing, show a placeholder box instead of failing compilation.
        \IfFileExists{figs/ro_vs_echem.jpeg}{%
            \includegraphics[width=0.6\textwidth]{ro_vs_echem.jpeg}%
        }{%
            \fbox{\parbox[b][0.4\textheight][c]{0.7\textwidth}{\centering Image `ro_vs_echem.jpeg' not found}}%
        }
    \end{figure}
\end{frame}
\section{Thermodynamic efficiency of different methods}
\begin{frame}
    \frametitle{Thermodynamic efficiency of different methods}
We can plot $\mathcal{P}=\frac{V_{\mathrm{D}}}{n A}$ vs. $\eta=\Delta \hat{G} / \hat{E}$
    \begin{figure}
        \centering
        \includegraphics[width=0.7\textwidth]{thermo.jpeg}
    \end{figure}
\end{frame}
\begin{frame}
    \frametitle{What electrosorption looks like}
    \begin{figure}
        \centering
        \includegraphics[width=0.5\textwidth]{sorp.png}
    \end{figure}
\end{frame}
\section{Kinetics of capacitive deionization}

\begin{frame}
    \frametitle{A coupled ion-electron transfer mechanism for CDI}
    \begin{figure}
        \centering
        \includegraphics[width=\textwidth]{marcus.jpg}
    \end{figure}
\centering
$\lambda_{\mathrm{o}}=\frac{e^2}{8 \pi \varepsilon_0 k_{\mathrm{B}} T}\left(\frac{1}{a_0}-\frac{1}{2 d}\right)\left(\frac{1}{\varepsilon_{\mathrm{op}}}-\frac{1}{\varepsilon_{\mathrm{s}}}\right)$
\end{frame}
\section{Transport: Identifying mechanisms of electrosorption}
\begin{frame}
    \frametitle{Identifying Faradaic vs. electrostatic electrosorption}
    \begin{figure}
        \centering
        \includegraphics[width=0.7\textwidth]{volt.jpg}
    \end{figure}
\end{frame}

\end{document}