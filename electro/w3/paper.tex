\documentclass[12pt]{article}
\usepackage[margin=1in]{geometry}
\usepackage{setspace}
\usepackage{parskip}
\usepackage{enumitem}
\usepackage{titlesec}
\usepackage{hyperref}

\setstretch{1.1}
\titleformat{\section}{\large\bfseries}{}{0pt}{}

\begin{document}

\begin{center}
    {\LARGE \textbf{Paper Analysis Notes}}\\[6pt]
    \today
\end{center}


\section*{Authors}
\textit{What fields are they from? How will that inform their approach to data collection, analysis, and interpretation?}\\[4pt]
The principal investigator's area of focus is heterogeneous catalysis. Another author's expertise seems to be in XPS. So they seem to be coming to electrochemistry from a slightly different perspective, in contrast with the authors we read before who came from this field.

\section*{Summary}
\textit{Provide a 2--3 sentence summary of the paper and what it is trying to communicate.}\\[4pt]
The paper is trying to communicate the results of a study on the electrochemical reduction of carbon dioxide on metallic copper surfaces. Many possible products are observed, with their quantity depending on the potential used. Deriving from the data, possible mechanisms are proposed.

\section*{Main points}
\textit{What are the main points that the paper is trying to convey? What data is provided to support those points? What data is provided that may contradict/support alternative hypotheses? What additional data would help convey their points?}\\[4pt]
The main points of the paper are that the electrochemical reduction of carbon dioxide on metallic copper surfaces can yield a variety of products, and that the distribution of these products is influenced by the applied potential. Plots of the partial current density are used to support their hypothetical mechanisms. Incorporating more computational simulations could turn the speculation about the mechanisms into more solid conclusions.

\section*{Motivation}
\textit{What is the motivation for this paper? Is the motivation convincing or do you think there are alternative motivations for this work?}\\[4pt]
The stated motivation for this paper is that they propose a novel experimental mechanism for investigating the electrochemical reduction of carbon dioxide on copper surfaces. Perhaps I have not read enough the literature in the field, but it seems to me like they also want to showcase the utility of a analysis based on surface science to electrochemical systems.

\section*{Extraneous Information}
\textit{What data is provided that is unnecessary to the main points of the paper? Are there specific figures or discussion points that should be moved from the main text to the SI? Is there anything in the SI that should be moved to the main text?}\\[4pt]
Details about the experimental setup could be moved to an SI. Personally, I did not find much benefit in reading about this as I am not a practitioner of the field, but I imagine for some experimentalists that would be very useful, and therefore they would be motivated to look through this SI.

\section*{Flow}
\textit{What jumped out to you (good or bad) about how this paper was written?}\\[4pt]
As stated above, too much time was spent on detailing the experimental setup, which was not useful for me. The plotting analysis was very well done and easy to follow.

\section*{Future Directions}
\textit{As researchers, we should always be thinking about next steps. What, if any, next papers should follow this one? Is there any additional analysis and conclusions that can be drawn from their data?}\\[4pt]
A more extensive computational modeling would give a definitive answer as to the validity of their proposed mechanisms. I imagine that it would be best to first do the low-hanging fruit of molecular dynamics simulations, then to do DFT if there is any ambiguity, and finally some post-mean field studies if necessary for a particular step.

\end{document}
