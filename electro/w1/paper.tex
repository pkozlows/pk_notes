\documentclass[12pt]{article}
\usepackage[margin=1in]{geometry}
\usepackage{setspace}
\usepackage{parskip}
\usepackage{enumitem}
\usepackage{titlesec}
\usepackage{hyperref}

\setstretch{1.1}
\titleformat{\section}{\large\bfseries}{}{0pt}{}

\begin{document}

\begin{center}
    {\LARGE \textbf{Paper Analysis Notes}}\\[6pt]
    \today
\end{center}


\section*{Authors}
\textit{What fields are they from? How will that inform their approach to data collection, analysis, and interpretation?}\\[4pt]
The authors are both practitioners in the field of electrochemistry, as professors within this specific field. One of them owns an electrochemical company.

\section*{Summary}
\textit{Provide a 2--3 sentence summary of the paper and what it is trying to communicate.}\\[4pt]
The paper provides a review of the electrochemistry field as of 2020. It starts of by explaining the advantages of electrochemistry with respect to the pursuit of decarbonizing chemical industry. It goes on to talk about the open problems in the field.

\section*{Main points}
\textit{What are the main points that the paper is trying to convey? What data is provided to support those points? What data is provided that may contradict/support alternative hypotheses? What additional data would help convey their points?}\\[4pt]
The paper is trying to convey the utility of electrochemistry probably to a technical audience, but not necessarily field experts. As this is a review article that attempts to take a birds eye view, the data provided most goes to show how the electoral chemistry processes can provide savings on the large scale, i.e. so it is not telling, for example, what voltage is needed for every particular process.

\section*{Motivation}
\textit{What is the motivation for this paper? Is the motivation convincing or do you think there are alternative motivations for this work?}\\[4pt]
The motivation for this paper is to provide a review of the field of electrochemistry and to highlight the open problems in the field. For example, a computational chemist like myself might read this paper to learn about a problem that is worth spending time on. For instance, my PI is currently doing a project that aims to compute quantities relevant for electrocatalysis accurately and with low computational cost. This type of paper might give some ideas about what the relevant electrochemical processes are, whose quantities deserve the quantum chemist's attention.

\section*{Extraneous Information}
\textit{What data is provided that is unnecessary to the main points of the paper? Are there specific figures or discussion points that should be moved from the main text to the SI? Is there anything in the SI that should be moved to the main text?}\\[4pt]
I did not see any reference to a supplemental information section. However, I think the last paragraph, in particular, was kind of long and rambling, so either it should have been split into subsections or a version of it should have been moved into another document.

\section*{Flow}
\textit{What jumped out to you (good or bad) about how this paper was written?}\\[4pt]
I thought they did a good job summarizing what the advantages of electrochemistry are for someone who is not an expert in the field, like me. As I mentioned in the previous point, the last section was tough to digest, and I think it would have been better to break it up into more manageable subsections with headings. I am accustomed to reading papers in my field where there are occasional breaks provided by headings, figures, or tables, but the final stretch had none of them.

\section*{Future Directions}
\textit{As researchers, we should always be thinking about next steps. What, if any, next papers should follow this one? Is there any additional analysis and conclusions that can be drawn from their data?}\\[4pt]
A more comprehensive discussion of the future directions for the field should follow this paper. It devoted only three-quarters of a page to this, but I think the topic warrants a more in-depth discussion.

\end{document}
