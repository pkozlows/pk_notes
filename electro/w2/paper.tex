\documentclass[12pt]{article}
\usepackage[margin=1in]{geometry}
\usepackage{setspace}
\usepackage{parskip}
\usepackage{enumitem}
\usepackage{titlesec}
\usepackage{hyperref}

\setstretch{1.1}
\titleformat{\section}{\large\bfseries}{}{0pt}{}

\begin{document}

\begin{center}
    {\LARGE \textbf{Paper Analysis Notes}}\\[6pt]
    \today
\end{center}


\section*{Authors}
\textit{What fields are they from? How will that inform their approach to data collection, analysis, and interpretation?}\\[4pt]
The authors are both practitioners in the field of electrochemistry. I would say that because they are at primarily scientific research institutions like Caltech and MIT, they have a desire for quantitative rigor.

\section*{Summary}
\textit{Provide a 2--3 sentence summary of the paper and what it is trying to communicate.}\\[4pt]
This paper starts by introducing a set of equations that only involve dimensionless variables, with the aim of providing a unified framework for describing the thermodynamics of industrial chemical processes. It then goes on to use this framework to produce plots that provide an efficient visualization of the thermodynamic landscape of such processes. The claim is that making such plots is only made possible through the introduction of these dimensionless variables.

\section*{Main points}
\textit{What are the main points that the paper is trying to convey? What data is provided to support those points? What data is provided that may contradict/support alternative hypotheses? What additional data would help convey their points?}\\[4pt]
The paper is trying to convey that by writing the traditional thermodynamic equations in terms of dimensionless variables, one is able to gain insight into the thermodynamic landscape of reactions, which would be useful in the early stages of research for new reactions, when the technoeconomic analyses are not yet available. The plots support this point. Giving examples of how this kind of plotting analysis has enabled the efficient translation of lab-scale research to industrial scale would help convey their points, but I understand that this might not be possible if this was the original work to propose this kind of thing.

\section*{Motivation}
\textit{What is the motivation for this paper? Is the motivation convincing or do you think there are alternative motivations for this work?}\\[4pt]
The motivation for this paper is to introduce a new way of visualizing the thermodynamic landscape of new chemical reactions, to provide a foresight into whether it would be worth pursuing them for industrial applications. This is convincing, as the authors make a good case that this kind of analysis would be useful in the early stages of research for new reactions, when the technoeconomic analyses are not yet available.

\section*{Extraneous Information}
\textit{What data is provided that is unnecessary to the main points of the paper? Are there specific figures or discussion points that should be moved from the main text to the SI? Is there anything in the SI that should be moved to the main text?}\\[4pt]

\section*{Flow}
\textit{What jumped out to you (good or bad) about how this paper was written?}\\[4pt]

\section*{Future Directions}
\textit{As researchers, we should always be thinking about next steps. What, if any, next papers should follow this one? Is there any additional analysis and conclusions that can be drawn from their data?}\\[4pt]

\end{document}
