\documentclass[12pt]{article}
\usepackage[margin=1in]{geometry}
\usepackage{setspace}
\usepackage{parskip}
\usepackage{enumitem}
\usepackage{titlesec}
\usepackage{hyperref}

\setstretch{1.1}
\titleformat{\section}{\large\bfseries}{}{0pt}{}

\begin{document}

\begin{center}
    {\LARGE \textbf{Paper Analysis Notes}}\\[6pt]
    \today
\end{center}


\section*{Authors}
\textit{What fields are they from? How will that inform their approach to data collection, analysis, and interpretation?}\\[4pt]
There are two corresponding authors; one does electrochemistry and one specializes generally in catalysis.

\section*{Summary}
\textit{Provide a 2--3 sentence summary of the paper and what it is trying to communicate.}\\[4pt]
Three electrochemical reactions are presented, each with a different mechanism. It is hypothesized how the convective and diffusive mass transport phenomena influence the rate and yield of each reaction. Then, the reaction is run in an advanced electrochemical cell that is designed to enhance the hypothesized mass transport, leading to noticeable improvements in performance, thus validating the hypotheses.

\section*{Main points}
\textit{What are the main points that the paper is trying to convey? What data is provided to support those points? What data is provided that may contradict/support alternative hypotheses? What additional data would help convey their points?}\\[4pt]
That consideration of how mass transport affects the mechanism of an electrochemical reaction can lead to improved performance. From some friends doing organic chemistry, I have heard that these days, usually a paper is not published without a supporting theory section. Having a theory section in this paper would greatly strengthen their hypotheses.

\section*{Motivation}
\textit{What is the motivation for this paper? Is the motivation convincing or do you think there are alternative motivations for this work?}\\[4pt]
The stated motivation is to show how consideration of mass transport phenomena can lead to improved electrochemical reaction performance.

\section*{Extraneous Information}
\textit{What data is provided that is unnecessary to the main points of the paper? Are there specific figures or discussion points that should be moved from the main text to the SI? Is there anything in the SI that should be moved to the main text?}\\[4pt]
The figures take up a lot of space. Maybe the figure 5 could be moved to the SI, because it does not seem relevant unless somebody is really interested in the specifications of the instrumentation come in which case they would analyze the SI anyway.

\section*{Flow}
\textit{What jumped out to you (good or bad) about how this paper was written?}\\[4pt]
There was a good logical flow, in that first the proposed reaction mechanism was introduced in the potential effect of mass transport considered. Then, it was demonstrated that intelligent design of the reactor could enhance the production rate of the given chemical by up to an order of magnitude; this final section really served to drive their point home.

\section*{Future Directions}
\textit{As researchers, we should always be thinking about next steps. What, if any, next papers should follow this one? Is there any additional analysis and conclusions that can be drawn from their data?}\\[4pt]
Practically speaking, a next step would be to run a bunch of industrial reactions where a mass transport mechanism is known in these enhanced reactors to see if the performance improvements can be achieved at scale, as this paper would lead you to believe it should be the case. Theoretically speaking, a next step would be to develop a more rigorous ab initio model of how mass transport phenomena interact with electrochemical reaction mechanisms to better predict which reactions would benefit most from enhanced mass transport.

\end{document}
