\documentclass[12pt]{article}
\usepackage[margin=1in]{geometry}
\usepackage{setspace}
\usepackage{parskip}
\usepackage{enumitem}
\usepackage{titlesec}
\usepackage{hyperref}

\setstretch{1.1}
\titleformat{\section}{\large\bfseries}{}{0pt}{}

\begin{document}

\begin{center}
    {\LARGE \textbf{Paper Analysis Notes}}\\[6pt]
    \today
\end{center}


\section*{Authors}
\textit{What fields are they from? How will that inform their approach to data collection, analysis, and interpretation?}\\[4pt]
They are electro chemists, but it seems that they put more of a focus on understanding the fundamentals of this process, and not focusing so much on industrial applications. One of them is a photo chemist so he probably knows what experimental techniques should be used in what needed to be developed in this work.

\section*{Summary}
\textit{Provide a 2--3 sentence summary of the paper and what it is trying to communicate.}\\[4pt]
The Co2R is studied when taking place on aGDE. It is verified, that the morphology of the electrode is important, and specifically in the electrode these churches are observed, which are shown to play a bit roll in decreasing the poh, thus increasing the selectivity towards C2+ products. 

\section*{Main points}
\textit{What are the main points that the paper is trying to convey? What data is provided to support those points? What data is provided that may contradict/support alternative hypotheses? What additional data would help convey their points?}\\[4pt]
That one should pay attention to the microenvironment of an electrode in heterogeneous catalysis because doing so can lead to novel insights about the mechanism. With a theoretical paper like this one, I can see how if somebody more application minded were to read, it would shape their thinking and possibly lead them to new ideas about how to do better electrochemistry.

\section*{Motivation}
\textit{What is the motivation for this paper? Is the motivation convincing or do you think there are alternative motivations for this work?}\\[4pt]

\section*{Extraneous Information}
\textit{What data is provided that is unnecessary to the main points of the paper? Are there specific figures or discussion points that should be moved from the main text to the SI? Is there anything in the SI that should be moved to the main text?}\\[4pt]
Much of the letter parts of the paper were devoted to demonstrating what kinds of insights can be gained for computational simulation, but based of what we just learned about the electrode double layer, which they mention themselves the simulations don't take into account, I don't know if all of this information is very useful.

\section*{Flow}
\textit{What jumped out to you (good or bad) about how this paper was written?}\\[4pt]
I agree with most of their choices about what to include in the paper, and what to relegate to the SI. For the computational simulations only the term metaphysics simulations was used in the paper and if the reader wanted to learn more, they were referred to the SI. This is a good choice because most readers would not be interested in the details of the simulations, but those who are can find them in the SI.

\section*{Future Directions}
\textit{As researchers, we should always be thinking about next steps. What, if any, next papers should follow this one? Is there any additional analysis and conclusions that can be drawn from their data?}\\[4pt]

\end{document}
