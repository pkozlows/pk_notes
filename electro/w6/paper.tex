\documentclass[12pt]{article}
\usepackage[margin=1in]{geometry}
\usepackage{setspace}
\usepackage{parskip}
\usepackage{enumitem}
\usepackage{titlesec}
\usepackage{hyperref}

\setstretch{1.1}
\titleformat{\section}{\large\bfseries}{}{0pt}{}

\begin{document}

\begin{center}
    {\LARGE \textbf{Paper Analysis Notes}}\\[6pt]
    \today
\end{center}


\section*{Authors}
\textit{What fields are they from? How will that inform their approach to data collection, analysis, and interpretation?}\\[4pt]
The first two others are electro chemists. The third does a lot of electrochemistry, but his training is in engineering, so maybe this explains the abundance of numbers in the text.

\section*{Summary}
\textit{Provide a 2--3 sentence summary of the paper and what it is trying to communicate.}\\[4pt]
This paper is giving an analysis of the kinetics of some chosen important reactions important to electrochemical conversion. It is operating at the microscopic scale and those to showcase the utility of a Tafel slope analysis to understand the reaction mechanisms.

\section*{Main points}
\textit{What are the main points that the paper is trying to convey? What data is provided to support those points? What data is provided that may contradict/support alternative hypotheses? What additional data would help convey their points?}\\[4pt]
It describe the dependence of the Tafel slope on the coverage of the formed surface species in the reactions considered and address
the applicability of the Butler-Volmer equation in describing electrocatalytic kinetics. Numerous case studies of reaction examples, complete with calculation of the Tafel slopes, are given to support these points. This paper would benefit from improved and more organized presentation of the data.

\section*{Motivation}
\textit{What is the motivation for this paper? Is the motivation convincing or do you think there are alternative motivations for this work?}\\[4pt]
The stated motivation is to show the utility of an analysis of Tafel slopes to understand the mechanisms of electrochemical reactions. The goal of the field is for the microkinetic analyses to establish a direct connection between catalyst design and function. As an example consider the final sentence of the first paragraph on page 18: the development of improved electrocatalysts should aim to identify catalysts that proceed via unexpected elementary steps, which breaks the volcano plot trend with one activity descriptor, such as metal-adsorbate bond strength.  Tafel slope analysis is a proposed way to do this. 

\section*{Extraneous Information}
\textit{What data is provided that is unnecessary to the main points of the paper? Are there specific figures or discussion points that should be moved from the main text to the SI? Is there anything in the SI that should be moved to the main text?}\\[4pt]
In the discussion of overall reaction rates, they could have just given the chemical equation, a few sentences of discussion, and then the expression of electric current, moving the rest into the SI. Question 96 is a great example; the average reader doesn't need to know the details of how this expression came about. Additionally, in numerous places in the text, they quote a lot of numbers, and this information could be better presented in a table.

\section*{Flow}
\textit{What jumped out to you (good or bad) about how this paper was written?}\\[4pt]
The presentation of the results left much to be desired. I wonder who the intended audience of this paper is; does the practicing electrochemist find this detailed analysis helpful? I guess it does provide a connection between the microscopic reaction mechanism and the macroscopic data that would be observed in experiment. Is there a better way to do this though?

\section*{Future Directions}
\textit{As researchers, we should always be thinking about next steps. What, if any, next papers should follow this one? Is there any additional analysis and conclusions that can be drawn from their data?}\\[4pt]
Examining the dependence of the Tafel slope on temperature is left to be explored in a different study.

\end{document}
