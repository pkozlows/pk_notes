\section{Formulation for the TDA}
\subsection{Definitions}
For simplicity, we will just work with a single channel, the lesser one. In the dTDA case, Booth's formulation for the upfolded Hamiltonian is
\begin{equation}
    \bm{H} = \begin{pmatrix} \bm{F} & \bm{W} \\ \bm{W}^{\dagger} & \bm{d} \end{pmatrix}
\label{eq:booth_hamiltonian}
\end{equation}
where we have the definitions
\begin{equation}
    W_{pkv} = \sum_{ia} (pk|ia) X_{ia}^{v} \quad \text{and} \quad d_{kv,lv'} = \left(\epsilon_k - \Omega_v\right) \delta_{k,l} \delta_{v,v'}
\label{eq:booth_definitions}
\end{equation}
Now, Tim's version of the Hamiltonian is given by
\begin{equation}
    \bm{H} = \begin{pmatrix} \bm{F} & \bm{V}^{2 \mathrm{h1p}}\\ \left(\bm{V}^{2 \mathrm{~h} 1 \mathrm{p}}\right)^{\dagger} & \bm{C}^{2 \mathrm{hlp}} \end{pmatrix}
\label{eq:tim_hamiltonian}
\end{equation}
where the definitions of the matrix elements are
\begin{align}
    V_{p, k[l c]}^{2 \mathrm{~h} 1 \mathrm{p}} &= \langle p c | k l \rangle \equiv (pk|lc) \\
    C_{i[j a], k[l c]}^{2 \mathrm{~h} 1 \mathrm{p}} &= \left[\left(\epsilon_i+\epsilon_j-\epsilon_a\right) \delta_{j l} \delta_{a c}-\langle j c | a l \rangle\right] \delta_{i k}
\label{c}
\end{align}
and in particular, we have a definition 
\begin{equation}
    \bm{C}^{2 \mathrm{hlp}} = \epsilon^{1 \mathrm{~h}} \oplus (-\bm{A}) = \epsilon^{1 \mathrm{~h}} \otimes \bm{1} + \bm{1} \otimes (-\bm{A})
\label{c_kron}
\end{equation}
Let us show how \ref{c} comes from \ref{c_kron}. If $A_{[ja],[lc]} = (\epsilon_j - \epsilon_a)\, \delta_{jl} \delta_{ac} + \langle jc | al \rangle$, then we can write
\begin{align}
    C_{i[ja],k[lc]}^{2 \mathrm{~h} 1 \mathrm{p}} &= \left[\epsilon_i \delta_{ik} \delta_{jl} \delta_{ac} + (-A)_{[ja],[lc]}\right] \\
    &= \delta_{ik} \left[\epsilon_i\, \delta_{jl} \delta_{ac} + (-A)_{[ja],[lc]}\right] \\
    &= \delta_{ik} \left[ \left(\epsilon_i + \epsilon_j - \epsilon_a \right) \delta_{jl} \delta_{ac} - \langle jc | al \rangle \right]
\label{tda_kron}
\end{align}
% Let us define:
% \begin{itemize}
%     \item \( \epsilon^{1\mathrm{h}} \): a diagonal matrix with eigenvalues \( \epsilon_i \), acting on the hole index \( i \),
%     \item \( \bm{A} \): a matrix acting on the composite index \( [ja] \), with elements:
%     \[
%     A_{[ja],[lc]} = (\epsilon_j - \epsilon_a)\, \delta_{jl} \delta_{ac} + \langle jc | al \rangle
%     \]
% \end{itemize}

% Then:
% \[
% - \bm{A}_{[ja],[lc]} = -(\epsilon_j - \epsilon_a)\, \delta_{jl} \delta_{ac} - \langle jc | al \rangle
% \]

% So the full matrix \( \bm{C}^{2\mathrm{h}1\mathrm{p}} \) becomes:
% \[
% C_{i[ja],k[lc]} = \delta_{ik} \left[ \epsilon_i\, \delta_{jl} \delta_{ac} + (-\bm{A})_{[ja],[lc]} \right]
% \]

% This has the structure of a Kronecker sum:
% \[
% \bm{C}^{2\mathrm{h}1\mathrm{p}} = \epsilon^{1\mathrm{h}} \otimes \bm{1}_{ja} + \bm{1}_i \otimes (-\bm{A})
% \]
\subsection{Similarity transformation}
\label{sec:gwtda_sim}
Let us define a unitary rotation $\bm{U} = \bm{1}_P \oplus_{\text{diag}} \left( \bm{1}_O \otimes \bm{X}_{OV} \right)$. Application of this unitary to the Hamiltonian will not change the spectrum and actually transforms the problem into
\begin{align}
    \bm{H}^{2h1p,\prime} &= \bm{U}^\dag \bm{H}^{2h1p} \bm{U} = \begin{pmatrix} \bm{1}_{P} & \bm{0}\\ \bm{0}&\left( \bm{1}_O \otimes \bm{X }_{OV} \right) \end{pmatrix}^\dag \begin{pmatrix} \bm{F}_P & \bm{V}^{2 \mathrm{h1p}}_{P,O^2V}\\ \left(\bm{V}^{2 \mathrm{~h} 1 \mathrm{p}}\right)^{\dagger}_{O^2V,P} & \bm{C}^{2 \mathrm{hlp}}_{O^2V,O^2V} \end{pmatrix} \begin{pmatrix} \bm{1}_P & \bm{0}\\ \bm{0}&\left( \bm{1}_O \otimes \bm{X }_{OV} \right) \end{pmatrix}\\
& = \begin{pmatrix} \bm{F}_P & \bm{V}^{2 \mathrm{h1p}}_{P,O^2V}\left( \bm{1}\otimes \bm{X }\right)_{O^2V,O^2V}\\ \left(\bm{V}^{2 \mathrm{h1p}}_{P,O^2V}\left( \bm{1} \otimes \bm{X } \right)_{O^2V,O^2V}\right)^{\dagger} & \left( \left( \bm{1} \otimes \bm{X } \right)_{O^2V,O^2V}\right)^\dagger \bm{C}^{2 \mathrm{hlp}}_{O^2V,O^2V} \left( \left( \bm{1} \otimes \bm{X } \right)_{O^2V,O^2V} \right) \end{pmatrix}.
\label{eq:booth_upfolded_hamiltonian}
\end{align}
Now let us evaluate $\left( \bm{1} \otimes \bm{X } \right)^\dagger \bm{C}^{2 \mathrm{hlp}} \left( \bm{1} \otimes \bm{X } \right)$ in the TDA case. We have
\begin{align}
    \left( \bm{1} \otimes \bm{X } \right)^\dagger \bm{C}^{2 \mathrm{hlp}} \left( \bm{1} \otimes \bm{X } \right) &= \left( \bm{1} \otimes \bm{X } \right)^\dagger \left[\epsilon^{1 \mathrm{~h}} \otimes \bm{1} + \bm{1} \otimes (-\bm{A})\right] \left( \bm{1} \otimes \bm{X } \right) \\
&= \left( \bm{1} \otimes \bm{X }^\dagger \right) \left[\bm{\epsilon}^{1 \mathrm{~h}} \otimes \bm{1}\right] \left( \bm{1} \otimes \bm{X } \right) + \left( \bm{1} \otimes \bm{X }^\dagger \right) \left[\bm{1} \otimes (-\bm{A})\right] \left( \bm{1} \otimes \bm{X } \right)\\
&= \bm{1}\bm{\epsilon}^{1 \mathrm{~h}}\bm{1} \otimes \bm{X }^\dagger \bm{1} \bm{X } + \bm{1} \bm{1} \bm{1} \otimes (-\bm{X }^\dagger \bm{A} \bm{X }) \\
&= \bm{\epsilon}^{1 \mathrm{~h}} \otimes \bm{1} + \bm{1} \otimes (-\bm{X }^\dagger \bm{A} \bm{X }) \\
&= \bm{\epsilon}^{1 \mathrm{~h}} \otimes \bm{1} + \bm{1} \otimes (-\bm{\Omega }) \\
&= \bm{\epsilon}^{1 \mathrm{~h}} \oplus (-\bm{\Omega }) 
% &= \bm{X}^\dagger \left(\epsilon^{1 \mathrm{~h}} \otimes \bm{1}\right) \bm{X } + \bm{X}^\dagger \left(\bm{1} \otimes (-\bm{A})\right) \bm{X } \\
% &= \left(\epsilon^{1 \mathrm{~h}} \underbrace{\bm{X}^\dag \bm{X}}_{\bm{I}} \right) \otimes \left(\underbrace{\bm{X}^\dag \bm{X}}_{\bm{I}}\right) + \left(\underbrace{\bm{X}^\dag \bm{X}}_{\bm{I}}\right) \otimes (-\underbrace{\bm{X}^\dag \bm{A} \bm{X}}_{\bm{\Omega }}) \\
% &= \epsilon^{1 \mathrm{~h}} \oplus (-\bm{\Omega }) \equiv \bm{d}^{<}
\end{align}
where we have used the fact that $\bm{X}^\dag \bm{X} = \bm{1}$, since $\bm{X}$ is unitary. Similarly, we can evaluate the other term
\begin{align}
    \bm{V}^{2 \mathrm{h1p}}_{P,O^2V}\left( \bm{1}\otimes \bm{X }\right)_{O^2V,O^2V} &= \sum_{k'lc} (pk|lc) X_{lc}^{v} \delta_{kk'}\\
&= \sum_{lc} (pk|lc) X_{lc}^{v}
 \equiv \bm{W}^< \\
\end{align}
So because Booth's and Tim's forms for TDA are related by a similarity transformation, we know that they have the same spectrum.

% \subsubsection{Downfolding the upfolded Hamiltonian}
% Application of downfolding on George's supermatrix gave us the expression
% \begin{equation}
%     \mathbf{\Sigma }(\omega )=\mathbf{W}^<(\omega \mathbf{I}-{\mathbf{d}^<})^{-1} {\mathbf{W}}^{<,\dagger} + \mathbf{W}^>(\omega \mathbf{I}-{\mathbf{d}^>})^{-1} {\mathbf{W}}^{>,\dagger}
% \label{eq:booth_self_energy}
% \end{equation}
% Meanwhile, downfolding Tim's formulation gives us the expression for the correlation self energy
% \begin{equation}
%     \Sigma(\omega)= \mathbf{V}^{2 \mathrm{hlp}}\left[\omega \mathbf{1}-\mathbf{C}^{2 \mathrm{hlp}}\right]^{-1}\left[\mathbf{V}^{2 \mathrm{hlp}}\right]^{\dagger} +\mathbf{V}^{2 \mathrm{plh}}\left[\omega \mathbf{1}-\mathbf{C}^{2 \mathrm{plh}}\right]^{-1}\left[\mathbf{V}^{2 \mathrm{plh}}\right]^{\dagger}
% \end{equation}
% Above, we showed that the forms are actually equivalent for the TDA case.
\subsection{Deriving the matrix vector products}
Now, we can define a vector $\bm{R} = ( r_i,\; r_a,\; r_{i[jb]},\; r_{[jb]a} )$. Application of the Hamiltonian to this vector gives us the matrix-vector product $\bm{H} \bm{R} = \bm{\sigma }$, where $\bm{\sigma} = ( \sigma_i,\; \sigma_a,\; \sigma_{i[jb]},\; \sigma_{[jb]a} )$. Consider
\begin{align}
\bm{H} \bm{R} &= \begin{pmatrix} \bm{F} & \bm{V}^{2 \mathrm{h1p}} & \bm{V}^{2 \mathrm{p} 1 \mathrm{~h}} \\ \left(\bm{V}^{2 \mathrm{~h} 1 \mathrm{p}}\right)^{\dagger} & \bm{C}^{2 \mathrm{hlp}} & \bm{0} \\ \left(\bm{V}^{2 \mathrm{plh}}\right)^{\dagger} & \bm{0} & \bm{C}^{2 \mathrm{plh}} \end{pmatrix} \begin{pmatrix} r_i \\ r_a \\ r_{i[jb]} \\ r_{[jb]a} \end{pmatrix} = \begin{pmatrix} \sigma_i \\ \sigma_a \\ \sigma_{i[jb]} \\ \sigma_{[jb]a} \end{pmatrix} \\[6pt]
\end{align}
Let us enumerate now what we actually will get:
\begin{align}
\sigma_i &= 
  \sum_{j} f_{i j}\,r_j
  + \sum_{b} f_{i b}\,r_b
  + \sum_{k l c} \bigl\langle i\,c | k\,l \bigr\rangle\,r_{k[l c]}
  + \sum_{k c d} \bigl\langle i\,k | d\,c \bigr\rangle\,r_{[k c]d}, \\[6pt]
\sigma_a &=
    \sum_{j} f_{a j}\,r_j
    + \sum_{b} f_{a b}\,r_b
    + \sum_{k l c} \bigl\langle a\,c | k\,l \bigr\rangle\,r_{k[l c]}
    + \sum_{k c d} \bigl\langle a\,k | d\,c \bigr\rangle\,r_{[k c]d}, \\[6pt]
\sigma_{i[ja]} &
    \sum_{k} \bigl\langle k\,a | i\,j \bigr\rangle\,r_k
    + \sum_{b} \bigl\langle b\,a | i\,j \bigr\rangle\,r_b
    + \left(\epsilon _{i} + \epsilon _{j} - \epsilon _{a}\right) r_{i[j a]}
    - \sum_{lc} \bigl\langle j\,c | a\,l \bigr\rangle\,r_{i[l c]} \\[6pt]
\sigma_{[ia]b} &= 
    \sum_{j} \bigl\langle j\,i | b\,a \bigr\rangle\,r_j
    + \sum_{c} \bigl\langle c\,i | b\,a \bigr\rangle\,r_c
    + \left(\epsilon _{a} + \epsilon _{b} - \epsilon _{i}\right) r_{[i a] b}
    + \sum_{k c} \bigl\langle a\,k | i\,c \bigr\rangle\,r_{[k c] b}.
\end{align}
\section{Formulation for the dRPA}
\subsubsection{Exact form}
The exact form is
\begin{equation}
    \bm{H}_{\text{Upfolded}}^{G_0W_0} = \begin{pmatrix} \bm{F} & \bm{W}^< & \bm{W}^> \\ \bm{W}^{<,\dagger} & \bm{d}^< & 0 \\ \bm{W}^{>, \dagger} & 0 & \bm{d}^> \end{pmatrix}
\label{eq:booth_hamiltonian}
\end{equation}
where $\bm{F}$ is the HF Fock matrix
\begin{equation}
    \bm{W}^<_{pk\nu} = \sum_{ia} (pk|ia) \left( X_{ia}^{\nu} + Y_{ia}^{\nu} \right) \quad \text{and} \quad \bm{W}^>_{pc\nu} = \sum_{ia} (pc|ia) \left( X_{ia}^{\nu} + Y_{ia}^{\nu} \right)
\end{equation}
and 
\begin{equation}
    \bm{d}^<_{k\nu , l\nu'} = \left(\epsilon_k - \Omega_\nu\right) \delta_{k,l} \delta_{\nu,\nu'} \quad \text{and} \quad \bm{d}^>_{c\nu , d\nu'} = \left(\epsilon_c + \Omega_\nu\right) \delta_{c,d} \delta_{\nu,\nu'}.
\end{equation}
\begin{tcolorbox}[colback=red!10!white, colframe=red!50!black, title=Downfolding exercise]
Let us confirm that this downfolds to the correct result. First, we can define a excitation vector $\bm{R}$ as
\begin{equation}
    \bm{R} = \begin{pmatrix} \bm{R}^{1\mathrm{h}+1\mathrm{p}} \\ \bm{R}^{2\mathrm{h}1\mathrm{p}} \\ \bm{R}^{2\mathrm{p}1\mathrm{h}} \end{pmatrix}
\end{equation}
and we can write
\begin{align}
    \begin{pmatrix} \bm{F} & \bm{W}^< & \bm{W}^> \\ \bm{W}^{<,\dagger} & \bm{d}^< & 0 \\ \bm{W}^{>, \dagger} & 0 & \bm{d}^> \end{pmatrix} \begin{pmatrix} \bm{R}^{1\mathrm{h}+1\mathrm{p}} \\ \bm{R}^{2\mathrm{h}1\mathrm{p}} \\ \bm{R}^{2\mathrm{p}1\mathrm{h}} \end{pmatrix} &= E \begin{pmatrix} \bm{R}^{1\mathrm{h}+1\mathrm{p}} \\ \bm{R}^{2\mathrm{h}1\mathrm{p}} \\ \bm{R}^{2\mathrm{p}1\mathrm{h}} \end{pmatrix} \\
\begin{pmatrix}
    \bm{F} \bm{R}^{1\mathrm{h}+1\mathrm{p}} + \bm{W}^< \bm{R}^{2\mathrm{h}1\mathrm{p}} + \bm{W}^> \bm{R}^{2\mathrm{p}1\mathrm{h}} \\
    \bm{W}^{<,\dagger} \bm{R}^{1\mathrm{h}+1\mathrm{p}} + \bm{d}^< \bm{R}^{2\mathrm{h}1\mathrm{p}} \\
    \bm{W}^{>, \dagger} \bm{R}^{1\mathrm{h}+1\mathrm{p}} + \bm{d}^> \bm{R}^{2\mathrm{p}1\mathrm{h}} \\
\end{pmatrix}
&= E \begin{pmatrix} \bm{R}^{1\mathrm{h}+1\mathrm{p}} \\ \bm{R}^{2\mathrm{h}1\mathrm{p}} \\ \bm{R}^{2\mathrm{p}1\mathrm{h}} \end{pmatrix}
\end{align}
This implies three coupled equations
\begin{align}
    \bm{F} \bm{R}^{1\mathrm{h}+1\mathrm{p}} + \bm{W}^< \bm{R}^{2\mathrm{h}1\mathrm{p}} + \bm{W}^> \bm{R}^{2\mathrm{p}1\mathrm{h}} &= E \bm{R}^{1\mathrm{h}+1\mathrm{p}} \\
    \bm{W}^{<,\dagger} \bm{R}^{1\mathrm{h}+1\mathrm{p}} + \bm{d}^< \bm{R}^{2\mathrm{h}1\mathrm{p}} &= E \bm{R}^{2\mathrm{h}1\mathrm{p}} \\
    \bm{W}^{>, \dagger} \bm{R}^{1\mathrm{h}+1\mathrm{p}} + \bm{d}^> \bm{R}^{2\mathrm{p}1\mathrm{h}} &= E \bm{R}^{2\mathrm{p}1\mathrm{h}}
\end{align}
Solving the latter two equations gives
\begin{align}
    \bm{R}^{2\mathrm{h}1\mathrm{p}} &= \left(E \bm{1} - \bm{d}^<\right)^{-1} \bm{W}^{<,\dagger} \bm{R}^{1\mathrm{h}+1\mathrm{p}} \\
    \bm{R}^{2\mathrm{p}1\mathrm{h}} &= \left(E \bm{1} - \bm{d}^>\right)^{-1} \bm{W}^{>,\dagger} \bm{R}^{1\mathrm{h}+1\mathrm{p}}
\end{align}
Substituting these into the first equation gives us
\begin{align}
    \bm{F} \bm{R}^{1\mathrm{h}+1\mathrm{p}} + \bm{W}^< \left(E \bm{1} - \bm{d}^<\right)^{-1} \bm{W}^{<,\dagger} \bm{R}^{1\mathrm{h}+1\mathrm{p}} + \bm{W}^> \left(E \bm{1} - \bm{d}^>\right)^{-1} \bm{W}^{>,\dagger} \bm{R}^{1\mathrm{h}+1\mathrm{p}} &= E \bm{R}^{1\mathrm{h}+1\mathrm{p}}
\end{align}
from which we can get the eigenvalue equation
\begin{equation}
    \left(\bm{F} + \bm{W}^< \left(E \bm{1} - \bm{d}^<\right)^{-1} \bm{W}^{<,\dagger} + \bm{W}^> \left(E \bm{1} - \bm{d}^>\right)^{-1} \bm{W}^{>,\dagger}\right) \bm{R}^{1\mathrm{h}+1\mathrm{p}} = E \bm{R}^{1\mathrm{h}+1\mathrm{p}}
\end{equation}
so we can identify that the correlation self energy in this approach is
\begin{equation}
    \bm{\Sigma} _c = \bm{W}^< \left(\omega \bm{1} - \bm{d}^<\right)^{-1} \bm{W}^{<,\dagger} + \bm{W}^> \left(\omega \bm{1} - \bm{d}^>\right)^{-1} \bm{W}^{>,\dagger}
\end{equation}
which is the same as the known frequency dependent form for the real part of the correlation self energy:
\begin{equation}
    \Sigma _{pq}^{\text{corr}}(\omega) = \sum_{\mu }^{\text{RPA}}\left(\sum_{i}^{\text{occupied}} \frac{w_{pi}^{\mu }w_{iq}^{\mu }}{\omega -(\epsilon _{i}-\Omega  _{\mu })}+ \sum_{a}^{\text{virtual}} \frac{w_{pa}^{\mu }w_{aq}^{\mu }}{\omega -(\epsilon _{a}+\Omega  _{\mu })}\right)
\end{equation}
with $w_{pq}^{\mu } = \sum_{ia} (pq|ia) \left(X_{ia}^{\mu} + Y_{ai}^{\mu}\right)$.    
\end{tcolorbox}

\subsubsection{Showing equivalence between excitation energies of M and Mtilde}
So we start with this generalized eigenvalue equation
$$
\mathbf{M}\bm{Z}=\mathbf{N}\bm{Z}\left(\begin{array}{cc}
\boldsymbol{\Omega}_{+} & 0 \\
0 & -\boldsymbol{\Omega}_{+}
\end{array}\right)
$$
where
$$
\mathbf{M} =\left(\begin{array}{ll}
\mathbf{A} & \mathbf{B} \\
\mathbf{B} & \mathbf{A}
\end{array}\right) \quad 
\mathbf{N} =\left(\begin{array}{cc}
\mathbf{1} & \mathbf{0} \\
\mathbf{0} & -\mathbf{1}
\end{array}\right) \quad
\bm{Z} =\left(\begin{array}{ll}
\bm{X} & \bm{Y} \\
\bm{Y} & \bm{X}
\end{array}\right)
$$
and $\boldsymbol{\Omega}_{+}$is a diagonal matrix of positive excitation energies. Left multiplying both sides by $\bm{N}$ and right multiplying by $\bm{Z}^{-1}$ gives us
\begin{equation}
    \bm{N}\bm{M}\underbrace{\bm{Z} \bm{Z}^{-1}}_{\bm{1}} = \underbrace{\bm{N}\bm{N}}_{\bm{1}}\bm{Z}\left(\begin{array}{cc}
\boldsymbol{\Omega}_{+} & 0 \\
0 & -\boldsymbol{\Omega}_{+}
\end{array}\right) \bm{Z^{-1}} \implies - \bm{N}\bm{M} = -\bm{Z}\left(\begin{array}{cc}\boldsymbol{\Omega}_{+} & 0 \\
0 & -\boldsymbol{\Omega}_{+}
\end{array}\right)\bm{Z}^{-1}
\end{equation}
Now we can use the fact that the action of a scalar function $f$, such as the step function, on a diagonalizable matrix $\bm{X} \equiv \bm{Y} \bm{\Lambda} \bm{Y}^{-1}$ can be expressed as
$$
f(\bm{X}) = \bm{Y} f(\bm{\Lambda}) \bm{Y}^{-1}
$$
so we can write
\begin{equation}
    \Theta(-\bm{N}\bm{M}) = \bm{Z} \left(\begin{array}{cc}\Theta(-\boldsymbol{\Omega}_{+}) & 0 \\ 0 & \Theta(\boldsymbol{\Omega}_{+})\end{array}\right)\bm{Z}^{-1} = \bm{Z} \left(\begin{array}{cc}\bm{0} & 0 \\ 0 & \bm{1}\end{array}\right)\bm{Z}^{-1}
\end{equation}
and so it becomes clear that if we define $\tilde{\bm{M}} = \bm{M} + \eta \bm{N}\Theta(-\bm{N}\bm{M})$, we can write
\begin{equation}
    \tilde{\bm{M}} \bm{Z} = \bm{M} \bm{Z} + \eta \bm{N}\Theta(-\bm{N}\bm{M}) \bm{Z} = \mathbf{N}\bm{Z}\left(\begin{array}{cc}
\boldsymbol{\Omega}_{+} & 0 \\
0 & -\boldsymbol{\Omega}_{+}
\end{array}\right) + \bm{N}\bm{Z} \left(\begin{array}{cc}\bm{0} & 0 \\ 0 & \bm{\eta}\end{array}\right) \underbrace{\bm{Z}^{-1} \bm{Z}}_{ \bm{1}} = \bm{N}\bm{Z}\left(\begin{array}{cc}\boldsymbol{\Omega}_{+} & 0 \\ 0 & -\boldsymbol{\Omega}_{+} + \bm{\eta}\end{array}\right).
\end{equation}


\subsection{Considering the supermatrices they give}
It is understood that the notations $<$,$>$ and 2h1p, 2p1h can be used interchangeably, respectively. Tim's supermatrix is given by
\begin{equation}
\bm{H} =
\begin{pmatrix}
\bm{F} & \bm{V}^{2\mathrm{h1p}} & \bm{V}^{2\mathrm{h1p}} & \bm{V}^{2\mathrm{plh}} & \bm{V}^{2\mathrm{plh}} \\
\left(\bm{V}^{2\mathrm{h1p}}\right)^{\dagger} &  &  & & \\
\left(\bm{V}^{2\mathrm{h1p}}\right)^{\dagger} &  & \bm{C}^{2\mathrm{hlp}} & & \bm{0} \\
\left(\bm{V}^{2\mathrm{plh}}\right)^{\dagger} & & & & \\
\left(\bm{V}^{2\mathrm{plh}}\right)^{\dagger} &  & \bm{0} & & \bm{C}^{2\mathrm{plh}}
\end{pmatrix}
\end{equation}
We are told that $\mathbf{C}^{2 \mathrm{hlp}}=\varepsilon^{1 \mathrm{~h}} \oplus(-\tilde{\mathbf{M}})$ and $\mathbf{C}^{2 \mathrm{plh}}=\varepsilon^{1 \mathrm{p}} \oplus \tilde{\mathbf{M}}$ and further, that the super-metric is
\begin{equation}
    \bm{\mathcal{N}} = \begin{pmatrix}
\bm{1} & 0 & 0 \\
0 & \bm{1} \oplus \bm{N} & 0 \\
0 & 0 & \bm{1} \oplus \bm{N}
    \end{pmatrix}
\end{equation}
Let's just focus on the 2h1p sector for now, so the supermatrices are
\begin{equation}
    \bm{H}^{2 \mathrm{hlp}} = 
    \begin{pmatrix}
        \bm{F}_{P,P} & \bm{V}^{2\mathrm{h1p}}_{P,O^2V} & \bm{V}^{2\mathrm{h1p}}_{P,O^2V} \\
        \left(\bm{V}^{2\mathrm{h1p}}_{P,O^2V}\right)^{\dagger} &  &  \\
        \left(\bm{V}^{2\mathrm{h1p}}_{P,O^2V}\right)^{\dagger} &  & \bm{C}^{2\mathrm{hlp}}_{2O^2V,2O^2V}
    \end{pmatrix}, \quad
    \bm{\mathcal{N}}^{2 \mathrm{hlp}} =
    \begin{pmatrix}
        \bm{1}_{P,P} & 0 \\
        0 & \bm{1}_O \oplus \bm{N}_{2OV}
    \end{pmatrix}, \quad
\bm{R}^{2 \mathrm{hlp}} =
    \begin{pmatrix}
        r_i \\
        r_a \\
        r_{i[jb]} \\
        \bar{r}_{i[jb]}
    \end{pmatrix}
\end{equation}
In the following, we might sometimes use the notation $P=O+V$, $A=O^2V$ and $T$ as the column dimension of the excitation vector.

\subsubsection{Multiplication of the Hamiltonian by the excitation vector}
First, we can take
\begin{align}
&\bm{H}^{2 \mathrm{hlp}} \bm{R}^{2 \mathrm{hlp}}\\
&=
    \begin{pmatrix}
        \bm{F}_{P,P} & \bm{V}^{2\mathrm{h1p}}_{P,O^2V} & \bm{V}^{2\mathrm{h1p}}_{P,O^2V} \\
        \left(\bm{V}^{2\mathrm{h1p}}_{P,O^2V}\right)^{\dagger} &  &  \\
        \left(\bm{V}^{2\mathrm{h1p}}_{P,O^2V}\right)^{\dagger} &  & \bm{C}^{2\mathrm{hlp}}_{2O^2V,2O^2V}
    \end{pmatrix}
    \begin{pmatrix}
        r_i \\
        r_a \\
        r_{i[jb]} \\
        \bar{r}_{i[jb]}
    \end{pmatrix} \\
&=
    \begin{pmatrix}
\begin{pmatrix}
        \bm{F}_{OO} & \bm{F}_{OV} \\ \bm{F}_{VO} & \bm{F}_{VV}
\end{pmatrix}_{P,P} \begin{pmatrix}
\bm{r}_i \\ \bm{r}_a
\end{pmatrix}_{P,T} + \begin{pmatrix}
\bm{V}^{2 \mathrm{h1p}}_{O,i[jb]} & \bm{V}^{2 \mathrm{h1p}}_{O,i[{jb}]} \\ \bm{V}^{2 \mathrm{h1p}}_{V,i[jb]} & \bm{V}^{2 \mathrm{h1p}}_{V,i[{jb}]} 
\end{pmatrix}_{P,2A} \begin{pmatrix}
    \bm{r}_{i[jb]} \\
\bm{\bar{r}}_{i[\bar{jb}]}
\end{pmatrix}_{2A,T}  \\
        \begin{pmatrix}
\left(\bm{V}^{2 \mathrm{h1p}}_{O,i[jb]}\right)^{\dagger} & \left(\bm{V}^{2 \mathrm{h1p}}_{V,i[{jb}]}\right)^{\dagger} \\ \left(\bm{V}^{2 \mathrm{h1p}}_{O,i[jb]}\right)^{\dagger} & \left(\bm{V}^{2 \mathrm{h1p}}_{V,i[{jb}]}\right)^{\dagger} 
\end{pmatrix}_{2A,P} \begin{pmatrix}
\bm{r}_i \\ \bm{r}_a
\end{pmatrix}_{P,T} + \left[ \bm{\epsilon}^{1 \mathrm{~h}}_{O,O} \otimes \bm{1}_{2OV,2OV} + \bm{1}_{O,O} \otimes -\tilde{\bm{M}}_{2OV,2OV} \right]_{2A,2A} \begin{pmatrix}
    \bm{r}_{i[jb]} \\
\bm{\bar{r}}_{i[\bar{jb}]}
\end{pmatrix}_{2A,T} \\
\end{pmatrix}
\end{align}
Just evaluate
\begin{align}
    &\left[ \bm{\epsilon}^{1 \mathrm{~h}}_{O,O} \otimes \bm{1}_{2OV,2OV} + \bm{1}_{O,O} \otimes -\tilde{\bm{M}}_{2OV,2OV} \right]_{2A,2A} \begin{pmatrix}
    \bm{r}_{i[jb]} \\
\bm{\bar{r}}_{i[\bar{jb}]}
\end{pmatrix}_{2A,T} \\
% &= \left(\bm{\epsilon}^{1 \mathrm{~h}}_{O,O} \otimes \bm{1}_{2OV,2OV} \right)\begin{pmatrix}
%     \bm{r}_{i[jb]} \\
% \bm{\bar{r}}_{i[\bar{jb}]}
% \end{pmatrix}_{2A,T} - \left(\bm{1}_{O,O} \otimes \tilde{\bm{M}}_{2OV,2OV} \right) \begin{pmatrix}
%     \bm{r}_{i[jb]} \\
% \bm{\bar{r}}_{i[\bar{jb}]}
% \end{pmatrix}_{2A,T} \\
\end{align}
% We have the identity $\operatorname{vec}(ABC) = (C^\top \otimes A) \operatorname{vec}(B)$. If we choose that $C^\top = \bm{\epsilon}^{1 \mathrm{~h}}_{O} + \bm{1}_{O}$, $A = \bm{1}_{2OV} - \tilde{\bm{M}}_{2OV}$, and $\operatorname{vec}(B) =\begin{pmatrix}
%     \bm{r}_{i[jb]} \\
% \bm{\bar{r}}_{i[\bar{jb}]}
% \end{pmatrix}$ this implies that
% \begin{align}
%     &\left[ \bm{\epsilon}^{1 \mathrm{~h}}_{O,O} \otimes \bm{1}_{2OV,2OV} + \bm{1}_{O,O} \otimes -\tilde{\bm{M}}_{2OV,2OV} \right]_{2A,2A} \begin{pmatrix}
%     \bm{r}_{i[jb]} \\
% \bm{\bar{r}}_{i[\bar{jb}]}
% \end{pmatrix}_{2A,T} \\
% &= \left[ \left(\bm{\epsilon}^{1 \mathrm{~h}}_{O,O} + \bm{1}_{O,O}\right) \otimes \left(\bm{1}_{2OV,2OV} - \tilde{\bm{M}}_{2OV,2OV}\right) \right]_{2A,2A} \begin{pmatrix}
%     \bm{r}_{i[jb]} \\[6pt]
% \bm{\bar{r}}_{i[\bar{jb}]}
% \end{pmatrix}_{2A,T} \\
% &= \operatorname{vec}\left[\left(\bm{1}_{2OV,2OV} - \tilde{\bm{M}}_{2OV,2OV}\right) \begin{pmatrix}
%     \bm{r}_{i[jb]} \\[6pt]
% \bm{\bar{r}}_{i[\bar{jb}]}
% \end{pmatrix}_{2OV,T} \left(\bm{\epsilon}^{1 \mathrm{~h}}_{O,O} + \bm{1}_{O,O}\right)^\top \right] \\ 
% \end{align}
First, use the definition of the tensor product and also notice that $\bm{\epsilon}^{1 \mathrm{~h}}_{O,O}$ is a matrix with the hole energies on the diagonal and zeros elsewhere
\begin{align}
    \bm{\epsilon}^{1 \mathrm{~h}}_{O,O} \otimes \bm{1}_{2OV,2OV} &= \begin{pmatrix}
\epsilon^{1 \mathrm{~h}}_{11} \bm{1}_{2OV,2OV} & \epsilon^{1 \mathrm{~h}}_{12} \bm{1}_{2OV,2OV} & \dots & \epsilon^{1 \mathrm{~h}}_{1O} \bm{1}_{2OV,2OV} \\[6pt]
\epsilon^{1 \mathrm{~h}}_{21} \bm{1}_{2OV,2OV} & \epsilon^{1 \mathrm{~h}}_{22} \bm{1}_{2OV,2OV} & \dots & \epsilon^{1 \mathrm{~h}}_{2O} \bm{1}_{2OV,2OV} \\[6pt]
\vdots & \vdots & \ddots & \vdots \\[6pt]
\epsilon^{1 \mathrm{~h}}_{O1} \bm{1}_{2OV,2OV} & \epsilon^{1 \mathrm{~h}}_{O2} \bm{1}_{2OV,2OV} & \dots & \epsilon^{1 \mathrm{~h}}_{OO} \bm{1}_{2OV,2OV}
\end{pmatrix}_{2A,2A}\\
&= \begin{pmatrix}
\epsilon^{1 \mathrm{~h}}_{11} \bm{1}_{2OV,2OV} & \bm{0} & \dots & \bm{0} \\[6pt]
\bm{0} & \epsilon^{1 \mathrm{~h}}_{22} \bm{1}_{2OV,2OV} & \dots & \bm{0} \\[6pt]
\vdots & \vdots & \ddots & \vdots \\[6pt]
\bm{0} & \bm{0} & \dots & \epsilon^{1 \mathrm{~h}}_{OO} \bm{1}_{2OV,2OV}
\end{pmatrix}_{2A,2A}
\end{align}
\begin{align}
    &\implies \left(\bm{\epsilon}^{1 \mathrm{~h}}_{O,O} \otimes \bm{1}_{2OV,2OV} \right)_{2A,2A}\begin{pmatrix}
    \bm{r}_{i[jb]} \\
\bm{\bar{r}}_{i[\bar{jb}]}
\end{pmatrix}_{2A,T}\\
& = \begin{pmatrix}
\epsilon^{1 \mathrm{~h}}_{11} \bm{1}_{2OV,2OV} & \bm{0} & \dots & \bm{0} \\[6pt]
\bm{0} & \epsilon^{1 \mathrm{~h}}_{22} \bm{1}_{2OV,2OV} & \dots & \bm{0} \\[6pt]
\vdots & \vdots & \ddots & \vdots \\[6pt]
\bm{0} & \bm{0} & \dots & \epsilon^{1 \mathrm{~h}}_{OO} \bm{1}_{2OV,2OV}
\end{pmatrix}_{2A,2A} \begin{pmatrix}
    \bm{r}_{i[jb]} \\[6pt]
\bm{\bar{r}}_{i[\bar{jb}]}
\end{pmatrix}_{2A,T} \\
\end{align}
For these, we can use the formula 
\begin{align}
    \left(A \otimes B\right) = \begin{pmatrix}
a_{1 1} B & a_{1 2} B & \dots & a_{1 n} B \\[6pt]
a_{2 1} B & a_{2 2} B & \dots & a_{2 n} B \\[6pt]
\vdots & \vdots & \ddots & \vdots \\[6pt]
a_{m 1} B & a_{m 2} B & \dots & a_{m n} B   
    \end{pmatrix}\implies\left(A \otimes B\right) K=\left[\begin{array}{c}\sum_{j=1}^n a_{1 j} B K^{(j)} \\ \sum_{j=1}^n a_{2 j} B K^{(j)} \\ \vdots \\ \sum_{j=1}^n a_{m j} B K^{(j)}\end{array}\right]
\end{align}
where the $K^{(j)}$ are the rows of the matrix $K$. To simplify the notation let us define $t=O//2$.
First, with
\begin{align}
    &\bm{K} = \begin{pmatrix}
\bm{r}_{i[jb]} \\
\bm{\bar{r}}_{i[{jb}]} \\
\end{pmatrix} = \begin{pmatrix}
r_{1[jb]} \\[6pt]
\vdots \\[6pt]
r_{O[jb]} \\[6pt]
\bar{r}_{1[{j} b]} \\[6pt]
\vdots \\[6pt]
\bar{r}_{O[{j} b]}
\end{pmatrix} = \begin{pmatrix}
K^{(1)} \\[6pt]
\vdots \\[6pt]
K^{(t)} \\[6pt]
\vdots \\[6pt]
K^{(O)}
\end{pmatrix}\in \mathbb{R}^{2O^2V},\\
&K^{(m)} = \begin{pmatrix}
r_{(2m-1)[jb]} \\[3pt]
{r}_{(2m)[{j} b]}
\end{pmatrix} \in \mathbb{R}^{2OV} \forall m \leq t, \quad K^{(m)} = \begin{pmatrix}
\bar{r}_{(m-1-t)[jb]} \\[3pt]
\bar{r}_{(m-t)[{j} b]}
\end{pmatrix} \in \mathbb{R}^{2OV} \forall m>t+1
\end{align}
Depending on whether O is even or odd, $K^{(t)}$ will be $\begin{pmatrix}
r_{(O-1)[jb]} \\[3pt]
r_{O[jb]}
\end{pmatrix}$ or $\begin{pmatrix}r_{O[jb]} \\[3pt]
\bar{r}_{1[{j} b]}
\end{pmatrix}$, respectively. This version seems correct, but I am unsure about how to proceed. Next, I will consider a modification to lead to a familiar form, but I am not sure if it is correct.
\subsubsection{Incorrect but suggestive form}
\begin{align}
    \bm{K} = \begin{pmatrix}
\bm{r}_{i[jb]} \\
\bm{\bar{r}}_{i[{jb}]} \\
\end{pmatrix} \color{red} \stackrel{?}{=} \color{black} \begin{pmatrix}
r_{1[jb]} \\[6pt]
\bar{r}_{1[{j} b]} \\[6pt]
r_{2[jb]} \\[6pt]
\bar{r}_{2[{j} b]} \\[6pt]
\vdots \\[6pt]
r_{O[jb]} \\[6pt]
\bar{r}_{O[{j} b]}
\end{pmatrix} = \begin{pmatrix}
K^{(1)} \\[6pt]
K^{(2)} \\[6pt]
\vdots \\[6pt]
K^{(O)}
\end{pmatrix}\in \mathbb{R}^{2O^2V},
\quad
K^{(i)} = \begin{pmatrix}
r_{i[jb]} \\[3pt]
\bar{r}_{i[{j} b]}
\end{pmatrix} \in \mathbb{R}^{2OV}
\end{align} 
\begin{align}
    \implies \left(\bm{\epsilon}^{1 \mathrm{~h}}_{O,O} \otimes \bm{1}_{2OV,2OV} \right)_{2A,2A}\begin{pmatrix}
    \bm{r}_{i[jb]} \\
\bm{\bar{r}}_{i[\bar{jb}]}
\end{pmatrix}_{2A,T}
& \color{red} \stackrel{?}{=} \color{black} \begin{pmatrix}
\epsilon^{1\mathrm{h}}_{11} K^{(1)} + \epsilon^{1\mathrm{h}}_{12} K^{(2)} + \dots + \epsilon^{1\mathrm{h}}_{1O} K^{(O)} \\[1.2em]
\epsilon^{1\mathrm{h}}_{21} K^{(1)} + \epsilon^{1\mathrm{h}}_{22} K^{(2)} + \dots + \epsilon^{1\mathrm{h}}_{2O} K^{(O)} \\[1em]
\vdots \\[6pt]
\epsilon^{1\mathrm{h}}_{O1} K^{(1)} + \epsilon^{1\mathrm{h}}_{O2} K^{(2)} + \dots + \epsilon^{1\mathrm{h}}_{OO} K^{(O)}
\end{pmatrix}\\
& =\begin{pmatrix}
\epsilon^{1\mathrm{h}}_{11} K^{(1)} + \dots +0 \\[1.2em]
0 + \epsilon^{1\mathrm{h}}_{22} K^{(2)} + \dots + 0 \\[1em]
\vdots \\[6pt]
0 + \dots + \epsilon^{1\mathrm{h}}_{OO} K^{(O)}
\end{pmatrix} =\begin{pmatrix}
\epsilon^{1\mathrm{h}}_{11} \begin{pmatrix} r_{1[jb]} \\[6pt]
\bar{r}_{1[{j} b]} \end{pmatrix}\\
\epsilon^{1\mathrm{h}}_{22} \begin{pmatrix} r_{2[jb]} \\[6pt]
\bar{r}_{2[{j} b]} \end{pmatrix}\\
\vdots \\[6pt]
\epsilon^{1\mathrm{h}}_{OO} \begin{pmatrix} r_{O[jb]} \\[6pt]
\bar{r}_{O[{j} b]} \end{pmatrix}
\end{pmatrix} = \begin{pmatrix}
\bm{\epsilon}^{1\mathrm{h}} \bm{r}_{i[jb]} \\[6pt]
\bm{\epsilon}^{1\mathrm{h}} \bar{\bm{r}}_{i[{jb}]}
\end{pmatrix}
\end{align}
where we have used the fact that $\bm{\epsilon}^{1\mathrm{h}}$ is a diagonal matrix with the hole energies. Next, it is useful to define (where the composite index $\mu \equiv OV$; it implies that the occupied-virtual pair is fixed)
\begin{align}
    \bm{K} = \begin{pmatrix}
\bm{r}_{i[jb]} \\
\bm{\bar{r}}_{i[{jb}]} \\
\end{pmatrix}  \color{red} \stackrel{?}{=} \color{black} \begin{pmatrix}
r_{i[\mu_1]} \\[6pt]
\bar{r}_{i[\mu_2]} \\[6pt]
r_{i[\mu_3]} \\[6pt]
\bar{r}_{i[\mu_4]} \\[6pt]
\vdots \\[6pt]
r_{i[\mu_{2OV-1}]} \\[6pt]
\bar{r}_{i[\mu_{2OV}]}
\end{pmatrix} = \begin{pmatrix}
K^{(1)} \\[6pt]
K^{(2)} \\[6pt]
\vdots \\[6pt]
K^{(2OV)}
\end{pmatrix} \in \mathbb{R}^{2O^2V},
\quad
K^{(i)} = \begin{pmatrix}
r_{i[OV]}/\bar{r}_{i[OV]}
\end{pmatrix} \in \mathbb{R}^{O}
\end{align} 
\begin{align}
    &\implies \left(\bm{1}_{O,O} \otimes \tilde{\bm{M}}_{2OV,2OV} \right)_{2A,2A} \begin{pmatrix}    
    \bm{r}_{i[jb]} \\[6pt]
\bm{\bar{r}}_{i[\bar{jb}]}
\end{pmatrix}_{2A,T} \color{red} \stackrel{?}{=} \color{black}\left( \tilde{\bm{M}}_{2OV,2OV} \otimes \bm{1}_{O,O}\right)_{2A,2A} \begin{pmatrix}    
    \bm{r}_{i[jb]} \\[6pt]
\bm{\bar{r}}_{i[\bar{jb}]}
\end{pmatrix}_{2A,T}\\
& \color{red} \stackrel{?}{=} \color{black}\begin{pmatrix}
\tilde{M}_{11} K^{(1)} + \dots + \tilde{M}_{1,OV} K^{(OV)} + \tilde{M}_{1,OV+1} K^{(OV+1)}+\dots + \tilde{M}_{1,2OV} K^{(2OV)} \\
\tilde{M}_{2,1} K^{(1)} +  \dots + \tilde{M}_{2,OV} K^{(OV)} + \tilde{M}_{2,OV+1} K^{(OV+1)}+\dots + \tilde{M}_{2,2OV} K^{(2OV)} \\[1.2em]
\vdots \\[6pt]
\tilde{M}_{OV,1} K^{(1)} + \dots + \tilde{M}_{OV,OV} K^{(OV)} + \tilde{M}_{OV,OV+1} K^{(OV+1)}+\dots + \tilde{M}_{OV,2OV} K^{(2OV)} \\[1em]
\tilde{M}_{OV+1,1} K^{(1)} + \dots + \tilde{M}_{OV+1,OV} K^{(OV)} + \tilde{M}_{OV+1,OV+1} K^{(OV+1)} + \dots + \tilde{M}_{OV+1,2OV} K^{(2OV)} \\[1em]
\vdots \\[6pt]
\tilde{M}_{2OV,1} K^{(1)} + \dots + \tilde{M}_{2OV,OV} K^{(OV)} + \tilde{M}_{2OV,OV+1} K^{(OV+1)} + \dots + \tilde{M}_{2OV,2OV} K^{(2OV)}\\
\end{pmatrix}\\
&=\begin{pmatrix}
\tilde{\bm{M}}_{xx} \bm{K}_{1\rightarrow OV} + \tilde{\bm{M}}_{xd} \bm{K}_{OV+1\rightarrow 2OV} \\
\tilde{\bm{M}}_{dx} \bm{K}_{1\rightarrow OV} + \tilde{\bm{M}}_{dd} \bm{K}_{OV+1\rightarrow 2OV} \\
\end{pmatrix}
\end{align}
where $\tilde{\bm{M}}_{xx}$, $\tilde{\bm{M}}_{xd}$, $\tilde{\bm{M}}_{dx}$, and $\tilde{\bm{M}}_{dd}$ are the excitation-excitation, excitation-de-excitation, de-excitation-excitation, and de-excitation-de-excitation blocks of the matrix $\tilde{\bm{M}}$ respectively. $\bm{K}_{1\rightarrow OV}$ and $\bm{K}_{OV+1\rightarrow 2OV}$ are defined as
\begin{align}
\bm{K}_{1\rightarrow OV} = \begin{pmatrix} K^{(1)} & K^{(2)} & \dots & K^{(OV)} \end{pmatrix} = \begin{pmatrix}
r_{i[\mu_1]} & \bar{r}_{i[\mu_2]} & \dots & r_{i[\mu_{OV}]}
\end{pmatrix} \\
\bm{K}_{OV+1\rightarrow 2OV} = \begin{pmatrix} K^{(OV+1)} & K^{(OV+2)} & \dots & K^{(2OV)} \end{pmatrix} = \begin{pmatrix}
r_{i[\mu_{OV+1}]} & \bar{r}_{i[\mu_{OV+2}]} & \dots & r_{i[\mu_{2OV}]}
\end{pmatrix}
\end{align}
%  $\bm{K}_{1\rightarrow OV} = \begin{pmatrix} K^{(1)} & K^{(2)} & \dots & K^{(OV)} \end{pmatrix}$ and $\bm{K}_{OV+1\rightarrow 2OV}$ horizontally stacks the last $OV$ rows of $\bm{K}$. So putting this back in gives
\begin{align}
    &\bm{H}^{2 \mathrm{hlp}} \bm{R}^{2 \mathrm{hlp}}\\
& \color{red} \stackrel{?}{=} \color{black} \begin{pmatrix}
\bm{F}_{OO} \bm{r}_i + \bm{F}_{OV} \bm{r}_a + \bm{V}^{2 \mathrm{h1p}}_{O,i[jb]} \bm{r}_{i[jb]} + \bm{V}^{2 \mathrm{h1p}}_{O,i[jb]} \bm{\bar{r}}_{i[\bar{jb}]} \\
\bm{F}_{VO} \bm{r}_i + \bm{F}_{VV} \bm{r}_a + \bm{V}^{2 \mathrm{h1p}}_{V,i[jb]} \bm{r}_{i[jb]} + \bm{V}^{2 \mathrm{h1p}}_{V,i[jb]} \bm{\bar{r}}_{i[\bar{jb}]}  \\
\left(\bm{V}^{2 \mathrm{h1p}}_{O,i[jb]}\right)^{\dagger} \bm{r}_i + \left(\bm{V}^{2 \mathrm{h1p}}_{O,i[jb]}\right)^{\dagger} \bm{r}_a + \bm{\epsilon}^{1 \mathrm{~h}}_{O,O} \bm{r}_{i[jb]} - \tilde{\bm{M}}_{xx} \bm{K}_{1\rightarrow OV} - \tilde{\bm{M}}_{xd} \bm{K}_{OV+1\rightarrow 2OV} \\
\left(\bm{V}^{2 \mathrm{h1p}}_{V,i[jb]}\right)^{\dagger} \bm{r}_i + \left(\bm{V}^{2 \mathrm{h1p}}_{V,i[jb]}\right)^{\dagger} \bm{r}_a + \bm{\epsilon}^{1 \mathrm{~h}}_{O,O} \bm{\bar{r}}_{i[jb]} - \tilde{\bm{M}}_{dx} \bm{K}_{1\rightarrow OV} - \tilde{\bm{M}}_{dd} \bm{K}_{OV+1\rightarrow 2OV} \\
\end{pmatrix}_{P+2A,T} \\% \bm{V}^{2 \mathrm{h1p}}_{O,i[jb]}^{\dagger} \bm{r}_i + \bm{V}^{2 \mathrm{h1p}}_{O,i[jb]}^{\dagger} \bm{r}_a + \bm{\epsilon}^{1 \mathrm{~h}}_{O,O} \bm{r}_{i[jb]} - \tilde{\bm{M}}_{xx} \bm{K}_{1\rightarrow OV} - \tilde{\bm{M}}_{xd} \bm{K}_{OV+1\rightarrow 2OV} \\
% \bm{V}^{2 \mathrm{h1p}}_{V,i[jb]}^{\dagger} \bm{r}_i + \bm{V}^{2 \mathrm{h1p}}_{V,i[jb]}^{\dagger} \bm{r}_a + \bm{\epsilon}^{1 \mathrm{~h}}_{V,V} \bm{r}_{i[jb]} - \tilde{\bm{M}}_{dx} \bm{K}_{1\rightarrow OV} - \tilde{\bm{M}}_{dd} \bm{K}_{OV+1\rightarrow 2OV} \\
% \end{pmatrix}_{P+2A,T} \\
&= \begin{pmatrix}
\sum_{j} f_{i j}\,r_j + \sum_{b} f_{i b}\,r_b + \sum_{k l c} \bigl\langle i\,c | k\,l \bigr\rangle\,r_{k[l c]} + \sum_{k l c} \bigl\langle i\,c | k\,l \bigr\rangle\,\bar{r}_{k[l c]} \\
\sum_{j} f_{a j}\,r_j + \sum_{b} f_{a b}\,r_b + \sum_{k l c} \bigl\langle a\,c | k\,l \bigr\rangle\,r_{k[l c]} + \sum_{k l c} \bigl\langle a\,c | k\,l \bigr\rangle\,\bar{r}_{k[l c]} \\
\vdots
\end{pmatrix}_{P+2A,T}
\end{align}

\subsubsection{Multiplying the supermetric by the Hamiltonian}
Alternatively, we can consider the LHS matrix multiplication
\begin{align}
    \bm{\mathcal{N}}^{2 \mathrm{hlp}} \bm{H}^{2 \mathrm{hlp}} &= \begin{pmatrix}
        \bm{1}_{P,P} & 0 \\
        0 & \bm{1}_{O,O} \oplus_{\text{kron}} \bm{N}_{2OV,2OV}
    \end{pmatrix}
    \begin{pmatrix}
        \bm{F}_{P,P} & \bm{V}^{2\mathrm{h1p}}_{P,O^2V} & \bm{V}^{2\mathrm{h1p}}_{P,O^2V} \\
        \left(\bm{V}^{2\mathrm{h1p}}_{P,O^2V}\right)^{\dagger} &  &  \\
        \left(\bm{V}^{2\mathrm{h1p}}_{P,O^2V}\right)^{\dagger} &  & \bm{C}^{2\mathrm{hlp}}_{2O^2V,2O^2V}
    \end{pmatrix} \\
&= \begin{pmatrix}        \bm{F}_{P,P} & \begin{pmatrix} \bm{V}^{2\mathrm{h1p}} & \bm{V}^{2\mathrm{h1p}}\end{pmatrix}_{P,2O^2V} \\
        \left( \bm{1} \oplus_{\text{kron}} \bm{N} \right)_{2O^2V,2O^2V}\begin{pmatrix} \left(\bm{V}^{2\mathrm{h1p}}\right)^{\dagger} \\ \left( \bm{V}^{2\mathrm{h1p}}\right)^{\dagger} \end{pmatrix}_{2O^2V,P} &   \left( \bm{1} \oplus_{\text{kron}} \bm{N} \right)_{2O^2V,2O^2V} \bm{C}^{2\mathrm{hlp}}_{2O^2V,2O^2V}  \\
    \end{pmatrix} \\
\end{align}
First, we can consider the off diagonal term
\begin{align}
    \left( \bm{1} \oplus_{\text{kron}} \bm{N}_{2OV,2OV} \right) \begin{pmatrix}
        \left(\bm{V}^{2\mathrm{h1p}}\right)^{\dagger} \\ \left( \bm{V}^{2\mathrm{h1p}}\right)^{\dagger}
    \end{pmatrix} _{2O^2V,P} &=\left( \bm{1}_O \otimes \bm{1}_{2OV} + \bm{1}_O \otimes \bm{N}_{2OV} \right) \begin{pmatrix}
        \left(\bm{V}^{2\mathrm{h1p}}\right)^{\dagger} \\ \left( \bm{V}^{2\mathrm{h1p}}\right)^{\dagger}
    \end{pmatrix} _{2O^2V,P}\\
& =\begin{pmatrix}
        \left(\bm{V}^{2\mathrm{h1p}}\right)^{\dagger} \\ \left( \bm{V}^{2\mathrm{h1p}}\right)^{\dagger}
    \end{pmatrix} _{2O^2V,P} + \begin{pmatrix}
        \left(\bm{V}^{2\mathrm{h1p}}\right)^{\dagger} \\ -\left( \bm{V}^{2\mathrm{h1p}}\right)^{\dagger}
    \end{pmatrix} _{2O^2V,P} \\
&= \begin{pmatrix}
        2\left(\bm{V}^{2\mathrm{h1p}}\right)^{\dagger} \\ 0
    \end{pmatrix} _{2O^2V,P}
\end{align}
Next, we can consider the auxiliary block
\begin{align}
 \left( \bm{1} \oplus_{\text{kron}} \bm{N} \right)\bm{C}^{2\mathrm{hlp}} &= \left(\bm{1}_{O} \oplus_{\text{kron}} \bm{N}_{2OV}\right) \left( \bm{\epsilon}^{1 \mathrm{~h}}_{O} \oplus_{\text{kron}} (-\bm{\tilde{M}})_{2OV} \right)\\
&= \left( \bm{1}_O \otimes \bm{1}_{2OV} + \bm{1}_O \otimes \bm{N}_{2OV} \right) \left( \bm{\epsilon}^{1 \mathrm{~h}}_{O} \otimes \bm{1}_{2OV} + \bm{1}_{O} \otimes (-\bm{\tilde{M}})_{2OV} \right)\\
&= \bm{\epsilon}^{1 \mathrm{~h}}_O \otimes \bm{1}_{2OV} + \underbrace{\bm{1}_O \otimes -\bm{\tilde{M}}_{2OV} + \bm{\epsilon}^{1 \mathrm{~h}}_O \otimes \bm{N}_{2OV}}_{\text{cross terms}} + \bm{1}_O \otimes -\bm{N}\bm{\tilde{M}}_{2OV}
\end{align}
\begin{align}
 &= \begin{pmatrix}
\epsilon^{1 \mathrm{~h}}_{1} \bm{1}_{2OV} & 0 & \dots & 0 \\[6pt]
0 & \epsilon^{1 \mathrm{~h}}_{2} \bm{1}_{2OV} & \dots & 0 \\[6pt]
\vdots & \vdots & \ddots & \vdots \\[6pt]
0 & 0 & \dots & \epsilon^{1 \mathrm{~h}}_{O} \bm{1}_{2OV}
\end{pmatrix} -  \begin{pmatrix}
1 \tilde{\bm{M}}_{2OV} & 0 & \dots & 0\\[6pt]
0 & 1 \tilde{\bm{M}}_{2OV} & \dots & 0\\[6pt]
\vdots & \vdots & \ddots & \vdots \\[6pt]
0 & 0 & \dots & 1 \tilde{\bm{M}}_{2OV}
\end{pmatrix}\\
&+ \begin{pmatrix}
\epsilon^{1 \mathrm{~h}}_{1} \bm{N}_{2OV} & 0 & \dots & 0 \\[6pt]
0 & \epsilon^{1 \mathrm{~h}}_{2} \bm{N}_{2OV} & \dots & 0 \\[6pt]
\vdots & \vdots & \ddots & \vdots \\[6pt]
0 & 0 & \dots & \epsilon^{1 \mathrm{~h}}_{O} \bm{N}_{2OV}   
\end{pmatrix} - \begin{pmatrix}
1 \left(\bm{N} \tilde{\bm{M}}\right)_{2OV} & 0 & \dots & 0\\[6pt]
0 & 1 \left(\bm{N} \tilde{\bm{M}}\right)_{2OV} & \dots & 0\\[6pt]
\vdots & \vdots & \ddots & \vdots \\[6pt]
0 & 0 & \dots &  1
    \left(\bm{N} \tilde{\bm{M}}\right)_{2OV}
\end{pmatrix}\\
&= \begin{pmatrix}
\epsilon^{1 \mathrm{~h}}_{1} \left(\bm{1} + \bm{N}\right)_{2OV} & \dots & 0 \\[6pt]
\vdots  & \ddots & \vdots \\[6pt]
0 & \dots & \epsilon^{1 \mathrm{~h}}_{O} \left(\bm{1} + \bm{N}\right)_{2OV}
\end{pmatrix} - \begin{pmatrix}
1 \left(\tilde{\bm{M}}+\bm{N} \tilde{\bm{M}}\right)_{2OV} & \dots & 0\\[6pt]
\vdots & \ddots & \vdots \\[6pt]
0 & \dots &  1 \left(\tilde{\bm{M}}+\bm{N} \tilde{\bm{M}}\right)_{2OV}
\end{pmatrix}\\
&= \begin{pmatrix}
\epsilon^{1 \mathrm{~h}}_{1} \left(\bm{1} + \bm{N}\right)_{2OV} - \left(\tilde{\bm{M}}+\bm{N} \tilde{\bm{M}}\right)_{2OV} & \dots & 0 \\[6pt]
\vdots  & \ddots & \vdots \\[6pt]
0 & \dots & \epsilon^{1 \mathrm{~h}}_{O} \left(\bm{1} + \bm{N}\right)_{2OV} - \left(\tilde{\bm{M}}+\bm{N} \tilde{\bm{M}}\right)_{2OV}
\end{pmatrix}\\
&= \begin{pmatrix}
\epsilon^{1 \mathrm{~h}}_{1} \left(\bm{1} + \bm{N}\right)_{2OV} -  \left[\left(\bm{1} + \bm{N}\right)\tilde{\bm{M}}\right]_{2OV} & \dots & 0 \\[6pt]
\vdots  & \ddots & \vdots \\[6pt]
0 & \dots & \epsilon^{1 \mathrm{~h}}_{O} \left(\bm{1} + \bm{N}\right)_{2OV} -  \left[\left(\bm{1} + \bm{N}\right)\tilde{\bm{M}}\right]_{2OV}
\end{pmatrix}
\end{align}
I have plotted 3 different data points from each molecular calculation. The blue bar shows the average deviation of the Ritz values from the exact solution using the expressions for the matrix vector products given in the main paper. The orange bar shows the average deviation of the Ritz values from the exact solution using the expressions for the matrix vector products given in the supplemental material (Supp. MV), where we set B=0. The green bar is just showing the norm of the final residual (when I reach the maximum number of iterations) I get when I run the Davidson iteration using Supp. MV. I am running for up to 75 iterations without doing any restarting. I interpret this as there not being a bug in my implementation since I see the same behavior across different molecules, but perhaps the expressions for the matrix vector products that he proposes in the supplemental material are just not exact. Do you have a different opinion?
\subsubsection{Reverse engineering}
We have the problem $\bm{\sigma} = \bm{\mathcal{N}}\bm{H} \bm{R}$. I know that $\bm{\sigma}$ is given by
\begin{align}
\sigma_i &= \sum_{j} f_{i j}\,r_j + \sum_{b} f_{i b}\,r_b + \sum_{k l c} \bigl\langle i\,c | k\,l \bigr\rangle\,r_{k[l c]} + \sum_{k c d} \bigl\langle i\,k | d\,c \bigr\rangle\,r_{[k c]d} + \sum_{k l c} \bigl\langle i\,c | k\,l \bigr\rangle\,\bar{r}_{k[l c]} + \sum_{k c d} \bigl\langle i\,k | d\,c \bigr\rangle\,\bar{r}_{[k c]d} \\
\sigma_a &= \sum_{j} f_{a j}\,r_j + \sum_{b} f_{a b}\,r_b + \sum_{k l c} \bigl\langle a\,c | k\,l \bigr\rangle\,r_{k[l c]} + \sum_{k c d} \bigl\langle a\,k | d\,c \bigr\rangle\,r_{[k c]d} + \sum_{k l c} \bigl\langle a\,c | k\,l \bigr\rangle\,\bar{r}_{k[l c]} + \sum_{k c d} \bigl\langle a\,k | d\,c \bigr\rangle\,\bar{r}_{[k c]d} \\
    \sigma _{i[ja]} &= \sum_{k} \bigl\langle k\,a | i\,j \bigr\rangle\,r_k + \sum_{b} \bigl\langle b\,a | i\,j \bigr\rangle\,r_b + \varepsilon_i r_{i[j a]} - \sum_{k b} \bigl[\mathbf{N} \tilde{\mathbf{M}}\bigr]_{j a k b}^{\mathrm{xx}} r_{i[k b]} - \sum_{k b} \bigl[\mathbf{N} \tilde{\mathbf{M}}\bigr]_{j a k b}^{\mathrm{xd}} \bar{r}_{i[k b]}
\label{sija} \\
    \bar{\sigma}_{i[ja]} &= -\sum_{k} \bigl\langle k\,a | i\,j \bigr\rangle\,r_k - \sum_{b} \bigl\langle b\,a | i\,j \bigr\rangle\,r_b + \varepsilon_i \bar{r}_{i[j a]} - \sum_{k b} \bigl[\mathbf{N} \tilde{\mathbf{M}}\bigr]_{j a, k b}^{\mathrm{dx}} r_{i[k b]} - \sum_{k b} \bigl[\mathbf{N} \tilde{\mathbf{M}}\bigr]_{j a, k b}^{\mathrm{dd}} \bar{r}_{i[k b]} \\
\sigma _{[ia]b} &= \sum_{j} \bigl\langle j\,i | b\,a \bigr\rangle\,r_j + \sum_{c} \bigl\langle c\,i | b\,a \bigr\rangle\,r_c + \varepsilon_b r_{[i a] b} + \sum_{jc}\bigl[ \mathbf{N} \tilde{\mathbf{M}}\bigr]_{i a, j c}^{\mathrm{xx}} r_{[j c] b} + \sum_{jc}\bigl[ \mathbf{N} \tilde{\mathbf{M}}\bigr]_{i a, j c}^{\mathrm{xd}} \bar{r}_{[j c] b} \\
\bar{\sigma}_{[ia]b} &= -\sum_{j} \bigl\langle j\,i | b\,a \bigr\rangle\,r_j - \sum_{c} \bigl\langle c\,i | b\,a \bigr\rangle\,r_c + \varepsilon_b \bar{r}_{[i a] b} + \sum_{jc}\bigl[ \mathbf{N} \tilde{\mathbf{M}}\bigr]_{i a, j c}^{\mathrm{dx}} r_{[j c] b} + \sum_{jc}\bigl[ \mathbf{N} \tilde{\mathbf{M}}\bigr]_{i a, j c}^{\mathrm{dd}} \bar{r}_{[j c] b}
\end{align}
$\bm{R}^{2h1p}= \begin{pmatrix}
r_i \\ r_a \\ r_{i[jb]} \\ \bar{r}_{i[\bar{jb}]} \\
\end{pmatrix} $ is the excitation vector. We will assume the form $\bm{\mathcal{N}}^{2h1p} = \begin{pmatrix}
\bm{1} & 0 \\
0 & \bm{X} \\
\end{pmatrix}$. We assume that $\bm{H}^{2h1p}$ is given by
\begin{equation}
\bm{H}^{2h1p} = \begin{pmatrix}
\bm{F} & \bm{V}^{2\mathrm{h1p}} & \bm{V}^{2\mathrm{h1p}}  \\
\left(\bm{V}^{2\mathrm{h1p}}\right)^{\dagger} &  &   \\
\left(\bm{V}^{2\mathrm{h1p}}\right)^{\dagger} &  & \bm{C}^{2\mathrm{hlp}} \\
\end{pmatrix}
\end{equation}
In our notation, $P=O+V$ is the number of molecular orbitals, $A=O^2V$ and $T$ is the column dimension of our excitation vector.
\begin{align}
&    \bm{\mathcal{N}}^{2h1p} \bm{H}^{2h1p} \bm{R}^{2h1p}\\
&= 
\begin{pmatrix}\bm{1}_{P,P} & 0 \\
0 & \bm{X}_{2A,2A}
\end{pmatrix}_{P+2A, P+2A}
\begin{pmatrix}\bm{F}_{P,P} & \bm{V}^{2\mathrm{h1p}}_{P,A} & \bm{V}^{2\mathrm{h1p}}_{P,A} \\
\left(\bm{V}^{2\mathrm{h1p}}\right)^{\dagger}_{A,P} &  &   \\
\left(\bm{V}^{2\mathrm{h1p}}\right)^{\dagger}_{A,P} &  & \bm{C}^{2\mathrm{hlp}}_{2A,2A} \\
\end{pmatrix}_{P+2A, P+2A}
\begin{pmatrix}\bm{r}_i \\ \bm{r}_a \\ \bm{r}_{i[jb]} \\ \bm{\bar{r}}_{i[\bar{jb}]}
\end{pmatrix}_{P+2A,T} \\
&= 
\begin{pmatrix}
\bm{1}_{P,P}
 \begin{pmatrix}
\bm{F}_{OO} & \bm{F}_{OV} \\ \bm{F}_{VO} & \bm{F}_{VV}
\end{pmatrix}_{P,P} \begin{pmatrix}
\bm{r}_i \\ \bm{r}_a
\end{pmatrix}_{P,T} + \bm{1}_{P,P} \begin{pmatrix}
\bm{V}^{2 \mathrm{h1p}}_{O,i[jb]} & \bm{V}^{2 \mathrm{h1p}}_{O,i[{jb}]} \\ \bm{V}^{2 \mathrm{h1p}}_{V,i[jb]} & \bm{V}^{2 \mathrm{h1p}}_{V,i[{jb}]} 
\end{pmatrix}_{P,2A} \begin{pmatrix}
    \bm{r}_{i[jb]} \\
\bm{\bar{r}}_{i[\bar{jb}]}
\end{pmatrix}_{2A,T} \\
\bm{X}_{2A,2A} \begin{pmatrix}\left(\bm{V}^{2 \mathrm{h1p}}_{O,i[jb]}\right)^\dag & \left(\bm{V}^{2 \mathrm{h1p}}_{V,i[{jb}]}\right)^\dag \\ \left(\bm{V}^{2 \mathrm{h1p}}_{O,i[jb]}\right)^\dag & \left(\bm{V}^{2 \mathrm{h1p}}_{V,i[{jb}]}\right)^\dag\end{pmatrix}_{2A,P} \begin{pmatrix}
\bm{r}_i \\ \bm{r}_a
\end{pmatrix}_{P,T} + \bm{X}_{2A,2A} \bm{C}^{2 \mathrm{hlp}}_{2A,2A} \begin{pmatrix}
    \bm{r}_{i[jb]} \\
\bm{\bar{r}}_{i[\bar{jb}]}
\end{pmatrix}_{2A,T}\\
%  \bm{N}_{2A,2A}\begin{pmatrix}
%     \left(\bm{V}^{2 \mathrm{h1p}}_{p,i[jb]}\right)^\dag \\ \left(\bm{V}^{2 \mathrm{h1p}}_{p,i[{jb}]}\right)^\dag \\
% \end{pmatrix}_{2A,P} \begin{pmatrix}
% \bm{r}_i \\ \bm{r}_a
% \end{pmatrix}_{P,A} +
% \bm{N}_{2A,2A}\left(\left(\varepsilon^{1h}\right) \oplus_{\text{direct}} -\bm{\tilde{M}}_{jb,jb;lc,lc}\right)_{2A,2A} \begin{pmatrix}
%     \bm{r}_{i[jb]} \\
% \bm{\bar{r}}_{i[\bar{jb}]}
% \end{pmatrix}_{2A,A}
\end{pmatrix} \\
&= \begin{pmatrix}
\bm{F}_{OO} \bm{r}_i + \bm{F}_{OV} \bm{r}_a + \bm{V}^{2 \mathrm{h1p}}_{O,i[jb]} \bm{r}_{i[jb]} + \bm{V}^{2 \mathrm{h1p}}_{O,i[jb]} \bm{\bar{r}}_{i[\bar{jb}]} \\
\bm{F}_{VO} \bm{r}_i + \bm{F}_{VV} \bm{r}_a + \bm{V}^{2 \mathrm{h1p}}_{V,i[jb]} \bm{r}_{i[jb]} + \bm{V}^{2 \mathrm{h1p}}_{V,i[jb]} \bm{\bar{r}}_{i[\bar{jb}]} \\[6pt]
?\\
\end{pmatrix}_{P+2A,T} \\
&= \begin{pmatrix}
\sum_j f_{ij} r_j + \sum_b f_{ib} r_b + \sum_{k l c} \bigl\langle i\,c | k\,l \bigr\rangle r_{k[l c]} + \sum_{k l c} \bigl\langle i\,c | k\,l \bigr\rangle \bar{r}_{k[l c]}  \\
\sum_j f_{aj} r_j + \sum_b f_{ab} r_b + \sum_{k l c} \bigl\langle a\,c | k\,l \bigr\rangle r_{k[l c]} + \sum_{k l c} \bigl\langle a\,c | k\,l \bigr\rangle \bar{r}_{k[l c]} \\
?\\
% \sum_{k} \bigl\langle k\,a | i\,j \bigr\rangle\,r_k + \sum_{b} \bigl\langle b\,a | i\,j \bigr\rangle\,r_b + \varepsilon_i r_{i[j a]} - \sum_{k b} \bigl[\mathbf{N} \tilde{\mathbf{M}}\bigr]_{j a k b}^{\mathrm{xx}} r_{i[k b]} - \sum_{k b} \bigl[\mathbf{N} \tilde{\mathbf{M}}\bigr]_{j a k b}^{\mathrm{xd}} \bar{r}_{i[k b]}\\
%  -\sum_{k} \bigl\langle k\,a | i\,j \bigr\rangle\,r_k - \sum_{b} \bigl\langle b\,a | i\,j \bigr\rangle\,r_b + \varepsilon_i \bar{r}_{i[j a]} - \sum_{k b} \bigl[\mathbf{N} \tilde{\mathbf{M}}\bigr]_{j a, k b}^{\mathrm{dx}} r_{i[k b]} - \sum_{k b} \bigl[\mathbf{N} \tilde{\mathbf{M}}\bigr]_{j a, k b}^{\mathrm{dd}} \bar{r}_{i[k b]}
\end{pmatrix} \\
\end{align}
Now let's consider that $\bm{X}_{2A,2A} = \bm{1}_O \otimes \bm{N}_{2OV,2OV} =\bm{1}_O \otimes \begin{pmatrix}
\bm{1}_{OV,OV} & 0 \\
0 & -\bm{1}_{OV,OV}
\end{pmatrix}
$.Then, it becomes true that
\begin{align}
&\left( \bm{1}_O \otimes \bm{N}_{2OV,2OV} \right) \begin{pmatrix}\left(\bm{V}^{2 \mathrm{h1p}}_{O,i[jb]}\right)^\dag & \left(\bm{V}^{2 \mathrm{h1p}}_{V,i[{jb}]}\right)^\dag \\ \left(\bm{V}^{2 \mathrm{h1p}}_{O,i[jb]}\right)^\dag & \left(\bm{V}^{2 \mathrm{h1p}}_{V,i[{jb}]}\right)^\dag\end{pmatrix}_{2A,P} \begin{pmatrix}
\bm{r}_i \\ \bm{r}_a
\end{pmatrix}_{P,T}\\
 &= \begin{pmatrix}\left(\bm{V}^{2 \mathrm{h1p}}_{O,i[jb]}\right)^\dag & \left(\bm{V}^{2 \mathrm{h1p}}_{V,i[{jb}]}\right)^\dag \\ \left(-\bm{V}^{2 \mathrm{h1p}}_{O,i[jb]}\right)^\dag & -\left(\bm{V}^{2 \mathrm{h1p}}_{V,i[{jb}]}\right)^\dag\end{pmatrix}_{2A,P} \begin{pmatrix}
\bm{r}_i \\ \bm{r}_a
\end{pmatrix}_{P,T} \\
&= \begin{pmatrix}
\sum_{k} \bigl\langle k\,a | i\,j \bigr\rangle\,r_k + \sum_{b} \bigl\langle b\,a | i\,j \bigr\rangle\,r_b \\
-\sum_{k} \bigl\langle k\,a | i\,j \bigr\rangle\,r_k - \sum_{b} \bigl\langle b\,a | i\,j \bigr\rangle\,r_b
\end{pmatrix}_{2A,T}
\end{align}
But it is still unclear how terms like \ref{sija}, arising from the auxiliary space, come about.
\subsection{My hypothesis for the supermatrix}
By reverse engineering from the matrix vector products given in the supplemental material, I have found the form (we have defined $P=O+V$)
\begin{align}
    \bm{H} = \begin{pmatrix}
\bm{F}_{P,P} & \begin{pmatrix} \bm{V}^{2\mathrm{h1p}} & \bm{V}^{2\mathrm{h1p}} \end{pmatrix}_{P,2O^2V} & \begin{pmatrix} \bm{V}^{2\mathrm{p1h}} & \bm{V}^{2\mathrm{p1h}} \end{pmatrix}_{P,2OV^2}\\
\begin{pmatrix} \left(\bm{V}^{2\mathrm{h1p}}\right)^{\dagger} \\ -\left(\bm{V}^{2\mathrm{h1p}}\right)^{\dagger} \end{pmatrix}_{2O^2V,P} & \bm{C}^{2\mathrm{hlp}}_{2O^2V,2O^2V} & \bm{0} \\
\begin{pmatrix} \left(\bm{V}^{2\mathrm{p1h}}\right)^{\dagger} \\ -\left(\bm{V}^{2\mathrm{p1h}}\right)^{\dagger} \end{pmatrix}_{2OV^2,P} & \bm{0} & \bm{C}^{2\mathrm{plh}}_{2OV^2,2OV^2} \\
\end{pmatrix}
\label{supermatrix}
\end{align}
with 
\begin{align}
    V_{p, k[ia]}^{2h1p} &= \langle p a | k i \rangle \equiv (pk|ia) \\
    V_{p, [ia] c}^{2p1h} &= \langle p i | a c \rangle \equiv (pc|ia) \\
    % C_{i[j a], k[l c]}^{2h1p} &= \left[\left(\epsilon_i+\epsilon_j-\epsilon_a\right) \delta_{j l} \delta_{a c}-\langle j c | a l \rangle\right] \delta_{i k}
\label{c}
\end{align}
and 
\begin{align}
\label{c2h1p}
    {C}^{2 \mathrm{hlp}}_{i[ka],j[lb]} &= \begin{pmatrix}
{\varepsilon}^{1h}_{ij}\delta_{kl} \delta_{ab} - \bigl[\mathbf{N} \tilde{\mathbf{M}}\bigr]^{\mathrm{xx}}_{ka,lb} \delta _{ij} & - \bigl[\mathbf{N} \tilde{\mathbf{M}}\bigr]^{\mathrm{xd}}_{ka,lb} \delta _{ij} \\
- \bigl[\mathbf{N} \tilde{\mathbf{M}}\bigr]^{\mathrm{dx}}_{ka,lb}\delta _{ij} & {\varepsilon}^{1h}_{ij}\delta_{kl} \delta_{ab} - \bigl[\mathbf{N} \tilde{\mathbf{M}}\bigr]^{\mathrm{dd}}_{ja,lb} \delta _{ij} \\
\end{pmatrix} = \delta_{kl}\delta_{ab}\left(\bm{\varepsilon}^{1h}\oplus_{\text{diag}} \bm{\varepsilon}^{1h}\right) -\delta_{ij} \bm{N}\tilde{\mathbf{M}}\\
\label{c2p1h}
{C}^{2 \mathrm{plh}}_{[kc]a,[ld]b} &= \begin{pmatrix}
{\varepsilon}^{1p}_{ab}\delta_{kl} \delta_{cd} + \bigl[\mathbf{N} \tilde{\mathbf{M}}\bigr]^{\mathrm{xx}}_{kc,ld} \delta _{ab} &  \bigl[\mathbf{N} \tilde{\mathbf{M}}\bigr]^{\mathrm{xd}}_{kc,ld} \delta _{ab} \\ 
    \bigl[\mathbf{N} \tilde{\mathbf{M}}\bigr]^{\mathrm{dx}}_{kc,ld}\delta _{ab} & {\varepsilon}^{1p}_{ab}\delta_{kl} \delta_{cd} + \bigl[\mathbf{N} \tilde{\mathbf{M}}\bigr]^{\mathrm{dd}}_{kc,ld} \delta _{ab} \\
\end{pmatrix} = \delta_{kl}\delta_{cd}\left(\bm{\varepsilon}^{1p}\oplus_{\text{diag}} \bm{\varepsilon}^{1p}\right) +\delta_{ab} \bm{N}\tilde{\mathbf{M}}
\end{align}
with $\bm{\varepsilon}^{1h}$ and $\bm{\varepsilon}^{1p}$ diagonal matrices with the 1h and 1p energies on the diagonal, so actually $\varepsilon^{1h}_{ij} = \varepsilon^{1h}_{ii} \delta_{ij}$ and $\varepsilon^{1p}_{ab} = \varepsilon^{1p}_{aa} \delta_{ab}$, respectively and $\tilde{\bm{M}} = \bm{M} + \eta \bm{N}\Theta(-\bm{N}\bm{M})$ 
where
\begin{align}
\bm{M} &= \begin{pmatrix}
\bm{A} & \bm{B} \\
\bm{B} & \bm{A}
\end{pmatrix}_{2OV,2OV} \quad
\bm{N} = \begin{pmatrix}
\bm{1} & \bm{0} \\
\bm{0} & -\bm{1}
\end{pmatrix}_{2OV,2OV}
\end{align}
and $\eta$ is some very large number.
\subsubsection{Connecting to the exact form of GW-RPA}
We know the eigendecomposition $\bm{N}\tilde{\mathbf{M}}=\bm{Z}\begin{pmatrix} \bm{\Omega}_{+} & 0 \\ 0 & -\bm{\Omega}_{+} + \bm{\eta} \end{pmatrix}\bm{Z}^{-1}$. This means we can rewrite \ref{c2h1p} and \ref{c2p1h} as
\begin{align}
    {C}^{2 \mathrm{hlp}}_{i[ka],j[lb]} &= \delta_{kl}\delta_{ab}\left(\bm{\varepsilon}^{1h}\oplus_{\text{diag}} \bm{\varepsilon}^{1h}\right) -\delta_{ij} \bm{Z}\begin{pmatrix} \bm{\Omega}_{+} & 0 \\ 0 & -\bm{\Omega}_{+} + \bm{\eta} \end{pmatrix}\bm{Z}^{-1}\\
    {C}^{2 \mathrm{plh}}_{[kc]a,[ld]b} &= \delta_{kl}\delta_{cd}\left(\bm{\varepsilon}^{1p}\oplus_{\text{diag}} \bm{\varepsilon}^{1p}\right) +\delta_{ab} \bm{Z}\begin{pmatrix} \bm{\Omega}_{+} & 0 \\ 0 & -\bm{\Omega}_{+} + \bm{\eta} \end{pmatrix}\bm{Z}^{-1}
\end{align}
\subsubsection{Why would we want an Mtilde of this form?}
Now, we know the following identity to be true
\begin{equation}
\bm{M} \bm{Z} = \bm{N} \bm{Z} \begin{pmatrix}
\boldsymbol{\Omega}_{+} & 0 \\
0 & -\boldsymbol{\Omega}_{+}
\end{pmatrix}
\end{equation}
% $$
with eigenvectors $\bm{Z}=\begin{pmatrix}
\bm{X} & \bm{Y} \\
\bm{Y} & \bm{X}
\end{pmatrix}_{2OV,2OV}$ and eigenvalues $\begin{pmatrix}
\bm{\Omega}_{+} & 0 \\
0 & -\bm{\Omega}_{+}
\end{pmatrix}$.
Now we can use the fact that the action of a scalar function $f$, such as the step function, on a diagonalizable matrix $\bm{X} \equiv \bm{Y} \bm{\Lambda} \bm{Y}^{-1}$ can be expressed as $f(\bm{X}) = \bm{Y} f(\bm{\Lambda}) \bm{Y}^{-1}$
so we can write
\begin{align}
    &\Theta(-\bm{N}\bm{M}) = \bm{Z} \left(\begin{array}{cc}\Theta(-\boldsymbol{\Omega}_{+}) & 0 \\ 0 & \Theta(\boldsymbol{\Omega}_{+})\end{array}\right)\bm{Z}^{-1} = \bm{Z} \left(\begin{array}{cc}\bm{0} & 0 \\ 0 & \bm{1}\end{array}\right)\bm{Z}^{-1} \\
& \implies \tilde{\bm{M}} \bm{Z} = \bm{N}\bm{Z}\left(\begin{array}{cc}\boldsymbol{\Omega}_{+} & 0 \\ 0 & -\boldsymbol{\Omega}_{+} + \bm{\eta}\end{array}\right) \quad \text{and} \quad \tilde{\bm{M}} = \bm{Z}\bm{N}\left(\begin{array}{cc}\boldsymbol{\Omega}_{+} & 0 \\ 0 & -\boldsymbol{\Omega}_{+} + \bm{\eta}\end{array}\right)\bm{Z}^{-1}
\end{align}
So now the purpose of the $\eta$ becomes clear; it is removing the influence of the negative RPA roots, so that in the GW supermatrix we can easily disregard the portion of its spectrum which inherits from the negative RPA roots. After all, the expression for the GW correlation self-energy is
\begin{equation}
\Sigma_{pq}^{c}(\omega) = \sum_{\mu}^{RPA} \left[\underbrace{\sum_{i}^{occ} \frac{\rho_{pi}^\mu \rho_{qi}^\mu}{\omega - (\epsilon_i - \Omega_\mu) - i\eta}}_{2h1p} + \underbrace{\sum_{a}^{virt} \frac{\rho_{pa}^n \rho_{qa}^n}{\omega - (\epsilon_a + \Omega_\mu) + i\eta}}_{1h2p}\right]
\end{equation}
with the transition densities $\rho_{rs}^\nu = \sum_{ia} (rs|ia) \left(X_{ia}^\nu + Y_{ia}^\nu\right)$, where $\nu$ is the RPA index. 
The standard way to show that this maps to GW-RPA is via similarity transformations, like we did with GW-TDA in \ref{sec:gwtda_sim}, but we can't do this her because $\bm{N}\tilde{\bm{M}}$ is not a Hermitian.
\subsubsection{Avoid constructing Theta}
Right now in order to construct $\Theta(-\bm{N}{\bm{M}})$, I have to diagonalize $\bm{N}{\bm{M}}$ first, which is $O(N^6)$ cost. To avoid this, we can approximate the step function by Chebyshev polynomials. But we don't want to form this Chebyshev polynomial at the beginning of the Davidson iteration, but rather just make an approximation, that will keep improving as we iterate. It is true, though, that we can efficiently estimate the bounds of the spectrum of $\bm{N}{\bm{M}}$ using a few iterations of Arnoldi, getting $\lambda_{\text{max}}$ (because this is RPA, we know that $\lambda_{\text{min}} = -\lambda_{\text{max}} $). So we can rescale the matrix so that its spectrum lies within the interval $[-1,1]$; this is accomplished by dividing by $\lambda_{\text{max}}$. Then, at each iteration of Davidson, we can apply this polynomial to the current guess vectors. This will be $O(N^4)$ cost per iteration, which is not too bad. The hope is that as we iterate, the guess vectors will span more and more of the space spanned by the negative eigenvalues of $\bm{N}{\bm{M}}$, so that we are effectively applying the step function to the relevant subspace.
\subsubsection{Attempting to transform to a Hermitian matrix}
\label{subsub:attempting_to_transform_to_a_hermitian_matrix}
The supermatrix \ref{supermatrix} is not pseudohermitian, or Hermitian under the super metric, so it really is just a similarity transform
\begin{align}
&\bm{\mathcal{N}} = \begin{pmatrix}
\bm{1}_{P,P} & \bm{0} & \bm{0} \\
\bm{0} & \bm{1}_{O,O} \otimes \bm{N}_{2OV,2OV} & \bm{0} \\
\bm{0} & \bm{0} & \bm{1}_{V,V} \otimes \bm{N}_{2OV,2OV}
\end{pmatrix}\\
&\implies \bm{H}^\dagger = \bm{\mathcal{N}} \bm{H} \bm{\mathcal{N}}^{-1}\\
& = \begin{pmatrix}
\bm{1} & \bm{0} & \bm{0} \\
\bm{0} & \bm{1} \otimes \bm{N} & \bm{0} \\
\bm{0} & \bm{0} & \bm{1} \otimes \bm{N}
\end{pmatrix}
\begin{pmatrix}
\bm{F} & \begin{pmatrix} \bm{V}^{2\mathrm{h1p}} & \bm{V}^{2\mathrm{h1p}} \end{pmatrix} & \begin{pmatrix} \bm{V}^{2\mathrm{p1h}} & \bm{V}^{2\mathrm{p1h}} \end{pmatrix} \\
\begin{pmatrix} \left(\bm{V}^{2\mathrm{h1p}}\right)^{\dagger} \\ -\left(\bm{V}^{2\mathrm{h1p}}\right)^{\dagger} \end{pmatrix} & \bm{C}^{2\mathrm{hlp}} & \bm{0} \\
\begin{pmatrix} \left(\bm{V}^{2\mathrm{p1h}}\right)^{\dagger} \\ -\left(\bm{V}^{2\mathrm{p1h}}\right)^{\dagger} \end{pmatrix} & \bm{0} & \bm{C}^{2\mathrm{plh}} \\
\end{pmatrix}
\begin{pmatrix}
\bm{1} & \bm{0} & \bm{0} \\
\bm{0} & \bm{1} \otimes \bm{N} & \bm{0} \\
\bm{0} & \bm{0} & \bm{1} \otimes \bm{N}
\end{pmatrix} \\
&=
\begin{pmatrix}
\bm{F} & \begin{pmatrix} \bm{V}^{2\mathrm{h1p}} & \bm{V}^{2\mathrm{h1p}} \end{pmatrix} & \begin{pmatrix} \bm{V}^{2\mathrm{p1h}} & \bm{V}^{2\mathrm{p1h}} \end{pmatrix} \\
\left(\bm{1}\otimes \bm{N}\right)\begin{pmatrix} \left(\bm{V}^{2\mathrm{h1p}}\right)^{\dagger} \\ -\left(\bm{V}^{2\mathrm{h1p}}\right)^{\dagger} \end{pmatrix} & \left(\bm{1}\otimes \bm{N}\right)\bm{C}^{2\mathrm{hlp}} & \bm{0} \\
\left(\bm{1}\otimes \bm{N}\right)\begin{pmatrix} \left(\bm{V}^{2\mathrm{p1h}}\right)^{\dagger} \\ -\left(\bm{V}^{2\mathrm{p1h}}\right)^{\dagger} \end{pmatrix} & \bm{0} & \left(\bm{1}\otimes \bm{N}\right)\bm{C}^{2\mathrm{plh}} \\
\end{pmatrix}
\begin{pmatrix}
\bm{1} & \bm{0} & \bm{0} \\
\bm{0} & \bm{1} \otimes \bm{N} & \bm{0} \\
\bm{0} & \bm{0} & \bm{1} \otimes \bm{N}
\end{pmatrix} \\
\end{align}
To simplify notation, from here on, we will just consider the 2h1p sector
\begin{align}
&\bm{\mathcal{N}}^{2\mathrm{h1p}} \bm{H}^{2\mathrm{h1p}} \left(\bm{\mathcal{N}}^{2\mathrm{h1p}}\right)^{-1} \\
&= \begin{pmatrix}
\bm{F}_{P} & 
    \left[\begin{pmatrix} 
        \bm{V}^{2\mathrm{h1p}} & \bm{V}^{2\mathrm{h1p}}
    \end{pmatrix}  \left(\bm{1}\otimes \bm{N}\right)\right]_{P,2O^2V}  \\[1em]
\left[\left(\bm{1}\otimes \bm{N}\right)
    \begin{pmatrix} 
        \left(\bm{V}^{2\mathrm{h1p}}\right)^{\dagger} \\ 
        -\left(\bm{V}^{2\mathrm{h1p}}\right)^{\dagger} 
    \end{pmatrix}\right]_{2O^2V,P} & 
    \left(\bm{1}\otimes \bm{N}\right)_{2O^2V}\bm{C}^{2\mathrm{hlp}}_{2O^2V}\left(\bm{1}\otimes \bm{N}\right)_{2O^2V} \\
\end{pmatrix}
\end{align}
Consider just the multiplication 
\begin{align}
&\begin{pmatrix} 
        \bm{V}^{2\mathrm{h1p}} & \bm{V}^{2\mathrm{h1p}}
    \end{pmatrix}  \left(\bm{1}\otimes \bm{N}\right) =\begin{pmatrix} 
        \bm{V}^{2\mathrm{h1p}} & -\bm{V}^{2\mathrm{h1p}}
    \end{pmatrix} \\
\end{align}
and
\begin{align}
\left(\bm{1}\otimes \bm{N}\right)
    \begin{pmatrix} 
        \left(\bm{V}^{2\mathrm{h1p}}\right)^{\dagger} \\ 
        -\left(\bm{V}^{2\mathrm{h1p}}\right)^{\dagger} 
    \end{pmatrix} =
    \begin{pmatrix} 
        \left(\bm{V}^{2\mathrm{h1p}}\right)^{\dagger} \\ 
        \left(\bm{V}^{2\mathrm{h1p}}\right)^{\dagger} 
    \end{pmatrix}
\end{align}
But now we are left to evaluate
\begin{align}
    &\left(\bm{1}\otimes \bm{N}\right)_{2O^2V,2O^2V} \bm{C}^{2\mathrm{hlp}}_{2O^2V,2O^2V} \left(\bm{1}\otimes \bm{N}\right)_{2O^2V,2O^2V}\\
&= \sum_{i'[k'a']}\sum_{j'[l'b']}\left(\delta_{ii'}N_{k'a',ka}\right)\left[ \begin{pmatrix}
{\varepsilon}^{1h}_{i'j'}\delta_{k'l'} \delta_{a'b'}  & 0 \\
0 & {\varepsilon}^{1h}_{i'j'}\delta_{k'l'} \delta_{a'b'}
\end{pmatrix} - \begin{pmatrix}
\delta _{i'j'}\bigl[\mathbf{N} \tilde{\mathbf{M}}\bigr]^{\mathrm{xx}}_{k'a',l'b'} & \delta _{i'j'}\bigl[\mathbf{N} \tilde{\mathbf{M}}\bigr]^{\mathrm{xd}}_{k'a',l'b'} \\
\delta _{i'j'}\bigl[\mathbf{N} \tilde{\mathbf{M}}\bigr]^{\mathrm{dx}}_{k'a',l'b'} & \delta _{i'j'}\bigl[\mathbf{N} \tilde{\mathbf{M}}\bigr]^{\mathrm{dd}}_{k'a',l'b'}
\end{pmatrix}\right] \\
& \times \left(\delta_{jj'}N_{lb,l'b'}\right) \\
&= \sum_{i'[k'a']}\sum_{j'[l'b']} \begin{pmatrix} \delta _{ii'}\delta_{k'k}\delta_{a'a}&0\\
0&-\delta _{ii'}\delta_{k'k}\delta_{a'a}
\end{pmatrix}
\left[ \begin{pmatrix}
{\varepsilon}^{1h}_{i'j'}\delta_{k'l'} \delta_{a'b'}  & 0 \\
0 & {\varepsilon}^{1h}_{i'j'}\delta_{k'l'} \delta_{a'b'}
\end{pmatrix} - \begin{pmatrix}
\delta _{i'j'}\bigl[\mathbf{N} \tilde{\mathbf{M}}\bigr]^{\mathrm{xx}}_{k'a',l'b'} & \delta _{i'j'}\bigl[\mathbf{N} \tilde{\mathbf{M}}\bigr]^{\mathrm{xd}}_{k'a',l'b'} \\
\delta _{i'j'}\bigl[\mathbf{N} \tilde{\mathbf{M}}\bigr]^{\mathrm{dx}}_{k'a',l'b'} & \delta _{i'j'}\bigl[\mathbf{N} \tilde{\mathbf{M}}\bigr]^{\mathrm{dd}}_{k'a',l'b'}
\end{pmatrix}\right]\\
&\times  \begin{pmatrix} \delta _{jj'}\delta_{ll'}\delta_{aa'}&0\\
0&-\delta _{jj'}\delta_{ll'}\delta_{aa'}
\end{pmatrix} \\
&= \left[ \begin{pmatrix}
{\varepsilon}^{1h}_{ij}\delta_{kl} \delta_{ab}  & 0 \\
0 & {\varepsilon}^{1h}_{ij}\delta_{kl} \delta_{ab}
\end{pmatrix} - \sum_{i'[k'a']}\sum_{j'[l'b']}\begin{pmatrix} \delta _{ii'}\delta_{k'k}\delta_{a'a}&0\\
0&-\delta _{ii'}\delta_{k'k}\delta_{a'a}
\end{pmatrix}\begin{pmatrix}
\delta _{i'j'}\bigl[\mathbf{N} \tilde{\mathbf{M}}\bigr]^{\mathrm{xx}}_{k'a',l'b'} & \delta _{i'j'}\bigl[\mathbf{N} \tilde{\mathbf{M}}\bigr]^{\mathrm{xd}}_{k'a',l'b'} \\
\delta _{i'j'}\bigl[\mathbf{N} \tilde{\mathbf{M}}\bigr]^{\mathrm{dx}}_{k'a',l'b'} & \delta _{i'j'}\bigl[\mathbf{N} \tilde{\mathbf{M}}\bigr]^{\mathrm{dd}}_{k'a',l'b'}
\end{pmatrix}\right.\\
&\left.\times\begin{pmatrix} \delta _{jj'}\delta_{ll'}\delta_{aa'}&0\\
0&-\delta _{jj'}\delta_{ll'}\delta_{aa'}
\end{pmatrix}\right]\\
&=  \left[ \begin{pmatrix}
{\varepsilon}^{1h}_{ij}\delta_{kl} \delta_{ab}  & 0 \\
0 & {\varepsilon}^{1h}_{ij}\delta_{kl} \delta_{ab}
\end{pmatrix} - \delta _{ij}\begin{pmatrix}
\bigl[\mathbf{N} \tilde{\mathbf{M}}\bigr]^{\mathrm{xx}}_{ka,lb} & -\bigl[\mathbf{N} \tilde{\mathbf{M}}\bigr]^{\mathrm{xd}}_{ka,lb} \\
-\bigl[\mathbf{N} \tilde{\mathbf{M}}\bigr]^{\mathrm{dx}}_{ka,lb} & \bigl[\mathbf{N} \tilde{\mathbf{M}}\bigr]^{\mathrm{dd}}_{ka,lb}
\end{pmatrix}\right]
\end{align}
This is nothing new, but I have left it here in case it becomes useful.
%  \left[\begin{pmatrix} 
%         \bm{V}^{2\mathrm{h1p}}_{P,O^2V} & \bm{V}^{2\mathrm{h1p}}
%     \end{pmatrix}  \left(\bm{1}\otimes \bm{N}\right)\right] = \begin{pmatrix}
%     V^{2\mathrm{h1p}}_{n,\,k[ia]} & V^{2\mathrm{h1p}}_{n,\,k[ia]}\\
%     V^{2\mathrm{h1p}}_{d,\,k[ia]} & V^{2\mathrm{h1p}}_{d,\,k[ia]}
% \end{pmatrix} \left(\bm{1}_O\otimes \bm{N}_{2OV}\right)$.
% \subsubsection{Starting from the downfolding}
% To recap the relevant equations, we have
% \begin{align}
% \begin{pmatrix}
% \omega - \bm{H}_{\mathcal{SS}} & -\bm{H}_{\mathcal{SL}} \\
% -\bm{H}_{\mathcal{LS}} & \omega - \bm{H}_{\mathcal{LL}}
% \end{pmatrix}
% \begin{pmatrix}
% \bm{G}_{\mathcal{SS}} & \bm{G}_{\mathcal{SL}} \\
% \bm{G}_{\mathcal{LS}} & \bm{G}_{\mathcal{LL}}
% \end{pmatrix}
% =
% \begin{pmatrix}
% \bm{I}_{\mathcal{SS}} & \bm{0} \\
% \bm{0} & \bm{I}_{\mathcal{LL}}
% \end{pmatrix}
% \end{align}
% leading to 
% \begin{align}
%     \bm{G}_{\mathcal{SS}} = \left[(\omega - \bm{H}_{\mathcal{SS}}) - \bm{H}_{\mathcal{SL}}\,(\omega - \bm{H}_{\mathcal{LL}})^{-1}\,\bm{H}_{\mathcal{LS}}\right]^{-1} = \left[(\omega - \left[\bm{F} + \bm{\Sigma }(\infty)\right]) - \bm{H}_{\mathcal{SL}}\,(\omega - \bm{H}_{\mathcal{LL}})^{-1}\,\bm{H}_{\mathcal{LS}}\right]^{-1} 
% \end{align}
% Now, the correlation self-energy can be defined by $\bm{H}_{\mathcal{SL}}\,(\omega - \bm{H}_{\mathcal{LL}})^{-1}\,\bm{H}_{\mathcal{LS}}$. In the GW approximation, it is
% \subsubsection{Contour integration}
% We can define a Green's function $\mathcal{\bm{G}}$ for the matrix $\bm{M}=(\bm{A}-\bm{B})^{1 / 2}(\bm{A}+\bm{B})(\bm{A}-\bm{B})^{1 / 2}$ as
% \begin{equation}
% \mathcal{\bm{G}}(\omega) = \frac{1}{\omega - \bm{M}} = \frac{1}{\omega - (\bm{A}-\bm{B})^{1 / 2}(\bm{A}+\bm{B})(\bm{A}-\bm{B})^{1 / 2}}
% \end{equation}
% As shown in the original FEAST paper, if $\bm{T}$ is the matrix of eigenvectors of $\bm{M}$ and $\bm{\Lambda}$ is the diagonal matrix of eigenvalues, we can write
% \begin{equation}
% \bm{T}\bm{T}^{\dagger} = \oint_{\mathcal{C}} \frac{\mathcal{\bm{G}}(\omega)}{2\pi i} d\omega
% \end{equation}
% The eigenvalues of $\bm{M}$ are the same as those of
% % \begin{equation}
% %     \bm{M}=(\bm{A}-\bm{B})^{1 / 2}(\bm{A}+\bm{B})(\bm{A}-\bm{B})^{1 / 2}
% % \label{eq:berk_gw_M}
% % \end{equation}
\subsubsection{Identification analysis}
Some explanation of the terminology is in order. When I talk about a priori identification, I mean that I identify eigenvalues within the spectrum as QPEs by identifying the eigenpair whose eigenvector has the largest overlap with the MO. This is what I have and still am always doing to identify QPEs in the exact ED spectrum; mathematically, it seemed like the most straightforward approach. However, because there is not a unitary transformation to relate the Supp. supermatrix to the exact one, I suspect that this method may not work for the Supp. supermatrix. In a posteriori identification, I choose which eigenvalue of Supp. supermatrix is closest to the exact QPE, by computing all the differences after both diagonalizations have proceeded. Note that this is not a sustainable approach, but I have included it to highlight that my a priori identification method is not working for the Supp. supermatrix. Alternatively, I have seen in this GW supermatrix literature sometimes people adopt different methods for selecting QPEs from the spectrum, so maybe I can try out some other ones. Attached is a screenshot from Tim's paper. Maybe I can try one of the alternatives he proposes. So I see two things that I need to do:
\begin{enumerate}
    \item Explore different methods for selecting QPEs from the spectrum of the Supp. supermatrix, like the ones proposed in Tim's paper.
    \item Find a way to mathematically bound the difference of the Supp. supermatrix with the exact one. From Googling, I see that there are some different ways, for instance, from the idea for the pseudospectrum. 
\end{enumerate}
The first plot that I have attached has the exact same format as what I sent you earlier today, but with CO included, as you requested. The following plots look through all the roots for certain molecule/parameter pairings for CO and N2; these are the parameters that seem to be presenting me with the most identification trouble.
\subsubsection{What to do next?}
I find numerically that all the eigenvalues of the Supp. supermatrix \ref{supermatrix} are real. This implies that there is some metric with respect to which the matrix is pseudo-Hermitian. I have not been able to find it yet. I have tried
\begin{equation}
    \bm{\mathcal{N}} = \begin{pmatrix}
\bm{1}_{P,P} & \bm{0} & \bm{0} \\
\bm{0} & \bm{1}_{O,O} \otimes \bm{N}_{2OV,2OV} & \bm{0} \\
\bm{0} & \bm{0} & \bm{1}_{V,V} \otimes \bm{N}_{2OV,2OV}
\end{pmatrix}
\end{equation}
but this doesn't work (see \ref{subsub:attempting_to_transform_to_a_hermitian_matrix}). I probably want to inverstigate this further first, before I move on to a non-Hermitian Davidson, because if I can find the metric, I can use a Hermitian Davidson, which is simpler.
% To figure out how to recreate them consider the matrix multiplication for just the 2h1p channel
% \begin{align}
% &\bm{\mathcal{N}}^{2\mathrm{h1p}} \bm{H}^{2\mathrm{h1p}} \bm{R}^{2\mathrm{h1p}} \\
% &= \begin{pmatrix}
% \bm{1}_{P,P} & 0 \\
% 0 & \bm{X} \\
% \end{pmatrix}
% \begin{pmatrix}
% \bm{F}_{P,P} & \bm{V}^{2\mathrm{h1p}} & \bm{V}^{2\mathrm{h1p}} \\
% \left(\bm{V}^{2\mathrm{h1p}}\right)^{\dagger} &  &  \\
% \left(\bm{V}^{2\mathrm{h1p}}\right)^{\dagger} &  & \bm{C}^{2\mathrm{hlp}}
% \end{pmatrix}
% \begin{pmatrix}
% r_i \\
% r_a \\
% r_{i[j b]} \\
% \bar{r}_{i[j b]} \\
% \end{pmatrix}\\
% &= \begin{pmatrix}
% \bm{1}_{P,P} & 0 \\
% 0 & \bm{X} \\
% \end{pmatrix}
% \begin{pmatrix}
% \bm{f}_{O,O}r^i_{O,\bar{A}} + \bm{f}_{O,V}r^a_{V,\bar{A}} + \bm{v}^{2\mathrm{h1p}}_{O,O^2V} r^{i[j b]}_{O^2V, \bar{A}} + \bm{v}^{2\mathrm{h1p}}_{O,O^2V} \bar{r}^{i[j b]}_{O^2V, \bar{A}} \\
% \bm{f}_{V,O}r^i_{O,\bar{A}} + \bm{f}_{V,V}r^a_{V,\bar{A}} + \bm{v}^{2\mathrm{h1p}}_{V,O^2V} r^{i[j b]}_{O^2V, \bar{A}} + \bm{v}^{2\mathrm{h1p}}_{V,O^2V} \bar{r}^{i[j b]}_{O^2V, \bar{A}} \\
% \bm{v}^{2\mathrm{h1p},\dagger}_{O^2V,O} r^i_{O,\bar{A}} + \bm{v}^{2\mathrm{h1p},\dagger}_{O^2V,V} r^a_{V,\bar{A}} + \bm{c}^{2\mathrm{hlp},xx}_{O^2V,O^2V} r^{i[j b]}_{O^2V, \bar{A}} + \bm{c}^{2\mathrm{hlp},xd}_{O^2V,O^2V} \bar{r}^{i[j b]}_{O^2V, \bar{A}} \\
% \bm{v}^{2\mathrm{h1p},\dagger}_{O^2V,O} r^i_{O,\bar{A}} + \bm{v}^{2\mathrm{h1p},\dagger}_{O^2V,V} r^a_{V,\bar{A}} + \bm{c}^{2\mathrm{hlp},dx}_{O^2V,O^2V} r^{i[j b]}_{O^2V, \bar{A}} + \bm{c}^{2\mathrm{hlp},dd}_{O^2V,O^2V} \bar{r}^{i[j b]}_{O^2V, \bar{A}} \\
% \end{pmatrix}
% \\
% &= \begin{pmatrix}
% \sigma_i \\
% \sigma_a \\
% \sigma_{i[j a]} \\
% \bar{\sigma}_{i[j a]} \\
% \end{pmatrix}
% =\begin{pmatrix}
% \sum_{j} f_{i j}\,r_j + \sum_{b} f_{i b}\,r_b + \sum_{k l c} \bigl\langle i\,c | k\,l \bigr\rangle\,r_{k[l c]}+ \sum_{k l c} \bigl\langle i\,c | k\,l \bigr\rangle\,\bar{r}_{k[l c]}  \\
% \sum_{j} f_{a j}\,r_j + \sum_{b} f_{a b}\,r_b + \sum_{k l c} \bigl\langle a\,c | k\,l \bigr\rangle\,r_{k[l c]} + \sum_{k l c} \bigl\langle a\,c | k\,l \bigr\rangle\,\bar{r}_{k[l c]} \\
% \sum_{k} \bigl\langle k\,a | i\,j \bigr\rangle\,r_k + \sum_{b} \bigl\langle b\,a | i\,j \bigr\rangle\,r_b + \varepsilon_i r_{i[j a]} - \sum_{k b} \bigl[\mathbf{N} \tilde{\mathbf{M}}\bigr]_{j a k b}^{\mathrm{xx}} r_{i[k b]} - \sum_{k b} \bigl[\mathbf{N} \tilde{\mathbf{M}}\bigr]_{j a k b}^{\mathrm{xd}} \bar{r}_{i[k b]} \\
% -\sum_{k} \bigl\langle k\,a | i\,j \bigr\rangle\,r_k - \sum_{b} \bigl\langle b\,a | i\,j \bigr\rangle\,r_b + \varepsilon_i \bar{r}_{i[j a]} - \sum_{k b} \bigl[\mathbf{N} \tilde{\mathbf{M}}\bigr]_{j a, k b}^{\mathrm{dx}} r_{i[k b]} - \sum_{k b} \bigl[\mathbf{N} \tilde{\mathbf{M}}\bigr]_{j a, k b}^{\mathrm{dd}} \bar{r}_{i[k b]} \\
% \end{pmatrix}
% \end{align}

\subsection{Alternate possibilities (assuming a typo was made)}
\subsubsection{Supermetric: Kronecker product, Auxiliary: Kronecker sum}
Note that it makes the most sense for the auxiliary block to be a Kronecker sum, because this was the case for TDA as can be seen in \ref{tda_kron}. So here, I tried the case where the supermetric is a Kronecker product
\begin{equation}
    \bm{\mathcal{N}}^{2 \mathrm{hlp}} = \begin{pmatrix}
        \bm{1}_{P,P} & 0 \\
        0 & \bm{1}_{O,O} \otimes \bm{N}_{2OV,2OV}
    \end{pmatrix}
\end{equation}
and the auxiliary block is a Kronecker sum.
Now we can consider the matrix multiplication
\begin{align}
    \bm{\mathcal{N}}^{2 \mathrm{hlp}} \bm{H}^{2 \mathrm{hlp}} &= \begin{pmatrix}
        \bm{1}_{P,P} & 0 \\
        0 & \bm{1}_{O,O} \otimes \bm{N}_{2OV,2OV}
    \end{pmatrix}
    \begin{pmatrix}
        \bm{F}_{P,P} & \bm{V}^{2\mathrm{h1p}}_{P,O^2V} & \bm{V}^{2\mathrm{h1p}}_{P,O^2V} \\
        \left(\bm{V}^{2\mathrm{h1p}}_{P,O^2V}\right)^{\dagger} &  &  \\
        \left(\bm{V}^{2\mathrm{h1p}}_{P,O^2V}\right)^{\dagger} &  & \bm{C}^{2\mathrm{hlp}}_{2O^2V,2O^2V}
    \end{pmatrix} \\
&= \begin{pmatrix}
        \bm{F}_{P,P} & \begin{pmatrix} \bm{V}^{2\mathrm{h1p}} & \bm{V}^{2\mathrm{h1p}}\end{pmatrix}_{P,2O^2V} \\
        \left( \bm{1} \otimes \bm{N} \right)_{2O^2V,2O^2V}\begin{pmatrix} \left(\bm{V}^{2\mathrm{h1p}}\right)^{\dagger} \\ \left( \bm{V}^{2\mathrm{h1p}}\right)^{\dagger} \end{pmatrix}_{2O^2V,P} &   \left( \bm{1} \otimes \bm{N} \right)_{2O^2V,2O^2V} \bm{C}^{2\mathrm{hlp}}_{2O^2V,2O^2V}  \\
    \end{pmatrix} \\
\end{align}
Evaluating the off-diagonal term, we have ($\bm{V}^{2\mathrm{h1p}}$ has the  elements $V_{p, k[ia]}^{2 \mathrm{~h} 1 \mathrm{p}} = \langle p a | k i \rangle \equiv (pk|ia)$ and $\bm{N}_{2OV,2OV}=\bm{1}_{OV,OV} \oplus_{\text{direct}} -\bm{1}_{OV,OV}$):
\begin{align}
    \left( \bm{1}_{O,O} \otimes \bm{N}_{2OV,2OV} \right) \begin{pmatrix}
        \left(\bm{V}^{2\mathrm{h1p}}\right)^{\dagger} \\ \left( \bm{V}^{2\mathrm{h1p}}\right)^{\dagger}
    \end{pmatrix} _{2O^2V,P} =\begin{pmatrix}
        \left(\bm{V}^{2\mathrm{h1p}}\right)^{\dagger} \\ -\left( \bm{V}^{2\mathrm{h1p}}\right)^{\dagger}
    \end{pmatrix} _{2O^2V,P} 
\end{align} 
Evaluating the auxiliary term with $\bm{C}^{2\mathrm{hlp}}_{2O^2V,2O^2V} = \epsilon^{1 \mathrm{~h}} \oplus_{\text{kron}} (-\bm{\tilde{M}}) = \bm{\epsilon}^{1 \mathrm{~h}}_{O,O} \otimes \bm{1}_{2OV,2OV} + \bm{1}_{O,O} \otimes (-\bm{\tilde{M}})_{2OV,2OV}$ we have
\begin{align}
 \left( \bm{1} \otimes \bm{N} \right)\bm{C}^{2\mathrm{hlp}} &= \left( \bm{1}_{O,O} \otimes \bm{N}_{2OV,2OV} \right) \left( \bm{\epsilon}^{1 \mathrm{~h}}_{O,O} \otimes \bm{1}_{2OV,2OV} + \bm{1}_{O,O} \otimes (-\bm{\tilde{M}})_{2OV,2OV} \right)\\
&= \bm{\epsilon}^{1 \mathrm{~h}}_{O,O} \otimes \bm{N}_{2OV,2OV} + \bm{1}_{O,O} \otimes -\bm{N}_{2OV,2OV} \bm{\tilde{M}}_{2OV,2OV}\\
&= \left[\left( \bm{\epsilon}^{1 \mathrm{~h}}_{O,O} \otimes \bm{1}_{OV,OV} \right) \oplus_{\text{direct}} -\left( \bm{\epsilon}^{1 \mathrm{~h}}_{O,O} \otimes \bm{1}_{OV,OV} \right)\right] + \left( \bm{1}_{O,O} \otimes -\bm{N}_{2OV,2OV} \bm{\tilde{M}}_{2OV,2OV} \right)\\
% &= \left[\left( \bm{\epsilon}^{1 \mathrm{~h}}_{O} \otimes \bm{1}_{OV} \right) \oplus_{\text{direct}} -\left( \bm{\epsilon}^{1 \mathrm{~h}}_{O} \otimes \bm{1}_{OV} \right)\right] + \left( \bm{1}_{O} \otimes -\bm{Z}_{2OV} \begin{pmatrix}
% \bm{\Omega} & \bm{0} \\ \bm{0} & \eta - \bm{\Omega}\end{pmatrix}_{2OV}
% \bm{Z}^{-1}_{2OV}\right)\\
&= \begin{pmatrix} \bm{\epsilon}^{1 \mathrm{~h}} \otimes \bm{1} & \bm{0} \\ \bm{0} & -\left(\bm{\epsilon}^{1 \mathrm{~h}} \otimes \bm{1}\right) \end{pmatrix}_{2O^2V}
 + \left( \bm{1}_{O} \otimes -\bm{Z}_{2OV} \begin{pmatrix}
\bm{\Omega} & \bm{0} \\ \bm{0} & \eta - \bm{\Omega}\end{pmatrix}_{2OV}
\bm{Z}^{-1}_{2OV}\right)_{2O^2V}\\
&= \begin{pmatrix} \bm{\epsilon}^{1 \mathrm{~h}} \otimes \bm{1} & \bm{0} \\ \bm{0} & -\left(\bm{\epsilon}^{1 \mathrm{~h}} \otimes \bm{1}\right) \end{pmatrix}_{2O^2V}
 - \left[ \left( \bm{1} \otimes\bm{Z} \right) \begin{pmatrix}
\bm{1} \otimes \bm{\Omega} & \bm{0} \\ \bm{0} &\bm{1} \otimes( \eta - \bm{\Omega})\end{pmatrix}
\left( \bm{1} \otimes \bm{Z}^{-1}\right) \right]_{2O^2V}\\
&= \left( \bm{1} \otimes\bm{Z} \right) \begin{pmatrix} \bm{\epsilon}^{1 \mathrm{~h}} \otimes \bm{1} + \bm{1} \otimes -\bm{\Omega} & \bm{0} \\ \bm{0} & -\left(\bm{\epsilon}^{1 \mathrm{~h}} \otimes \bm{1}\right) -(\bm{1} \otimes( \eta - \bm{\Omega})) \end{pmatrix}
\left( \bm{1} \otimes \bm{Z}^{-1}\right)\\
% &= \begin{pmatrix} \bm{\epsilon}^{  1 \mathrm{~h}} \otimes \bm{1} + \bm{1} \otimes -\bm{Z}\bm{\Omega}\bm{Z}^{-1} & \bm{0} \\ \bm{0} & -\left(\bm{\epsilon}^{1 \mathrm{~h}} \otimes \bm{1}\right) + \bm{1} \otimes \bm{Z}\left[\eta -\bm{\Omega}\right]\bm{Z}^{-1} \end{pmatrix}_{2O^2V} \\
% &= \left(\bm{1} \otimes \bm{Z}\right) \begin{pmatrix}
% \bm{\epsilon}^{1 \mathrm{~h}} \otimes \bm{1} - \bm{1} \otimes \bm{\Omega} & \bm{0} \\
% \bm{0} & -\left(\bm{\epsilon}^{1 \mathrm{~h}} \otimes \bm{1}\right) + \bm{1} \otimes \left[\eta -\bm{\Omega}\right]
% \end{pmatrix} \left(\bm{1} \otimes \bm{Z}\right)^{-1}
\end{align}
where $\bm{Z}=\begin{pmatrix} \bm{X} & \bm{Y} \\ \bm{Y} & \bm{X} \end{pmatrix}$ is the eigenvector matrix and $\bm{\Omega}$ is the diagonal matrix of positive excitation energies.
\subsubsection{Similarity transformation}
 Now let us define the rotation $\bm{\zeta} = \begin{pmatrix} \bm{1} & 0 \\ 0& \bm{1} \otimes \bm{Z} \end{pmatrix}$. We can left multiply by $\bm{\zeta}^{-1}$ and right multiply by $\bm{\zeta}$ to get
\begin{align}
    &\bm{\zeta}^{-1} \bm{\mathcal{N}}^{2 \mathrm{hlp}} \bm{H}^{2 \mathrm{hlp}} \bm{\zeta} = \begin{pmatrix}
\bm{1} & 0 \\
0 & \bm{1} \otimes \bm{Z}^{-1}
\end{pmatrix}
\begin{pmatrix}
\bm{F} & \begin{pmatrix} \bm{V}^{2\mathrm{h1p}} & \bm{V}^{2\mathrm{h1p}}\end{pmatrix} \\
\begin{pmatrix} \left(\bm{V}^{2\mathrm{h1p}}\right)^{\dagger} \\ -\left( \bm{V}^{2\mathrm{h1p}}\right)^{\dagger} \end{pmatrix} &   \left( \bm{1} \otimes \bm{N} \right) \bm{C}^{2\mathrm{hlp}}  \\
\end{pmatrix}
 \begin{pmatrix}\bm{1} & 0 \\
0 & \bm{1} \otimes \bm{Z}
\end{pmatrix} \\
&= \begin{pmatrix}\bm{F} & \begin{pmatrix} \bm{V}^{2\mathrm{h1p}} & \bm{V}^{2\mathrm{h1p}}\end{pmatrix} \left(\bm{1} \otimes \bm{Z}\right) \\
\left(\bm{1} \otimes \bm{Z}^{-1}\right) \begin{pmatrix} \left(\bm{V}^{2\mathrm{h1p}}\right)^{\dagger} \\ -\left( \bm{V}^{2\mathrm{h1p}}\right)^{\dagger} \end{pmatrix} &   \left( \bm{1} \otimes \bm{Z}^{-1} \bm{Z} \right) \begin{pmatrix}
\bm{\epsilon}^{1 \mathrm{~h}} \oplus_{\text{kron}} -\bm{\Omega} & \bm{0} \\
\bm{0} & -\bm{\epsilon}^{1 \mathrm{~h}}\oplus_{\text{kron}}\left[\eta -\bm{\Omega}\right]
\end{pmatrix} \left(\bm{1} \otimes \bm{Z}^{-1}\bm{Z}\right) \end{pmatrix}\\
&= \begin{pmatrix}\bm{F} & \begin{pmatrix} \bm{V}^{2\mathrm{h1p}} & \bm{V}^{2\mathrm{h1p}}\end{pmatrix} \left(\bm{1} \otimes \begin{pmatrix} \bm{X} & \bm{Y} \\ \bm{Y} & \bm{X} \end{pmatrix}\right) \\
\left(\bm{1} \otimes \begin{pmatrix} \bm{X} & \bm{Y} \\ \bm{Y} & \bm{X} \end{pmatrix}^{-1}\right) \left(\bm{1}\otimes \bm{N} \right)\begin{pmatrix} \left(\bm{V}^{2\mathrm{h1p}}\right)^{\dagger} \\ \left( \bm{V}^{2\mathrm{h1p}}\right)^{\dagger} \end{pmatrix} & \begin{pmatrix}
\bm{\epsilon}^{1 \mathrm{~h}} \oplus_{\text{kron}} -\bm{\Omega} & \bm{0} \\
\bm{0} & -\bm{\epsilon}^{1 \mathrm{~h}}\oplus_{\text{kron}}\left[\eta -\bm{\Omega}\right]
\end{pmatrix} \end{pmatrix} \\
&= \begin{pmatrix}\bm{F} & \begin{pmatrix} \bm{V}^{2\mathrm{h1p}}\left(\bm{X} + \bm{Y}\right) & \bm{V}^{2\mathrm{h1p}}\left(\bm{X} + \bm{Y}\right)\end{pmatrix} \\
 \left(\bm{1} \otimes \begin{pmatrix} \bm{X} & \bm{Y} \\ \bm{Y} & \bm{X} \end{pmatrix}^{-1} \right)\begin{pmatrix} \left(\bm{V}^{2\mathrm{h1p}}\right)^{\dagger} \\ -\left( \bm{V}^{2\mathrm{h1p}}\right)^{\dagger} \end{pmatrix} & \begin{pmatrix}
\bm{\epsilon}^{1 \mathrm{~h}} \oplus_{\text{kron}} -\bm{\Omega} & \bm{0} \\
\bm{0} & -\bm{\epsilon}^{1 \mathrm{~h}} \oplus_{\text{kron}}\left[\eta -\bm{\Omega}\right]
\end{pmatrix}
\end{pmatrix}
\end{align}
But because it is true that $\bm{Z}$ are the eigenvectors of a non-Hermitian eigenproblem, it is not true that $\bm{Z}^{-1} = \bm{Z}^{\dagger}$, so we cannot simplify the lower left block any further.
Accordingly, doing ED of this form for the supermatrices gives the wrong spectrum, but I have included this derivation of the similarity transformation because it seems like the most promising approach I have considered so far.
% So it is true that $\begin{pmatrix}
%     \bm{X} & \bm{Y} \\
%     \bm{Y} & \bm{X}
% \end{pmatrix}^\dag \bm{N} \begin{pmatrix}
%     \bm{X} & \bm{Y} \\
%     \bm{Y} & \bm{X}
% \end{pmatrix} = \bm{1}$. How does this help though?
% where the elements of $\bm{W}^<$ are given by $\bm{W}^<_{pk\nu} = \sum_{ia} (pk|ia) \left( X_{ia}^{\nu} + Y_{ia}^{\nu} \right)$.

\subsubsection{Supermetric: Kronecker sum, Auxiliary: Kronecker product}
Let us try the alternate case where $\bm{C}^{2\mathrm{hlp}} = \left( \bm{\epsilon}^{1 \mathrm{~h}} \otimes -\bm{\tilde{M}} \right)$, which gives
\begin{align}
 \left( \bm{1} \oplus_{\text{kron}} \bm{N} \right)\bm{C}^{2\mathrm{hlp}} &= \left(\bm{1}_{O,O} \oplus_{\text{kron}} \bm{N}_{2OV,2OV}\right) \left( \bm{\epsilon}^{1 \mathrm{~h}}_{O,O} \otimes (-\bm{\tilde{M}})_{2OV,2OV} \right)\\
&= \left( \bm{1}_O \otimes \bm{1}_{2OV} + \bm{1}_O \otimes \bm{N}_{2OV} \right) \left( \bm{\epsilon}^{1 \mathrm{~h}}_{,O} \otimes (-\bm{\tilde{M}})_{2OV,} \right)\\
&= \bm{\epsilon}^{1 \mathrm{~h}} \otimes -\bm{\tilde{M}} + \bm{\epsilon}^{1 \mathrm{~h}} \otimes -\bm{N}\bm{\tilde{M}}\\
\end{align}
This is not what we want and the form for the off-diagonal term is the same as in the previous subsubsection.
% where we have the definitions
% If we choose to assume that these are direct products, the shapes don't work out because $\mathbf{C}^{2 \mathrm{hlp}}$ has the dimension $O+2OV$ while its upstairs neighbor of $\begin{pmatrix}
% \bm{V}^{2\mathrm{h1p}} & \bm{V}^{2\mathrm{h1p}}
% \end{pmatrix}$ has the column dimension $2O^2V$. So we need to use the Kronecker product, which is what Tim does. The supermetric is given by
% % $$
% % \bm{\mathcal{N}}=\left(\begin{array}{ccc}
% % \bm{1} & 0 & 0 \\
% % 0 & \bm{1} \oplus \bm{N} & 0 \\
% % 0 & 0 & \bm{1} \oplus \bm{N}
% % \end{array}\right)
% % $$
% and the excitation vector is $\bm{R} = (\bm{R}^{1\mathrm{h}+1\mathrm{p}}, \bm{R}^{2\mathrm{h}1\mathrm{p}}, \bm{R}^{2\mathrm{p}1\mathrm{h}})^\dag$ with $\bm{R}^{2\mathrm{h}1\mathrm{p}}=(\bm{r}^{2\mathrm{h}1\mathrm{p}}, \bm{\bar{r}}^{2\mathrm{h}1\mathrm{p}})^\dag$ and $\bm{R}^{2\mathrm{p}1\mathrm{h}}=(\bm{r}^{2\mathrm{p}1\mathrm{h}}, \bm{\bar{r}}^{2\mathrm{p}1\mathrm{h}})^\dag$.
% \begin{equation}
%     \bm{C}^{2 \mathrm{hlp}}_{2O^2V,2O^2V} = \epsilon^{1 \mathrm{~h}} \oplus_{\text{kron}} (-\bm{\tilde{M}}) = \bm{\epsilon}^{1 \mathrm{~h}}_{O,O} \otimes \bm{1}_{2OV,2OV} + \bm{1}_{O,O} \otimes (-\bm{\tilde{M}})_{2OV,2OV}
% \end{equation}
% If we multiply by $\bm{N}$, only to the
% \begin{equation}
%     \bm{C}^{2 \mathrm{hlp}}_{2O^2V,2O^2V} = \bm{N} \left( \epsilon^{1 \mathrm{~h}} \oplus_{\text{kron}} (-\bm{\tilde{M}}) \right) = \bm{N} \bm{\epsilon}^{1 \mathrm{~h}}_{O,O} \otimes \bm{1}_{2OV,2OV} + \bm{N} \bm{1}_{O,O} \otimes (-\bm{\tilde{M}})_{2OV,2OV}
% \end{equation}

% \begin{equation}
%     w_{pq}^{\mu} = \sum_{ia} (pq|ia) \left(X_{ia}^{\mu} + Y_{ai}^{\mu}\right)
% \end{equation}
% where we have defined the excitation and de-excitation vectors at the excitation index $\mu$ as $X_{ia}^{\mu}$ and $Y_{ai}^{\mu}$, respectively.
% I am not sure how to connect this with the known expression $v\epsilon ^{-1}$; I see the similarities given that we are contracting an ERI with what we get from the RPA calculation that is connected to the polarizability, but can't connect exactly.
% We want to figure out how this matches with my previous $O(N^6)$ expression, which was
% \begin{equation}
%     \Sigma_{pp}^{\text{corr}}(\omega) = \sum_{\mu }^{\text{RPA}}\left(\sum_{i}^{\text{occupied}} \frac{w_{pi}^{\mu }w_{ip}^{\mu }}{\omega -(\epsilon _{i}-\Omega  _{\mu })}+ \sum_{a}^{\text{virtual}} \frac{w_{pa}^{\mu }w_{ap}^{\mu }}{\omega -(\epsilon _{a}+\Omega  _{\mu })}\right)
% % \end{equation}
% \subsection{Krylov procedure to approximate the step function}
% It does not seem like a good idea to me to start applying the Krylov procedure without first having ED. But we can start to think about what this would look like. The matrix vector products contain terms like $\sum_{jc}\bigl[ \mathbf{N} \tilde{\mathbf{M}}\bigr]_{i a, j c}^{\mathrm{dd}} \bar{r}_{[j c] b}$. Right now, I am explicitly constructing $\Theta(-\bm{N}\bm{M})$, which requires diagonalizing $\bm{N}\bm{M}$, naively scaling as $O(N^6)$. Using a Krylov subspace method, we can avoid this. The idea is to approximate the entirety of $\mathbf{N} \tilde{\mathbf{M}}$ once before the start of the Davidson iteration, and then we can extract the xx, xd, dx, and dd blocks from the approximate matrix. Specifically
% \begin{align}
% \bm{N}\tilde{\bm{M}} = \bm{N}\left(\bm{M}+\eta \bm{N} \Theta(-\bm{N}\bm{M})\right) = \bm{N}\bm{M} + \eta \Theta(-\bm{N}\bm{M})
% \end{align}
% The first term is not a problem, but then we approximate the spectrum of $-\bm{N}\bm{M}$ using a Krylov subspace method, to determine where are the negative eigenvalues, thus determining the step function $\Theta(-\bm{N}\bm{M})$.
% \subsubsection{Arnoldi iteration}
% They suggest an Arnoldi procedure to determine the spectrum of $-\bm{N}\bm{M}$, since it is not Hermitian. We might be able to do better by converting into a symmetric problem. 

% \subsubsection{Lanczos iteration}
% We start by considering $\bm{B} = \bm{N}\bm{M}\bm{N}$, which is symmetric since $\bm{N}$ is symmetric and $\bm{M}$ is symmetric. Then, we can do Lanczos to determine the spectrum of $\bm{B}$, giving the Ritz pairs $\{\bm{\lambda}_i, \bm{v}_i\}$. Then, we can determine the step function on the $\bm{B}$ space as $\bm{Q}\bm{V}_- \bm{V}_-^\dagger \bm{Q}^\dagger$, where $\bm{V}_-$ is the matrix of Ritz vectors with negative Ritz values and $\bm{Q}$ is the Krylov subspace. Then, we can transform back to the original space using $\bm{N}$, giving us $\Theta(-\bm{N}\bm{M})=\bm{N}\bm{Q}\bm{V}_- \bm{V}_-^\dagger \bm{Q}^\dagger\bm{N}$.
% \begin{tcolorbox}[colback=red!10!white, colframe=red!50!black]
%  I run into trouble when I try to do this because the spectrum of $\bm{B}$ is completely positive, so it no longer makes sense to use the step function.
% \end{tcolorbox}
% \subsection{Tried and failed using Z-vector trick for dRPA}
% The super matrix that downfolds into the GW self-energy is
% \begin{align}
% \begin{pmatrix} \bm{F} & \bm{V}^{2 \mathrm{h1p}} & \bm{V}^{2 \mathrm{p} 1 \mathrm{~h}} \\ \left(\bm{V}^{2 \mathrm{~h} 1 \mathrm{p}}\right)^{\dagger} & \bm{C}^{2 \mathrm{hlp}} & \bm{0} \\ \left(\bm{V}^{2 \mathrm{plh}}\right)^{\dagger} & \bm{0} & \bm{C}^{2 \mathrm{plh}} \end{pmatrix}
% \end{align}
% For dRPA, we will choose 
% \begin{equation}
%     \bm{C}^{2 \mathrm{hlp}} = \epsilon^{1 \mathrm{~h}} \oplus (-\bm{M}^{1/2})
% \implies C_{i[j a], k[l c]}^{2 \mathrm{~h} 1 \mathrm{p}} = \left[\epsilon _i - \bm{M^{1/2}}_{ja,lc}\right] \delta_{i k}
% \end{equation}
%  where $\bm{M}$ is the matrix defined in \eqref{eq:berk_gw_M}.
% If we choose our unitary transformation to be $\bm{U'} = \bm{1} \oplus_{\text{diag}} \left(\bm{1}\otimes \bm{T}\right)$, where $\bm{T}$ is the matrix of eigenvectors of the symmetric formulation as 
% \begin{equation}
%     \bm{M}\bm{T}= \bm{\lambda } \bm{T}
% \label{eq:berk_gw_eigenproblem}
% \end{equation}
% with the matrix $\bm{M}$ defined as
% \begin{equation}
%     \bm{M}=(\bm{A}-\bm{B})^{1 / 2}(\bm{A}+\bm{B})(\bm{A}-\bm{B})^{1 / 2}
% \label{eq:berk_gw_M}
% \end{equation}
% with the identities $\bm{\lambda } = \bm{\Omega }^2$ and $\bm{T} = \bm{\Omega }^{1/2} \left[\bm{A}-\bm{B}\right]^{-1/2}\left(\bm{X} + \bm{Y}\right)$, where $\bm{X}$ and $\bm{Y}$ are the excitation and de-excitation vectors, respectively. 
% \begin{tcolorbox}[colback=red!10!white, colframe=red!50!black, title=Exercise]
% \subsubsection{Casida transformation}
% The matrix problem is:
% \begin{equation}
% \begin{bmatrix}
% \textbf{A} & \textbf{B} \\
% -\textbf{B} & -\textbf{A}
% \end{bmatrix}
% \begin{bmatrix}
% \textbf{X}&\textbf{Y} \\
% \textbf{Y} &\textbf{X}
% \end{bmatrix}
% = 
% \begin{bmatrix}
% \textbf{X} & \textbf{Y}\\
% \textbf{Y} & \textbf{X}
% \end{bmatrix}
% \begin{bmatrix}
% \boldsymbol{\Omega } & 0 \\
% 0 & -\boldsymbol{\Omega }
% \end{bmatrix}
% .
% \end{equation}
% which becomes equivalent to the pair of linear equations for each excitation mode
% \begin{equation}
% \begin{split}
%     \left(\textbf{A} + \textbf{B}\right) \left(X + {Y}\right)_\nu = \Omega_\nu \left({X} - {Y}\right)_\nu \\
%     \left(\textbf{A} - \textbf{B}\right) \left(X - {Y}\right)_\nu = \Omega_\nu \left(X + {Y}\right)_\nu
% \label{eqn:A_B}
% \end{split}
% \end{equation}
% We know we have the symmetric eigenproblem $\bm{M}T_\nu = \lambda_\nu T_\nu$, where $\bm{M}=(\bm{A}-\bm{B})^{1/2}(\bm{A}+\bm{B})(\bm{A}-\bm{B})^{1/2}$. Let us make the ansatze 
% \begin{equation}
% \begin{split}
%     \left({X} + {Y}\right)_\nu &= {\Omega }_\nu^{-\frac{1}{2}} \left(\textbf{A}-\textbf{B}\right)^{\frac{1}{2}}T_\nu \\
%     \left({X} - {Y}\right)_\nu &= {\Omega }_\nu^{\frac{1}{2}} \left(\textbf{A}-\textbf{B}\right)^{-\frac{1}{2}}T_\nu
% \label{eqn:XY_ansatz}
% \end{split}
% \end{equation}
% We know that this satisfies the biorthogonality relation since
% \begin{equation}
%     \left({X} + {Y}\right)_\nu^\dagger \left({X} - {Y}\right)_\nu = \left({\Omega }_\nu^{-\frac{1}{2}} \left(\textbf{A}-\textbf{B}\right)^{\frac{1}{2}}T_\nu\right)^\dagger \left({\Omega }_\nu^{\frac{1}{2}} \left(\textbf{A}-\textbf{B}\right)^{-\frac{1}{2}}T_\nu\right) = T_\nu^\dagger T_\nu = {1}
% \end{equation}
% Now, we can plug the ansatz \ref{eqn:XY_ansatz} into the equations \ref{eqn:A_B} to get
% \begin{equation}
% \begin{split}
%     \left(\textbf{A} + \textbf{B}\right) \left({\Omega }_\nu^{-\frac{1}{2}} \left(\textbf{A}-\textbf{B}\right)^{\frac{1}{2}}T_\nu\right) &= {\Omega }_\nu \left( {\Omega }_\nu^{\frac{1}{2}} \left(\textbf{A}-\textbf{B}\right)^{-\frac{1}{2}}T_\nu\right) \\
%     \implies {\Omega }_\nu^{-\frac{1}{2}} \left(\textbf{A} + \textbf{B}\right) \left(\textbf{A}-\textbf{B}\right)^{\frac{1}{2}}T_\nu &= {\Omega }_\nu^{\frac{3}{2}} \left(\textbf{A}-\textbf{B}\right)^{-\frac{1}{2}}T_\nu \\
% \implies \left(\textbf{A}-\textbf{B}\right)^{\frac{1}{2}} \left(\textbf{A} + \textbf{B}\right) \left(\textbf{A}-\textbf{B}\right)^{\frac{1}{2}}T_\nu &= {\Omega }_\nu^{2} T_\nu
% \end{split}
% \end{equation}
% We make the ansatz that the eigenvectors of the symmetric formulation are given by $\textbf{T} = \boldsymbol{\Omega }^{\frac{1}{2}} \left(\textbf{A}-\textbf{B}\right)^{-\frac{1}{2}}\left(\textbf{X} + \textbf{Y}\right)$. Plugging into \ref{eqn:Aplus_B} gives us
% \begin{align}
%     \left(\textbf{A} + \textbf{B}\right) \left(\boldsymbol{\Omega }^{-\frac{1}{2}} \left(\textbf{A}-\textbf{B}\right)^{\frac{1}{2}}\bm{T}\right) &= \boldsymbol{\Omega } \left( \bm{X} - \bm{Y}\right) \\
% \bm{\Omega}^{-\frac{1}{2}} \left(\textbf{A} + \textbf{B}\right) \left(\textbf{A}-\textbf{B}\right)^{\frac{1}{2}}\bm{T} &= \boldsymbol{\Omega } \left( \bm{X} - \bm{Y}\right) \\
% \bm{\Omega}^{-\frac{1}{2}}\left(\textbf{A}-\textbf{B}\right)^{-\frac{1}{2}} \bm{M} \bm{T} &= \boldsymbol{\Omega } \left( \bm{X} - \bm{Y}\right) \\
% \end{align}
% \end{tcolorbox}
% Then, we can apply the unitary transformation to the Hamiltonian in Tim's formulation as
% \begin{align}
%     \bm{H'} &= {\bm{U}'}^\dagger \bm{H} \bm{U'} \\
%     &= \begin{pmatrix} \bm{1} & \bm{0}\\ \bm{0}&\bm{1}\otimes\bm{T} \end{pmatrix}^\dagger 
%        \begin{pmatrix} \bm{F} & \bm{V}^{2 \mathrm{h1p}}\\ 
%        \left( \bm{V}^{2 \mathrm{~h} 1 \mathrm{p}} \right)^{\dagger} & 
%        \bm{C}^{2 \mathrm{hlp}} \end{pmatrix} 
%        \begin{pmatrix} \bm{1} & \bm{0} \\ \bm{0}&\bm{1} \otimes \bm{T} \end{pmatrix} \\
%     &= \begin{pmatrix} 
%         \bm{F} & \bm{V}^{2 \mathrm{h1p}} \left( \bm{1} \otimes \bm{T} \right) \\
%         \left( \bm{1} \otimes \bm{T} \right)^\dagger 
%         \left( \bm{V}^{2 \mathrm{~h} 1 \mathrm{p}} \right)^{\dagger} & 
%         \left( \bm{1} \otimes \bm{T} \right)^\dagger 
%         \bm{C}^{2 \mathrm{hlp}} 
%         \left( \bm{1} \otimes \bm{T} \right) 
%     \end{pmatrix}.
% \label{eq:tim_upfolded_hamiltonian}
% \end{align}
% But now let us try evaluating the term $\left(\bm{1}\otimes\bm{T}\right)^\dagger \bm{C}^{2 \mathrm{hlp}} \left(\bm{1}\otimes\bm{T}\right)$ in the dRPA case. We have
% \begin{align}
%     \left(\bm{1}\otimes\bm{T}\right)^\dagger \bm{C}^{2 \mathrm{hlp}} \left(\bm{1}\otimes\bm{T}\right) &= \left(\bm{1}\otimes\bm{T}\right)^\dagger \left(\epsilon^{1 \mathrm{~h}} \otimes \bm{1} + \bm{1} \otimes (-{\bm{M}}^{1/2})\right) \left(\bm{1}\otimes\bm{T}\right) \\
% &= \left(\bm{1}\otimes\bm{T}\right)^\dagger \left(\epsilon^{1 \mathrm{~h}} \otimes \bm{1}\right) \left(\bm{1}\otimes\bm{T}\right) + \left(\bm{1}\otimes\bm{T}\right)^\dagger \left(\bm{1} \otimes (-{\bm{M}}^{1/2})\right) \left(\bm{1}\otimes\bm{T}\right) \\
% &= \epsilon^{1 \mathrm{~h}} \otimes\bm{T}^\dagger\bm{T} + \bm{1} \otimes \left(\underbrace{\bm{T}^\dagger (-{\bm{M}}^{1/2}) \bm{T}}_{-\boldsymbol{\Omega}}\right) \\
% \end{align}
% Next, let us consider the product $\bm{V}^{2 \mathrm{~h} 1 \mathrm{p}} \left(\bm{1}\otimes\bm{T}\right) $. We have
% \begin{align}
%     \bm{V}^{2 \mathrm{~h} 1 \mathrm{p}} \left(\bm{1}\otimes\bm{T}\right) &= \sum_{ia} (pk|ia) \left( X_{ia}^{v} + Y_{ia}^{v} \right) \Omega_{ia;v}^{1/2} \underbrace{\sum_{jb} \left(A_{ia,jb} - B_{ia,jb}\right)^{-1/2}}_{\Delta _{ia}^{-1/2}} \\
% \end{align}
% So we have
% \begin{align}
%     \bm{H'} &= \begin{pmatrix} \bm{F} & \sum_{ia} (pk|ia) \left( X_{ia}^{v} + Y_{ia}^{v} \right) \Omega_{ia;v}^{1/2} \Delta _{ia}^{-1/2} \\ \sum_{jb} (pk|jb)^\dagger \left( X_{jb}^{v} + Y_{jb}^{v} \right)^\dagger \Omega_{jb;v}^{1/2} \Delta _{jb}^{-1/2} & \epsilon^{1 \mathrm{~h}} \oplus_{\text{kron}} - \boldsymbol{\Omega }\end{pmatrix}
% \end{align}
% Now consider a second similarity transform of $\bm{U}'' = \bm{1} \oplus \left( \bm{1} \otimes \left(\frac{\Delta _{lc}}{\Omega _{lc;v}}\right)^{1/2} \right)$, which gives us
% \begin{align}
%     &\bm{H''} = {\bm{U}''}^\dagger \bm{H'} \bm{U}'' \\
%     &= \begin{pmatrix}
%         \bm{F} & \sum_{ia} (pk|ia) \left( X_{ia}^{v} + Y_{ia}^{v} \right) \\
%         \sum_{jb} (pk|jb)^\dagger \left( X_{jb}^{v} + Y_{jb}^{v} \right)^\dagger & 
%         \left( \bm{1} \otimes \left( \frac{\Delta_{lc}}{\Omega_{lc;v}} \right)^{1/2} \right)^\dagger
%         \left( \epsilon^{1\mathrm{~h}} \oplus_{\text{kron}} \bm{T}^\dagger (-{\bm{M}}^{1/2}) \bm{T} \right)
%         \left( \bm{1} \otimes \left( \frac{\Delta_{lc}}{\Omega_{lc;v}} \right)^{1/2} \right)
%     \end{pmatrix}
% \end{align}

% Just considering the auxiliary term gives
% \begin{align}
%     &\left( \bm{1} \otimes \left(\frac{\Delta _{lc}}{\Omega _{lc;v}}\right)^{1/2} \right)^\dag \left(\epsilon^{1 \mathrm{~h}} \oplus_{\text{kron}}\bm{T}^\dagger (-{\bm{M}}^{1/2}) \bm{T}\right)\left( \bm{1} \otimes \left(\frac{\Delta _{lc}}{\Omega _{lc;v}}\right)^{1/2} \right) \\
% &= \left( \bm{1} \otimes \left(\frac{\Delta _{lc}}{\Omega _{lc;v}}\right)^{1/2} \right)^\dag \left(\epsilon^{1 \mathrm{~h}} \otimes \bm{1} + \bm{1} \otimes \bm{T}^\dagger (-{\bm{M}}^{1/2})\bm{T}\right)\left( \bm{1} \otimes \left(\frac{\Delta _{lc}}{\Omega _{lc;v}}\right)^{1/2} \right) \\
% &= \epsilon^{1 \mathrm{~h}} \otimes \frac{\Delta _{lc}}{\Omega _{lc;v}} + \bm{1} \otimes \left(\bm{T}\left(\frac{\Delta _{lc}}{\Omega _{lc;v}}\right)^{1/2}\right)^\dagger (-{\bm{M}}^{1/2})\left(\bm{T}\left(\frac{\Delta _{lc}}{\Omega _{lc;v}}\right)^{1/2}\right) \\
% \end{align}
% So $\bm{T}$ are eigenvectors of the matrix $\bm{M}$, with eigenvalues $\bm{\Omega }^2$. Is there any alternate choice of $\bm{M}$ that would massage this into the desired form of $\epsilon^{1 \mathrm{~h}} \oplus_{\text{kron}} (-\bm{\Omega })$?
% This would only work if we could assume that $\left(\frac{\Omega _{ia;v}}{\Delta _{ia}}\right)^{1/2} \approx 1$. 
% \subsubsection{Evaluating the matrix-vector products}
% Now, we can define a vector $\bm{R} = ( r_i,\; r_a,\; r_{i[jb]},\; r_{[jb]a} )$. Application of the Hamiltonian to this vector gives us the matrix-vector product $\bm{H'} \bm{R} = \bm{\sigma }$, where $\bm{\sigma} = ( \sigma_i,\; \sigma_a,\; \sigma_{i[jb]},\; \sigma_{[jb]a} )$. So the one body matrix vector products are the same as in the dTDA case
% \begin{align}
% \sigma_i &= 
%   \sum_{j} f_{i j}\,r_j
%   + \sum_{b} f_{i b}\,r_b
%   + \sum_{k l c} \bigl\langle i\,c | k\,l \bigr\rangle\,r_{k[l c]}
%   + \sum_{k c d} \bigl\langle i\,k | d\,c \bigr\rangle\,r_{[k c]d}, \\[6pt]
% \sigma_a &=
%     \sum_{j} f_{a j}\,r_j
%     + \sum_{b} f_{a b}\,r_b
%     + \sum_{k l c} \bigl\langle a\,c | k\,l \bigr\rangle\,r_{k[l c]}
%     + \sum_{k c d} \bigl\langle a\,k | d\,c \bigr\rangle\,r_{[k c]d}, \\[6pt]
% \end{align}
% For the two-body matrix vector products, we have
% \begin{align}
% \sigma_{i[ja]} &=
%     \sum_{k} \bigl\langle k\,a | i\,j \bigr\rangle\,r_k
%     + \sum_{b} \bigl\langle b\,a | i\,j \bigr\rangle\,r_b
%     + \sum_{klc} C_{i[j a], k[l c]}^{2 \mathrm{~h} 1 \mathrm{p}} r_{i[l c]}
% \end{align}
% Now, consider just
% \begin{equation}
%     \sum_{klc} C_{i[j a], k[l c]}^{2 \mathrm{~h} 1 \mathrm{p}} r_{i[l c]}= \sum_{klc} \left[\left(\epsilon_i-{\bm{M}}^{1/2}_{ja,lc}\right) \delta_{ik} r_{i[l c]} \right] = \sum_{l c} \left(\epsilon_i - {\bm{M}}^{1/2}_{ja,lc}\right) r_{i[l c]}
% \end{equation}
% Consider now just the term 
% \begin{equation}
%     \sum_{l c} {\bm{M}}^{1/2}_{ja,lc} r_{i[l c]} =  \sum_{l c} \left[\underbrace{\Delta _{ja}^{1/2} \left(\Delta_{ja} + 2(ja|lc)\right)_{ja,lc} \Delta_{ja}^{1/2}}_{\bm{D}}\right]^{1/2} r_{i[l c]}.
% \end{equation}
% Let us consider instead the similarity transformed $\tilde{\bm{D}} = \Delta^{-1/2} \bm{D} \Delta^{-1/2} = \left(\Delta + 2K\right)$. This is useful because we have $D^{1/2} = \Delta^{1/2}\tilde{D}^{1/2}\Delta^{1/2}=\left(\Delta+2K\right)^{1/2}$ for our SPD D.Now make the change of variables $u = \Delta^{1/2}r \implies r = \Delta^{-1/2}u$. Then we have $\left(\Delta + 2K\right) \Delta^{-1/2}u = \Delta^{1/2} \left[ {I} + \underbrace{2\Delta^{-1/2}K\Delta^{-1/2}}_{L}\right] u$. Again, because we are working with weak to moderately correlated systems, we can assume that L is small, and neglect it for now. So we have
% \begin{equation}
%     \sum_{l c} {\bm{M}}^{1/2}_{ja,lc} r_{i[l c]} \approx \sum_{l c} \Delta^{1/2} I \Delta^{1/2} r_{i[l c]} \approx \sum_{l c} \Delta r_{i[l c]}
% \end{equation}

% We can rewrite this as
% So really we have reduced the scaling to $O(O^3V^2)$. If we do a similar thing for the other term, we have
% \begin{align}
% \sigma_{[ia]b} &= 
%     \sum_{j} \bigl\langle j\,i | b\,a \bigr\rangle\,r_j
%     + \sum_{c} \bigl\langle c\,i | b\,a \bigr\rangle\,r_c
%     + \sum_{kc} \left[\epsilon_b + \bm{M}^{1/2}_{ia,kc}\right] r_{[kc] b}
% \end{align}
% which gives the scaling of $O(O^2V^3)$.
% % \section{Extra}
% Recall that we are trying to recover the form 
% \begin{equation}
%     \bm{H} = \begin{pmatrix} \bm{F} & \bm{W}^<\\ \bm{W}^{<,\dagger} & \bm{d}^< \end{pmatrix}
% \label{eq:booth_upfolded_hamiltonian2}
% \end{equation}
% where now we have the definitions
% \begin{align}
%     W_{pkv}^{<} &= \sum_{ia} (pk|ia) \left( X_{ia}^{v} + Y_{ia}^{v} \right) \\
%     d_{kv,lv'}^{<} &= \left(\epsilon_k - \Omega_v\right) \delta_{k,l} \delta_{v,v'}
% \label{eq:booth_definitions2}
% \end{align}
% where now $\bm{\Omega }$ is taken fromm the eigenproblem $\bm{M}\bm{T}= \bm{\Omega}^2 \bm{T}$, which is the symmetric formulation of the RPA.

% where in the dTDA case the supermatrix is
% \begin{equation}
%     \mathbf{H}=\left(\begin{array}{ccc}
% \mathbf{f} & \mathbf{V}^{2 \mathrm{hlp}} & \mathbf{V}^{2 \mathrm{p} 1 \mathrm{~h}} \\
% \left(\mathbf{V}^{2 \mathrm{~h} 1 \mathrm{p}}\right)^{\dagger} & \mathbf{C}^{2 \mathrm{hlp}} & \mathbf{0} \\
% \left(\mathbf{V}^{2 \mathrm{plh}}\right)^{\dagger} & \mathbf{0} & \mathbf{C}^{2 \mathrm{plh}}
% \end{array}\right)
% \end{equation}
% with the definitions
% \begin{align}
%     V_{p, k[l c]}^{2 \mathrm{~h} 1 \mathrm{p}} &= \langle p c | k l \rangle \\
%     V_{p,[k c] d}^{2 \mathrm{p} 1 \mathrm{~h}} &= \langle p k | d c \rangle \\
%     C_{i[j a], k[l c]}^{2 \mathrm{~h} 1 \mathrm{p}} &= \left[\left(\epsilon_i+\epsilon_j-\epsilon_a\right) \delta_{j l} \delta_{a c}-\langle j c | a l \rangle\right] \delta_{i k} \\
%     C_{[i a] b,[k c] d}^{2 \mathrm{p} 1 \mathrm{~h}} &= \left[\left(\epsilon_a+\epsilon_b-\epsilon_i\right) \delta_{i k} \delta_{a c}+\langle a k | i c\rangle\right] \delta_{b d}
% \end{align}
% Presumably they should be equivalent but I am a little bet confused. In both kisses we have physical and auxiliary spaces. However, Tim uses a bare Coulomb interaction for the coupling matrix, while Booth uses the screened interaction. Then also the definitions of the auxiliary spaces $\bm{d}$ versus $\bm{C}$ are different, where the former makes use of the $\Omega $, assuming that the RPA calculation has already been performed, while the latter is just defined by the A matrix in the dTDA case.  Please show how to get from Tim's formulation to Booth's formulation, or vice versa, and how the two are equivalent. let's just focus on a single channel, so just the lesser channel. Let us start by proving for the dTDA case, where we have the definitions for Booth's formulation
% \begin{align}
%     W_{pkv}^{<} &= \sum_{ia} (pk|ia) \left( X_{ia}^{v} \right) \\
%     d_{kv,lv'}^{<} &= \left(\epsilon_k - \Omega_v\right) \delta_{k,l} \delta_{v,v'}
% \end{align}
% where in the dtda case, we are working with the eigenproblem $\bm{A}\bm{X} = \bm{\Omega }\bm{X}$.
% while in Tim's formulation we have
% \begin{align}
%     \bm{C}^{2 \mathrm{hlp}} &= \epsilon^{1 \mathrm{~h}} \oplus (-\bm{A}) \\
% \end{align}
% with the metrics aliments defined as
% \begin{align}
%     V_{p, k[l c]}^{2 \mathrm{~h} 1 \mathrm{p}} &= \langle p c | k l \rangle \\
%     C_{i[j a], k[l c]}^{2 \mathrm{~h} 1 \mathrm{p}} &= \left[\left(\epsilon_i+\epsilon_j-\epsilon_a\right) \delta_{j l} \delta_{a c}-\langle j c | a l \rangle\right] \delta_{i k}
% \end{align}
% \section{Derivation with RPA screening using Furche's symmetric formulation}
% \label{sec:berk_gw}
% In the symmetric formulation, we have the hermitian eigenproblem:
% \begin{equation}
%     \bm{M}\bm{T}= \bm{\lambda } \bm{T}
% \label{eq:berk_gw_eigenproblem}
% \end{equation}
% where we define
% \begin{equation}
%     \bm{M}=(\bm{A}-\bm{B})^{1 / 2}(\bm{A}+\bm{B})(\bm{A}-\bm{B})^{1 / 2}
% \end{equation}
% with the $\bm{A}$ and $\bm{B}$ matrices defined as
% \begin{align}
%     A_{i a, j b} &= \left(\epsilon_a-\epsilon_i\right) \delta_{i j} \delta_{a b}+\mathcal{K}_{i a, b j} \\
%     B_{i a, j b} &= \mathcal{K}_{i a, j b}.
% \end{align}
% From this eigenproblem, we can get the quantities that we need for GW as
% \begin{equation}
%     \begin{aligned}
% & \Omega_n=\sqrt{\lambda_n} \\
% & (X+Y)_n=\frac{1}{\sqrt{\Omega_n}}(A-B)^{1 / 2} T_n \\
% \end{aligned}
% \end{equation}
% In direct RPA, one defines a frequency-independent "super-matrix" H,
% $$
% \mathbf{H}=\left(\begin{array}{ccc}
% \mathbf{f} & \mathbf{V}^{2 \mathrm{hlp}} & \mathbf{V}^{2 \mathrm{p} 1 \mathrm{~h}} \\
% \left(\mathbf{V}^{2 \mathrm{~h} 1 \mathrm{p}}\right)^{\dagger} & \mathbf{C}^{2 \mathrm{hlp}} & \mathbf{0} \\
% \left(\mathbf{V}^{2 \mathrm{plh}}\right)^{\dagger} & \mathbf{0} & \mathbf{C}^{2 \mathrm{plh}}
% \end{array}\right)
% $$
% If the Tamm-Dancoff approximation is used, we have $\mathbf{C}^{2 \text { hlp }}=\varepsilon^{1 \mathrm{~h}} \oplus(-\mathbf{A})$ and $\mathbf{C}^{2 \mathrm{plh}}=\varepsilon^{1 \mathrm{p}} \oplus \mathbf{A}$. 

% with matrix elements
% $$
% \begin{gathered}
% V_{p, k[l c]}^{2 \mathrm{~h} 1 \mathrm{p}}=\langle p c | k l\rangle \\
% V_{p,[k c] d}^{2 \mathrm{p} 1 \mathrm{~h}}=\langle p k | d c\rangle \\
% C_{i[j a], k[l c]}^{2 \mathrm{~h} 1 \mathrm{p}}=\left[\left(\varepsilon_i+\varepsilon_j-\varepsilon_a\right) \delta_{j l} \delta_{a c}-\langle j c | a l\rangle\right] \delta_{i k} \\
% C_{[i a] b,[k c] d}^{2 \mathrm{p} 1 \mathrm{~h}}=\left[\left(\varepsilon_a+\varepsilon_b-\varepsilon_i\right) \delta_{i k} \delta_{a c}+\langle a k | i c\rangle\right] \delta_{b d}
% \end{gathered}
% $$
% But now, to incorporate the RPA screening, we can instead define $\bm{C}^{2 \text{h1p}} \equiv \varepsilon^{1 \mathrm{~h}} \oplus(-\bm{M})$ and $\bm{C}^{2 \mathrm{p} 1 \mathrm{~h}} \equiv \varepsilon^{1 \mathrm{p}} \oplus \bm{M}$, where $\bm{M}$ is the matrix defined above. We can get the matrix elements of $\bm{M}$ as
% \begin{align}
%     M_{i a, j b} &= \left(\bm{A}-\bm{B}\right)^{1 / 2}_{i a, j b} \left(\bm{A}+\bm{B}\right)_{i a, j b} \left(\bm{A}-\bm{B}\right)^{1 / 2}_{i a, j b} \\
% \\
% &= \sqrt{\Delta _{ia}}\left[\left(\epsilon_a-\epsilon_i\right) \delta_{i j} \delta_{a b}+2 \mathcal{K}_{i a, b j}
% \right] \sqrt{\Delta _{jb}}. \\
% \end{align}
% In the second line we have used the fact that in the direct formulation $\bm{A}-\bm{B}$ is purely diagonal, so in fact $\left(\bm{A}-\bm{B}\right)^{1/2}$ is just the square root of the diagonal elements, which we denote as $\sqrt{\Delta_{ia}}$. Furthermore, we have that the interaction kernels $\mathcal{K}_{ia,bj} \equiv \mathcal{K}_{i a, j b}$ for the HF reference, where in the physicist's notation we have $\mathcal{K}_{i a, j b} = \langle i j | a b \rangle$. So the expressions become
% \begin{align}
%     \bm{C}^{2 \text{h1p}} &= \epsilon^{1 \mathrm{~h}} \oplus (-\bm{M}) \\
% & = \begin{pmatrix}
%     \epsilon ^{1h} & 0 \\
%     0 & -\bm{M}
% \end{pmatrix}
% \end{align}
% and
% \begin{align}
%     \bm{C}^{2 \mathrm{p} 1 \mathrm{~h}} &= \epsilon^{1 \mathrm{p}} \oplus \bm{M} \\
% & = \begin{pmatrix}
%     \epsilon ^{1p} & 0 \\
%     0 & \bm{M}
% \end{pmatrix}
% \end{align}
% where $\epsilon^{1 \mathrm{~h}}$ and $\epsilon^{1 \mathrm{p}}$ are the diagonal matrices of the one-electron energies for the hole and particle states, respectively. 

% % Now, we actually want to evaluate the metrics vector products, so we define a excitation vector $\boldsymbol{R} = ( r_i,\; r_a,\; r_{i[jb]},\; r_{[jb]a} )$ and the matrix-vector product $\bm{H} \boldsymbol{R} = \bm{\sigma }$, where $\bm{\sigma} = ( \sigma_i,\; \sigma_a,\; \sigma_{i[jb]},\; \sigma_{[jb]a} )$. Let us enumerate now what we actually will get:
% % \begin{align}
% % \sigma_i &= 
% %   \sum_{j} f_{i j}\,r_j
% %   + \sum_{b} f_{i b}\,r_b
% %   + \sum_{k l c} \bigl\langle i\,c | k\,l \bigr\rangle\,r_{k[l c]}
% %   + \sum_{k c d} \bigl\langle i\,k | d\,c \bigr\rangle\,r_{[k c]d}, \\[6pt]
% % \sigma_a &= 
% %   \sum_{j} f_{a j}\,r_j
% %   + \sum_{b} f_{a b}\,r_b
% %   + \sum_{j l c} \bigl\langle a\,c | j\,l \bigr\rangle\,r_{j[l c]}
% %   + \sum_{k c d} \bigl\langle a\,k | d\,c \bigr\rangle\,r_{[k c]d}, \\[6pt]
% % \sigma_{i[jb]} &
% %   \sum_{k} \bigl\langle k\,b | i\,j \bigr\rangle\,r_k
% %   + \sum_{a} \bigl\langle a\,j | b\,i \bigr\rangle\,r_a
% %   + M_{i a,\,j b}\,r_{i[jb]}, \\[6pt]
% % \sigma_{[jb]a} &= 
% %   \sum_{k} \bigl\langle k\,j | a\,b \bigr\rangle\,r_k
% %   + \sum_{c} \bigl\langle c\,b | j\,a \bigr\rangle\,r_c
% %   + M_{i a,\,j b}\,r_{[jb]a}.
% % \end{align}
% % The interesting part will be to figure out how to evaluate elements like $M_{i a, j b}r_{i[jb]}$. We can write this as
% % \begin{align}
% %     M_{i a, j b} r_{i[jb]} &= \sqrt{\Delta_{ia}} \left[\left(\epsilon_a-\epsilon_i\right) \delta_{i j} \delta_{a b}+2 \mathcal{K}_{i a, b j}\right] \sqrt{\Delta_{jb}} r_{i[jb]} \\
% % &= \sqrt{\Delta_{ia}} \left[\left(\epsilon_a-\epsilon_i\right) \delta_{i j} \delta_{a b}+2 \mathcal{K}_{i a, b j}\right] r_{i[jb]} \sqrt{\Delta_{jb}}.
% % \end{align}
They say that $\bm{V}^{2\mathrm{h1p}}$ has the  elements $V_{p, k[ia]}^{2 \mathrm{~h} 1 \mathrm{p}} = \langle p a | k i \rangle \equiv (pk|ia)$ so we say that $\bm{v}^{2\mathrm{h1p}}_{O,O^2V}$ is the first O rows of $\bm{V}^{2\mathrm{h1p}}$, while $\bm{v}^{2\mathrm{h1p}}_{V,O^2V}$ is the latter V rows of $\bm{V}^{2\mathrm{h1p}}$.
% $\bm{C}^{2\mathrm{hlp}}$ is defined as $\varepsilon^{1h} \oplus -\tilde{\bm{M}}.$ They also say that $\bm{X}=1\oplus \bm{N}$, where $\bm{N}=\begin{pmatrix} \bm{1}_{OV} & 0 \\ 0 & -\bm{1}_{OV} \end{pmatrix}$, but I am doubting this because the shapes in the matrix multiplication don't work out then.
% Define $\begin{pmatrix} \bm{1}_{OV} & 0 \\
% 0 & -\bm{1}_{OV} \end{pmatrix} = \bm{N}$ and we have $V_{p, k[ia]}^{2 \mathrm{~h} 1 \mathrm{p}} = \langle p a | k i \rangle \equiv (pk|ia)$. To simplify matters, let's just consider the $2\mathrm{h1p}$ sector. Writing this down gives
% \begin{align}
% &\bm{\mathcal{N}} \bm{H}^{2\mathrm{h1p}}\\ &= \begin{pmatrix}
% \bm{1}_{P,P} & 0 \\
% 0 & \bm{1}_{O,O} \oplus \bm{N}_{2OV,2OV} \\
% \end{pmatrix}
% \begin{pmatrix}
% \bm{F} & \bm{V}^{2\mathrm{h1p}} & \bm{V}^{2\mathrm{h1p}} \\
% \left(\bm{V}^{2\mathrm{h1p}}\right)^{\dagger} &  &  \\
% \left(\bm{V}^{2\mathrm{h1p}}\right)^{\dagger} &  & \bm{C}^{2\mathrm{hlp}}
% \end{pmatrix} \\
% &= \begin{pmatrix}\bm{F}_{P,P} & \bm{1}_{P,P} \bm{V}^{2\mathrm{h1p}}_{P,2O^2V} & \bm{1}_{P,P} \bm{V}^{2\mathrm{h1p}}_{P,2O^2V} \\
% & &  &  & & \\
% \begin{pmatrix}
% \bm{1}_{O,O} & 0 \\
% 0 & \bm{N}_{2OV,2OV}
% \end{pmatrix}
% \begin{bmatrix}
% \left( \bm{V}^{2\mathrm{h1p}} \right)^{\dagger} \\ \left( \bm{V}^{2\mathrm{h1p}}\right)^{\dagger}\end{bmatrix}_{2O^2V,P} &  &\begin{pmatrix}
% \bm{1}_{O,O} & 0 \\
% 0 & \bm{N}_{2OV,2OV}
% \end{pmatrix}\begin{pmatrix}
%     \epsilon^{1\mathrm{h}}_{O,O} & 0 \\
%     0 & -\tilde{\mathbf{M}}_{2OV,2OV}
% \end{pmatrix} \\
% \end{pmatrix}\\
% &= \begin{pmatrix}\bm{F}_{P,P} & \bm{1}_{P,P} \bm{V}^{2\mathrm{h1p}}_{P,2O^2V} & \bm{1}_{P,P} \bm{V}^{2\mathrm{h1p}}_{P,2O^2V} \\
% & &  &  & & \\
% \begin{pmatrix}
% \bm{1}_{O,O} & 0 \\
% 0 & \bm{N}_{2OV,2OV}
% \end{pmatrix}
% \begin{bmatrix}
% \left( \bm{V}^{2\mathrm{h1p}} \right)^{\dagger} \\ \left( \bm{V}^{2\mathrm{h1p}}\right)^{\dagger}\end{bmatrix}_{2O^2V,P} &  &\begin{pmatrix}
%     \epsilon^{1\mathrm{h}}_{O,O} & 0 \\
%     0 & -\mathbf{N}\tilde{\bm{M}}_{2OV,2OV}
% \end{pmatrix} \\
% \end{pmatrix}
% \end{align}
% % \begin{align}
% % &\bm{\mathcal{N}} \bm{H}\\ &= \begin{pmatrix}
% % \bm{1}_{P,P} & 0 & 0 \\
% % 0 & \bm{1}_{O,O} \oplus \bm{N}_{2OV,2OV} & 0 \\
% % 0 & 0 & \bm{1}_{V,V} \oplus \bm{N}_{2OV,2OV}
% % \end{pmatrix}
% % \begin{pmatrix}
% % \bm{F} & \bm{V}^{2\mathrm{h1p}} & \bm{V}^{2\mathrm{h1p}} & \bm{V}^{2\mathrm{plh}} & \bm{V}^{2\mathrm{plh}} \\
% % \left(\bm{V}^{2\mathrm{h1p}}\right)^{\dagger} &  &  & & \\
% % \left(\bm{V}^{2\mathrm{h1p}}\right)^{\dagger} &  & \bm{C}^{2\mathrm{hlp}} & & \bm{0} \\
% % \left(\bm{V}^{2\mathrm{plh}}\right)^{\dagger} & & & & \\
% % \left(\bm{V}^{2\mathrm{plh}}\right)^{\dagger} &  & \bm{0} & & \bm{C}^{2\mathrm{plh}}
% % \end{pmatrix} \\
% % &= \begin{pmatrix}\bm{F}_{P,P} & \bm{1}_{P,P} \bm{V}^{2\mathrm{h1p}}_{P,A} & \bm{1}_{P,P} \bm{V}^{2\mathrm{h1p}}_{P,A} & \bm{1}_{P,P} \bm{V}^{2\mathrm{plh}}_{P,\bar{A}} & \bm{1}_{P,P} \bm{V}^{2\mathrm{plh}}_{P,\bar{A}} \\
% % & &  &  & & \\
% % \left[ \bm{1} \oplus \bm{N} \right]_{2A,2A}\begin{bmatrix}
% % \left( \bm{V}^{2\mathrm{h1p}} \right)^{\dagger} \\ \left( \bm{V}^{2\mathrm{h1p}}\right)^{\dagger}\end{bmatrix}_{2A,P} &  & \left[ \bm{1} \oplus \bm{N} \right]_{2A,2A}\bm{C}^{2\mathrm{hlp}}_{2A,2A} & & \bm{0}\\
% % & & & & & \\
% % \left[ \bm{1} \oplus \bm{N} \right]_{2\bar{A},2\bar{A}}\begin{bmatrix}
% % \left( \bm{V}^{2\mathrm{plh}} \right)^{\dagger} \\ \left( \bm{V}^{2\mathrm{plh}}\right)^{\dagger}\end{bmatrix}_{2\bar{A},P} &  & \bm{0} & & \left[ \bm{1} \oplus \bm{N} \right]_{2\bar{A},2\bar{A}}\bm{C}^{2\mathrm{plh}}_{2\bar{A},2\bar{A}}
% % \end{pmatrix}
% % \end{align}
%  If we assume that the virtual space is largest, this means that the column width of the excitation vector is $\bar{A}$. The action of the super-metric on the excitation vector is given by
% $\bm{\mathcal{N}} \bm{H} \bm{R} = \bm{R} \bm{E}$ leads to:
% \begin{equation}
% \begin{pmatrix}
% \bm{1} & 0 & 0 \\
% 0 & \bm{1} \oplus \bm{N} & 0 \\
% 0 & 0 & \bm{1} \oplus \bm{N}
% \end{pmatrix}
% \begin{pmatrix}
% \bm{F} & \bm{V}^{2\mathrm{h1p}} & \bm{V}^{2\mathrm{h1p}} & \bm{V}^{2\mathrm{plh}} & \bm{V}^{2\mathrm{plh}} \\
% \left(\bm{V}^{2\mathrm{h1p}}\right)^{\dagger} &  &  & & \\
% \left(\bm{V}^{2\mathrm{h1p}}\right)^{\dagger} &  & \bm{C}^{2\mathrm{hlp}} & & \bm{0} \\
% \left(\bm{V}^{2\mathrm{plh}}\right)^{\dagger} & & & & \\
% \left(\bm{V}^{2\mathrm{plh}}\right)^{\dagger} &  & \bm{0} & & \bm{C}^{2\mathrm{plh}}
% \end{pmatrix}_{T,T}
% \begin{pmatrix}
% \bm{r}^{1\mathrm{h}+1\mathrm{p}} \\
% \bm{r}^{2\mathrm{h}1\mathrm{p}} \\
% \bm{\bar{r}}^{2\mathrm{h}1\mathrm{p}} \\
% \bm{r}^{2\mathrm{p}1\mathrm{h}} \\
% \bm{\bar{r}}^{2\mathrm{p}1\mathrm{h}}
% \end{pmatrix}_{T,\bar{A}} = E
% \begin{pmatrix}
% \bm{r}^{1\mathrm{h}+1\mathrm{p}} \\
% \bm{r}^{2\mathrm{h}1\mathrm{p}} \\
% \bm{\bar{r}}^{2\mathrm{h}1\mathrm{p}} \\
% \bm{r}^{2\mathrm{p}1\mathrm{h}} \\
% \bm{\bar{r}}^{2\mathrm{p}1\mathrm{h}}
% \end{pmatrix}_{T,\bar{A}}
% \end{equation}

% From this we get:
% \begin{align}
% &
% \begin{pmatrix}
% \bm{1}_{P,P} & 0 & 0 \\
% 0 & \left(\bm{1}_{O} \oplus_{\text{kron}} \bm{N}_{2OV}\right)_{2A,2A} & 0 \\
% 0 & 0 & \left(\bm{1}_V \oplus_{\text{kron}} \bm{N}_{2OV}\right)_{2\bar{A},2\bar{A}}
% \end{pmatrix}_{T,T} \\
% &
% \begin{pmatrix}
% \bm{F}\bm{r}^{1\mathrm{h}+1\mathrm{p}} + \bm{V}^{2\mathrm{h}1\mathrm{p}}\bm{r}^{2\mathrm{h}1\mathrm{p}} + \bm{V}^{2\mathrm{h}1\mathrm{p}}\bm{\bar{r}}^{2\mathrm{h}1\mathrm{p}} + \bm{V}^{2\mathrm{plh}}\bm{r}^{2\mathrm{p}1\mathrm{h}} + \bm{V}^{2\mathrm{plh}}\bm{\bar{r}}^{2\mathrm{p}1\mathrm{h}} \\
% \left(\bm{V}^{2\mathrm{h1p}}\right)^{\dagger} \bm{r}^{1\mathrm{h}+1\mathrm{p}} + [\bm{\tilde{C}}^{2\mathrm{hlp}}]^{xx} \bm{r}^{2\mathrm{h}1\mathrm{p}} + [\bm{\tilde{C}}^{2\mathrm{hlp}}]^{xd} \bm{\bar{r}}^{2\mathrm{h}1\mathrm{p}} + \bm{0} \\
% \left(\bm{V}^{2\mathrm{h1p}}\right)^{\dagger} \bm{r}^{1\mathrm{h}+1\mathrm{p}} + [\bm{\tilde{C}}^{2\mathrm{hlp}}]^{dx} \bm{r}^{2\mathrm{h}1\mathrm{p}} + [\bm{\tilde{C}}^{2\mathrm{hlp}}]^{dd} \bm{\bar{r}}^{2\mathrm{h}1\mathrm{p}} + \bm{0} \\
% \left(\bm{V}^{2\mathrm{plh}}\right)^{\dagger} \bm{r}^{1\mathrm{h}+1\mathrm{p}} + \bm{0} + [\bm{\tilde{C}}^{2\mathrm{plh}}]^{xx} \bm{r}^{2\mathrm{p}1\mathrm{h}} + [\bm{\tilde{C}}^{2\mathrm{plh}}]^{xd} \bm{\bar{r}}^{2\mathrm{p}1\mathrm{h}} \\
% \left(\bm{V}^{2\mathrm{plh}}\right)^{\dagger} \bm{r}^{1\mathrm{h}+1\mathrm{p}} + \bm{0} + [\bm{\tilde{C}}^{2\mathrm{plh}}]^{dx} \bm{r}^{2\mathrm{p}1\mathrm{h}} + [\bm{\tilde{C}}^{2\mathrm{plh}}]^{dd} \bm{\bar{r}}^{2\mathrm{p}1\mathrm{h}}
% \end{pmatrix}_{T,\bar{A}}\\ &= E
% \begin{pmatrix}
% \bm{r}^{1\mathrm{h}+1\mathrm{p}} \\
% \bm{r}^{2\mathrm{h}1\mathrm{p}} \\
% \bm{\bar{r}}^{2\mathrm{h}1\mathrm{p}} \\
% \bm{r}^{2\mathrm{p}1\mathrm{h}} \\
% \bm{\bar{r}}^{2\mathrm{p}1\mathrm{h}}
% \end{pmatrix}_{T,\bar{A}}
% \end{align}
% So notice that the dimensions inside the super-metric must be as indicated. The left hand side becomes
% \begin{align}
%     \begin{pmatrix}
% \bm{F}\bm{r}^{1\mathrm{h}+1\mathrm{p}} + \bm{V}^{2\mathrm{h}1\mathrm{p}}\bm{r}^{2\mathrm{h}1\mathrm{p}} + \bm{V}^{2\mathrm{h}1\mathrm{p}}\bm{\bar{r}}^{2\mathrm{h}1\mathrm{p}} + \bm{V}^{2\mathrm{plh}}\bm{r}^{2\mathrm{p}1\mathrm{h}} + \bm{V}^{2\mathrm{plh}}\bm{\bar{r}}^{2\mathrm{p}1\mathrm{h}} \\
% \left(\bm{V}^{2\mathrm{h1p}}\right)^{\dagger} \bm{r}^{1\mathrm{h}+1\mathrm{p}} + [\bm{\tilde{C}}^{2\mathrm{hlp}}]^{xx} \bm{r}^{2\mathrm{h}1\mathrm{p}} + [\bm{\tilde{C}}^{2\mathrm{hlp}}]^{xd} \bm{\bar{r}}^{2\mathrm{h}1\mathrm{p}} + \bm{0} \\
% \bm{N}_{A,A}\left(\bm{V}^{2\mathrm{h1p}}\right)^{\dagger}_{A,P} \bm{r}^{1\mathrm{h}+1\mathrm{p}}_{P,\bar{A}} + \bm{N}_{A,A}[\bm{\tilde{C}}^{2\mathrm{hlp}}]^{dx}_{A,A} \bm{r}^{2\mathrm{h}1\mathrm{p}}_{A,\bar{A}} + \bm{N}_{A,A}[\bm{\tilde{C}}^{2\mathrm{hlp}}]^{dd}_{A,A} \bm{\bar{r}}^{2\mathrm{h}1\mathrm{p}}_{A,\bar{A}} + \bm{0} \\
% \left(\bm{V}^{2\mathrm{plh}}\right)^{\dagger} \bm{r}^{1\mathrm{h}+1\mathrm{p}} + \bm{0} + [\bm{\tilde{C}}^{2\mathrm{plh}}]^{xx} \bm{r}^{2\mathrm{p}1\mathrm{h}} + [\bm{\tilde{C}}^{2\mathrm{plh}}]^{xd} \bm{\bar{r}}^{2\mathrm{p}1\mathrm{h}} \\
% \bm{N}_{\bar{A}, \bar{A}}\left(\bm{V}^{2\mathrm{plh}}\right)^{\dagger}_{\bar{A},P} \bm{r}^{1\mathrm{h}+1\mathrm{p}}_{P,\bar{A}} + \bm{0} + \bm{N}_{\bar{A}, \bar{A}}[\bm{\tilde{C}}^{2\mathrm{plh}}]^{dx}_{\bar{A},\bar{A}} \bm{r}^{2\mathrm{p}1\mathrm{h}}_{\bar{A},\bar{A}} + \bm{N}_{\bar{A}, \bar{A}}[\bm{\tilde{C}}^{2\mathrm{plh}}]^{dd}_{\bar{A},\bar{A}} \bm{\bar{r}}^{2\mathrm{p}1\mathrm{h}}_{\bar{A},\bar{A}}
% \end{pmatrix}_{T,\bar{A}}
% \end{align}
% This is the point at which we encounter issues in the derivation. See the alternate derivation below for a more straightforward approach that highlights this.
% \begin{equation}
% \begin{pmatrix}
% \bm{1} & 0 & 0 \\
% 0 & \bm{1} \oplus \bm{N} & 0 \\
% 0 & 0 & \bm{1} \oplus \bm{N}
% \end{pmatrix}
% \begin{pmatrix}
% \bm{F} & \bm{V}^{2\mathrm{h1p}} & \bm{V}^{2\mathrm{h1p}} & \bm{V}^{2\mathrm{plh}} & \bm{V}^{2\mathrm{plh}} \\
% \left(\bm{V}^{2\mathrm{h1p}}\right)^{\dagger} &  &  & & \\
% \left(\bm{V}^{2\mathrm{h1p}}\right)^{\dagger} &  & \bm{C}^{2\mathrm{hlp}} & & \bm{0} \\
% \left(\bm{V}^{2\mathrm{plh}}\right)^{\dagger} & & & & \\
% \left(\bm{V}^{2\mathrm{plh}}\right)^{\dagger} &  & \bm{0} & & \bm{C}^{2\mathrm{plh}}
% \end{pmatrix}
% \begin{pmatrix}
% \bm{R}^{1\mathrm{h}+1\mathrm{p}} \\
% \bm{R}^{2\mathrm{h}1\mathrm{p}} \\
% \bm{R}^{2\mathrm{p}1\mathrm{h}} \\
% \end{pmatrix} = E
% \begin{pmatrix}
% \bm{R}^{1\mathrm{h}+1\mathrm{p}} \\
% \bm{R}^{2\mathrm{h}1\mathrm{p}} \\
% \bm{R}^{2\mathrm{p}1\mathrm{h}} \\
% \end{pmatrix}
% \end{equation}
% The 3 coupled equations can be written as
% \begin{align}
% \bm{F} \bm{R}^{1\mathrm{h}+1\mathrm{p}} +\bm{V}^{2\mathrm{h}1\mathrm{p}} \bm{R}^{2\mathrm{h}1\mathrm{p}}  + \bm{V}^{2\mathrm{plh}} \bm{R}^{2\mathrm{p}1\mathrm{h}} &= E \bm{R}^{1\mathrm{h}+1\mathrm{p}} \\
%  \left(1\oplus N\right)\left(\bm{V}^{2\mathrm{h}1\mathrm{p}}\right)^{\dagger} \bm{R}^{1\mathrm{h}+1\mathrm{p}} + \left(1\oplus N\right) \left(\varepsilon^{1\mathrm{~h}} \oplus (-\bm{\tilde{M}})\right)\bm{R}^{2\mathrm{h}1\mathrm{p}} &= E \bm{R}^{2\mathrm{h}1\mathrm{p}} \\
% \left(1\oplus N\right)\left(\bm{V}^{2\mathrm{p}1\mathrm{h}}\right)^\dagger \bm{R}^{1\mathrm{h}+1\mathrm{p}} + \left(1\oplus N\right)\left(\varepsilon^{1\mathrm{p}} \oplus \bm{\tilde{M}}\right)\bm{\tilde{C}}^{2\mathrm{p}1\mathrm{h}}\bm{R}^{2\mathrm{p}1\mathrm{h}} &= E \bm{R}^{2\mathrm{p}1\mathrm{h}}
% \end{align}
% and we now define $\bm{\tilde{C}}^{2\mathrm{hlp}} = \left(1\oplus N\right) \left(\varepsilon^{1\mathrm{~h}} \oplus (-\bm{\tilde{M}})\right) = \varepsilon^{1\mathrm{~h}} \oplus (-\bm{N}\bm{\tilde{M}})$ and $\bm{\tilde{C}}^{2\mathrm{plh}} = \left(1\oplus N\right)\left(\varepsilon^{1\mathrm{p}} \oplus \bm{\tilde{M}}\right) = \varepsilon^{1\mathrm{p}} \oplus \bm{N}\bm{\tilde{M}}$.
% Now solve the last two equations for $\bm{R}^{2\mathrm{h}1\mathrm{p}}$ and $\bm{\bar{R}}^{2\mathrm{p}1\mathrm{h}}$:
% \begin{align}
% \bm{R}^{2\mathrm{h}1\mathrm{p}} &= \left(\bm{1}E - \bm{\tilde{C}}^{2\mathrm{hlp}}\right)^{-1} \left( \left(1\oplus N\right)\left(\bm{V}^{2\mathrm{h}1\mathrm{p}}\right)^\dagger \bm{R}^{1\mathrm{h}+1\mathrm{p}} \right)\\
% \bm{\bar{R}}^{2\mathrm{p}1\mathrm{h}} &= \left(\bm{1}E - \bm{\tilde{C}}^{2\mathrm{plh}}\right)^{-1} \left( \left(1\oplus N\right)\left(\bm{V}^{2\mathrm{plh}}\right)^\dagger \bm{R}^{1\mathrm{h}+1\mathrm{p}} \right)
% \end{align}
% Substituting these into the first equation gives us
% \begin{align}
% \bm{F} \bm{R}^{1\mathrm{h}+1\mathrm{p}} &+ \bm{V}^{2\mathrm{h}1\mathrm{p}} \left(\bm{1}E - \bm{\tilde{C}}^{2\mathrm{hlp}}\right)^{-1} \left( \left(1\oplus N\right)\left(\bm{V}^{2\mathrm{h}1\mathrm{p}}\right)^\dagger \bm{R}^{1\mathrm{h}+1\mathrm{p}} \right)\\
% &+  \bm{V}^{2\mathrm{plh}} \left(\bm{1}E - \bm{\tilde{C}}^{2\mathrm{plh}}\right)^{-1} \left( \left(1\oplus N\right)\left(\bm{V}^{2\mathrm{plh}}\right)^\dagger \bm{R}^{1\mathrm{h}+1\mathrm{p}} \right) = E \bm{R}^{1\mathrm{h}+1\mathrm{p}}
% \end{align}
% This gives us a form for the frequency dependent correlation self energy as
% \begin{align}
% \bm{\Sigma }^c(\omega ) &= \bm{V}^{2\mathrm{h}1\mathrm{p}} \left(\bm{1}\omega  - \bm{\tilde{C}}^{2\mathrm{hlp}}\right)^{-1} \left(1\oplus N\right) \left(\bm{V}^{2\mathrm{h}1\mathrm{p}}\right)^\dagger +  \bm{V}^{2\mathrm{plh}} \left(\bm{1}\omega  - \bm{\tilde{C}}^{2\mathrm{plh}}\right)^{-1} \left(1\oplus N\right)\left(\bm{V}^{2\mathrm{plh}}\right)^\dagger
% \end{align}
%  The next thing we need to do is to write the spectral decomposition of the $\bm{\tilde{C}}$ where for the 2h1p sector, this is $\bm{\tilde{C}}^{2\mathrm{hlp}} = \bm{U} \left(\bm{1}\left(\epsilon_k - \Omega_{\nu}\right)\right) \bm{U}^{-1} \implies \left( \bm{1}\omega  - \bm{\tilde{C}}^{2\mathrm{hlp}}\right)^{-1} = \bm{U} \left(\bm{1}\left[ \omega - \left(\epsilon_k - \Omega_\nu\right)\right]\right)^{-1} \bm{U}^{-1}$ and $\bm{\tilde{C}}^{2\mathrm{plh}} = \bm{U} \left(\bm{1}\left(\epsilon_k + \Omega_{\nu}\right)\right) \bm{U}^{-1} \implies \left( \bm{1}\omega  - \bm{\tilde{C}}^{2\mathrm{plh}}\right)^{-1} = \bm{U} \left(\bm{1}\left[ \omega  - \left(\epsilon_k + \Omega_\nu\right)\right]\right)^{-1} \bm{U}^{-1}$. Due to the bioorthonormality of the problem, the columns of $\bm{U}$ are the left eigenvectors of the $\bm{\tilde{C}}$ matrices, while the rows are the right eigenvectors. Thus, can we write
% % \begin{align}
% % \bm{\Sigma }^c(\omega ) &= 4 \frac{\bm{M}^{2\mathrm{h}1\mathrm{p}}\bm{M}^{2\mathrm{h}1\mathrm{p}}}{\omega  - \epsilon_k - \Omega_\nu} + 4 \frac{\bm{M}^{2\mathrm{plh}}\bm{M}^{2\mathrm{plh}}}{\omega  - \epsilon_k + \Omega_\nu}
% % \end{align}
% % where we have defined the transition densities $\bm{M}$ as a contraction between the RPA eigenvectors and the $\bm{V}$ matrix. 
% \begin{align}
% \bm\Sigma^c_{pq}(\omega)&=\sum_\nu \biggl[
%     \sum_k\frac{\left(\sum_{[ia]}\bm{V^{2h1p}}_{p,k[ia]}|v_{[ia],\nu}^{R,2h1p}\rangle\right) \left(\sum_{[jb]} \langle v_{[jb],\nu}^{L,2h1p}|\,(\bm{V^{2h1p}}_{q,k[jb]})^\dagger\right)}
%          {\omega - (\epsilon_k - \Omega_\nu)}\\
%   &+ \sum_c\frac{\left(\sum_{[ia]}\bm{V^{2plh}}_{p,[ia]c}|v_{[ia],\nu}^{R,2plh}\rangle\right) \left(\sum_{[jb]} \langle v_{[jb],\nu}^{L,2plh}|\,(\bm{V^{2plh}}_{q,[jb]c})^\dagger\right)}
%          {\omega - (\epsilon_c + \Omega_\nu)} \biggr]?
% \end{align}
% % \begin{align}
% % \left(\bm(){V}^{2\mathrm{plh}}\right)^\dagger \bm{r}^{1\mathrm{h}+1\mathrm{p}} + [\bm{\tilde{C}}^{2\mathrm{plh}}]^{xx}\bm{r}^{2\mathrm{p}1\mathrm{h}}+ [\bm{\tilde{C}}^{2\mathrm{plh}}]^{xd}\bm{\bar{r}}^{2\mathrm{p}1\mathrm{h}} &= E \bm{r}^{2\mathrm{p}1\mathrm{h}} \\
% % \left(\bm{V}^{2\mathrm{plh}}\right)^\dagger \bm{r}^{1\mathrm{h}+1\mathrm{p}} + [\bm{\tilde{C}}^{2\mathrm{plh}}]^{dx}\bm{r}^{2\mathrm{p}1\mathrm{h}}+ [\bm{\tilde{C}}^{2\mathrm{plh}}]^{dd}\bm{\bar{r}}^{2\mathrm{p}1\mathrm{h}} &= E \bm{\bar{r}}^{2\mathrm{p}1\mathrm{h}}
% % \end{align}
% % If we make the definitions $\bm{R}_{\Sigma }^{2\mathrm{h}1\mathrm{p}} = \bm{r}^{2\mathrm{h}1\mathrm{p}} + \bm{\bar{r}}^{2\mathrm{h}1\mathrm{p}}$ and $\bm{R}_{\Sigma }^{2\mathrm{p}1\mathrm{h}} = \bm{r}^{2\mathrm{p}1\mathrm{h}} + \bm{\bar{r}}^{2\mathrm{p}1\mathrm{h}}$, we can rewrite:
% % \begin{align}
% % \bm{F} \bm{r}^{1\mathrm{h}+1\mathrm{p}} + \bm{V}^{2\mathrm{h}1\mathrm{p}} \bm{R}_{\Sigma }^{2\mathrm{h}1\mathrm{p}} + \bm{V}^{2\mathrm{plh}} \bm{R}_{\Sigma }^{2\mathrm{p}1\mathrm{h}} &= E \bm{r}^{1\mathrm{h}+1\mathrm{p}} \\
% % \left(\bm{V}^{2\mathrm{h}1\mathrm{p}}\right)^\dagger \bm{r}^{1\mathrm{h}+1\mathrm{p}} + \left([\bm{\tilde{C}}^{2\mathrm{hlp}}]^{xx} + [\bm{\tilde{C}}^{2\mathrm{hlp}}]^{xd} \right)\bm{R}_{\Sigma }^{2\mathrm{h}1\mathrm{p}} &= E \bm{r}^{2\mathrm{h}1\mathrm{p}} \\
% % \left(\bm{V}^{2\mathrm{h}1\mathrm{p}}\right)^\dagger \bm{r}^{1\mathrm{h}+1\mathrm{p}} - [\bm{\tilde{C}}^{2\mathrm{hlp}}]^{xd}\bm{r}^{2\mathrm{h}1\mathrm{p}} - [\bm{\tilde{C}}^{2\mathrm{hlp}}]^{xx}\bm{\bar{r}}^{2\mathrm{h}1\mathrm{p}} &= E \bm{\bar{r}}^{2\mathrm{h}1\mathrm{p}} \\
% % \left(\bm{V}^{2\mathrm{plh}}\right)^\dagger \bm{r}^{1\mathrm{h}+1\mathrm{p}} + [\bm{\tilde{C}}^{2\mathrm{plh}}]^{xx}\bm{r}^{2\mathrm{p}1\mathrm{h}}+ [\bm{\tilde{C}}^{2\mathrm{plh}}]^{xd}\bm{\bar{r}}^{2\mathrm{p}1\mathrm{h}} &= E \bm{r}^{2\mathrm{p}1\mathrm{h}} \\
% % \left(\bm{V}^{2\mathrm{plh}}\right)^\dagger \bm{r}^{1\mathrm{h}+1\mathrm{p}} - [\bm{\tilde{C}}^{2\mathrm{plh}}]^{xd}\bm{r}^{2\mathrm{p}1\mathrm{h}} - [\bm{\tilde{C}}^{2\mathrm{plh}}]^{xx}\bm{\bar{r}}^{2\mathrm{p}1\mathrm{h}} &= E \bm{\bar{r}}^{2\mathrm{p}1\mathrm{h}}
% % \end{align}
% % We can combine into 3 by adding the second to the third and the fourth to the fifth, which gives us
% % \begin{align}
% % \bm{F} \bm{r}^{1\mathrm{h}+1\mathrm{p}} + \bm{V}^{2\mathrm{h}1\mathrm{p}} \left(\bm{r}^{2\mathrm{h}1\mathrm{p}} + \bm{\bar{r}}^{2\mathrm{h}1\mathrm{p}}\right) + \bm{V}^{2\mathrm{plh}} \left(\bm{r}^{2\mathrm{p}1\mathrm{h}} + \bm{\bar{r}}^{2\mathrm{p}1\mathrm{h}}\right) &= E \bm{r}^{1\mathrm{h}+1\mathrm{p}} \\[6pt]
% % 2\left(\bm{V}^{2\mathrm{h}1\mathrm{p}}\right)^\dagger \bm{r}^{1\mathrm{h}+1\mathrm{p}} + \left([\bm{\tilde{C}}^{2\mathrm{hlp}}]^{xx}+[\bm{\tilde{C}}^{2\mathrm{hlp}}]^{xd}\right)\bm{r}^{2\mathrm{h}1\mathrm{p}}+ \left([\bm{\tilde{C}}^{2\mathrm{hlp}}]^{dx}+[\bm{\tilde{C}}^{2\mathrm{hlp}}]^{dd}\right)\bm{\bar{r}}^{2\mathrm{h}1\mathrm{p}} &= E \bm{r}^{2\mathrm{h}1\mathrm{p}} \\[6pt]
% % \end{align}

% % Solving the last two equations gives:
% % \begin{align}
% % \bm{r}^{2\mathrm{h}1\mathrm{p}} + \bm{\bar{r}}^{2\mathrm{h}1\mathrm{p}} &= \left(E\bm{1} - \bm{\tilde{C}}^{2\mathrm{hlp}}\right)^{-1} \left(\left(\bm{V}^{2\mathrm{h}1\mathrm{p}}\right)^\dagger \bm{r}^{1\mathrm{h}+1\mathrm{p}} - \bm{\tilde{C}}^{2\mathrm{plh}} \left(\bm{r}^{2\mathrm{p}1\mathrm{h}} + \bm{\bar{r}}^{2\mathrm{p}1\mathrm{h}}\right)\right) \\[6pt]
% % \bm{r}^{2\mathrm{p}1\mathrm{h}} + \bm{\bar{r}}^{2\mathrm{p}1\mathrm{h}} &= \left(\bm{\tilde{C}}^{2\mathrm{plh}} - E \bm{1}\right)^{-1} \left(\left(\bm{V}^{2\mathrm{plh}}\right)^\dagger \bm{r}^{1\mathrm{h}+1\mathrm{p}} + \bm{\tilde{C}}^{2\mathrm{hlp}} \left(\bm{r}^{2\mathrm{h}1\mathrm{p}} + \bm{\bar{r}}^{2\mathrm{h}1\mathrm{p}}\right)\right)
% % \end{align}
% % \subsubsection{Deriving matrix vector products for 2h1p sector}
% % The eigenvalue equation is then given by $\bm{H} \bm{R} = \bm{\mathcal{N}} \bm{R} \bm{E}$ and then leading to the non-Hermitian eigenvalue equation $\mathcal{\bm{N}} \bm{H} \bm{R} = \bm{R} \bm{E}$.
% % Let us carry this matrix multiplication out, just for the 2h1p sector. Collect the relevant components of the excitation vector into a block vector:
% % \[
% % \bm{R}_{T,A} = 
% % \begin{pmatrix}
% % \bm{r}_i &\bm{r}_a &
% % \bm{r}_{i[jb]} &
% % \bm{\bar{r}}_{i[jb]}
% % \end{pmatrix}^\dag
% % \]
% % We will define the total dimension of this excitation vector as $T=P+2A$, where $P=O+V$ and $A=O^2V$.
% % The super-Hamiltonian decomposes as:
% % \[
% % \bm{H} =
% % \begin{pmatrix}
% % \begin{pmatrix}
% %     \bm{F}_{oo} & \bm{F}_{ov} \\
% %     \bm{F}_{vo} & \bm{F}_{vv}
% % \end{pmatrix}_{P,P} &
% % \begin{pmatrix}
% %     \bm{V}^{2 \mathrm{h1p}}_{p,i[jb]} & \bm{V}^{2 \mathrm{h1p}}_{p,i[\bar{jb}]} \\
% % \end{pmatrix}_{P,2A} \\ 
% % \begin{pmatrix}
% %     \bm{V}^{2 \mathrm{h1p}}_{p,k[lc]} & \bm{V}^{2 \mathrm{h1p}}_{p,k[\bar{lc}]}
% % \end{pmatrix}^\dagger_{2A,P} &
% % \left(\left(\varepsilon^{1h}\right) \oplus_{\text{direct}} -\bm{\tilde{M}}_{jb,jb;lc,lc}\right)_{2A,2A}
% % \end{pmatrix}_{T,T}
% % \] 
% % The matrix-vector product becomes:
% % \[
% % \bm{\sigma}_{T,A}
% % = 
% % \bm{\mathcal{N}}_{T,T}
% % \begin{pmatrix}
% %  \begin{pmatrix}
% % \bm{F}_{oo} & \bm{F}_{ov} \\ \bm{F}_{vo} & \bm{F}_{vv}
% % \end{pmatrix}_{P,P} \begin{pmatrix}
% % \bm{r}_i \\ \bm{r}_a
% % \end{pmatrix}_{P,A} + \begin{pmatrix}
% %     \bm{V}^{2 \mathrm{h1p}}_{p,i[jb]} & \bm{V}^{2 \mathrm{h1p}}_{p,i[\bar{jb}]} \\
% % \end{pmatrix}_{P,2A} \begin{pmatrix}
% %     \bm{r}_{i[jb]} \\
% % \bm{\bar{r}}_{i[\bar{jb}]}
% % \end{pmatrix}_{2A,A} \\[6pt]
% %  \begin{pmatrix}
% %     \bm{V}^{2 \mathrm{h1p}}_{p,i[jb]} & \bm{V}^{2 \mathrm{h1p}}_{p,i[\bar{jb}]} \\
% % \end{pmatrix}^\dag_{2A,P} \begin{pmatrix}
% % \bm{r}_i \\ \bm{r}_a
% % \end{pmatrix}_{P,A} +
% % \left(\left(\varepsilon^{1h}\right) \oplus_{\text{direct}} -\bm{\tilde{M}}_{jb,jb;lc,lc}\right)_{2A,2A} \begin{pmatrix}
% %     \bm{r}_{i[jb]} \\
% % \bm{\bar{r}}_{i[\bar{jb}]}
% % \end{pmatrix}_{2A,A}
% % \end{pmatrix}
% % \]
% % Now we are left to decide what the super-metric $\bm{\mathcal{N}}$ is. We can define it as
% % \[\bm{\mathcal{N}} =
% % \begin{pmatrix}
% % \bm{I_{PP}} & 0 \\[3pt]
% % 0 & \bm{N}_{2A,2A}
% % \end{pmatrix}_{T,T}
% % \]
% %  with $\bm{N}_{2A,2A} \equiv \bm{1}_{O} \oplus_{\text{direct}} \bm{N}_{\text{2OV}}$. 

% % % But if we take this definiton, then $\bm{\mathcal{N}}\bm{\mathcal{N}} \neq \bm{1}$, so we can't do $\bm{H} \bm{R} = \bm{\mathcal{N}} \bm{R} \bm{E} \implies \bm{\mathcal{N}} \bm{H} \bm{R} = \bm{R} \bm{E}$, as they claim. Next, assume they made a typo, so do $\bm{N}_{AA} = \begin{pmatrix}
% % %  \bm{1}_{O^2V} & 0 \\
% % % 0 & \bm{-1}_{O^2V}
% % %  \end{pmatrix}$. Now, we have  $\bm{N}_{AA}\bm{N}_{AA} = \bm{1}_{AA}$ and
% % \begin{align}
% % &    \bm{\mathcal{N}} \bm{H} \bm{R}\\
% % &= 
% % \begin{pmatrix}
% % \bm{1}_{P,P}
% %  \begin{pmatrix}
% % \bm{F}_{oo} & \bm{F}_{ov} \\ \bm{F}_{vo} & \bm{F}_{vv}
% % \end{pmatrix}_{P,P} \begin{pmatrix}
% % \bm{r}_i \\ \bm{r}_a
% % \end{pmatrix}_{P,A} + \bm{1}_{P,P}\begin{pmatrix}
% %     \bm{V}^{2 \mathrm{h1p}}_{p,i[jb]} & \bm{V}^{2 \mathrm{h1p}}_{p,i[{jb}]} \\
% % \end{pmatrix}_{P,2A} \begin{pmatrix}
% %     \bm{r}_{i[jb]} \\
% % \bm{\bar{r}}_{i[\bar{jb}]}
% % \end{pmatrix}_{2A,A} \\[6pt]
% %  \bm{N}_{2A,2A}\begin{pmatrix}
% %     \left(\bm{V}^{2 \mathrm{h1p}}_{p,i[jb]}\right)^\dag \\ \left(\bm{V}^{2 \mathrm{h1p}}_{p,i[{jb}]}\right)^\dag \\
% % \end{pmatrix}_{2A,P} \begin{pmatrix}
% % \bm{r}_i \\ \bm{r}_a
% % \end{pmatrix}_{P,A} +
% % \bm{N}_{2A,2A}\left(\left(\varepsilon^{1h}\right) \oplus_{\text{direct}} -\bm{\tilde{M}}_{jb,jb;lc,lc}\right)_{2A,2A} \begin{pmatrix}
% %     \bm{r}_{i[jb]} \\
% % \bm{\bar{r}}_{i[\bar{jb}]}
% % \end{pmatrix}_{2A,A}
% % \end{pmatrix} \\
% % &= \begin{pmatrix}
% %  \begin{pmatrix}
% % \bm{F}_{oo}\bm{r}_i + \bm{F}_{ov}\bm{r}_a \\ \bm{F}_{vo}\bm{r}_i + \bm{F}_{vv}\bm{r}_a
% % \end{pmatrix}_{P,A} + \begin{pmatrix}
% %     \bm{V}^{2 \mathrm{h1p}}_{k,i[jb]}\bm{r}_{i[jb]} + \bm{V}^{2 \mathrm{h1p}}_{k,i[{jb}]}\bm{\bar{r}}_{i[\bar{jb}]}\\ \bm{V}^{2 \mathrm{h1p}}_{c,i[jb]}\bm{r}_{i[jb]} + \bm{V}^{2 \mathrm{h1p}}_{c,i[{jb}]}\bm{\bar{r}}_{i[{jb}]}
% % \end{pmatrix}_{P,A} \\[6pt]
% %  \bm{N}_{2A,2A}\begin{pmatrix}
% %     \bm{V}^{2 \mathrm{h1p}}_{k,i[jb]}\bm{r}_k+ \bm{V}^{2 \mathrm{h1p}}_{a,i[jb]}\bm{r}_a \\
% % \bm{V}^{2 \mathrm{h1p}}_{k,i[jb]}\bm{r}_k+ \bm{V}^{2 \mathrm{h1p}}_{a,i[jb]}\bm{r}_a \\
% % \end{pmatrix}_{2A,A} +
% % \left(\left(\varepsilon^{1h}\right) \oplus_{\text{direct}} -\begin{pmatrix}
% % \bm{\tilde{M}}^{xx}_{jb,lc} & \bm{\tilde{M}}^{xd}_{jb,lc} \\
% % -\bm{\tilde{M}}^{dx}_{jb,lc} & -\bm{\tilde{M}}^{dd}_{jb,lc}
% % \end{pmatrix}\right)_{2A,2A} \begin{pmatrix}
% %     \bm{r}_{i[jb]} \\
% % \bm{\bar{r}}_{i[\bar{jb}]}
% % \end{pmatrix}_{2A,A}
% % \end{pmatrix} \\
% % % &= \begin{pmatrix}
% % %  \begin{pmatrix}
% % % \sum_j f_{ij}r_j+ \sum_b f_{ib}r_b + \sum_{jlc} \langle i c | jl \rangle r_{j[lc]} + \sum_{jlc} \langle i c | jl \rangle \bar{r}_{j[lc]} \\ \sum_j f_{aj}r_j+ \sum_b f_{ab}r_b + \sum_{jlc} \langle a c | jl \rangle r_{j[lc]} + \sum_{jlc} \langle a c | jl \rangle \bar{r}_{j[lc]}
% % % \end{pmatrix}_{P,A} \\[6pt]
% % % \begin{pmatrix}
% % %     \sum_k \langle ka | ij  \rangle {r}_k + \sum_b \langle ba | ij  \rangle {r}_b \\
% % % -\sum_k \langle ka | ij  \rangle {r}_k - \sum_b \langle ba | ij  \rangle {r}_b
% % % \end{pmatrix}_{2A,A} +
% % % \bm{N}_{AA}\left(\left(\varepsilon^{1h}\right) \oplus_{\text{kron}} -\begin{pmatrix}
% % % \bm{\tilde{M}}^{xx}_{jb,lc} & \bm{\tilde{M}}^{xd}_{jb,lc} \\
% % % \bm{\tilde{M}}^{dx}_{jb,lc} & \bm{\tilde{M}}^{dd}_{jb,lc}
% % % \end{pmatrix}\right)_{AA} \begin{pmatrix}
% % %     \bm{r}_{i[jb]} \\
% % % \bm{\bar{r}}_{i[\bar{jb}]}
% % % \end{pmatrix}_{2A,A}
% % % \end{pmatrix} \\
% % &= \begin{pmatrix}
% %  \begin{pmatrix}
% % \sum_j f_{ij}r_j+ \sum_b f_{ib}r_b + \sum_{jlc} \langle i c | jl \rangle r_{j[lc]} + \sum_{jlc} \langle i c | jl \rangle \bar{r}_{j[lc]} \\ \sum_j f_{aj}r_j+ \sum_b f_{ab}r_b + \sum_{jlc} \langle a c | jl \rangle r_{j[lc]} + \sum_{jlc} \langle a c | jl \rangle \bar{r}_{j[lc]}
% % \end{pmatrix}_{P,A} \\[6pt]
% % \begin{pmatrix}
% %     ?\\
% %  ?
% % \end{pmatrix}_{2A,A} +
% %  \begin{pmatrix}
% % ? \\
% % ?
% % \end{pmatrix}_{2A,A}
% % \end{pmatrix} \\
% % \end{align}
% % % \subsubsection{Proving equivalence to Booth's ED}
% % % Again, we will just work with a single channel, the lesser one. In the dRPA case, Booth's formulation for the upfolded Hamiltonian is
% % % \begin{equation}
% % %     \bm{H} = \begin{pmatrix} \bm{F} & \bm{W} \\ \bm{W}^{\dagger} & \bm{d} \end{pmatrix}
% % % \label{eq:booth_upfolded_hamiltonian}
% % % \end{equation}
% % % where we have the definitions
% % % \begin{equation}
% % %     W_{pkv} = \sum_{ia} (pk|ia) \left( X_{ia}^{v} + Y_{ia}^{v} \right) \quad \text{and} \quad d_{kv,lv'} = \left(\epsilon_k - \Omega_v\right) \delta_{k,l} \delta_{v,v'}
% % % \end{equation}
% % % Now Tim's version of the Hamiltonian for the lesser channel is given by
% % % \begin{equation}
% % %     \bm{H} = \begin{pmatrix} \bm{F} & \bm{V}^{2 \mathrm{h1p}} & \bm{V}^{2 \mathrm{h1p}} \\ \left(\bm{V}^{2 \mathrm{~h} 1 \mathrm{p}}\right)^{\dagger} & &\\ \left(\bm{V}^{2 \mathrm{~h} 1 \mathrm{p}}\right)^{\dagger} && \bm{C}^{2 \mathrm{hlp}} \end{pmatrix}
% % % \label{eq:tim_upfolded_hamiltonian}
% % % \end{equation}
% % % where the formal defintions are $\bm{C}^{2 \mathrm{hlp}} = \epsilon^{1 \mathrm{~h}} \oplus (-\tilde{\bm{M}})$ and $\bm{V}^{2 \mathrm{~h} 1 \mathrm{p}} = \langle p c | k l \rangle $. Note that now the auxiliary block contain two seperate paricle-hole excitations, which is why the coupling blocks are duplicated. Let us define a unitary rotation $\bm{U} = \bm{1} \otimes \bm{Z}$. Application of this unitary to the Hamiltonian will not change the spectrum and actually transforms the problem into
% % % \begin{equation}
% % %     \bm{H}' = \bm{U}^\dag \bm{H} \bm{U} = \begin{pmatrix} \bm{1} & \bm{0}\\ \bm{0}&\bm{Z }^\dag \end{pmatrix} \begin{pmatrix} \bm{F} & \bm{V}^{2 \mathrm{h1p}} & \bm{V}^{2 \mathrm{h1p}} \\ \left(\bm{V}^{2 \mathrm{~h} 1 \mathrm{p}}\right)^{\dagger} & &\\ \left(\bm{V}^{2 \mathrm{~h} 1 \mathrm{p}}\right)^{\dagger} && \bm{C}^{2 \mathrm{hlp}} \end{pmatrix} \begin{pmatrix} \bm{1} & \bm{0} \\ \bm{0}&\bm{Z } \end{pmatrix}
% % % = \begin{pmatrix} \bm{F} & \bm{V}^{2 \mathrm{h1p}}\bm{Z} & \bm{V}^{2 \mathrm{h1p}}\bm{Z} \\ \bm{Z}^\dag \left(\bm{V}^{2 \mathrm{~h} 1 \mathrm{p}}\right)^{\dagger} & &\\ \bm{Z}^\dag \left(\bm{V}^{2 \mathrm{~h} 1 \mathrm{p}}\right)^{\dagger} && \bm{Z }^\dagger \bm{C}^{2 \mathrm{hlp}} \bm{Z } \end{pmatrix}.
% % % \label{eq:tim_upfolded_hamiltonian}
% % \end{equation}