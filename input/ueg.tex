Throughout, I am working in the retarded time formalism. We are working with a plane wave basis, where each state has its own wavevector.
\section{MF}
Even before doing HF, I generate my basis of plane waves, and their wave vectors in 3D are just a list of integers, three entries long. From HF, I get a set of MO coefficients $C_{\bm{k}_\mu \bm{k}_p}$ and orbital energies $\epsilon_{\bm{k}_p}$.
\section{RPA}
For the UEG, I am first looping over all momentum transfer vectors $\bm{k}_q$. To learn which excitations are valid for the given $\bm{k}_q$, we first loop over occupied states $\bm{k}_i$. Virtual states $\bm{k}_a$ are already determined by that point; we compute $\bm{k}_a \equiv \bm{k}_i + \bm{k}_q$, and based of of whether it is a valid virtual index, we either add it to our list or continue. The criterion for validity is that the wave vector must be outside the Fermi sphere, but still within our plane wave basis, i.e. within the KE cutoff. We do the same for the second pair of indices $\bm{k}_j, \bm{k}_b$. Then, we can form the Casida eigenproblem for each $\bm{k}_q$.
\begin{equation}
    \begin{pmatrix}
        \bm{A}(\bm{k}_q) & \bm{B}(\bm{k}_q) \\
        -\bm{B}(\bm{k}_q) & -\bm{A}(\bm{k}_q)
    \end{pmatrix}
    \begin{pmatrix}
        \bm{X}(\bm{k}_q) & \bm{Y}(\bm{k}_q) \\
        \bm{Y}(\bm{k}_q) & \bm{X}(\bm{k}_q)
    \end{pmatrix}
    \begin{pmatrix}
        \bm{\Omega}(\bm{k}_q) & \bm{0} \\
        \bm{0} & -\bm{\Omega}(\bm{k}_q)
    \end{pmatrix}
    = \begin{pmatrix}
        \bm{X}(\bm{k}_q) & \bm{Y}(\bm{k}_q) \\
        \bm{Y}(\bm{k}_q) & \bm{X}(\bm{k}_q)
    \end{pmatrix}
    \label{eqn:casida_eq}
\end{equation}
The $\bm{A}(\bm{k}_q)$ and $\bm{B}(\bm{k}_q)$ matrices are defined as
\begin{align}
    A_{\bm{k}_i \bm{k}_a,\, \bm{k}_b \bm{k}_j }(\bm{k}_q) &= \left(\epsilon_{\bm{k}_a}-\epsilon_{\bm{k}_i}\right) \delta_{\bm{k}_i \bm{k}_j} \delta_{\bm{k}_a \bm{k}_b} + \frac{4\pi}{|\bm{k}_q|^2} \\
    B_{\bm{k}_i \bm{k}_a,\, \bm{k}_j \bm{k}_b}(\bm{k}_q) &= \frac{4\pi}{|\bm{k}_q|^2}
\end{align}
After solving the Casida eigenvalue problem, I perform the bioorthogonalization procedure, which ensures that
\begin{equation}
    \left(\bm{X}(\bm{k}_q) - \bm{Y}(\bm{k}_q)\right)^\dagger \left(\bm{X}(\bm{k}_q) + \bm{Y}(\bm{k}_q)\right) = \bm{I}
    % \left(\bm{X} - \bm{Y}\right)^\dagger \left(\bm{X} + \bm{Y}\right) = \bm{I}
\end{equation}
Now, I have a list of excitation energies $\Omega$ and corresponding right eigenvectors $\bm{X}, \bm{Y}$ for each momentum transfer $\bm{k}_q$. But for the self energy, we actually need rather the transition densities $\rho(\bm{k}_q)$, which are lists (with length $\mu$) of scalars, defined as
\begin{equation}
    \rho_{\nu}(\bm{k}_q) = \sum_{\bm{k}_i \bm{k}_a} \left(X_{\bm{k}_i \bm{k}_a}^\nu(\bm{k}_q) + Y_{\bm{k}_i \bm{k}_a}^\nu(\bm{k}_q)\right).
\end{equation}
So to get them, I just need to contract over the indices $\bm{k}_i, \bm{k}_a$.
\section{GW}

% Now, I can determine the transition densities $w_{\bm{k}_p \bm{k}_r}^\nu(\bm{k}_q)$. In the Ab-Initio case, these are given by
% \begin{equation}
%     w_{\bm{k}_p \bm{k}_r}^\nu = \sum_{\bm{k}_i \bm{k}_a} (X_{\bm{k}_i \bm{k}_a}^\nu + Y_{\bm{k}_i \bm{k}_a}^\nu) \langle \bm{k}_p \bm{k}_i || \bm{k}_r \bm{k}_a \rangle
% \end{equation}
% However, in the uniform electron gas (UEG) with a plane-wave basis, we have
% \begin{align}
%     w_{\bm{k}_p \bm{k}_r}^\nu(\bm{k}_q) &= 
%     \sum_{\bm{k}_i,\bm{k}_a} (X_{\bm{k}_i \bm{k}_a}^\nu(\bm{k}_q) + Y_{\bm{k}_i \bm{k}_a}^\nu(\bm{k}_q)) 
%     \frac{4\pi}{|\bm{k}_q|^2}
%     \left[\, \delta_{\bm{k}_p - \bm{k}_q, \bm{k}_r }
%     \, \delta_{\bm{k}_i - \bm{k}_q,\, \bm{k}_a}\right] \\
%  &= \frac{4\pi}{ |\bm{k}_q|^2} \delta_{\bm{k}_p - \bm{k}_q, \bm{k}_r } \sum_{\bm{k}_i} 
%     \bigl(X_{\bm{k}_i,\,\bm{k}_a}^\nu(\bm{k}_q) + Y_{\bm{k}_i,\,\bm{k}_a}^\nu(\bm{k}_q)\bigr)
% \end{align}
% % The direct term is always going to be 0, because we cannot have 
% The second Kronecker delta in the first line is redundant because we have already enforced that condition (I defined earlier that $\bm{k}_i - \bm{k}_q = \bm{k}_a$), but now we can see that the external indices $(\bm{k}_p,\bm{k}_r)$ must also differ by the same $\bm{k}_q$.

\subsection{Correlation self energy}
Because I am only interested in computing the diagonal spectral function for a certain index $\bm{k}_p$, I only need to compute the diagonal correlation self energy $\Sigma^{\text{corr}}(\bm{k}_p,\omega)$, for the given frequency $\omega$, which is given by
\begin{equation}
    \Sigma^{\text{corr}}(\bm{k}_p,\omega) = \sum_{\bm{k}_q}\sum_{\nu }^{\text{RPA}} \frac{4\pi}{|\bm{k}_q|^2} |\rho_\nu(\bm{k}_q)|^2 \left[ \frac{f_{\bm{k}_p - \bm{k}_q}}{\omega + \Omega_\nu(\bm{k}_q) - \epsilon_{\bm{k}_p - \bm{k}_q} + i\eta} + \frac{1 - f_{\bm{k}_p - \bm{k}_q}}{\omega - \Omega_\nu(\bm{k}_q) - \epsilon_{\bm{k}_p - \bm{k}_q} + i\eta} \right]
\end{equation}
where $f_{\bm{k}}$ is the Fermi occupation (1 for occupied, 0 for virtual). The terms correspond to hole and particle contributions, respectively. Note that all of the $\bm{k}_q$ may not be taken into account in this summation. To generate the $\bm{k}_q$ list, our criterion was that it needs to connect an occupied state $\bm{k}_i$ to a virtual state $\bm{k}_a$. Suppose $\bm{k}_i = [-1,-1,-1]$ and $\bm{k}_a = [0,0,2]$, then $\bm{k}_q = \bm{k}_i - \bm{k}_a = [-1,-1,-3]$. But if we now consider the state $\bm{k}_p = [0,0,1]$, then $\bm{k}_p -\bm{k}_q = [1,1,4]$, which may be beyond the KE cutoff, and thus not in our basis. So we must skip this $\bm{k}_q$ in the summation for the correlation self energy.
\subsection{Spectral function}
Finally, the spectral function is given by
\begin{equation}
    A(\bm{k}_p,\omega) = \frac{1}{\pi} \frac{|\text{Im}\,\Sigma(\bm{k}_p,\omega)|}{\left(\omega - \epsilon_{\bm{k}_p} - \text{Re}\,\Sigma(\bm{k}_p,\omega)\right)^2 + \left(\text{Im}\,\Sigma(\bm{k}_p,\omega)\right)^2}
\end{equation}
\section{GW+C}
\subsection{Cumulant}
 By relating the Dyson equation to the Taylor series expansion of the exponential (both to first order), we can write:
\begin{equation}
    \bm{G}^0(t) \bm{C}(t) = \iint \dd t_1 \dd t_2 \bm{G}^0(t-t_1) \bm{\Sigma}^c(t_1 - t_2) \bm{G}^0(t_2)
\end{equation}
We know that all of the operators are diagonal in momentum space for the UEG, so insertion of the resolution of the identity just gives us:
\begin{align}
\bm{G}^{0}(\bm{p}, t) \bm{C}(\bm{p}, t) &= {\iint \dd t_1 \dd t_2 \bm{G}^{0}(\bm{p}, t-t_1) \bm{\Sigma}^c(\bm{p}, t_1 - t_2) \bm{G}^{0}(\bm{p}, t_2)} \\
&= \int \frac{\dd \omega}{2\pi} e^{-i\omega t} \bm{G}^{0}(\bm{p},\omega)\bm{\Sigma}^c(\bm{p},\omega)\bm{G}^{0}(\bm{p},\omega) \\
\implies \bm{C}(\bm{p}, t) &= i \int \frac{\dd \omega}{2\pi} \frac{ \bm{\Sigma}^c(\bm{p},\omega+\epsilon_{\bm{p}}^{HF})}{(\omega + i \eta)^2} e^{-i \omega t} \\
&= i \int \frac{d\omega}{2\pi} \frac{1}{(\omega + i \eta)^2} e^{-i \omega t} \Biggl\{\sum_{\bm{q}}\sum_{\nu }^{\text{RPA}} \underbrace{\frac{4\pi}{|\bm{q}|^2} |\rho_\nu(\bm{q})|^2 }_{M_{\bm{q}\nu}^2}\left[ \frac{f_{\bm{p} - \bm{q}}}{\omega \underbrace{+ \Omega_\nu(\bm{q}) + (\epsilon_{\bm{p}} - \epsilon_{\bm{p} - \bm{q}}) + i\eta}_{-\Delta ^{\text{occ}}_{\bm{q}\nu}}}\right. \\
& \left. + \frac{1 - f_{\bm{p} - \bm{q}}}{\omega \underbrace{- \Omega_\nu(\bm{q})+ (\epsilon_{\bm{p}} - \epsilon_{\bm{p} - \bm{q}} )+ i\eta}_{-\Delta^{\text{virt}}_{\bm{q}\nu}}} \right]\Biggr\} \notag \\
&= i \sum_{\bm{q}\nu} M_{\bm{q}\nu}^2 \int \frac{d\omega}{2\pi} e^{-i \omega t} \left[\frac{f_{\bm{p} - \bm{q}}}{(\omega + i \eta)^2(\omega - \Delta^{\text{occ}}_{\bm{q}\nu})} + \frac{1 - f_{\bm{p} - \bm{q}}}{(\omega + i \eta)^2(\omega - \Delta^{\text{virt}}_{\bm{q}\nu})}\right] \\
&= \sum_{\bm{q}\nu} \zeta_{\bm{p}\bm{q}\nu}^{\text{occ}} \left[e^{-i \Delta^{\text{occ}}_{\bm{q}\nu} t} - 1 + i \Delta^{\text{occ}}_{\bm{q}\nu} t\right] + \sum_{\bm{q}\nu} \zeta_{\bm{p}\bm{q}\nu}^{\text{virt}} \left[e^{-i \Delta^{\text{virt}}_{\bm{q}\nu} t} - 1 + i \Delta^{\text{virt}}_{\bm{q}\nu} t\right]
\end{align}
For the final expression, we have defined $\zeta_{\bm{p}\bm{q}\nu}^{\text{occ}} = \left(\frac{f_{\bm{p}-\bm{q}} M_{\bm{q}\nu}}{\Delta^{\text{occ}}_{\bm{q}\nu}}\right)^2$ and $\zeta_{\bm{p}\bm{q}\nu}^{\text{virt}} = \left(\frac{(1-f_{\bm{p}-\bm{q}}) M_{\bm{q}\nu}}{\Delta^{\text{virt}}_{\bm{q}\nu}}\right)^2$. This allows us to arrive at the something similar to the Landau form of the cumulant.
%  For the final expression, we have defined $\zeta_{pi\nu} = \left(\frac{M_{pi\nu}}{\Delta_{pi\nu}}\right)^2$ and $\zeta_{pa\nu} = \left(\frac{M_{pa\nu}}{\Delta_{pa\nu}}\right)^2$. This allows us to arrive at the something similar to the Landau form of the cumulant.
\begin{tcolorbox}
A few notes on how to evaluate the contour integral: there is a double pole at $\omega_1 = -i\eta$ and a simple pole at $\omega_2 = -\Delta - i\eta$. Closing the contour in the lower half-plane because $\operatorname{Im}\left(\omega_1\right), \operatorname{Im}\left(\omega_2\right)<0$, and applying Cauchy's residue theorem, leads to
\begin{align}
    \int \frac{\mathrm{d} \omega}{2 \pi} e^{-\mathrm{i} \omega t} \frac{1}{\left(\omega-\omega_1\right)^2} \frac{1}{\omega-\omega_2} & =(-\mathrm{i})\left\{\left[\partial_\omega\left(\frac{e^{-\mathrm{i} \omega t}}{\omega-\omega_2}\right)\right]_{\omega=\omega_1}+\left[\frac{e^{-\mathrm{i} \omega t}}{\left(\omega-\omega_1\right)^2}\right]_{\omega=\omega_2}\right\} \\
    & =\frac{(-\mathrm{i})}{\left(\omega_1-\omega_2\right)^2}\left\{\left[(-\mathrm{i} t)\left(\omega_1-\omega_2\right)-1\right] e^{-\mathrm{i} \omega_1 t}+e^{-\mathrm{i} \omega_2 t}\right\}\\
\implies \int \frac{\mathrm{d} \omega}{2 \pi} e^{-i \omega t} \frac{1}{[\omega-(0-\mathrm{i} \eta)]^2} \frac{f}{\omega-\Delta}&=\frac{-\mathrm{i}f}{\Delta^2}\left(e^{-\mathrm{i} \Delta t}+\mathrm{i} \Delta t-1\right)
\end{align}
\end{tcolorbox}
\subsection{Green's function}
Now, we plug in our derived expression for $C(\bm{p}, t)$ into the cumulant ansatz for the retarded Green's function:
\begin{align}
G^{GW+C}(\bm{p}, t) & = G^{HF}(\bm{p}, t) e^{C(\bm{p}, t)} \\
& = -i \Theta(t) e^{-i \epsilon_{\bm{p}}^{HF} t + C(\bm{p}, t)} \\
& = -i \Theta(t) e^{-i \epsilon_{\bm{p}}^{HF} t + \sum_{\bm{q}\nu} \zeta_{\bm{p}\bm{q}\nu}^{\text{occ}}\left(e^{-i\Delta^{\text{occ}}_{\bm{q}\nu} t} + i\Delta^{\text{occ}}_{\bm{q}\nu} t - 1\right) + \sum_{\bm{q}\nu} \zeta_{\bm{p}\bm{q}\nu}^{\text{virt}} \left(e^{-i\Delta^{\text{virt}}_{\bm{q}\nu} t} + i\Delta^{\text{virt}}_{\bm{q}\nu} t - 1\right)}\\
&= -i \Theta (t) Z_{\bm{p}}^{QP} e^{-i \epsilon_{\bm{p}}^{QP} t} e^{\sum_{\bm{q}\nu} \zeta_{\bm{p}\bm{q}\nu}^{\text{occ}} e^{-i\Delta^{\text{occ}}_{\bm{q}\nu} t} + \sum_{\bm{q}\nu} \zeta_{\bm{p}\bm{q}\nu}^{\text{virt}} e^{-i\Delta^{\text{virt}}_{\bm{q}\nu} t}} \\
\end{align}
where we have the weight of the quasiparticle peak $Z_{\bm{p}}^{QP} = \exp\left(-\sum_{\bm{q}\nu} \zeta_{\bm{p}\bm{q}\nu}^{\text{occ}} - \sum_{\bm{q}\nu} \zeta_{\bm{p}\bm{q}\nu}^{\text{virt}}\right)$ and the quasiparticle energy $\epsilon_{\bm{p}}^{QP} = \epsilon_{\bm{p}}^{HF} - \left(\sum_{\bm{q}\nu} \zeta_{\bm{p}\bm{q}\nu}^{\text{occ}}\Delta^{\text{occ}}_{\bm{q}\nu} + \sum_{\bm{q}\nu} \zeta_{\bm{p}\bm{q}\nu}^{\text{virt}}\Delta^{\text{virt}}_{\bm{q}\nu}\right)$.
\begin{tcolorbox}
We pause to make some important connections. Notice
\begin{align}
    Z_{\bm{p}}^{QP} & = \exp\left(-\sum_{\bm{q}\nu} \zeta_{\bm{p}\bm{q}\nu}^{\text{occ}} - \sum_{\bm{q}\nu} \zeta_{\bm{p}\bm{q}\nu}^{\text{virt}}\right) \\
    & = \exp\left(-\sum_{\bm{q}\nu} \left(\frac{f_{\bm{p}-\bm{q}} M_{\bm{q}\nu}}{\Delta^{\text{occ}}_{\bm{q}\nu}}\right)^2 - \sum_{\bm{q}\nu} \left(\frac{(1-f_{\bm{p}-\bm{q}}) M_{\bm{q}\nu}}{\Delta^{\text{virt}}_{\bm{q}\nu}}\right)^2\right) \\
    & = \exp\left(\left[\frac{\partial \Sigma^c(\bm{p},\omega)}{\partial \omega}\right]_{\omega = \epsilon_{\bm{p}}^{HF}}\right)
\end{align}
and
\begin{align}
    \epsilon_{\bm{p}}^{QP} & = \epsilon_{\bm{p}}^{HF} - \left(\sum_{\bm{q}\nu} \zeta_{\bm{p}\bm{q}\nu}^{\text{occ}}\Delta^{\text{occ}}_{\bm{q}\nu} + \sum_{\bm{q}\nu} \zeta_{\bm{p}\bm{q}\nu}^{\text{virt}}\Delta^{\text{virt}}_{\bm{q}\nu}\right) \\
    & = \epsilon_{\bm{p}}^{HF} - \left(\sum_{\bm{q}\nu} \frac{f_{\bm{p}-\bm{q}} M_{\bm{q}\nu}^2}{\Delta^{\text{occ}}_{\bm{q}\nu}} + \sum_{\bm{q}\nu} \frac{(1-f_{\bm{p}-\bm{q}}) M_{\bm{q}\nu}^2}{\Delta^{\text{virt}}_{\bm{q}\nu}}\right) \\
    & = \epsilon_{\bm{p}}^{HF} + \Sigma^c(\bm{p},\epsilon_{\bm{p}}^{HF})
\end{align}
where we have used the fact that $\Sigma^c(\bm{p},\omega) = \sum_{\bm{q}\nu} \frac{f_{\bm{p}-\bm{q}} M_{\bm{q}\nu}^2}{\omega + \Omega_\nu(\bm{q}) - \epsilon_{\bm{p}-\bm{q}}} + \sum_{\bm{q}\nu} \frac{(1-f_{\bm{p}-\bm{q}}) M_{\bm{q}\nu}^2}{\omega - \Omega_\nu(\bm{q}) - \epsilon_{\bm{p}-\bm{q}}} \implies \left[\frac{\partial \Sigma^c(\bm{p},\omega)}{\partial \omega}\right]_{\omega = \epsilon_{\bm{p}}^{HF}} = -\sum_{\bm{q}\nu} \left(\frac{f_{\bm{p}-\bm{q}} M_{\bm{q}\nu}}{\Delta^{\text{occ}}_{\bm{q}\nu}}\right)^2 - \sum_{\bm{q}\nu} \left(\frac{(1-f_{\bm{p}-\bm{q}}) M_{\bm{q}\nu}}{\Delta^{\text{virt}}_{\bm{q}\nu}}\right)^2$.
\end{tcolorbox}
Next, we want to perform a Fourier transform.
\begin{align}
G^{GW+C}(\bm{p}, \omega) & = \int_{-\infty}^{\infty} dt\, e^{i \omega t} G^{GW+C}(\bm{p}, t) \\
& = -i Z_{\bm{p}}^{QP} \int_0^{\infty} dt\, e^{i(\omega - \epsilon_{\bm{p}}^{QP}) t} e^{\sum_{\bm{q}\nu} \zeta_{\bm{p}\bm{q}\nu}^{\text{occ}} e^{-i\Delta^{\text{occ}}_{\bm{q}\nu} t} + \sum_{\bm{q}\nu} \zeta_{\bm{p}\bm{q}\nu}^{\text{virt}} e^{-i\Delta^{\text{virt}}_{\bm{q}\nu} t}} \\
& = -i Z_{\bm{p}}^{QP} \int_0^{\infty} dt\, e^{i(\omega - \epsilon_{\bm{p}}^{QP}) t} \left(1 + \sum_{\bm{q}\nu} \zeta_{\bm{p}\bm{q}\nu}^{\text{occ}} e^{-i\Delta^{\text{occ}}_{\bm{q}\nu} t} + \sum_{\bm{q}\nu} \zeta_{\bm{p}\bm{q}\nu}^{\text{virt}} e^{-i\Delta^{\text{virt}}_{\bm{q}\nu} t} + \ldots \right) \\
&= -i Z_{\bm{p}}^{QP} \int_0^{\infty} dt\, e^{[-\eta + i(\omega - \epsilon_{\bm{p}}^{QP}) t]}\label{Gpp3} \\
&\quad - i Z_{\bm{p}}^{QP} \sum_{\bm{q}\nu} \zeta_{\bm{p}\bm{q}\nu}^{\text{occ}} \int_0^{\infty} dt\, e^{[-\eta + i(\omega - \epsilon_{\bm{p}}^{QP} - \Delta^{\text{occ}}_{\bm{q}\nu}) t]} \notag \\
&\quad - i Z_{\bm{p}}^{QP} \sum_{\bm{q}\nu} \zeta_{\bm{p}\bm{q}\nu}^{\text{virt}} \int_0^{\infty} dt\, e^{[-\eta + i(\omega - \epsilon_{\bm{p}}^{QP} - \Delta^{\text{virt}}_{\bm{q}\nu}) t]} + \ldots \notag \\
&= \frac{Z_{\bm{p}}^{QP}}{\omega - \epsilon_{\bm{p}}^{QP} + i\eta} + \sum_{\bm{q}\nu} \frac{Z_{\bm{p}}^{QP} \zeta_{\bm{p}\bm{q}\nu}^{\text{occ}}}{\omega - \epsilon_{\bm{p}}^{QP} - \Delta^{\text{occ}}_{\bm{q}\nu} + i\eta} + \sum_{\bm{q}\nu} \frac{Z_{\bm{p}}^{QP} \zeta_{\bm{p}\bm{q}\nu}^{\text{virt}}}{\omega - \epsilon_{\bm{p}}^{QP} - \Delta^{\text{virt}}_{\bm{q}\nu} + i\eta} + \ldots\\
& = \frac{Z_{\bm{p}}^{QP}}{\omega - \epsilon_{\bm{p}}^{QP} + i\eta} + \sum_{\bm{q}\nu} \frac{Z_{\bm{p}\bm{q}\nu}^{\text{occ-sat}}}{\omega - \epsilon_{\bm{p}\bm{q}\nu}^{\text{occ-sat}} + i\eta} + \sum_{\bm{q}\nu} \frac{Z_{\bm{p}\bm{q}\nu}^{\text{virt-sat}}}{\omega - \epsilon_{\bm{p}\bm{q}\nu}^{\text{virt-sat}} + i\eta} + \ldots
\end{align}
where we define the satellite energies $\epsilon_{\bm{p}\bm{q}\nu}^{\text{occ-sat}} = \epsilon_{\bm{p}}^{QP} + \Delta^{\text{occ}}_{\bm{q}\nu}$ and $\epsilon_{\bm{p}\bm{q}\nu}^{\text{virt-sat}} = \epsilon_{\bm{p}}^{QP} + \Delta^{\text{virt}}_{\bm{q}\nu}$, as well as the satellite weights $Z_{\bm{p}\bm{q}\nu}^{\text{occ-sat}} = Z_{\bm{p}}^{QP} \zeta_{\bm{p}\bm{q}\nu}^{\text{occ}}$ and $Z_{\bm{p}\bm{q}\nu}^{\text{virt-sat}} = Z_{\bm{p}}^{QP} \zeta_{\bm{p}\bm{q}\nu}^{\text{virt}}$.
\subsection{Spectral function}
The spectral function is obtained as (the virtual satellites spectral function, whose derivation will mirror that of the occupied satellites, are omitted for the sake of brevity)
\begin{align}
A^{GW+C}(\bm{p}, \omega) & = -\frac{1}{\pi} \operatorname{Im} G^{GW+C}(\bm{p}, \omega) \\
& = -\frac{1}{\pi} \operatorname{Im}\left[\frac{Z_{\bm{p}}^{QP}}{\omega - \epsilon_{\bm{p}}^{QP} + i\eta} + \sum_{\bm{q}\nu} \frac{Z_{\bm{p}\bm{q}\nu}^{\text{occ-sat}}}{\omega - \epsilon_{\bm{p}\bm{q}\nu}^{\text{occ-sat}} + i\eta} + \ldots \right] \\
& = -\frac{1}{\pi} \operatorname{Im}\left[\frac{\operatorname{Re} Z_{\bm{p}}^{QP} + i \operatorname{Im} Z_{\bm{p}}^{QP}}{\omega - \operatorname{Re} \epsilon_{\bm{p}}^{QP} + i\left(\eta - \operatorname{Im} \epsilon_{\bm{p}}^{QP}\right)} + \sum_{\bm{q}\nu} \frac{\operatorname{Re} Z_{\bm{p}\bm{q}\nu}^{\text{occ-sat}} + i \operatorname{Im} Z_{\bm{p}\bm{q}\nu}^{\text{occ-sat}}}{\omega - \operatorname{Re} \epsilon_{\bm{p}\bm{q}\nu}^{\text{occ-sat}} + i\left(\eta - \operatorname{Im} \epsilon_{\bm{p}\bm{q}\nu}^{\text{occ-sat}}\right)} + \ldots \right] \\
& = -\frac{1}{\pi} \operatorname{Im}\left[\frac{\left(\operatorname{Re} Z_{\bm{p}}^{QP} + i \operatorname{Im} Z_{\bm{p}}^{QP}\right)\left(\omega - \operatorname{Re} \epsilon_{\bm{p}}^{QP} - i\left(\eta - \operatorname{Im} \epsilon_{\bm{p}}^{QP}\right)\right)}{\left(\omega - \operatorname{Re} \epsilon_{\bm{p}}^{QP}\right)^2 + \left(\operatorname{Im} \epsilon_{\bm{p}}^{QP}\right)^2}\right.\\
& \left. + \sum_{\bm{q}\nu} \frac{\left(\operatorname{Re} Z_{\bm{p}\bm{q}\nu}^{\text{occ-sat}} + i \operatorname{Im} Z_{\bm{p}\bm{q}\nu}^{\text{occ-sat}}\right)\left(\omega - \operatorname{Re} \epsilon_{\bm{p}\bm{q}\nu}^{\text{occ-sat}} - i\left(\eta - \operatorname{Im} \epsilon_{\bm{p}\bm{q}\nu}^{\text{occ-sat}}\right)\right)}{\left(\omega - \operatorname{Re} \epsilon_{\bm{p}\bm{q}\nu}^{\text{occ-sat}}\right)^2 + \left(\operatorname{Im} \epsilon_{\bm{p}\bm{q}\nu}^{\text{occ-sat}}\right)^2} + \ldots \right] \\
& = -\frac{1}{\pi}\left[\frac{\left(\operatorname{Re} Z_{\bm{p}}^{QP}\right)\left(\operatorname{Im} \epsilon_{\bm{p}}^{QP}\right) + \left(\operatorname{Im} Z_{\bm{p}}^{QP}\right)\left(\omega - \operatorname{Re} \epsilon_{\bm{p}}^{QP}\right)}{\left(\omega - \operatorname{Re} \epsilon_{\bm{p}}^{QP}\right)^2 + \left(\operatorname{Im} \epsilon_{\bm{p}}^{QP}\right)^2}\right.\\
& \left. + \sum_{\bm{q}\nu} \frac{\left(\operatorname{Re} Z_{\bm{p}\bm{q}\nu}^{\text{occ-sat}}\right)\left(\operatorname{Im} \epsilon_{\bm{p}\bm{q}\nu}^{\text{occ-sat}}\right) + \left(\operatorname{Im} Z_{\bm{p}\bm{q}\nu}^{\text{occ-sat}}\right)\left(\omega - \operatorname{Re} \epsilon_{\bm{p}\bm{q}\nu}^{\text{occ-sat}}\right)}{\left(\omega - \operatorname{Re} \epsilon_{\bm{p}\bm{q}\nu}^{\text{occ-sat}}\right)^2 + \left(\operatorname{Im} \epsilon_{\bm{p}\bm{q}\nu}^{\text{occ-sat}}\right)^2} + \ldots \right]
\end{align}
