\section{Background}
\subsection{The qualitative interpretation}
The cumulant has the ability to enhance spectral features with respect to $GW$. The prototypical example that is given to motivate its utility is the analysis of the self-consistent $GW$ spectral function by von Barth and Holm for the uniform electron gas. There they show that scGW predicts an unphysical peak, known as the plasmaron, which is off from the expected plasmon frequency, as the only feature in the satellite spectrum. Cumulant plus $GW$, on the other hand, gives a more accurate answer. 
The way I think about this is as follows. $GW$ (with or without self-consistency) yields similar quasiparticle energies, while self-consistency mostly affects spectral weight distribution. However, $GW$ often produces an artificial plasmaron satellite. The cumulant expansion not only redistributes weight but also corrects the satellite structure, eliminating the spurious plasmaron and aligning the features with the physical plasmon frequency.\\\\
As we know, the ansatz for the cumulant $C(t)$ is
\begin{equation}
    G(t) = G_0(t) e^{C(t)}
\end{equation}
where we have the non-interacting and fully interacting Green's functions $G_0$ and $G$, respectively. After some derivation, we arrive at the Landau form for the cumulant
\begin{equation}
    C(t)=\int d \omega \frac{\beta(\omega)}{\omega^2}\left[e^{-i \omega t}+i \omega t-1\right]
\end{equation}
where the cumulant kernel is defined as
\begin{equation}
    \beta(\omega)=-\frac{1}{\pi} \operatorname{Im} \Sigma^{\mathrm{c}}(\omega)
\end{equation}
This form is amenable to physical interpretation if we partition into
\begin{equation}
    C(t)=-a+i \Delta t+\tilde{C}(t)
\end{equation}
where $a=\int d \omega \beta(\omega) / \omega^2$ is the net satellite strength, $\Delta=\int d \omega \beta(\omega) / \omega$ is the quasiparticle shift, or core-level "relaxation energy", and $\tilde{C}(t)$ is the remainder of the cumulant, which contains the information about the satellites.

\subsection{Alternate derivation starting from Dyson equation}
We can start from where Schwinger's fictitious potential $u$ has already been introduced:
\begin{align}
G_{u}\left(1,1^{\prime}\right)= & G^{0}\left(1,1^{\prime}\right)+G^{0}(1, \overline{2}) \left\{\left[u(\overline{2})+v_{H u}(\overline{2})\right] G_{u}\left(\overline{2}, 1^{\prime}\right)+i v_{c}(\overline{2}, \overline{3}) \frac{\delta G_{u}\left(\overline{2}, 1^{\prime}\right)}{\delta u\left(\overline{3}^{+}\right)}\right\} 
\end{align}
where $G^{0}$ is the non-interacting Green's function and $v_{c}$ is the bare Coulomb interaction. $v_{H u}$ is the Hartree potential built with the density $n_{u}(\mathbf{x})=-i G_{u}\left(\mathbf{x}, \mathbf{x}, t, t^{+}\right)$. One of the complications of the equations is the fact that the density $n_{u}$ in the Hartree potential introduces a term that is quadratic in $G_{u}$ for the first term in the above equation. To overcome this problem, we introduce the total classical potential
\begin{equation}
u_{\mathrm{cl}}(1)=u(1)+v_{H u}(1)
\end{equation}
which allows us to rewrite the equation for $G_{u}\equiv G_u^{\mathrm{cl}}$ as
\begin{align}
G_{u}\left(1,1^{\prime}\right)= & G^{0}\left(1,1^{\prime}\right)+G^{0}(1, \overline{2}) u_{\mathrm{cl}}(\overline{2}) G_{u}\left(\overline{2}, 1^{\prime}\right) +i G^{0}(1, \overline{2}) v_{c}(\overline{2}, \overline{3}) \frac{\delta G_{u}\left(\overline{2}, 1^{\prime}\right)}{\delta u\left(\overline{3}^{+}\right)} \\
= & G^{0}\left(1,1^{\prime}\right)+G^{0}(1, \overline{2}) u_{\mathrm{cl}}(\overline{2}) G_{u}\left(\overline{2}, 1^{\prime}\right) +i G^{0}(1, \overline{2}) W_{u}(\overline{2}, \overline{3}) \frac{\delta G_{u}\left(\overline{2}, 1^{\prime}\right)}{\delta u_{\mathrm{cl}}\left(\overline{3}^{+}\right)}
\end{align}
where we have defined the screened Coulomb interaction $W_{u} =\epsilon_{u}^{-1} v_{c}$ with the time-ordered inverse dielectric function $\epsilon_{u}^{-1} =\delta u_{\mathrm{cl}} / \delta u$. Note that $\epsilon_{u}^{-1}$ is not the usual linear response dielectric function, since it depends on the perturbing potential. However, since we are interested in the solution for vanishing $u$, a reasonable approximation is to evaluate the equation using $\epsilon_{u}^{-1} \approx \epsilon^{-1}$ at $u=0$. \emph{This corresponds to a linear-response approximation.} Written in a basis, the resulting equation reads
\begin{align}
G_{i j}^{u_{cl}}\left(t_{12}\right)= & G_{i j}^{0}\left(t_{12}\right)+G_{i m}^{0}\left(t_{13}\right) u_{\mathrm{cl}, m k}\left(t_{3}\right) G_{k j}^{u_{cl}}\left(t_{32}\right) +i G_{i k}^{0}\left(t_{13}\right) W_{k l m n}\left(t_{34}\right) \frac{\partial G_{n j}^{u_{cl}}\left(t_{32}\right)}{\partial u_{\mathrm{cl}, l m}\left(t_{4}\right)}
\label{eq:gw_start}
\end{align}
where $t_{12} \equiv\left(t_{1}, t_{2}\right)$ or $\left(t_{1}-t_{2}\right)$ in equilibrium, $W_{k l m n}$ is a matrix element of the screened Coulomb interaction $W$, and we have replaced functional derivatives by partial derivatives, supposing the basis to be discrete, which corresponds to calculations in practice. Repeated indices are summed over.
\subsubsection{GW approximation}
 The $GW$ approximation sets
\begin{equation}
\frac{\partial G_{n j}^{u}\left(t_{32}\right)}{\partial u_{\mathrm{cl}, l m}\left(t_{4}\right)} \approx G_{n l}\left(t_{34}\right) G_{m j}\left(t_{42}\right)
\end{equation}
At $u=0$, this yields the Dyson equation $G =G^{0}+G^{0} \Sigma^{G W} G$ with the $GW$ approximation for the self-energy,
\begin{equation}
\Sigma_{i m}^{G W}\left(t_{34}\right)=v_{H, i m}+i G_{n l}\left(t_{34}\right) W_{i l m n}\left(t_{34}\right) .
\end{equation}
But we will take a different route here, in order to obtain the cumulant expression for the Green's function.
\subsubsection{Cumulant expansion}
The basic idea is to introduce a quasi-particle Green's function $G^{Q P, u}$, defined as
\begin{align}
\left[G^{Q P, u}\right]_{i j}^{-1}&=\left[G_{0}^{-1}\right]_{i j}-\left[u_{\mathrm{cl}}\right]_{i j}-\Sigma_{i j}^{G W}\left(\frac{\varepsilon_{i}+\varepsilon_{j}}{2}\right)\\
 \implies \left[G_0\right]_{ij} &= G^{Q P, u}_{i j} - G^{Q P, u}_{i m} \left( u_{\mathrm{cl}, m k} + \Sigma_{m k}^{G W}\left(\frac{\varepsilon_{m}+\varepsilon_{k}}{2}\right) \right) \left[ G_0\right]_{kj}
\end{align}
Plugging this into \ref{eq:gw_start} (suppressing the time and orbital indices for ease of notation) gives
\begin{align}
G^{u}&= G^{Q P, u} - G^{Q P, u} \left( u_{\mathrm{cl}} +  \Sigma  \right) G_0 \notag \\
& + \left(G^{Q P, u} - G^{Q P, u} \left( u_{\mathrm{cl}} +  \Sigma \right) G_0\right) u_{\mathrm{cl}} G^{u} \notag \\
& + i \left(G^{Q P, u} - G^{Q P, u} \left( u_{\mathrm{cl}} +  \Sigma \right) G_0\right) W \frac{\partial G^{u}}{\partial u_{\mathrm{cl}}} \\
&= G^{Q P, u} + G^{Q P, u} u_{\mathrm{cl}} G^{u} + i G^{Q P, u} W \frac{\partial G^{u}}{\partial u_{\mathrm{cl}}} \notag \\
& - G^{Q P, u} \left( u_{\mathrm{cl}} +  \Sigma \right) \left[ \underbrace{G_0 + G_0 u_{\mathrm{cl}} G^{u} + i G_0 W \frac{\partial G^{u}}{\partial u_{\mathrm{cl}}}}_{G^{u}} \right] \notag \\
&= G^{Q P, u} + i G^{Q P, u} W \frac{\partial G^{u}}{\partial u_{\mathrm{cl}}} - G^{Q P, u}  \Sigma  G^{u} \\
\implies G^{u}_{ij}(t_{12}) &= G^{Q P, u}_{ij}(t_{12}) + i G^{Q P, u}_{ik}(t_{13}) W_{klmn}(t_{34}) \frac{\partial G^{u}_{mj}(t_{32})}{\partial u_{\mathrm{cl}, lm}(t_4)} - G^{Q P, u}_{ik}(t_{13}) \Sigma^{GW}_{kl}\left(\frac{\varepsilon_k + \varepsilon_l}{2}\right) G^{u}_{lj}(t_{32})
\end{align}
\textcolor{orange}{%
We have made use of the $GW$ self-energy. It is known that in the $GW$ approximation, because of a fortuitous cancellation of errors, it is most wise to use the RPA approximation for the dielectric function that goes into the screened Coloumb interaction. I wonder if this is also the smartest decision for the cumulant expansion; we could try using a more accurate approximation to the dielectric function and see what the effect is.} Now one decouples the equations by supposing that $G^{u}$ and $G^{Q P, u}$ are diagonal in the same $u$-independent basis. So we take the diagonal components of all operators in the above equation, so $G^{u}_{ij} \rightarrow G^{u}_{ii}\equiv \mathcal{G}^{u}$, $G^{Q P, u}_{ij} \rightarrow G^{Q P, u}_{ii} \equiv \mathcal{G}^{Q P, u}$, $\Sigma^{GW}_{ij}\left(\frac{\varepsilon_{i}+\varepsilon_{j}}{2}\right) \rightarrow \Sigma^{GW}_{ii}\left(\varepsilon_i\right) \equiv \tilde{\Sigma}^{GW}$, $W_{ijkl} \rightarrow W_{iiii} \equiv \mathcal{W}$, and $u_{\mathrm{cl}, ij} \rightarrow u_{\mathrm{cl}, ii} \equiv u$. This is a strong assumption, but nevertheless, it allows us to get to the conventional form for the cumulant expansion. The resulting equation is
\begin{equation}
\mathcal{G}^{u}(t_{12}) = \mathcal{G}^{Q P, u}(t_{12}) + i \mathcal{G}^{Q P, u}(t_{13}) \mathcal{W}(t_{34}) \frac{\partial \mathcal{G}^{u}(t_{32})}{\partial {u}(t_4)} - \mathcal{G}^{Q P, u}(t_{13}) \tilde{\Sigma}^{GW} \mathcal{G}^{u}(t_{32})
\end{equation}
In the paper they claim that the solution for $u \rightarrow 0$ is
\begin{align}
\mathcal{G}\left(t_{12}\right)= & \mathcal{G}_{Q P}^{0}\left(t_{12}\right) e^{i\left(t_{1}-t_{2}\right) \Sigma_{i i}^{G W}\left(\varepsilon_{i}\right)} \exp \left[-i \int_{t_{1}}^{t_{2}} d t^{\prime} \int_{t^{\prime}}^{t_{2}} d t^{\prime \prime} \mathcal{W}\left(t^{\prime}-t^{\prime \prime}\right)\right].
\end{align}
 I have not been able to derive, but my attempts are below, and we will just continue for now. The double integral can be evaluated as
\begin{align}
-i \int_{t_{1}}^{t_{2}} d t^{\prime} \int_{t^{\prime}}^{t_{2}} d t^{\prime \prime} \mathcal{W}\left(t^{\prime}-t^{\prime \prime}\right) 
\label{a}
& =-\frac{i}{2\pi} \int d\omega\, \mathcal W(\omega)
\int_{t_1}^{t_2} dt' \int_{t'}^{t_2} dt''\, e^{-i\omega (t' - t'')}\\
&= -\frac{1}{2\pi} \int d\omega\, \frac{\mathcal W(\omega)}{\omega}
\int_{t_1}^{t_2} dt' \Big( e^{i\omega (t_2-t')} - 1 \Big) 
\label{b}
\\
&= -\frac{1}{2\pi} \int d\omega\, \frac{\mathcal W(\omega)}{\omega}
\left[ \frac{e^{i\omega (t_2-t_1)} - 1}{i\omega} - (t_2-t_1) \right] 
\label{c}
\\
&=-\left(t_{1}-t_{2}\right) \frac{1}{2 \pi} \int d \omega \frac{\mathcal{W}(\omega)}{\omega} +\frac{i}{2 \pi} \int d \omega \frac{\mathcal{W}(\omega)}{\omega^{2}}\left(e^{-i \omega\left(t_{1}-t_{2}\right)}-1\right)
\label{d}
\end{align}
where in going from Eq.~(\ref{a}) to Eq.~(\ref{b}) we have used $\int_{t'}^{t_2} dt'' \, e^{-i\omega(t'-t'')}  = \frac{e^{i\omega (t_2-t')} - 1}{i\omega}$, and in going from Eq.~(\ref{b}) to Eq.~(\ref{c}) we have used $\int_{t_1}^{t_2} dt' \big(e^{i\omega(t_2-t')} - 1\big)
= \frac{e^{i\omega \Delta}-1}{i\omega} - \Delta$ with $\Delta = t_2-t_1$.
Let us first examine the term proportional to $\left(t_{1}-t_{2}\right)$ in Eq.~(\ref{d}), by comparing it to a GW quasi-particle shift. In the decoupling approximation, $\Sigma_{k k}^{G W} \approx i G_{k k} W_{k k k k}$. Evaluated at the quasi-particle energy, this yields exactly the term we are interested in. This means that $e^{-\left(t_{1}-t_{2}\right) \frac{1}{2 \pi} \int d \omega \frac{\mathcal{W}(\omega)}{\omega}}$ approximately cancels with the GW shift in $e^{i\left(t_{1}-t_{2}\right) \Sigma_{i i}^{G W}\left(\varepsilon_{i}\right)}$, and we are left with
\begin{equation*}
\mathcal{G}\left(t_{12}\right)=\mathcal{G}_{Q P}^{0}\left(t_{12}\right) \exp \left[\frac{i}{2 \pi} \int d \omega \frac{\mathcal{W}(\omega)}{\omega^{2}}\left(e^{i \omega\left(t_{1}-t_{2}\right)}-1\right)\right] \tag{17}
\end{equation*}


\begin{tcolorbox}
Using the functional derivative identity $\frac{\partial \mathcal{G}^{u}(t_{32})}{\partial u(t_4)}|_{u=0} = \mathcal{G}(t_{34}) \mathcal{G}(t_{42})$ and correspondingly setting $\mathcal{G}^{u}\rightarrow \mathcal{G}$, $\mathcal{G}^{Q P, u}\rightarrow \mathcal{G}^{Q P}$ gives
\begin{equation}
\mathcal{G}(t_{12}) = \mathcal{G}^{Q P}(t_{12}) + i \mathcal{G}^{Q P}(t_{13}) \mathcal{W}(t_{34}) \mathcal{G}(t_{34}) \mathcal{G}(t_{42}) - \mathcal{G}^{Q P}(t_{13}) \tilde{\Sigma}^{GW} \mathcal{G}(t_{32})
\end{equation}
Now define $\tilde{\mathcal{G}}(t_{12}) = e^{i (t_1 - t_2) \tilde{\Sigma}^{GW}} \mathcal{G}(t_{12}) \implies \mathcal{G}(t_{12}) = e^{-i (t_1 - t_2) \tilde{\Sigma}^{GW}} \tilde{\mathcal{G}}(t_{12})$. Plugging this in gives
\begin{align}
\tilde{\mathcal{G}}(t_{12}) &= e^{i (t_1 - t_2) \tilde{\Sigma}^{GW}} \mathcal{G}^{Q P}(t_{12}) + i e^{i (t_1 - t_2) \tilde{\Sigma}^{GW}} \mathcal{G}^{Q P}(t_{13}) \mathcal{W}(t_{34}) e^{-i (t_3 - t_4) \tilde{\Sigma}^{GW}} \tilde{\mathcal{G}}(t_{34}) e^{-i (t_4 - t_2) \tilde{\Sigma}^{GW}} \tilde{\mathcal{G}}(t_{42}) \notag \\
& - e^{i (t_1 - t_2) \tilde{\Sigma}^{GW}} \mathcal{G}^{Q P}(t_{13}) \tilde{\Sigma}^{GW} e^{-i (t_3 - t_2) \tilde{\Sigma}^{GW}} \tilde{\mathcal{G}}(t_{32}) \notag \\
\end{align}
\subsubsection{Inegration factor method}
Define $A(u)=-\mathcal{G}^{Q P, u}\left(t_{13}\right) \tilde{\Sigma}^{G W}, \quad B(u)=\mathcal{G}^{Q P, u}\left(t_{12}\right), \quad C(u)=i \mathcal{G}^{Q P, u}\left(t_{13}\right) \mathcal{W}\left(t_{34}\right) .$ Then the equation can be written as
\begin{equation}
C(u) \frac{\partial \mathcal{G}^u}{\partial u}+A(u) \mathcal{G}^u=B(u) .
\end{equation}
Divide through by $C(u)$, we get
\begin{equation}
\frac{\partial \mathcal{G}^u}{\partial u}+\frac{A(u)}{C(u)} \mathcal{G}^u=\frac{B(u)}{C(u)} .
\end{equation}
So our integrating factor is
\begin{equation}
\mu(u)=\exp \left(\int^u \frac{A(s)}{C(s)} d s\right) = \exp \left(+i \int^u d s \frac{\tilde{\Sigma}^{G W}}{\mathcal{W}}\right)
\end{equation}
and the solution is
\begin{align}
\mathcal{G}^u&=\frac{1}{\mu(u)}\left[C_0+\int^u \mu(s) \frac{B(s)}{C(s)} d s\right]\\
& = \exp \left(-i \int^u d s \frac{\tilde{\Sigma}^{G W}}{\mathcal{W}}\right)\left[C_0+\int^u d s \exp \left(+i \int^s d s' \frac{\tilde{\Sigma}^{G W}}{\mathcal{W}}\right) \frac{\mathcal{G}^{Q P, s}}{i \mathcal{G}^{Q P, s} \mathcal{W}}\right] \\
& = \exp \left(-i \int^u d s \frac{\tilde{\Sigma}^{G W}}{\mathcal{W}}\right)\left[C_0+\int^u d s \exp \left(+i\int^s d s' \frac{\tilde{\Sigma}^{G W}}{\mathcal{W}}\right) \frac{-i}{ \mathcal{W}}\right] \\
\end{align}    
\end{tcolorbox}


% Introduce shorthand:
% $$
% A(u)=-\mathcal{G}^{Q P, u}\left(t_{13}\right) \tilde{\Sigma}^{G W}, \quad B(u)=\mathcal{G}^{Q P, u}\left(t_{12}\right), \quad C(u)=i \mathcal{G}^{Q P, u}\left(t_{13}\right) \mathcal{W}\left(t_{34}\right) .
% $$

% Equation (1) is equivalent to
% $$
% C(u) \frac{\partial \mathcal{G}^u}{\partial u}+A(u) \mathcal{G}^u=B(u) .
% $$

% Divide through by $C(u)$ :
% $$
% \frac{\partial \mathcal{G}^u}{\partial u}+\frac{A(u)}{C(u)} \mathcal{G}^u=\frac{B(u)}{C(u)} .
% $$

% Now this is in the standard form $\frac{d Y}{d u}+\alpha(u) Y=\beta(u)$.
% 5. Integrating factor

% Define
% $$
% \mu(u)=\exp \left(\int^u \frac{A(s)}{C(s)} d s\right)
% $$

% Then the solution is
% $$
% g^u=\frac{1}{\mu(u)}\left[C_0+\int^u \mu(s) \frac{B(s)}{C(s)} d s\right]
% $$
% with constant $C_0$ fixed by the boundary condition.
% 6. Boundary condition and physical solution

% The physical requirement is:
% - As the Coulomb interaction $v_c \rightarrow 0$, or equivalently $W \rightarrow 0$, the solution must reduce to the non-interacting Green's function.
% - This selects the correct $C_0$.

% So you discard spurious solutions (phase transitions) and keep the branch that connects continuously to the free Green's function.
% 7. Set $u \rightarrow 0$

% Finally, evaluate (5) at $u=0$. Since $\mathcal{G}^{Q P, u}$ is linear in $u$, the coefficients $A(u), B(u), C(u)$ reduce to constants when $u=0$. The integral in the exponent then gives you exactly the exponentials in the cumulant solution:
% $$
% \mathcal{G}\left(t_{12}\right)=\mathcal{G}_{Q P}^0\left(t_{12}\right) \exp \left[i\left(t_1-t_2\right) \Sigma_{i i}^{G W}\left(\varepsilon_i\right)\right] \exp \left(-i \int_{t_1}^{t_2} d t^{\prime} \int_{t^{\prime}}^{t_2} d t^{\prime \prime} W\left(t^{\prime}-t^{\prime \prime}\right)\right)
% $$




\section{Cumulant expansion for electronic structure}

\subsection{Annotation of Loos paper}
The definition of the cumulant ansatz for the retarded Green's function is given by:
\begin{equation}
    \bm{G}(t) = \bm{G}^0(t)e^{\bm{C}(t)}
\end{equation}
where $\bm{C}(t)$ is the retarded cumulant and $\bm{G}^0(t)$ is the retarded HF Green's function. By relating the Dyson equation to the Taylor series expansion of the exponential (both to first order), we can write:
\begin{equation}
    \bm{G}^0(t) \bm{C}(t) = \iint \dd t_1 \dd t_2 \bm{G}^0(t-t_1) \bm{\Sigma}^c(t_1 - t_2) \bm{G}^0(t_2)
\end{equation}
\textbf{This first order expansion is exact up to the first order in the screened Coulomb interaction $W$. If we choose instead to use a second order self-energy, now the cumulant will be exact to second order in the bare Coulomb interaction $v$.}
Projecting to the spin-orbital basis and inserting the resolution of the identity, we get:
\begin{align}
    \sum_{r}\bra{p}\bm{G}^0(t) \ket{r}\bra{r}\bm{C}(t)\ket{q} &= \sum_{rs}\iint \dd t_1 \dd t_2 \bra{p}\bm{G}^0(t-t_1)\ket{r} \bra{r}\bm{\Sigma}^c(t_1 - t_2)\ket{s} \bra{s}\bm{G}^0(t_2)\ket{q} \\
    \sum_{r}\bm{G}_{pr}^0(t) \bm{C}_{rq}(t) &= \sum_{rs}\iint \dd t_1 \dd t_2 \bm{G}_{ps}^0(t-t_1) \bm{\Sigma}_{sr}^c(t_1 - t_2) \bm{G}_{rq}^0(t_2)\\
    \bm{G}_{pp}^{0}(t) \bm{C}_{pq}(t) &= \underbrace{\iint \dd t_1 \dd t_2 \bm{G}_{pp}^{0}(t-t_1) \bm{\Sigma}_{pq}^c(t_1 - t_2) \bm{G}_{qq}^{0}(t_2)}_{*}
    \label{eqn:cumulant_connection}
\end{align}
where $\bm{G}^0(t)$ is the retarded HF Green's function, which is diagonal in the spin-orbital basis, specifically $\bm{G}_{pp}^{0}(t) = -i\Theta(t)e^{-i\epsilon_p t}$, where $\epsilon_p$ is the HF energy of the $p$-th spin-orbital.
The formula for the inverse Fourier transform is given by:
\begin{equation}
    f(t) = \int \frac{\dd \omega}{2\pi} e^{-i\omega t} f(\omega)
\end{equation}
which implies that
\begin{align}
    \bm{G}_{pp}^{0}(t-t_1) = \int \frac{\dd \omega}{2\pi} e^{-i\omega (t-t_1)} \bm{G}_{pp}^{0}(\omega)\\
    \bm{\Sigma}_{pq}^c(t_1 - t_2) = \int \frac{\dd \omega'}{2\pi} e^{-i\omega' (t_1 - t_2)} \bm{\Sigma}_{pq}^c(\omega')\\
    \bm{G}_{qq}^{0}(t_2) = \int \frac{\dd \omega''}{2\pi} e^{-i\omega'' t_2} \bm{G}_{qq}^{0}(\omega'')
\end{align}
and plugging into the double time integral *, we get:
\begin{align}
    * &= \iint \dd t_1 \dd t_2 \left[\int \frac{\dd \omega}{2\pi} e^{-i\omega (t-t_1)} \bm{G}_{pp}^{0}(\omega)\right] \left[\int \frac{\dd \omega'}{2\pi} e^{-i\omega' (t_1 - t_2)} \bm{\Sigma}_{pq}^c(\omega')\right] \left[\int \frac{\dd \omega''}{2\pi} e^{-i\omega'' t_2} \bm{G}_{qq}^{0}(\omega'')\right]\\
    &= \underbrace{\int \dd t_1 e^{-i \left(\omega' -\omega\right)t_1} \int \dd t_2 e^{-i \left(\omega'' -\omega'\right)t_2}}_{4\pi^2 \delta(\omega' -\omega)\delta(\omega'' -\omega')} \iiint \dd \omega \dd \omega' \dd \omega'' \frac{e^{-i\omega t}}{8\pi^3} \bm{G}_{pp}^{0}(\omega)\bm{\Sigma}_{pq}^c(\omega')\bm{G}_{qq}^{0}(\omega'')\\
    &= \int \frac{\dd \omega}{2\pi} e^{-i\omega t} \bm{G}_{pp}^{0}(\omega)\bm{\Sigma}_{pq}^c(\omega)\bm{G}_{qq}^{0}(\omega)
\end{align}
Multiplying both sides of eqn.~\ref{eqn:cumulant_connection} by $ie^{\epsilon_p t} $ and recalling that we are dealing with retarded quantities that vanish for negative times, we get:
\emph{We are also interested in what happens if we use an exact self-energy in \ref{eq:Cpp_0} instead of the $GW$ one. Also can we get away with not using the diagonal approximation?} 
\subsection{Standard route: diagonal cumulant with GW self-energy}
\begin{align}
	 C_{pp}(t) &= i \int \frac{d\omega}{2\pi} \frac{ \Sigma_{pp}^c\left(\omega+\epsilon_p^{HF}\right)}{(\omega + i \eta)^2} e^{-i \omega t}
\label{eq:Cpp_0} \\
&= i \int \frac{d\omega}{2\pi} \frac{1}{(\omega + i \eta)^2} e^{-i \omega t} \left[ \sum_{i\nu} \frac{M_{pi\nu}^2}{\omega + \underbrace{\epsilon_p^{HF} - \epsilon_i + \Omega_\nu + i \eta}_{-\Delta_{pi\nu}}} + \sum_{a\nu} \frac{M_{pa\nu}^2}{\omega + \underbrace{\epsilon_p^{HF} - \epsilon_a - \Omega_\nu + i \eta}_{-\Delta_{pa\nu}}} \right] \\
& =\mathrm{i} \sum_{i v} M_{p i v}^2 \int \frac{\mathrm{~d} \omega}{2 \pi} e^{-\mathrm{i} \omega t} \frac{1}{[\omega+\mathrm{i} \eta]^2} \frac{1}{\omega-\Delta_{p i v}} +\mathrm{i} \sum_{a v} M_{p a v}^2 \int \frac{\mathrm{~d} \omega}{2 \pi} e^{-\mathrm{i} \omega t} \frac{1}{[\omega+\mathrm{i} \eta]^2} \frac{1}{\omega-\Delta_{p a v}}
\label{eq:Cpp_1}
\\
&= \sum_{i\nu} \zeta_{pi\nu} \left[ e^{-i\Delta_{pi\nu} t} - 1 + i\Delta_{pi\nu} t \right] + \sum_{a\nu} \zeta_{pa\nu} \left[ e^{-i\Delta_{pa\nu} t} - 1 + i\Delta_{pa\nu} t \right]
\label{eq:Cpp_2}
\end{align}
where in going from eqn.~\ref{eq:Cpp_m1} to eqn.~\ref{eq:Cpp_0}, where we made a diagonal approximation for the self-energy and introduced the frequency shift $\omega \to \omega + \epsilon_p^{HF}$, and then from eqn.~\ref{eq:Cpp_1} to eqn.~\ref{eq:Cpp_2}, we have evaluated a contour integral. For the final expression, we have defined $\zeta_{pi\nu} = \left(\frac{M_{pi\nu}}{\Delta_{pi\nu}}\right)^2$ and $\zeta_{pa\nu} = \left(\frac{M_{pa\nu}}{\Delta_{pa\nu}}\right)^2$. This allows us to arrive at the something similar to the Landau form of the cumulant.
\begin{tcolorbox}
A few notes on how to evaluate this integral: there is a double pole at $\omega_1 = -i\eta$ and a simple pole at $\omega_2 = -\Delta - i\eta$. Closing the contour in the lower half-plane because $\operatorname{Im}\left(\omega_1\right), \operatorname{Im}\left(\omega_2\right)<0$, and applying Cauchy's residue theorem, leads to
\begin{align}
    \int \frac{\mathrm{d} \omega}{2 \pi} e^{-\mathrm{i} \omega t} \frac{1}{\left(\omega-\omega_1\right)^2} \frac{1}{\omega-\omega_2} & =(-\mathrm{i})\left\{\left[\partial_\omega\left(\frac{e^{-\mathrm{i} \omega t}}{\omega-\omega_2}\right)\right]_{\omega=\omega_1}+\left[\frac{e^{-\mathrm{i} \omega t}}{\left(\omega-\omega_1\right)^2}\right]_{\omega=\omega_2}\right\} \\
    & =\frac{(-\mathrm{i})}{\left(\omega_1-\omega_2\right)^2}\left\{\left[(-\mathrm{i} t)\left(\omega_1-\omega_2\right)-1\right] e^{-\mathrm{i} \omega_1 t}+e^{-\mathrm{i} \omega_2 t}\right\}\\
\implies \int \frac{\mathrm{d} \omega}{2 \pi} e^{-i \omega t} \frac{1}{[\omega-(0-\mathrm{i} \eta)]^2} \frac{1}{\omega-\Delta}&=\frac{-\mathrm{i}}{\Delta^2}\left(e^{-\mathrm{i} \Delta t}+\mathrm{i} \Delta t-1\right)
\end{align}
\end{tcolorbox}
Now we plug in our derived expression for $C_{pp}(t)$ into the cumulant ansatz for the retarded Green's function:
\begin{align}
G_{pp}(t) & = G_{pp}^{HF}(t) e^{C_{pp}(t)} \\
& = -i \Theta(t) e^{-i \epsilon_p^{HF} t + C_{pp}(t)} \\
& = -i \Theta(t) e^{-i \epsilon_p^{HF} t + \sum_{i\nu} \zeta_{pi\nu}\left(e^{-i\Delta_{pi\nu} t} + i\Delta_{pi\nu} t - 1\right) + \sum_{a\nu} \zeta_{pa\nu} \left(e^{-i\Delta_{pa\nu} t} + i\Delta_{pa\nu} t - 1\right)}\\
& = -i \Theta(t) \underbrace{e^{-\sum_{i\nu} \zeta_{pi\nu} - \sum_{a\nu} \zeta_{pa\nu}}}_{Z_p^{QP}} e^{-i\overbrace{\left(\epsilon_p^{HF} - \sum_{i\nu} \zeta_{pi\nu}\Delta_{pi\nu} - \sum_{a\nu} \zeta_{pa\nu}\Delta_{pa\nu}\right)}^{\epsilon_p^{QP}} t} e^{\sum_{i\nu} \zeta_{pi\nu} e^{-i\Delta_{pi\nu} t} + \sum_{a\nu} \zeta_{pa\nu} e^{-i\Delta_{pa\nu} t}} \\ 
& = -i \Theta(t) Z_p^{QP} e^{-i \epsilon_p^{QP} t} e^{\sum_{i\nu} \zeta_{pi\nu} e^{-i\Delta_{pi\nu} t} + \sum_{a\nu} \zeta_{pa\nu} e^{-i\Delta_{pa\nu} t}} \\
\end{align}
where we have the weight of the quasiparticle peak $Z_p^{QP} = \exp\left(-\sum_{i\nu} \zeta_{pi\nu} - \sum_{a\nu} \zeta_{pa\nu}\right)$ and the quasiparticle energy $\epsilon_p^{QP} = \epsilon_p^{HF} - \left(\sum_{i\nu} \zeta_{pi\nu} \Delta_{pi\nu} + \sum_{a\nu} \zeta_{pa\nu} \Delta_{pa\nu}\right)$. 
\begin{tcolorbox}
We pause to make some important connections. Notice
\begin{align}
\epsilon_p^{QP} & = \epsilon_p^{HF} - \left(\sum_{i\nu} \zeta_{pi\nu} \Delta_{pi\nu} + \sum_{a\nu} \zeta_{pa\nu} \Delta_{pa\nu}\right) \\
& = \epsilon_p^{HF} - \left(\sum_{i\nu} \frac{M_{pi\nu}^2}{\Delta_{pi\nu}} + \sum_{a\nu} \frac{M_{pa\nu}^2}{\Delta_{pa\nu}}\right) \\
& = \epsilon_p^{HF} + \Sigma_{pp}^c\left(\epsilon_p^{HF}\right)
\end{align}
and
\begin{align}
Z_p^{QP} & = \exp\left(-\sum_{i\nu} \zeta_{pi\nu} - \sum_{a\nu} \zeta_{pa\nu}\right) \\
& = \exp\left(-\sum_{i\nu} \left(\frac{M_{pi\nu}}{\Delta_{pi\nu}}\right)^2 - \sum_{a\nu} \left(\frac{M_{pa\nu}}{\Delta_{pa\nu}}\right)^2\right) \\
& = \exp\left(\left[\frac{\partial \Sigma_{pp}^c(\omega)}{\partial \omega}\right]_{\omega = \epsilon_p^{HF}}\right) \\
\end{align}
where we have used the fact that $\Sigma_{pp}^c(\omega) = \sum_{i\nu} \frac{M_{pi\nu}^2}{\omega - \epsilon_i + \Omega_\nu} + \sum_{a\nu} \frac{M_{pa\nu}^2}{\omega - \epsilon_a - \Omega_\nu} \implies \left[\frac{\partial \Sigma_{pp}^c(\omega)}{\partial \omega}\right]_{\omega = \epsilon_p^{HF}} = -\sum_{i\nu} \left(\frac{M_{pi\nu}}{\Delta_{pi\nu}}\right)^2 - \sum_{a\nu} \left(\frac{M_{pa\nu}}{\Delta_{pa\nu}}\right)^2$.
\end{tcolorbox}
Next, we want to perform a Fourier transform.
\begin{align}
G_{pp}(\omega) & = \int_{-\infty}^{\infty} dt e^{i \omega t} G_{pp}(t)
\label{Gpp1}
\\
& = -i Z_p^{QP} \int_0^{\infty} dt e^{i(\omega - \epsilon_p^{QP}) t} e^{\sum_{i\nu} \zeta_{pi\nu} e^{-i\Delta_{pi\nu} t} + \sum_{a\nu} \zeta_{pa\nu} e^{-i\Delta_{pa\nu} t}} 
\label{Gpp2}
\\
& = -i Z_p^{QP} \int_0^{\infty} dt\, e^{i(\omega - \epsilon_p^{QP}) t} \left(1 + \sum_{i\nu} \zeta_{pi\nu} e^{-i\Delta_{pi\nu} t} + \sum_{a\nu} \zeta_{pa\nu} e^{-i\Delta_{pa\nu} t}4 \right) \\
&= -i Z_p^{QP} \int_0^{\infty} dt\, e^{[-\eta + i(\omega - \epsilon_p^{QP}) t]}\label{Gpp3} \\
&\quad - i Z_p^{QP} \sum_{i\nu} \zeta_{pi\nu} \int_0^{\infty} dt\, e^{[-\eta + i(\omega - \epsilon_p^{QP} - \Delta_{pi\nu}) t]} \\
&\quad - i Z_p^{QP} \sum_{a\nu} \zeta_{pa\nu} \int_0^{\infty} dt\, e^{[-\eta + i(\omega - \epsilon_p^{QP} - \Delta_{pa\nu}) t]} + \ldots \\
&= \frac{Z_p^{QP}}{\omega - \epsilon_p^{QP} + i\eta} + \sum_{i\nu} \frac{Z_p^{QP} \zeta_{pi\nu}}{\omega - \epsilon_p^{QP} - \Delta_{pi\nu} + i\eta} + \sum_{a\nu} \frac{Z_p^{QP} \zeta_{pa\nu}}{\omega - \epsilon_p^{QP} - \Delta_{pa\nu} + i\eta} + \ldots\\
% & =-i Z_p^{Q P} \int_0^{+\infty} d r e^{\left[-\eta+i\left(\omega-\varepsilon_p^{Q P}\right)\right]} \\
% & -\mathrm{i} Z_p^{\mathrm{OP}} \sum_{i v} \zeta_{p i v} \int_0^{+\infty} \mathrm{d} r e^{\left\{-\eta+i\left[\omega-\left(\epsilon_p^{\mathrm{OP}}+\Delta_{p i v}\right)\right]\right\}^{\prime}}-\mathrm{i} Z_p^{\mathrm{OP}} \sum_{a v} \zeta_{p a v} \int_0^{+\infty} \mathrm{d} r e^{\left(-\eta+i\left[\omega-\left(\epsilon_p^{\mathrm{OP}}+\Delta_{p \omega v}\right)\right]\right\}^{\prime}} \\
& = \frac{Z_p^{QP}}{\omega - \epsilon_p^{QP} + i\eta} + \sum_{i\nu} \frac{Z_{pi\nu}^{sat}}{\omega - \epsilon_{pi\nu}^{sat} + i\eta} + \sum_{a\nu} \frac{Z_{pa\nu}^{sat}}{\omega - \epsilon_{pa\nu}^{sat} + i\eta} + \ldots
\end{align}
In going from eqn.~\ref{Gpp1} to eqn.~\ref{Gpp2}, we used the step function to restrict the lower bound of the integral to $0$ and then we end by defining the satellite energies $\epsilon_{pi\nu}^{sat} = \epsilon_p^{QP} + \Delta_{pi\nu}$ and $\epsilon_{pa\nu}^{sat} = \epsilon_p^{QP} + \Delta_{pa\nu}$, as well as the satellite weights $Z_{pi\nu}^{sat} = Z_p^{QP} \zeta_{pi\nu}$ and $Z_{pa\nu}^{sat} = Z_p^{QP} \zeta_{pa\nu}$. \emph{What happens when we treat the exponential more than just up to the first order? Is this useful?}
\subsubsection{Spectral function}
The diagonal elements of the spectral function are obtained as (the virtual satellites spectral function, whose derivation will mirror that of the occupied satellites, are omitted for the sake of brevity)
\begin{align}
A_{pp}^{GW+C}(\omega) & = -\frac{1}{\pi} \operatorname{Im} G_{pp}(\omega) \\
& = -\frac{1}{\pi} \operatorname{Im}\left[\frac{Z_p^{QP}}{\omega - \epsilon_p^{QP} + i\eta} + \sum_{i\nu} \frac{Z_{pi\nu}^{sat}}{\omega - \epsilon_{pi\nu}^{sat} + i\eta} + \sum_{a\nu} \frac{Z_{pa\nu}^{sat}}{\omega - \epsilon_{pa\nu}^{sat} + i\eta}\right] \\
& = -\frac{1}{\pi} \operatorname{Im}\left[\frac{\operatorname{Re} Z_p^{QP} + i \operatorname{Im} Z_p^{QP}}{\omega - \operatorname{Re} \epsilon_p^{QP} + i\left(\eta - \operatorname{Im} \epsilon_p^{QP}\right)} + \sum_{i\nu} \frac{\operatorname{Re} Z_{pi\nu}^{sat} + i \operatorname{Im} Z_{pi\nu}^{sat}}{\omega - \operatorname{Re} \epsilon_{pi\nu}^{sat} + i\left(\eta - \operatorname{Im} \epsilon_{pi\nu}^{sat}\right)} + \ldots \right] \\
& = -\frac{1}{\pi} \operatorname{Im}\left[\frac{\left(\operatorname{Re} Z_p^{QP} + i \operatorname{Im} Z_p^{QP}\right)\left(\omega - \operatorname{Re} \epsilon_p^{QP} - i\left(\eta - \operatorname{Im} \epsilon_p^{QP}\right)\right)}{\left(\omega - \operatorname{Re} \epsilon_p^{QP}\right)^2 + \left(\operatorname{Im} \epsilon_p^{QP}\right)^2}\right.\\
& \left. + \sum_{i\nu} \frac{\left(\operatorname{Re} Z_{pi\nu}^{sat} + i \operatorname{Im} Z_{pi\nu}^{sat}\right)\left(\omega - \operatorname{Re} \epsilon_{pi\nu}^{sat} - i\left(\eta - \operatorname{Im} \epsilon_{pi\nu}^{sat}\right)\right)}{\left(\omega - \operatorname{Re} \epsilon_{pi\nu}^{sat}\right)^2 + \left(\operatorname{Im} \epsilon_{pi\nu}^{sat}\right)^2} + \ldots \right] \\
& = -\frac{1}{\pi}\left[\frac{\left(\operatorname{Re} Z_p^{QP}\right)\left(\operatorname{Im} \epsilon_p^{QP}\right) + \left(\operatorname{Im} Z_p^{QP}\right)\left(\omega - \operatorname{Re} \epsilon_p^{QP}\right)}{\left(\omega - \operatorname{Re} \epsilon_p^{QP}\right)^2 + \left(\operatorname{Im} \epsilon_p^{QP}\right)^2}\right.\\
& \left. + \sum_{i\nu} \frac{\left(\operatorname{Re} Z_{pi\nu}^{sat}\right)\left(\operatorname{Im} \epsilon_{pi\nu}^{sat}\right) + \left(\operatorname{Im} Z_{pi\nu}^{sat}\right)\left(\omega - \operatorname{Re} \epsilon_{pi\nu}^{sat}\right)}{\left(\omega - \operatorname{Re} \epsilon_{pi\nu}^{sat}\right)^2 + \left(\operatorname{Im} \epsilon_{pi\nu}^{sat}\right)^2} + \ldots \right]
\end{align}
\subsection{Annotation of CC Cumulant Green's function paper}
\subsubsection{Cumulant from second order self-energy}
The second order self-energy, which uses a HF reference, can be written as
\begin{align}
\Sigma_{pq}^{(2)}(\omega) =& \frac{1}{2} \sum_{iab} \frac{\langle pi || ab \rangle \langle ab || qi \rangle}{\omega + \epsilon_i - \epsilon_a - \epsilon_b} + \frac{1}{2} \sum_{ija} \frac{\langle pa || ij \rangle \langle ij || qa \rangle}{\omega + \epsilon_a - \epsilon_i - \epsilon_j} \\
\implies \Sigma_{pp}^{(2)}(\omega+\epsilon_p) =&
\frac{1}{2} \sum_{iab}
\frac{\left< pi \left| \right| ab \right>^2}
{\omega-\epsilon_{pi}^{ab}} + \frac{1}{2} \sum_{ija}
\frac{\left< pa \left| \right| ij \right>^2}
{\omega-\epsilon_{pa}^{ij}}
\end{align}
where
$\epsilon_{pi}^{ab} = \epsilon_{a}+\epsilon_{b}-\epsilon_{p}-\epsilon_{i}$
and
$\epsilon_{pa}^{ij} = \epsilon_{i}+\epsilon_{j}-\epsilon_{p}-\epsilon_{a}$. So we can write the diagonal cumulant\footnote{ 
Note that in order to get from eqn.~\ref{eq:cumulant_2nd_1} to eqn.~\ref{eq:cumulant_2nd} we have used the identity $\int \frac{d\omega}{2\pi} \frac{i e^{-i \omega t}}{\omega^2\left(\omega-\epsilon\right)} = \frac{1}{\epsilon^2} \left(e^{-i \epsilon t} +i \epsilon t -1 \right) \mathrm{sgn}(t)$, and since we are dealing with retarded quantities, we only care about $t>0$ so $\mathrm{sgn}(t) = 1$.} as
\begin{align}
C_{pp}(t) &\equiv i \int \frac{d\omega}{2\pi} \frac{ \Sigma_{pp}^c\left(\omega+\epsilon_p\right)}{(\omega + i \eta)^2} e^{-i \omega t} \\
\implies C_{pp}^{(2)}(t) &= \frac{1}{2} \sum_{iab} \left< pi \left| \right| ab \right>^2 \int \frac{d\omega}{2\pi} \frac{i e^{-i \omega t}}{\omega^2\left(\omega-\epsilon_{pi}^{ab}\right)} + \frac{1}{2} \sum_{ija} \left< pa \left| \right| ij \right>^2 \int \frac{d\omega}{2\pi} \frac{i e^{-i \omega t}}{\omega^2\left(\omega-\epsilon_{pa}^{ij}\right)} 
\label{eq:cumulant_2nd_1}
\\
&= \frac{1}{2}\sum_{iab} \frac{\left< pi \left| \right| ab \right>^2}{\left(\epsilon_{pi}^{ab}\right)^2}
\left(e^{-i \epsilon_{pi}^{ab} t} +i \epsilon_{pi}^{ab} t -1 \right)
+\frac{1}{2}\sum_{ija} \frac{\left< pa \left| \right| ij \right>^2}{\left(\epsilon_{pa}^{ij}\right)^2}
\left(e^{-i \epsilon_{pa}^{ij} t} +i \epsilon_{pa}^{ij} t -1 \right) 
\label{eq:cumulant_2nd}
\\
&= \frac{1}{2} \sum_{i a b}\langle p i \| a b\rangle^2 f\left(\epsilon_{p i}^{a b}\right)+\frac{1}{2} \sum_{i j a}\langle p a \| i j\rangle^2 f\left(\epsilon_{p a}^{i j}\right)\\
&= \int d\omega\, \beta(\omega) f(\omega)
\end{align}
where $f(\omega) \equiv \frac{e^{-i \omega t}+i \omega t-1}{\omega^2} $ and we identify the cumulant kernel as
\footnote{To get from eqn.~\ref{eq:beta_def} to eqn.~\ref{eq:beta_2nd}, we have used the identity $\operatorname{Im} \frac{1}{x + i \eta} = -\pi \delta(x)$ as $\eta \to 0^+$.}
\begin{align}
    \beta(\omega) &= -\frac{1}{\pi} \operatorname{Im} \Sigma_{pp}^{(2)}\left(\omega+\epsilon_p\right) 
\label{eq:beta_def}\\
\\
&=\frac{1}{2} \sum_{i a b}\langle p i \| a b\rangle^2 \delta\left(\omega-\epsilon_{p i}^{a b}\right)+\frac{1}{2} \sum_{i j a}\langle p a \| i j\rangle^2 \delta\left(\omega-\epsilon_{p a}^{i j}\right) \label{eq:beta_2nd}
\end{align}
A connection can be made between the cumulant kernel $\beta(\omega)=\sum_q g_q^2 \delta(\omega-\omega_q)$ for the 2nd-order self energy  to that for electrons coupled to
bosonic excitations at the frequencies $\omega_q\equiv \epsilon_{pq}^{rs}$ in the quasi-boson approximation
with coupling coefficients $g_q \equiv \langle p q \| r s\rangle$.
\subsubsection{Equivalence between the second order self-energy and the CC retarded Green's function}
When both the IP and EA branches are included, the CC GF can be written in frequency space as
\footnote{A diagrammatic analysis shows that only three non-zero terms when $p, q \in o c c$ survive, which allows us to move from eqn.~\ref{eq:ccgf_expanded_2} to eqn.~\ref{eq:ccgf_expanded_3}.}
\begin{tcolorbox}
    The retarded single-particle Green's function in the time domain is defined as
\begin{equation}
    G_{pq}^R(t-t') = -i \, \Theta(t-t') 
    \langle 0 | \{ a_p(t), a_q^\dagger(t') \} |0\rangle .
\label{spgf}
\end{equation}

Fourier transforming and inserting a resolution of identity over $(N\!\pm\!1)$-electron eigenstates gives the Lehmann representation for both the IP and EA parts:
\begin{align}
    G_{pq}^R(\omega) 
    &= \sum_n \frac{ \langle 0 | a_p(t) | n^{N+1} \rangle \langle n^{N+1} | a_q^\dagger(t') |0\rangle }
      {\omega - (E_n^{N+1} - E_0^N) + i\delta} 
    + \sum_m \frac{ \langle 0 | a_q^\dagger(t') | m^{N-1} \rangle \langle m^{N-1} | a_p(t) |0\rangle }
      {\omega + (E_m^{N-1} - E_0^N) + i\delta}.
\end{align}

In coupled-cluster theory, the exact, correlated ground state can be written as $|\Psi_0\rangle = e^T |\Phi\rangle$ and $\langle \Psi_0| = \langle \Phi | (1+\Lambda) e^{-T}$ and defining the similarity-transformed operators
\begin{equation}
    \bar{H}_N = e^{-T} H_N e^{T}, \qquad
    \bar{a}_p(t) = e^{-T} a_p(t) e^{T}, \qquad
    \bar{a}_q^\dagger(t') = e^{-T} a_q^\dagger(t') e^{T}.
\end{equation}

we can write
\begin{align}
    G_{pq}^R(\omega) 
    &= \langle \Phi | (1+\Lambda)\,
    \bar{a}_q^\dagger(t') \,
    \frac{1}{\omega + i\delta + \bar{H}_N} \,
    \bar{a}_p(t)
    |\Phi\rangle
    + \langle \Phi | (1+\Lambda)\,
    \bar{a}_p(t) \,
    \frac{1}{\omega + i\delta - \bar{H}_N} \,
    \bar{a}_q^\dagger(t')
    |\Phi\rangle .
\end{align}
Now, if we start again with Eqn.~\ref{spgf} and decide just to consider the retarded Green's function for one core orbital $c$, it is just diagonal and reads
\begin{align}
    G_{c}^{R}(t) & = -i \Theta(t) e^{iE_0t} \left<0\left| a_c e^{-iHt} a_c^\dagger \right| 0 \right> -i \Theta(t) e^{-iE_0t}\left<0\left| a_c^\dagger e^{iHt} a_c \right| 0 \right> \\
    & = -i \Theta(t) e^{-iE_0t}\left<N-1\left| e^{iHt} \right| N-1 \right>
\end{align}
Note that we could only do this because we were able to make the separable approximation to the ground state $\left| 0 \right> \simeq a_c^\dagger \left| N-1 \right>$, where $\left| N-1 \right>$ is the exact $N-1$ wavefunction with the core electron separated from it. We are just able to make this approximation for core states because they are localized and thus weakly correlated, so their removal won't relax the exact wavefunction too much. This is not the case for valence states, which are highly correlated, so the separable approximation would not make sense.
\end{tcolorbox}
\begin{align}
G_{pq}^R (\omega) &= \left\langle \Phi \left| (1+\Lambda) \bar{a^{\dagger}_q}
(\omega + \bar{H}_N + i\delta)^{-1} \bar{a_p} \right| \Phi \right\rangle + \left\langle \Phi \left| (1+\Lambda) \bar{a_p}
(\omega - \bar{H}_N + i\delta)^{-1} \bar{a^{\dagger}_q} \right| \Phi \right\rangle \\
\langle\Phi|\left(1+\Lambda_2\right)\left(a_q^{\dagger}+\left(a_q^{\dagger} T_2\right)_C\right) X_p(\omega)|\Phi\rangle+\langle\Phi|\left(1+\Lambda_2\right)\left(a_p+\left(a_p T_2\right)_C\right) Y_q(\omega)|\Phi\rangle \label{eq:ccgf_expanded_2}\\
&= \langle\Phi| a_q^{\dagger} X_{1, p}(\omega)|\Phi\rangle+\langle\Phi| \Lambda_2\left(a_q^{\dagger} T_2\right)_C X_{1, p}(\omega)|\Phi\rangle+\langle\Phi| \Lambda_2 a_p Y_{2, q}(\omega)|\Phi\rangle \label{eq:ccgf_expanded_3} \\
&= x^q(\omega)_p-\frac{1}{2} \sum_{i j a b} \lambda_{i j}^{a b} t_{a b}^{q j} x^i(\omega)_p-\frac{1}{2} \sum_{i a b} \lambda_{p i}^{a b} y_{a b}^i(\omega)_q
\end{align}
where to get eqn.~\ref{eq:ccgf_expanded_2} we limited the CC expansion to doubles, i.e. $T\approx T_2=\frac{1}{4} \sum_{i j a b} t_{i j}^{a b} a_a^{\dagger} a_b^{\dagger} a_j a_i$ and $\Lambda\approx\Lambda_2=\frac{1}{4} \sum_{i j a b} \lambda_{i j}^{a b} a_i^{\dagger} a_j^{\dagger} a_b a_a$, and expanded the $\bar{a}_p = a_p + [a_p,T_2]$ and $\bar{a}_q^{\dagger} = a_q^{\dagger} + [a_q^{\dagger},T_2]$ operators into their connected forms. We also defined the $X_p(\omega)\equiv (\omega + \bar{H}_N + i\delta)^{-1} \bar{a_p}$ and $Y_q(\omega)\equiv (\omega - \bar{H}_N + i\delta)^{-1} \bar{a_q}^{\dagger}$ operators that have the components
\begin{align}
    X_p(\omega) & =\sum_i x^i(\omega)_p a_i+\frac{1}{2!} \sum_{i j, a} x_a^{i j}(\omega)_p a_a^{\dagger} a_j a_i=X_{1, p}(\omega)+X_{2, p}(\omega) \\
Y_q(\omega) & =\sum_a y_a(\omega)_q a_a^{\dagger}+\frac{1}{2!} \sum_{i, a b} y_{a b}^i(\omega)_q a_a^{\dagger} a_b^{\dagger} a_i=Y_{1, q}(\omega)+Y_{2, q}(\omega)
\end{align}
\emph{This seems similar to second RPA theory, probably because it is.}
At this point, we can form a perturbation series defined by a perturbation parameter $\xi$, e.g. $t_{p q}^{r s}=t_{p q}^{(0) r s}+\xi t_{p q}^{(1) r s}+\xi^2 t_{p q}^{(2) r s}$, etc. By keeping only terms up to second order we get:
\begin{align}
G_{p q}^{(2)R}(\omega) &= x^{(2) q}(\omega)_p-\frac{1}{2} \sum_{i j a b} \lambda_{i j}^{(1) a b} t_{a b}^{(1) q j} x^{(0) i}(\omega)_p-\frac{1}{2} \sum_{i a b}\left(\lambda_{p i}^{(1) a b} y_{a b}^{(1) i}(\omega)_q+\lambda_{p i}^{(2) a b} y_{a b}^{(0) i}(\omega)_q\right) \\
&= x^{(0) p}(\omega)_p \delta_{p q}+x^{(2) q}(\omega)_p-\frac{1}{2} \sum_{i a b} \lambda_{p i}^{(1) a b}\left(t_{a b}^{(1) q i} x^{(0) p}(\omega)_p+y_{a b}^{(1) i}(\omega)_q\right) \\
&= \frac{\delta_{pq}}{(\omega-\epsilon_p)} + \frac{1}{(\omega-\epsilon_p)} \left[ \frac{1}{2}\sum_{ija} \frac{v^{qa}_{ij}v^{pa}_{ij}}{(\omega+\epsilon_a-\epsilon_i-\epsilon_j)} + \frac{1}{2}\sum_{iab} \frac{v^{pi}_{ab}v^{qi}_{ab}}{(\omega+\epsilon_i-\epsilon_a-\epsilon_b)} \right] \frac{1}{(\omega-\epsilon_q)} \\
&\equiv G_{pq}^{R(0)} (\omega) + G_{pq}^{R(0)} (\omega) \Sigma_{pq}^{(2)}(\omega) G_{pq}^{R(0)} (\omega)
\end{align}
 where we used the simplifications $x^{(1) q}(\omega)_p=0$ and $x^{(0) q}(\omega)_p=x^{(0) p}(\omega)_p \delta_{p q}$ and in the perturbation analysis, we identified that $G_{p q}^{(0) R}(\omega)=x^{(0) q}(\omega)_p= \frac{1}{(\omega-\epsilon_q)}$.
% Here we skip the derivation of each of the coefficients in term of the two-particle integrals and HF eigenvalues and simply list expressions:
% $$
% \begin{gathered}
% x^{(0) p}(\omega)_p=\frac{1}{\left(\omega-\epsilon_p\right)} \\
% x^{(2) q}(\omega)_p=\frac{1}{2\left(\omega-\epsilon_q\right)}\left[\sum_{i j a} v_{i j}^{q a} x_a^{(1) i j}(\omega)_p+\frac{1}{\left(\omega-\epsilon_p\right)} \sum_{i a b} v_{a b}^{p i} t_{a b}^{(1) q i}\right] \\
% x_a^{(1) i j}(\omega)_p=\frac{v_{p a}^{i j}}{\left(\omega-\epsilon_p\right)\left(\omega+\epsilon_a-\epsilon_i-\epsilon_j\right)} \\
% t_{a b}^{(1) i j}=\lambda_{i j}^{(1) a b}=\frac{v_{a b}^{i j}}{\left(\epsilon_i+\epsilon_j-\epsilon_a-\epsilon_b\right)}=\frac{v_{a b}^{i j}}{\epsilon_{a b}^{i j}} \\
% x_a^{(1) i j}(\omega)_p=\frac{v_{p a}^{i j}}{\left(\omega-\epsilon_p\right)\left(\omega+\epsilon_a-\epsilon_i-\epsilon_j\right)}
% \end{gathered}
% $$
% \begin{align}
% \implies G_{pq}^{R(2)} (\omega) &= \frac{\delta_{pq}}{(\omega-\epsilon_p)} + \frac{1}{(\omega-\epsilon_p)} \left[ \frac{1}{2}\sum_{ija} \frac{v^{qa}_{ij}v^{pa}_{ij}}{(\omega+\epsilon_a-\epsilon_i-\epsilon_j)} + \frac{1}{2}\sum_{iab} \frac{v^{pi}_{ab}v^{qi}_{ab}}{(\omega+\epsilon_i-\epsilon_a-\epsilon_b)} \right] \frac{1}{(\omega-\epsilon_q)}.
% \end{align}

\subsubsection{Real-time EOM-CC Cumulant GF}
We restrict the discussion to the retarded
core-hole Green's function for a given deep core level $p=c$, $G^R_c=G_{cc}$
given by
\begin{align}
G_{c}^{R}(t) &= -i \Theta(t-t')
\left<0\left| \left\{a_c(t), a_c^\dagger(t') \right\} \right| 0 \right>\\
&= -i \Theta(t) 
  e^{iE_0t} \left<0\left| a_c e^{-iHt} a_c^\dagger \right| 0 \right> + -i \Theta(t) 
e^{-iE_0t}\left<0\left| a_c^\dagger e^{iHt} a_c \right| 0 \right> \\
&= -i \Theta(t) 
  e^{iE_0t} \left<N-1\left| a_c a_c e^{-iHt} a_c^\dagger a_c^\dagger \right| N-1 \right> + -i \Theta(t) 
e^{-iE_0t}\left<N-1\left| a_c a_c^\dagger e^{iHt} a_c a_c^\dagger \right| N-1 \right> \\
&= -i \Theta(t) 
  e^{-iE_0t}\left<N-1\left| e^{iHt} \right| N-1 \right> \\
&= -i \Theta(t) e^{-iE_0 t} \left<N-1 | N-1, t \right> 
\label{eq:gc_t} 
\end{align}
where we have used the separable approximation to the ground state $\left| 0
\right> \simeq a_c^\dagger \left| N-1 \right>$, which is reasonable for
core excitations. And then at the end we have defined $\left| N-1, t \right> =
e^{iHt} \left| N-1 \right>$, which is a solution to $-i \frac{d\left| N-1, t \right>}{dt} = H \left| N-1, t \right>$. The next step is to assume a time-dependent, CC ansatz for
$\left| N-1, t \right> = N(t) e^{T(t)} \left| \phi \right>$, where $\left| \phi \right> = a_c \left| N \right>$ is the reference determinant with a core hole, $N(t)$ is a \emph{scalar} normalization factor, and $T(t)$ is the time-dependent cluster operator acting only on the $N-1$ electron Fock space. Inserting this ansatz into
the differential equation for $\left| N-1, t \right>$ and left multiplying 
by $e^{-T(t)}$, we obtain the coupled EOM 
\footnote{%
In going eqn.~\ref{de_1} to eqn.~\ref{de_2}, we used:
\begin{align*}
e^{-T(t)}\frac{d}{dt}\left( N(t) e^{T(t)} \right)
&= e^{-T(t)}\left( \dot{N}(t) e^{T(t)} + N(t) \dot{T}(t) e^{T(t)} \right) \\
&= \dot{N}(t) e^{-T(t)} e^{T(t)} + N(t) \left( \dot{T}(t) + \frac{1}{2!} [\dot{T}(t), T(t)] + \frac{1}{3!} [[\dot{T}(t), T(t)], T(t)] + \ldots \right) \\
& \approx \dot{N}(t) + N(t) \dot{T}(t) \\
&= N(t)\left( \frac{\dot{N}(t)}{N(t)} + \dot{T}(t) \right) \\
&= N(t)\left( \frac{d}{dt}\ln N(t) + \dot{T}(t) \right)
\end{align*}
where we have used the Baker-Campbell-Hausdorff expansion and the truncation is consistent with CCSD.
}
\begin{align}
    -i \frac{d\left[N(t) e^{T(t)} \left| \phi \right>\right]}{dt} & = H N(t) e^{T(t)} \left| \phi \right> \\
    -i e^{-T(t)}{\frac{d\left[N(t) e^{T(t)} \right]}{dt}} \left| \phi \right>& = N(t) \underbrace{e^{-T(t)} H e^{T(t)}}_{\bar{H}(t)} \left| \phi \right> 
\label{de_1}\\
    -i N(t) \left( \frac{d\ln N(t)}{dt} + \frac{d T(t)}{dt} \right) \left| \phi \right>& = N(t) \bar{H}(t) \left| \phi \right> 
\label{de_2}
\\
    -i \left( \frac{d\ln N(t)}{dt} + \frac{d T(t)}{dt} \right) \left| \phi \right>& = \left( \bar{H}_N(t) + E^{N-1} \right) \left| \phi \right>
\label{de_3}
\end{align}
 where we have defined a normal-order, similarity transformed Hamiltonian as $\bar{H}_N(t) = \bar{H}(t) - E^{N-1}$, where $E^{N-1} = \left< \phi | \bar{H} | \phi \right>$. Projecting on the reference $\left< \phi \right|$ and excited determinants $\left< \phi_{ij}^{ab} \right|$, respectively, we obtain the coupled EOMs for the normalization factor and the cluster amplitudes, respectively:
\begin{align}
\label{eq-dlnndt}
-i \frac{d \ln N(t)}{dt} = \left< \phi \left| \bar{H}_N(t) \right| \phi \right>
+ E^{N-1} \\
\implies N(t) = e^{i E^{N-1} t} e^{i \int_0^t \left< \phi \left| \bar{H}_N(t') \right| \phi \right> dt'}
\label{eq:nt}
\end{align}
\begin{equation}
\label{eq-dtdt}
-i \left< \phi_{ij...}^{ab...} \left| \frac{d T(t)}{dt} \right| \phi \right> =
\langle \phi_{ij...}^{ab...} \left| \bar{H}_N(t) \right| \phi \rangle.
\end{equation}
In order to further evaluate, we need to introduce some additional approximations. First, we assume that the ground state is uncorrelated, i.e.
$\left| N-1 \right> \simeq a_c \left| \Phi \right> = \left| \phi \right>$,
so that
\begin{equation}
\begin{split}
\left<N-1\right| \left. N-1, t \right> =&
N(t) \left<N-1\left| e^{iHt}\right| \phi \right> \\
=&N(t) \left< \phi \left| \left( 1+ iHt + \frac{(iHt)^2}{2!} + \ldots \right) \right| \phi \right> \\
=&N(t) \left( 1+ \left< \phi \left| R(t) \right| \phi \right> \right) \\
=&N(t).
\end{split}
\end{equation}
where $R(t)$ is the excitation operator that collects all excited terms in the series expansion, so it has expectation value $\left< \phi \left| R(t) \right| \phi \right> = 0$. Now, if we insert eqn.~\ref{eq:nt} into eqn.~\ref{eq:gc_t}, we see that the Green's function is given by
\begin{align}
G_c^R(t) &= -i \Theta(t) e^{-iE_0 t} N(t) \\
&= -i \Theta(t) e^{-i(E_0 - E^{N-1}) t} e^{i \int_0^t \left< \phi \left| \bar{H}_N(t') \right| \phi \right> dt'} \\
&= -i \Theta(t) e^{-i \epsilon_c t} e^{C_c^R(t)}
\label{eq:gc_final}
\end{align}
In the last line, we take $E_0 \equiv E_{HF}$ as the reference energy of the $N$ electron system and so we have $E_0-E^{N-1} \simeq \epsilon_c$, as expected from Koopmans' theorem. And importantly, we have identified the cumulant as
\begin{equation}
\label{eqn:cum_t_f}
C_c^{R}(t) = i \int_0^t \left< \phi \left| \bar{H}_N(t) \right| \phi\right> dt'.
\end{equation}
\begin{tcolorbox}
Now, the known Landau form for the cumulant is
\begin{align}
C_L(t) & = \int d\omega \frac{\beta(\omega)}{\omega^2} \left(e^{-i\omega t} + i\omega t - 1\right) \\
\implies C_L''(t) & = -\int d\omega \beta(\omega) e^{-i\omega t} \\
\end{align}
But now if we do the same thing by differentiating eqn.~\ref{eqn:cum_t_f} twice, we have
\begin{align}
C_c'' (t) & = i \frac{d}{dt} \left< \phi \left| \bar{H}_N(t) \right| \phi \right> \\
\end{align}
So we can identify that
\begin{equation}
\int d\omega \beta(\omega) e^{-i\omega t} = -i \frac{d}{dt} \left< \phi \left| \bar{H}_N(t) \right| \phi \right>
\end{equation}
Taking the inverse Fourier transform, we find a form for the CC cumulant kernel as
\begin{align}
\beta(\omega) &= \frac{1}{2\pi} \int_{-\infty}^{\infty} dt e^{i\omega t} \left[-i \frac{d}{dt} \left< \phi \left| \bar{H}_N(t) \right| \phi \right> \right] \\
&= \frac{1}{\pi} \mathrm{Re} \int_0^{\infty} dt e^{-i\omega t} \left[-i \frac{d}{dt} \left< \phi \left| \bar{H}_N(t) \right| \phi \right> \right]
\end{align}
\color{red}{The above relation is not derived yet.}
\color{black}
 Recall, that in the $GW$+C approach, we had $\beta(\omega)=\frac{1}{\pi}\left|\operatorname{Im} \Sigma_{p p}(\omega+\epsilon_p)\right|$. \emph{Therefore, we can identify that there is a connection, but it is unclear currently how to explore it.}
\end{tcolorbox}
So given the differential equation for the logarithm of the normalization factor in eqn.~\ref{eq-dlnndt} and comparing with eqn.~\ref{eq:gc_final}, we see that the cumulant obeys the differential equation
\begin{equation}
\label{eqn:matel1}
\begin{split}
-i\frac{d C_c^R(t)}{dt} &= \left< \phi \left| \bar{H}_N(t) \right| \phi \right> \\
&= \sum_{ia} f_{ia} t_i^a +
\frac{1}{2} \sum_{ijab} v_{ij}^{ab} t_j^b t_i^a,
\end{split}
\end{equation}
\color{red} {The above relation is not derived yet. Supposedly it requires some tedious
algebra and diagrammatic analysis.} \color{black} We can also use eqn.~\ref{eq-dtdt} to write the equation of motion for the cluster amplitudes, where for the singles, it is given by
% \begin{equation}
% \label{eq:rt_eom_ccs}
% -i \dot {t}_i^a(t) = \left< \phi_{i}^{a} \left| \frac{d T(t)}{dt} \right| \phi \right>
% = \left< \phi_{i}^{a} \right| \bar{H}_N(t) \left| \phi \right>.
% \end{equation}
% but shouldn't this be
\begin{equation}
\label{eq:rt_eom_ccs}
-i \dot {t}_i^a(t) \equiv -i \left< \phi_{i}^{a} \left| \frac{d T(t)}{dt} \right| \phi \right>
= \left< \phi_{i}^{a} \right| \bar{H}_N(t) \left| \phi \right>.
\end{equation}
I choose to not continue further for now.


\section{Cumulant expansion for electron-phonon interactions}
\subsection{Annotating PJ's SC-CE paper}
 The time-ordered form of the electron Green's function for one electron can be written as the power series expansion of an exponential ansatz
\begin{align}
\mathcal{G}_k(t)&=\mathcal{G}_k^{(0)}(t)\left\langle a_k T e^{-i \int_0^t d \tau V({\tau})} a_k^\dag\right\rangle
\label{6.5}\\
&=\mathcal{G}_k^{(0)}(t)\left\langle T e^{-i \int_0^t d \tau V({\tau})}\right\rangle \\
&=\mathcal{G}_k^{(0)}(t) \left\langle \sum_{n=0}^{\infty} \frac{(-i)^n}{n!} \int_0^t d t_1 \cdots \int_0^t d t_n T\left[V ( t _ { 1 } ) \cdots V \left(t_n\right)\right] \right\rangle \\
% &=\mathcal{G}_k^{(0)}(t) \left\langle \sum_{n=0}^{\infty} \frac{(-i)^n}{n!} \int_0^t d t_1 \cdots \int_0^t d t_n T\left[ V\left(t_j\right)\right] \right\rangle \\
&=\mathcal{G}_k^{(0)}(t) \left\langle \sum_{n=0}^{\infty} \frac{(-i)^n}{n!} \int_0^t d t_1 \cdots \int_0^t d t_n T\left[\prod_{j=1}^n \sum_{q_j} g_{q_j, k_j} a_{k_j+q_j}^{\dagger}\left(t_j\right) a_{k_j}\left(t_j\right) A_{q_j}\left(t_j\right)\right] \right\rangle \\
&=\mathcal{G}_k^{(0)}(t) \left\langle \sum_{n=0}^{\infty} T \left[\frac{(-i)^n}{n!} \int_0^t d t_1 \cdots \int_0^t d t_n \prod_{j=1}^n \sum_{q_j} g_{q_j, k_j} a_{k_j+q_j}^{\dagger} a_{k_j}e^{-it_j(\epsilon_{k_j}-\epsilon_{k_j + q_j})} A_{q_j}\left(t_j\right)\right] \right\rangle 
\end{align}
where $k_j = k + \sum_{i=j+1}^n q_i$ and $\sum_i q_i=0$. The interpretation of the above expression is that the electron starts in state $k$ at time $0$, and then scatters off $n$ phonons, each with momentum $q_j$, before returning to state $k$ at time $t$. $A_q(t)=b_q e^{-i \omega_q t}+b_{-q}^\dagger e^{i \omega_q t}$, where $b_q$ and $b_q^\dagger$ are the annihilation and creation operators for a phonon of momentum $q$ and frequency $\omega_q$ with $[b_q, b_{q^{\prime}}^{\dagger}]=\delta_{q q^{\prime}}$. If we define the vertex operator $\Gamma_{q k}(t)=g_{q k} e^{-i t \epsilon_k} e^{q \cdot \frac{d}{d k}} e^{i t \epsilon_k}$ we can rewrite the electron Green's function as an exponential,
\begin{equation}
\mathcal{G}_k(t)=\mathcal{G}_k^{(0)}(t)\left\langle T e^{-i \sum_q \int_0^t d \tau \Gamma_{q k}(\tau) A_q(\tau)}\right\rangle
\end{equation}
 where the remaining trace is only over the bosonic degrees of freedom. Assuming harmonic bosons, we can use $\left\langle e^B \right\rangle = e^{\left\langle B^2\right\rangle / 2}$ with the definition
\begin{equation}
    B=-i \sum_q \int_0^t d \tau \Gamma_{q k}(\tau) A_q(\tau)
\end{equation}
where our $B$ is linear in the bosonic operators $A_q$, so the identity applies. Then, 
\begin{align}
    \ln \left\langle T e^B\right\rangle & =\frac{1}{2}\left\langle T B^2\right\rangle \\
& =\frac{1}{2}(-i)^2 \sum_{q q^{\prime}} \int_0^t d \tau \int_0^t d \tau^{\prime} \Gamma_{q k}(\tau) \Gamma_{q^{\prime} k}\left(\tau^{\prime}\right)\left\langle T A_q(\tau) A_{q^{\prime}}\left(\tau^{\prime}\right)\right\rangle \\
& =-\frac{1}{2} \sum_{q q^{\prime}} \int_0^t d \tau \int_0^t d \tau^{\prime} \Gamma_{q k}(\tau) \Gamma_{q^{\prime} k}\left(\tau^{\prime}\right)\left[i D_q^{(0)}\left(\tau-\tau^{\prime}\right) \delta_{q,-q^{\prime}}\right] \\
& =-\frac{i}{2} \sum_q \int_0^t d \tau \int_0^t d \tau^{\prime} D_q^{(0)}\left(\tau-\tau^{\prime}\right) \Gamma_{q k}(\tau) \Gamma_{-q, k}\left(\tau^{\prime}\right) \\
& = -i \sum_q \int_0^t d \tau \int_0^t d \tau^{\prime} D_q^{(0)}\left(\tau-\tau^{\prime}\right) \Gamma_{-q k}(\tau) \Gamma_{q k}\left(\tau^{\prime}\right) \equiv S
\end{align}
To find a self consistent expression for $S$, we use the Feynman operator ordering theorem, which states that given a functional (the Green's function) of time-dependent operators (vertices) $F[\hat{A}(\tau), \hat{B}(\tau), \ldots]$,  where $\tau \in[0, t]$ and with a unitary operator of the form $\hat{U}\left(\tau^{\prime}\right)=T \exp \left[\int_0^{\tau^{\prime}} d \tau \hat{P}(\tau)\right]$, then
\begin{equation}
\hat{U}(t) F[\hat{A}(\tau), \hat{B}(\tau), \ldots] =T\left[\exp \left(\int_0^t d \tau \hat{P}(\tau)\right) F\left[\hat{U}(\tau) \hat{A}(\tau) \hat{U}^{-1}(\tau), \ldots\right]\right] .
\end{equation}
We choose $\hat{P}(\tau)=-i\left[\epsilon_k+\frac{d}{d \tau} \phi_k(\tau)\right]$ and
so the transformed $S$ is given by
\begin{align}
    \bar{S}
&= -i \int_0^t d \tau\left(\epsilon_k+\frac{d}{d \tau} \phi_k(\tau)\right) -i \sum_q \int_0^t d \tau \int_0^\tau d \tau^{\prime} D_q^{(0)}\left(\tau-\tau^{\prime}\right) \bar{\Gamma}_{-q k}(\tau) \bar{\Gamma}_{q k}\left(\tau^{\prime}\right)
\end{align}
with the transformed vertex operators $\bar{\Gamma}_{q k}(\tau)\equiv  \hat{U}(\tau)\Gamma_{q k}(\tau)\hat{U}^{-1}(\tau)=g_{q k} e^{-i \phi_k(\tau)} e^{q \cdot \frac{d}{d k}} e^{+i \phi_k(\tau)}$. The Green's function can then be written as
\begin{align}
    \mathcal{G}_k(t)&=i\Theta(t) e^{i \phi_k(t)} T \left[\exp (-i \int_0^t d \tau(\epsilon_k+\frac{d}{d \tau} \phi_k(\tau)))\right. \\
& \left.\times \exp \left(-i \sum_q \int_0^t d \tau \int_0^\tau d \tau^{\prime} D_q^{(0)}\left(\tau-\tau^{\prime}\right) \bar{\Gamma}_{-q k}(\tau) \bar{\Gamma}_{q k}\left(\tau^{\prime}\right)\right)\right] \notag \\
&=i\Theta(t) e^{C_k(t)}
\end{align}
where
\begin{equation}
    C_k(t)=i\phi_k(t) + \sum_{n=1}^{\infty} \frac{1}{n!} T\left[\bar{S}^n\right]_c
\label{a7}
\end{equation}
where the notation $[\cdot \cdot \cdot]_c$ denotes the cumulant of an operator, and $T[\bar{S}]_c=T[\bar{S}]$ with subsequent terms given by $T\left[\bar{S}^n\right]_c=T\left[\bar{S}^n\right]-\sum_{m=1}^{n-1} \frac{(n-1)!}{m!(n-m-1)!} T\left[\bar{S}^m\right] T\left[\bar{S}^{n-m}\right]_c$. Then, evaluating \ref{a7} with $\phi_k(t)=C_k(t)$ to leading order produces
\begin{align}
    C_k(t)&=-i \epsilon_k t - i \frac{g^2}{N}\sum_q \int_0^t d \tau \int_0^\tau d \tau^{\prime} D_q^{(0)}\left(\tau-\tau^{\prime}\right) e^{-C_k(\tau)+C_{k-q}(\tau)-C_{k-q}\left(\tau^{\prime}\right)+C_k\left(\tau^{\prime}\right)} \label{12} \\
\implies \dot{C}_k(t)&=-i \epsilon_k - i \frac{g^2}{N} \sum_q \int_0^t d \tau D_q^{(0)}(t-\tau) e^{-C_k(t)+C_{k-q}(t)-C_{k-q}(\tau)+C_k(\tau)} \label{13}
\end{align}
with the bare phonon propagator $D_q^{(0)}(t)= -i\left[\left(1+N_q\right) e^{-i \omega_q|l|}+N_q e^{\left|\omega_q\right| l \mid}\right]$ and the Bose occupation factor $N_q=\left[\exp \left(\beta \omega_q\right)-1\right]^{-1}$.  Eqn~\ref{12} can be recast as a self-consistent equation for the interacting Green's function by considering
\begin{align}
\frac{d \mathcal{G}_k(t)}{d t}&= \frac{d}{d t}\left[-i \Theta(t) e^{C_k(t)}\right] \\
&= \dot{C_k}(t) \mathcal{G}_k(t) \\
&= -i \epsilon_k \mathcal{G}_k(t) - i \frac{g^2}{N} \sum_q \int_0^t d \tau \mathcal{G}_k(t) D_q^{(0)}(t-\tau) e^{-C_k(t)+C_{k-q}(t)-C_{k-q}(\tau)+C_k(\tau)} \\
&= -i \epsilon_k \mathcal{G}_k(t) - i \frac{g^2}{N} \sum_q \int_0^t d \tau \mathcal{G}_k(t) D_q^{(0)}(t-\tau) \frac{\mathcal{G}_{k-q}(t) \mathcal{G}_k(\tau)}{\mathcal{G}_{k-q}(\tau) \mathcal{G}_k(t)} \\
&=-i \epsilon_k \mathcal{G}_k(t) - i \frac{g^2}{N} \sum_q \int_0^t d \tau D_q^{(0)}(t-\tau) \frac{\mathcal{G}_k(\tau) \mathcal{G}_{k-q}(t)}{\mathcal{G}_{k-q}(\tau)}\\
\implies \frac{1}{\mathcal{G}_k(t)} \frac{d \mathcal{G}_k(t)}{d t}&={-i \epsilon_k} - i \frac{g^2}{N} \sum_q \int_0^t d \tau D_q^{(0)}(t-\tau) \frac{\mathcal{G}_{k-q}(t)}{\mathcal{G}_{k-q}(\tau)} \frac{\mathcal{G}_k(\tau)}{\mathcal{G}_k(t)} \label{a8} \\
\implies \frac{d \ln \mathcal{G}_k(t)}{d t}-\frac{d \ln \mathcal{G}_k^{(0)}(t)}{d t}&=- i \frac{g^2}{N} \sum_q \int_0^t d \tau D_q^{(0)}(t-\tau) \frac{\mathcal{G}_{k-q}(t)}{\mathcal{G}_{k-q}(\tau)} \frac{\mathcal{G}_k(\tau)}{\mathcal{G}_k(t)} \\
\implies \frac{d \ln \left[\mathcal{G}_k(t) / \mathcal{G}_k^{(0)}(t)\right]}{d t}&=- i \frac{g^2}{N} \sum_q \int_0^t d \tau D_q^{(0)}(t-\tau) \frac{\mathcal{G}_{k-q}(t)}{\mathcal{G}_{k-q}(\tau)} \frac{\mathcal{G}_k(\tau)}{\mathcal{G}_k(t)} \\
\implies ln \left[\mathcal{G}_k(t) / \mathcal{G}_k^{(0)}(t)\right] &=- i \frac{g^2}{N} \sum_q \int_0^t d \sigma \int_0^\sigma d \tau D_q^{(0)}(\sigma-\tau) \frac{\mathcal{G}_{k-q}(\sigma)}{\mathcal{G}_{k-q}(\tau)} \frac{\mathcal{G}_k(\tau)}{\mathcal{G}_k(\sigma)} \\
\implies \mathcal{G}_k(t) &=\mathcal{G}_k^{(0)}(t) \exp \left[i \sum_q \int_0^t d \sigma \int_0^\sigma d \tau\left|g_{q k}\right|^2 D_q^{(0)}(\sigma-\tau) \frac{\mathcal{G}_{k-q}(\sigma)}{\mathcal{G}_{k-q}(\tau)} \frac{\mathcal{G}_k(\tau)}{\mathcal{G}_k(\sigma)}\right] \\
\implies \mathcal{G}_k^{(n+1)}(t)
&=\mathcal{G}_k^{(0)}(t)
\exp\!\left[
i\!\sum_q
\int_0^t d\sigma\!\!\int_0^\sigma d\tau\,
\left|g_{q k}\right|^2 D_q^{(0)}(\sigma\!-\!\tau)
\frac{\mathcal{G}_{k-q}^{(n)}(\sigma)}{\mathcal{G}_{k-q}^{(n)}(\tau)}
\frac{\mathcal{G}_k^{(n)}(\tau)}{\mathcal{G}_k^{(n)}(\sigma)}
\right]
\end{align}
where the last equation admits a form for an iterative solution of this self-consistent equation. For $n=0$, we insert the bare Green's function, and get the second order cumulant,
\begin{align}
    \mathcal{G}_k^{(1)}(t)&=\mathcal{G}_k^{(0)}(t) \exp \left[i \sum_q \int_0^t d \sigma \int_0^\sigma d \tau\left|g_{q k}\right|^2 D_q^{(0)}(\sigma-\tau) \frac{\mathcal{G}_{k-q}^{(0)}(\sigma)}{\mathcal{G}_{k-q}^{(0)}(\tau)} \frac{\mathcal{G}_k^{(0)}(\tau)}{\mathcal{G}_k^{(0)}(\sigma)}\right] \\
&=\mathcal{G}_k^{(0)}(t) \exp \left[i \sum_q \int_0^t d \sigma \int_0^\sigma d \tau\left|g_{q k}\right|^2 D_q^{(0)}(\sigma-\tau) e^{-i\left(\epsilon_{k-q}-\epsilon_k\right)(\sigma-\tau)}\right] \\
&\equiv\mathcal{G}_k^{(0)}(t) \exp \left[C_2(k, t)\right]
\end{align}
For $n=1$, we have
\begin{align}
    \mathcal{G}_k^{(2)}(t)&=\mathcal{G}_k^{(0)}(t) \exp \left[i \sum_q \int_0^t d \sigma \int_0^\sigma d \tau\left|g_{q k}\right|^2 D_q^{(0)}(\sigma-\tau) \frac{\mathcal{G}_{k-q}^{(1)}(\sigma)}{\mathcal{G}_{k-q}^{(1)}(\tau)} \frac{\mathcal{G}_k^{(1)}(\tau)}{\mathcal{G}_k^{(1)}(\sigma)}\right] \label{iterative_step} \\
&=\mathcal{G}_k^{(0)}(t) \exp \left[i \sum_q \int_0^t d \sigma \int_0^\sigma d \tau\left|g_{q k}\right|^2 D_q^{(0)}(\sigma-\tau) \frac{\mathcal{G}_{k-q}^{(0)}(\sigma)}{\mathcal{G}_{k-q}^{(0)}(\tau)} \frac{\mathcal{G}_k^{(0)}(\tau)}{\mathcal{G}_k^{(0)}(\sigma)}\right. \notag \\
&\left.\times e^{C_2(k, \tau)+C_2(k-q, \sigma)-C_2(k, \sigma)-C_2(k-q, \tau)}\right] \\
&=\mathcal{G}_k^{(0)}(t) \exp \left[\sum_{n=0}^{\infty} \frac{-i}{n!} \sum_q \int_0^t d \sigma \int_0^\sigma d \tau \left|g_{q k}\right|^2 D_q^{(0)}(\sigma-\tau) \mathcal{G}_{k-q}^{(0)}(\sigma-\tau) \mathcal{G}_k^{(0)}(\tau-\sigma)\right. \\
& \left.\times\left[C_2(k, \tau)+C_2(k-q, \sigma)-C_2(k, \sigma)-C_2(k-q, \tau)\right]^n\right]\\
&=\mathcal{G}_k^{(0)}(t) \exp \left(\sum_{m=1}^{\infty} C_{2n}^{(2)}(k, t)\right)
\end{align}
where $C_{2n}^{(2)}(k, t)$ is the approximation to the $2n$th order cumulant produced by the second iteration.
Because the second iteration produces cumulants of all orders, further iterations require expanding the exponent into moments. For iterations $n \geq 2$, we use
\begin{align}
    \mathcal{G}_{\mathbf{k}}^{(n+1)}(t)= & \mathcal{G}_{\mathbf{k}}^{(0)}(t) \exp \left(-i \sum_{\mathbf{q}} \int_0^t d \sigma \int_0^\sigma d \tau\left|g_{q k}\right|^2\right. \\
& \left.\times D_{\mathbf{q}}^0(\sigma-\tau) \mathcal{G}_{\mathbf{k}}^{(0)}(\tau-\sigma) \mathcal{G}_{\mathbf{k}-\mathbf{q}}^{(0)}(\sigma-\tau) e^{F^{(n)}(\mathbf{k}, \mathbf{q}, \sigma, \tau)}\right) \notag
\end{align}
introducing the notation $F^{(n)}(\mathbf{k}, \mathbf{q}, \sigma, \tau) = F_2^{(n)}(\mathbf{k}, \mathbf{q}, \sigma, \tau)+F_4^{(n)}(\mathbf{k}, \mathbf{q}, \sigma, \tau)+\cdots$ where $F_i^{(n)}(\mathbf{k}, \mathbf{q}, \sigma, \tau) = C_i^{(n)}(\mathbf{k}-\mathbf{q}, \sigma)-C_i^{(n)}(\mathbf{k}-\mathbf{q}, \tau) -C_i^{(n)}(\mathbf{k}, \sigma)+C_i^{(n)}(\mathbf{k}, \tau)$.
The procedure for constructing the approximation to the $n$th iteration of the $m$ th cumulant is given by the moment expansion
$$
\sum_{m=0}^{\infty} \lambda^{2 m} W_{2 m}^{(n)}(\mathbf{k}, \mathbf{q}, \sigma, \tau)=e^{\sum_{m-1}^{\infty}, \lambda^{2 m}, F_{2 m}^{(2)}(\mathbf{k}, \mathbf{q}, \sigma, \tau)},
$$
where the first few moments are given by
$$
\begin{gathered}
W_0^{(n)}(\mathbf{k}, \mathbf{q}, \sigma, \tau)=1, \\
W_2^{(n)}(\mathbf{k}, \mathbf{q}, \sigma, \tau)=F_2^{(n)}(\mathbf{k}, \mathbf{q}, \sigma, \tau), \\
W_4^{(n)}(\mathbf{k}, \mathbf{q}, \sigma, \tau)=F_4^{(n)}(\mathbf{k}, \mathbf{q}, \sigma, \tau)+\frac{1}{2}\left(F_2^{(n)}(\mathbf{k}, \mathbf{q}, \sigma, \tau)\right)^2 .
\end{gathered}
$$

Because each iteration introduces another power of $g^2$ to all of the approximated cumulants, and the $n$ th-order cumulant is defined as being proportional to $g^n$, the $n$th iteration cannot alter any of the approximated cumulants of order lower than $n$. Because of this, and the fact that the approximate cumulant for a given order is an integral over lower-order approximations, the iterative method is actually recursive. Dropping the superscripts indicating iteration, the general form for the SC-CE approximation to the $n$th cumulant is
$$
\begin{aligned}
C_{2 n}^{\mathrm{SC}-\mathrm{CE}}(\mathbf{k}, t)= & -\left.i \sum_{\mathbf{q}} \int_0^t d \sigma \int_0^\sigma d \tau\left|g_{\mathbf{q}}\right|\right|^2 D_{\mathbf{q}}(\sigma-\tau) \\
& \times \mathcal{G}_{\mathbf{k}}^{(0)}(\tau-\sigma) \mathcal{G}_{\mathbf{k}-\mathbf{q}}^{(0)}(\sigma-\tau) W_{2 n-2}(\mathbf{k}, \mathbf{q}, \sigma, \tau)
\end{aligned}
$$

APPENDIX C: NUMERICAL VIDE SOLUTION
Equation (13) is not convenient for numerical calculations. Instead, we define $y_k(t)=e^{C_\lambda(t)}$ and solve directly for $y_k(t)$,
$$
\begin{aligned}
\frac{d y_k(t)}{d t}= & -i \sum_q \int_0^t d \tau\left|g_{q k}\right|^2 D_q^{(0)}(t-\tau) \\
& \times e^{d\left(\epsilon-\epsilon \epsilon_{k-q}\right)(t-\tau)} \frac{y_k(\tau) y_{k-q}(t)}{y_{k-q}(\tau)}
\end{aligned}
$$

Numerical solutions to Eq. (C1) were carried out with a multistep predictor corrector method as derived by Linz [55]. His method is general for first-order Volterra integrodifferential equations (VIDEs) of the form
$$
y^{\prime}(x)=F\left(x, y(x), \int_0^x d t K[x, t, y(t)]\right)
$$

The form of Eq. (Cl) implies that $F$ can be written purely as a function of $\int_0^x d t K[x, t, y(t)]$. We used an Adams-Bashforth second-order predictor step and a third-order Adams-Moulton corrector step. Following Linz, for the starting procedure of the Holstein model we applied the self-consistent Simpson's method iterated five times [55].

Because each iteration introduces another power of $g^2$ to all of the approximated cumulants, and the $n$ th-order cumulant is defined as being proportional to $g^{\prime \prime}$, the $n$th iteration cannot alter any of the approximated cumulants of order lower than $n$. Because of this, and the fact that the approximate cumulant for a given order is an integral over lower-order approximations, the iterative method is actually recursive. Dropping the superscripts indicating iteration, the general form for the SC-CE approximation to the $n$th cumulant is
$$
\begin{aligned}
C_{2 n}^{\mathrm{SCCE}}(\mathbf{k}, t)= & -i \sum_{\mathbf{q}} \int_0^t d \sigma \int_0^\sigma d \tau\left|g_{q \mathbf{k}}\right|^2 D_{\mathbf{q}}(\sigma-\tau) \\
& \times \mathcal{G}_{\mathbf{k}}^{(0)}(\tau-\sigma) \mathcal{G}_{\mathbf{k}-\mathbf{q}}^{(0)}(\sigma-\tau) W_{2 n-2}(\mathbf{k}, \mathbf{q}, \sigma, \tau)
\end{aligned}
$$
APPENDIX C: NUMERICAL VIDE SOLUTION
Equation (13) is not convenient for numerical calculations. Instead, we define $y_k(t)=e^{C_\lambda(t)}$ and solve directly for $y_k(t)$,
$$
\begin{aligned}
\frac{d y_k(t)}{d t}= & -i \sum_q \int_0^t d \tau\left|g_{q k}\right|^2 D_q^{(0)}(t-\tau) \\
& \times e^{i(\epsilon-\epsilon-\epsilon-q)(t-\tau)} \frac{y_k(\tau) y_{k-q}(t)}{y_{k-q}(\tau)}
\end{aligned}
$$

Numerical solutions to Eq. (C1) were carried out with a multistep predictor corrector method as derived by Linz [55]. His method is general for first-order Volterra integrodifferential equations (VIDEs) of the form
$$
y^{\prime}(x)=F\left(x, y(x), \int_0^x d t K[x, t, y(t)]\right)
$$

The form of Eq. (Cl) implies that $F$ can be written purely as a function of $\int_0^x d t K[x, t, y(t)]$. We used an Adams-Bashforth second-order predictor step and a third-order Adams-Moulton corrector step. Following Linz, for the starting procedure of the Holstein model we applied the self-consistent Simpson's method iterated five times [55].
\section{My derivations}
\subsection{SC-CE}

\subsubsection{My derivation}
 The cumulant ansatz for the retarded Green's function is:
\begin{equation}
    G^R(t)=\underbrace{-i \theta(t) e^{-i \epsilon_0 t}}_{G_0^R(t)} e^{C^R(t)} .
\end{equation}
Note that because this is for the retarded Green's function, we can assume that $t>0$ when differentiating, so $\theta(t)=1$. 
Differentiating with respect to time gives:
\begin{align}
    \partial_t G^R(t)&=-i\partial_t\left(e^{-i \epsilon_0 t} e^{C^R(t)}\right)\\
    &= -i \epsilon_0 G^R(t) - i \dot{C}^R(t) G^R(t)
\end{align}
% Notice that a product in the time domain is equivalent to a convolution integral in the frequency domain, so $\dot{C}^R(t) G^R(t) = G^R(t)\dot{C}^R(t) = \int_{-\infty}^\infty d \tau G^R\left(t-\tau\right) \dot{C}^R\left(\tau\right) = \int_0^t d \tau G^R\left(t-\tau\right) \dot{C}^R\left(\tau\right)$. We have been able to change the bounds of the integral from $[-\infty, \infty]\rightarrow [0,t]$ because we are dealing with retarded quantities, i.e. $\dot{C}^R\left(\tau\right) = 0 \quad \forall\quad  \tau<0$ and $G^R\left(t-\tau\right) = 0 \quad \forall\quad  \tau>t$.
%  Plugging this in gives:
% \begin{equation}
%     \partial_t G^R(t)=-i \epsilon_0 G^R(t)-\int_0^t d \tau G^R\left(t-\tau\right) \dot{C}^R\left(\tau\right) .
% \end{equation}
 Now, the equation of motion
\footnote{
The Nakajima-Zwanzig equation for a correlation function $\mathcal{A}(t)$ is given by
\begin{equation}
\dot{\mathcal{A}}(t) = \mathcal{A}(t) {\Omega_1} - \int_{0}^{t} d \tau\, \mathcal{A}(t - \tau) \mathcal{K}_1(\tau) + D(t)
\label{eq:GQME}
\end{equation}
Since we know the connection $\int_{0}^{t} d \tau\, \mathcal{A}(t - \tau) \mathcal{K}_1(\tau) = i \dot{C}^R(t) G^R(t) = \int_0^t d \tau G^R(t-\tau) \Sigma^R(\tau)$ and the outer definitions involve convolutions, we might be able to make some progress by demanding equality of their Laplace transforms.}
for the retarded Green's function in the Dyson formulation with the self energy $\Sigma$ is given by
\begin{align}
 \partial_t G^R(t)&=-i \epsilon _0 G^R(t)-i \int_0^t d \tau G^R(t-\tau) \Sigma^R(\tau)
\end{align}
% Equating the two gives an expression for the cumulant derivative:
% \begin{align}
% \dot{C}^R(t)G^R(t) &= \int_0^t d \tau G^R(t-\tau) \Sigma^R(\tau)\\
% &= \int_0^t d \tau G^R(t-\tau) \left[\left(G_0^R(\tau)\right)^{-1} - \left(G^R(\tau)\right)^{-1} \right] \\
% \implies \dot{C}^R(t)&=\int_0^t d \tau G^R(t-\tau) \left[\left(G_0^R(\tau)\right)^{-1} - \left(G^R(\tau)\right)^{-1} \right] \left(G^R(t)\right)^{-1} \\
% &=\int_0^t d \tau G^R(t-\tau) \left(G_0^R(\tau)\right)^{-1} \left(G^R(t)\right)^{-1} - \int_0^t d \tau G^R(t-\tau) \left(G^R(\tau)\right)^{-1} \left(G^R(t)\right)^{-1} \\
% \end{align}
Equating the two gives an expression for the cumulant derivative:
\begin{align}
\dot{C}^R(t)G^R(t) = \int_0^t d \tau G^R(t-\tau) \Sigma^R(\tau)
\end{align}
From here there are two directions I have in mind. Firstly, the Kowalski paper gave the EOM for the retarded core-hole cumulant as $-i\frac{d C_c^R(t)}{dt} = \left< \phi \left| \bar{H}_N(t) \right| \phi \right> = \sum_{ia} f_{ia} t_i^a + \frac{1}{2} \sum_{ijab} v_{ij}^{ab} t_j^b t_i^a$. I could plug this in and see. Secondly, we also know that 
\begin{align}
i \dot{C}^R(t) G^R(t) &= \int_{0}^{t} d \tau\, \mathcal{A}(t - \tau) \mathcal{K}_1(\tau)  \\  
\implies \mathcal{L}\left\{i \dot{C}^R(t) G^R(t)\right\}(s) &= \mathcal{L}\left\{\int_{0}^{t} d \tau\, \mathcal{A}(t - \tau) \mathcal{K}_1(\tau)\right\}(s)
\end{align}
where the RHS comes from the Nakajima-Zwanzig equation \eqref{eq:GQME} and $\mathcal{L}\left\{i \dot{C}^R(t) G^R(t)\right\}(s) $ has a known analytical form.
\footnote{
\begin{align}
\mathcal{L}\left\{\dot{C}^R(t)G^R(t)\right\}(s)
&=\int_0^\infty dt\, e^{-s t} \dot{C}^R(t) \left(-i e^{-i \epsilon_0 t} e^{C^R(t)}\right) \\
&= -i \int_0^\infty e^{-t(s + i\epsilon_0)} \frac{d}{dt}\big(e^{C^R(t)}\big)\, dt \label{eq:integration_by_parts}\\
&= i - i(s + i\epsilon_0)\int_0^\infty e^{-t(s + i\epsilon_0)} e^{C^R(t)}\,dt \label{id}\\
&= i - i(s + i\epsilon_0)\mathcal{L}\{G^R(t)\}(s)
\end{align}
where to proceed from \eqref{eq:integration_by_parts} we take $u' = \frac{d}{dt}\big(e^{C^R(t)}\big) \implies u = e^{C^R(t)}$ and $v = e^{-t(s + i\epsilon_0)} \implies v' = - (s + i\epsilon_0)e^{-t(s + i\epsilon_0)}$, so that
\begin{align}
\int_0^\infty e^{-t(s + i\epsilon_0)} \frac{d}{dt}e^{C^R(t)}\,dt
&= \left[e^{-t(s + i\epsilon_0)} e^{C^R(t)}\right]_0^\infty 
+ (s + i\epsilon_0)\int_0^\infty e^{-t(s + i\epsilon_0)} e^{C^R(t)}\,dt \\
&= -1 + (s + i\epsilon_0)\int_0^\infty e^{-t(s + i\epsilon_0)} e^{C^R(t)}\,dt
\end{align}
and then in \eqref{id} we recognize 
\begin{align}
\tilde{G}^R(s)&=\int_0^\infty dt\, e^{-s t} \left(-i e^{-i \epsilon_0 t} e^{C^R(t)}\right) \\
&=-i \int_0^\infty dt\, e^{-t\left(s+i \epsilon_0\right)} e^{C^R(t)}
\end{align}
}
. This Laplace transform might be promising because the RHS is a convolution ($(f*g)(t)=\int_0^t d \tau f(t-\tau) g(\tau)$, with $f(t)=G^R(t)$ and $g(t)=\Sigma^R(t)$) and we know that there is a nice formula for the Laplace transform of a convolution as $\mathcal{L}\{(f*g)(t)\}(s)=\mathcal{L}\{f(t)\}(s) \mathcal{L}\{g(t)\}(s)$. From here on out, we will interchange with the notation that $\mathcal{L}\{f(t)\}(s)=\tilde{f}(s)$. 
Then
\begin{align}
\mathcal{L}\left\{\text{RHS}\right\}(s) &=\tilde{G}^R(s) \tilde{\Sigma}^R(s).
\end{align}
We already know the form of $\tilde{G}^R(s)$, but we need to find $\tilde{\Sigma}^R(s)$. We can start by taking the Laplace transform of $\Sigma^R(t)$:
\begin{align}
\mathcal{L}\{\Sigma^R(t)\}(s)&=\int_0^\infty dt\, e^{-s t} \Sigma^R(t) \\
&=\int_0^\infty dt\, e^{-s t} \left[ \left(G_0^R(t)\right)^{-1} - \left(G^R(t)\right)^{-1} \right]
\end{align}
Given that it is not simple to invert the Green's function analytically in the time domain, we can stop here. \color{red}{But perhaps you can think of certain limits where this inversion is possible to do numerically?} \color{black}Setting $\mathcal{L}\left\{\text{LHS}\right\}(s)=\mathcal{L}\left\{\text{RHS}\right\}(s)$, we have
\begin{align}
i - i(s + i\epsilon_0)\tilde{G}^R(s) &= \tilde{G}^R(s) \tilde{\Sigma}^R(s) \\
\implies \tilde{\Sigma}^R(s)
&= \frac{i}{\tilde{G}^R(s)} - i(s + i\epsilon_0)
\quad\Longleftrightarrow\quad
\tilde{G}^R(s)
= \frac{i}{\tilde{\Sigma}^R(s) + i(s + i\epsilon_0)}. \label{eq:boxed}
\end{align}
So the relationship of $\tilde{G}^R(s)$ to the $C^R(t) $ becomes clear if we define
\begin{align}
&F(z) \equiv \mathcal{L}\{e^{C^R(t)}\}(z)
= \int_0^\infty e^{-z t}\,e^{C^R(t)}\,dt\\
\implies \tilde{G}^R(s) &= -i \int_0^\infty dt\, e^{-t(s + i\epsilon_0)} e^{C^R(t)} = -i F(s + i\epsilon_0)
\end{align}
So with
$z = s + i\epsilon_0$ in \eqref{eq:boxed}, we have
\begin{align}
&-i\,F(z)
= \frac{i}{\tilde{\Sigma}^R(z - i\epsilon_0) + i z}\\
&\implies 
F(z)
= \mathcal{L}\{e^{C^R(t)}\}(z)
= -\,\frac{1}{\tilde{\Sigma}^R(z - i\epsilon_0) + i z}\\
& \implies
C^R(t)
= \ln\!\left[
\mathcal{L}^{-1}_z
\left\{
-\frac{1}{\tilde{\Sigma}^R(z - i\epsilon_0) + i z}
\right\}(t)
\right],
\end{align}
where $\mathcal{L}^{-1}_z$ is the inverse Laplace transform with respect to $z$.

If we pass to the frequency domain by setting
$z = \eta - i\omega$ (with $\eta \to 0^+$),
the boxed relation becomes equivalent to the familiar Dyson form:
\begin{align}
G^R(\omega)
= \frac{1}{\omega - \epsilon_0 - \Sigma^R(\omega)}.
\end{align}
Thus, the Laplace-space identity is the causal/Laplace version of
Dyson’s equation combined with the cumulant ansatz.

Under the usual GW+cumulant approximations,
one expands around the quasiparticle pole and rewrites
$C^R(t)$ in terms of the satellite spectral weight:
\begin{align}
C^R(t)
= \int_{-\infty}^{\infty}
\frac{d\omega}{\pi}\,
\frac{\Im\,\Sigma^R(\epsilon_0 + \omega)}{\omega^2}
\big(e^{-i\omega t} + i\omega t - 1\big),
\end{align}
which can be recovered by (i) expressing the inverse Laplace kernel
via a spectral representation of $\Sigma^R$, and
(ii) performing the Bromwich integral—this uses analyticity and
retarded boundary conditions.
The result is the well-known closed form used for plasmon satellites
and shake-up structures.

\subsection*{Practical use}

The compact identity
\begin{align}
F(z)
= -\frac{1}{\tilde{\Sigma}^R(z - i\epsilon_0) + i z}
\end{align}
is a convenient starting point for controlled approximations:
\begin{itemize}
  \item \textbf{Low-$t$ expansion:} expand the denominator for large $z$
        to obtain short-time moments of $C^R(t)$.
  \item \textbf{Long-$t$ (satellite) behavior:} analyze the singularity
        structure of the denominator (branch cuts from $\Sigma^R$)
        to extract asymptotics and satellite weights.
\end{itemize}

Maybe the best thing now would be to start with the Landau form of the cumulant and to simplify the notation it is assumed that we deal with retarded quantities only.
\begin{equation}
    C(t)=\int d \omega \frac{\beta(\omega)}{\omega^2}\left[e^{-i \omega t}+i \omega t-1\right]
\end{equation}
where the cumulant kernel is defined as
\begin{align}
    \beta(\omega)&=-\frac{1}{\pi} \operatorname{Im} \Sigma(\omega) \\
    &=\frac{1}{\pi} \operatorname{Im} \left(G_0^{-1}(\omega) - G^{-1}(\omega) \right) \\
\end{align}
 If we partition into a physical and auxiliary space, denoted by $\mathcal{S}$ and $\mathcal{A}$, respectively, $\bm{G}_0(\omega)= \begin{pmatrix} \bm{G}^0_{\mathcal{S}\mathcal{S}}(\omega) & 0 \\ 0 & \bm{G}^0_{\mathcal{A}\mathcal{A}}(\omega) \end{pmatrix}$ and $\bm{G}(\omega)= \begin{pmatrix} \bm{G}_{\mathcal{S}\mathcal{S}}(\omega) & \bm{G}_{\mathcal{S}\mathcal{A}}(\omega) \\ \bm{G}_{\mathcal{A}\mathcal{S}}(\omega) & \bm{G}_{\mathcal{A}\mathcal{A}}(\omega) \end{pmatrix}$, and plug in to get an expression for the cumulant kernel in the partitioned space:
\begin{align}
    \beta_{S S}(\omega)&=\frac{1}{\pi} \operatorname{Im}\left[\mathbf{G}_{S S}^{0,-1}(\omega)-\left(\mathbf{G}_{S S}(\omega)-\mathbf{G}_{S \mathcal{A}}(\omega) \mathbf{G}_{\mathcal{A} \mathcal{A}}^{-1}(\omega) \mathbf{G}_{\mathcal{A} S}(\omega)\right)^{-1}\right]
\end{align}
I don't know if this form is useful.
%     \frac{G^R(t-\tau)}{G^R(t)}=e^{i \epsilon_0 \tau} e^{C^R(t-\tau)-C^R(t)} = e^{i \epsilon_0 \tau} e^{-\int_{t-\tau}^t d t' \dot{C}^R(t')} .
% \end{equation}
% So the final expression for the cumulant derivative is a nonlinear Volterra integro-differential equation (VIDE):
% \begin{align}
%     \dot{C}^R(t)&=\int_0^t d \tau \Sigma^R(\tau) e^{i \epsilon_0 \tau} e^{-\int_{t-\tau}^t d t' \dot{C}^R(t')} \\
%     &=i\int_0^t d \tau \int_0^{\infty} d \tau' \left[G^R(\tau - \tau') W^{>}(\tau')+G^{<}(\tau-\tau') W^R(\tau')\right] e^{i \epsilon_0 \tau} e^{-\int_0^\tau d t' \dot{C}^R(t')} 
% \end{align}
% A VIDE also appeared in PJ's SC-CE and they solved it using differential equation solvers in the time domain.
\subsubsection{One-shot cumulant}
In the first iteration, we set the exponential factor to 1, which gives the time-derivative of the one-shot cumulant:
\begin{equation}
    \dot{C}^{R}_0(t)=\int_0^t d \tau \Sigma^R(\tau) e^{i \epsilon_0 \tau} .
\end{equation}
Note that we can express the second-derivative of the one-shot retarded cumulant as
\begin{equation}
    \ddot{C}^{R}_0= \Sigma^R(t) e^{i \epsilon_0 t} .
\end{equation}
We integrate with respect to time to get the cumulant itself:
\begin{equation}
    C^{R}_0(t)=\int_0^t d t' \int_0^{t'} d \tau \Sigma^R(\tau) e^{i \epsilon_0 \tau} = \int_0^t d \tau \Sigma^R(\tau) e^{i \epsilon_0 \tau} (t-\tau) .
\end{equation}
Now, we perform a Fourier transform to get the frequency-domain cumulant:
\begin{align}
    C^{R}_0(\omega) &= \int_0^\infty dt\, e^{i\omega t} \int_0^t d\tau\, \Sigma^R(\tau) e^{i\epsilon_0 \tau} (t-\tau) \\
    &= \int_0^\infty d\tau\, \Sigma^R(\tau) e^{i\epsilon_0 \tau} \int_\tau^\infty dt\, e^{i\omega t} (t-\tau)
\end{align}
Let $t' = t - \tau$, so $dt = dt'$, and when $t = \tau$, $t' = 0$, and as $t \to \infty$, $t' \to \infty$:
\begin{align}
    \int_\tau^\infty dt\, e^{i\omega t} (t-\tau) &= \int_0^\infty dt'\, e^{i\omega (t'+\tau)} t' \\
    &= e^{i\omega \tau} \int_0^\infty dt'\, e^{i\omega t'} t'
\end{align}
We can evaluate the integral over $t'$ as
\begin{align}
    \int_0^\infty dt'\, e^{i\omega t'} t' & = -\frac{1}{(\omega + i\eta)^2}
\end{align}
where where $\eta$ is the positive infinitesimal convergence factor.

Putting it all together,
\begin{align}
    C^{R}_0(\omega) &= \int_0^\infty d\tau\, \Sigma^R(\tau) e^{i(\omega+\epsilon_0)\tau} \left(-\frac{1}{(\omega + i\eta)^2}\right) \\
    &= -\frac{1}{(\omega + i\eta)^2} \int_0^\infty d\tau\, \Sigma^R(\tau) e^{i(\omega+\epsilon_0)\tau} \\
    &= -\frac{\Sigma^R(\omega+\epsilon_0)}{(\omega + i\eta)^2} 
\end{align}
Now, consider the inverse Fourier transform to get back to the time domain:
\begin{align}
    C^{R}_0(t) &= \int_{-\infty}^{\infty} d\omega\, e^{-i\omega t} C^{R}_0(\omega) \\
    &= \int_{-\infty}^{\infty} d\omega\, e^{-i\omega t} \left(-\frac{\Sigma^R(\omega+\epsilon_0)}{(\omega + i\eta)^2}\right)
\end{align}
\subsection{GW+C}
\subsubsection{Using exact self-energy}
We know that $\Sigma = G_0^{-1} - G^{-1}$, and inserting this gives\\ $G_0(\omega) \Sigma(\omega) G_0(\omega) = G_0(\omega) \left(G_0^{-1}(\omega) - G^{-1}(\omega)\right) G_0(\omega) = G_0(\omega) - G_0(\omega) G^{-1}(\omega) G_0(\omega)$
\begin{align}
	C_{pq}(t) &= i \int \frac{d\omega}{2\pi} e^{-i(\omega-\epsilon_p^{HF})t} \left[G_0(\omega) \Sigma(\omega) G_0(\omega)\right]_{pq} \\
&= i \int \frac{d\omega}{2\pi} e^{-i(\omega-\epsilon_p^{HF})t} \left[G_0(\omega) - G_0(\omega) G^{-
1}(\omega) G_0(\omega)\right]_{pq} \\
&= i \int \frac{d\omega}{2\pi} e^{-i(\omega-\epsilon_p^{HF})t} \left[\frac{1}{\omega - \epsilon_p^{HF} + i\eta} - \left[G_0(\omega) G^{-1}(\omega) G_0(\omega)\right]_{pq}\right] \\
&= \theta(t) - i \int \frac{d\omega}{2\pi} e^{-i(\omega-\epsilon_p^{HF})t}  \left[G_0(\omega) G^{-1}(\omega) G_0(\omega)\right]_{pq}
\label{restart} \\
&= i \int \frac{d\omega}{2\pi}  \left[\frac{e^{-i(\omega-\epsilon_p^{HF})t}}{\omega - \epsilon_p^{HF} + i\eta} - \frac{e^{-i(\omega-\epsilon_p^{HF})t}G^{-1}_{pq}(\omega)}{(\omega - \epsilon_p^{HF} + i\eta)(\omega - \epsilon_q^{HF} + i\eta)}\right] \\
&= \theta  (t) - i \int \frac{d\omega}{2\pi}  \frac{e^{-i(\omega-\epsilon_p^{HF})t}G^{-1}_{pq}(\omega)}{(\omega - \epsilon_p^{HF} + i\eta)(\omega - \epsilon_q^{HF} + i\eta)} 
\end{align}
Now we plug this into the ansatz for the retarded Green's function:
\begin{align}
    G_{pq}(t) &= G_{pp}^{HF}(t) e^{C_{pq}(t)} \\
&= -i \theta(t) e^{- i \epsilon_p^{HF} t} \exp \left[\theta(t) - i \int \frac{d\omega}{2\pi}  \frac{e^{-i(\omega-\epsilon_p^{HF})t}G^{-1}_{pq}(\omega)}{(\omega - \epsilon_p^{HF} + i\eta)(\omega - \epsilon_q^{HF} + i\eta)}\right] \\
&\approx -i \theta(t) e^{- i \epsilon_p^{HF} t} \left[1 - i \int \frac{d\omega}{2\pi}  \frac{e^{-i(\omega-\epsilon_p^{HF})t}G^{-1}_{pq}(\omega)}{(\omega - \epsilon_p^{HF} + i\eta)(\omega - \epsilon_q^{HF} + i\eta)}\right] 
\end{align}
This doesn't seem useful, so we can start again with eqn.~\ref{restart}.
We can use a projection technique and split into a HF space where the projector is defined as $\hat{P}=\sum_p^{\text{HF}}\ket{p}\bra{p}$, which we can identify with $\begin{pmatrix}1 \\ 0\end{pmatrix}$ and the rest space with projector $\hat{Q}=1-\hat{P}$, which we can identify with $\begin{pmatrix}0 \\ 1\end{pmatrix}$. Then, using the fact that $G^{-1} = G_0^{-1} - \Sigma$ and assuming the HF reference, we can write:
\begin{align}
    G^{-1}_{pq}(\omega) 
&= \left[\hat{P} G^{-1}(\omega) \hat{P} + \hat{P} G^{-1}(\omega) \hat{Q} + \hat{Q} G^{-1}(\omega) \hat{P} + \hat{Q} G^{-1}(\omega) \hat{Q}\right]_{pq} \\
 &= \left(\left[\omega - \epsilon_p^{HF} \right] \delta _{pq}- \Sigma_{pq}(\omega)\right) \hat{P}\otimes \hat{P} - \left(\Sigma^c_{pQ}\right) \hat{P}\otimes \hat{Q} - \left(\Sigma^c_{Pq}\right) \hat{Q}\otimes \hat{P}\\
& + \left( (\omega - \epsilon_P^{HF}) \delta _{PQ} - \Sigma_{PQ}(\omega)\right) \hat{Q}\otimes \hat{Q} \\
&= \begin{pmatrix} \left[\omega - \epsilon_p^{HF} \right] \delta _{pq}- \Sigma_{pq}(\omega) & -\Sigma^c_{pQ} \\ -\Sigma^c_{Pq} & (\omega - \epsilon_P^{HF}) \delta _{PQ} - \Sigma_{PQ}(\omega) \end{pmatrix} \\
\implies G _{pq}(\omega)&=  
% \\
% &= \left(\left[\omega - \epsilon_p^{HF} \right] \delta _{pq}- \Sigma_{pq}(\omega)\right)\begin{pmatrix} 1 & 0 \\ 0 & 0 \end{pmatrix} - \left(\Sigma^c_{pQ}\right) \begin{pmatrix} 0 & 0 \\ 0 & 1 \end{pmatrix} - \left(\Sigma^c_{Pq}\right) \begin{pmatrix} 1 & 0 \\ 0 & 0 \end{pmatrix} + \left( (\omega - \epsilon_P^{HF}) \delta _{PQ} - \Sigma_{PQ}(\omega)\right) \begin{pmatrix} 0 & 0 \\ 0 & 1 \end{pmatrix}
\end{align}
where lowercase letters indicate indices in the physical space, while uppercase letters indicate indices in the rest space.
\subsubsection{Off-diagonal cumulant with GW self-energy}
We can start with
\begin{align}
		C_{pq}(t) &= i \int \frac{d\omega}{2\pi} e^{-i(\omega-\epsilon_p^{HF})t} G_{pp}^{HF}(\omega) \Sigma_{pq}^c(\omega) G_{qq}^{HF}(\omega) \\
&= i \int \frac{d\omega}{2\pi} e^{-i(\omega-\epsilon_p^{HF})t} \frac{\Sigma_{pq}^c(\omega)}{(\omega - \epsilon_p^{HF} + i\eta)(\omega - \epsilon_q^{HF} + i\eta)} 
\label{eq:Cpp_m1}\\
\end{align}
At this point we make the frequency shift $\omega \to \omega + \epsilon_p^{HF}$, and then we can write
\begin{align}
    C_{pq}(t) &= i \int \frac{d\omega}{2\pi} e^{-i\omega t} \frac{\Sigma_{pq}^c(\omega + \epsilon_p^{HF})}{(\omega + i\eta)(\omega + \underbrace{\epsilon_p^{HF} - \epsilon_q^{HF}}_{\Delta_{pq}} + i\eta)}
\end{align}
Now using the partial fractions we can write $\frac{1}{(\omega + i\eta)(\omega + \Delta_{pq} + i\eta)} = \frac{1}{\Delta_{pq}}\left(\frac{1}{\omega + i\eta} - \frac{1}{\omega + \Delta_{pq} + i\eta}\right)$ and we can plug in the full form for the retarded GW self-energy as 
\begin{equation}
    \Sigma_{pq}^{c, \mathrm{G}_0 \mathrm{~W}_0}(\omega)= \sum_{i \nu}\left[W_{p i \nu} \frac{1}{\omega-\left(\epsilon_i-\Omega_\nu\right)+\mathrm{i} \eta} W_{q i \nu}\right]+ \sum_{a \nu}\left[W_{p a \nu} \frac{1}{\omega-\left(\epsilon_a+\Omega_\nu\right)+\mathrm{i} \eta} W_{q a \nu}\right]
\end{equation}
to get
\begin{align}
    C_{pq}(t) &= \frac{i}{\Delta_{pq}} \int \frac{d\omega}{2\pi} e^{-i\omega t} \left[ \sum_{i \nu} \frac{W_{p i \nu} W_{q i \nu}}{\omega+\epsilon_p-\left(\epsilon_i-\Omega_\nu\right)+\mathrm{i} \eta} + \sum_{a \nu}  \frac{W_{p a \nu} W_{q a \nu}}{\omega+\epsilon_p-\left(\epsilon_a+\Omega_\nu\right)+\mathrm{i} \eta} \right] \\
&\left(\frac{1}{\omega + i\eta} - \frac{1}{\omega + \left( \epsilon_p - \epsilon_q\right) + i \eta}\right) \notag \\ 
& = i \sum_{i \nu} \frac{W_{p i \nu} W_{q i \nu}}{\Delta_{pq}} \int \frac{d\omega}{2\pi} e^{-i\omega t} \times \\
&\left(\frac{1}{\left(\omega+\epsilon_p-\left(\epsilon_i-\Omega_\nu\right)+\mathrm{i} \eta\right)\left(\omega + i\eta\right)} - \frac{1}{\left(\omega+\epsilon_p-\left(\epsilon_i-\Omega_\nu\right)+\mathrm{i} \eta\right)\left(\omega + \left( \epsilon_p - \epsilon_q\right) + i\eta\right)}\right) \notag \\
& + i \sum_{a \nu} \frac{W_{p a \nu} W_{q a \nu}}{\Delta_{pq}} \int \frac{d\omega}{2\pi} e^{-i\omega t} \times \notag \\
&\left(\frac{1}{\left(\omega+\epsilon_p-\left(\epsilon_a+\Omega_\nu\right)+\mathrm{i} \eta\right)\left(\omega + i\eta\right)} - \frac{1}{\left(\omega+\epsilon_p-\left(\epsilon_a+\Omega_\nu\right)+\mathrm{i} \eta\right)\left(\omega + \left( \epsilon_p - \epsilon_q\right) + i\eta\right)}\right) \notag
\end{align}
To evaluate these contour integrals, recall that $\oint d\omega f(\omega) = \int_{-\infty}^{\infty} d\omega f(\omega) + \int_{\text{arc}} d\omega f(\omega)$, so in order to equate the real frequency integral to the contour integral, which will then allow us to use the residue theorem, we need to ensure that the integral over the arc vanishes. This only will happen (due to Jordan's Lemma) if the numerator $e^{-i\omega t}$ vanishes for $\omega \rightarrow \infty$ and $\omega \rightarrow -\infty$; because we have $t>0$, this is only true if $\operatorname{Im}(\omega) < 0$, and so we must close the contour in the lower half plane. 
\begin{align}
    &\int \frac{d\omega}{2\pi} e^{-i\omega t} \frac{1}{\left(\omega+\epsilon_p-\left(\epsilon_i-\Omega_\nu\right)+\mathrm{i} \eta\right)\left(\omega + i\eta\right)} = i\left(\frac{e^{-i(\epsilon_i - \Omega_\nu-\epsilon_p)t} - 1}{\epsilon_i - \Omega_\nu-\epsilon_p}\right) \\
    &\int \frac{d\omega}{2\pi} e^{-i\omega t} \frac{1}{\left(\omega+\epsilon_p-\left(\epsilon_i-\Omega_\nu\right)+\mathrm{i} \eta\right)\left(\omega + \Delta_{pq} + i\eta\right)} = i\left(\frac{e^{i(\Delta_{pq})t} - e^{-i(\epsilon_i - \Omega_\nu-\epsilon_p)t}}{-\epsilon_i + \Omega_\nu+\epsilon_p - \Delta_{pq}}\right) \\
&= i\left(\frac{e^{i(\epsilon_p - \epsilon_q)t} - e^{-i(\epsilon_i - \Omega_\nu-\epsilon_p)t}}{-\epsilon_i + \Omega_\nu+\epsilon_q}\right) \notag \\
    &\int \frac{d\omega}{2\pi} e^{-i\omega t} \frac{1}{\left(\omega+\epsilon_p-\left(\epsilon_a+\Omega_\nu\right)+\mathrm{i} \eta\right)\left(\omega + i\eta\right)} = i\left(\frac{e^{-i(\epsilon_a + \Omega_\nu-\epsilon_p)t} - 1}{\epsilon_a + \Omega_\nu-\epsilon_p}\right) \\
    &\int \frac{d\omega}{2\pi} e^{-i\omega t} \frac{1}{\left(\omega+\epsilon_p-\left(\epsilon_a+\Omega_\nu\right)+\mathrm{i} \eta\right)\left(\omega + \Delta_{pq} + i\eta\right)} = i\left(\frac{e^{i(\Delta_{pq})t} - e^{-i(\epsilon_a + \Omega_\nu-\epsilon_p)t}}{-\epsilon_a - \Omega_\nu+\epsilon_p - \Delta_{pq}}\right) \\
&= i\left(\frac{e^{i(\epsilon_p - \epsilon_q)t} - e^{-i(\epsilon_a + \Omega_\nu-\epsilon_p)t}}{-\epsilon_a - \Omega_\nu+\epsilon_q}\right) \notag
\end{align}
To ease the notation, we can introduce $\Xi_{i\nu} \equiv \epsilon_i - \Omega_\nu$, $\Xi_{a\nu} \equiv \epsilon_a + \Omega_\nu$. Then we can write
\begin{tcolorbox}
As a sanity check at this point, we can consider $\lim_{q \to p} T_{i\nu }(\Delta )$ and see if we recover the diagonal result. We can start by defining $f(\Delta ) = \frac{e^{-i\left(\Xi_{i\nu} - \epsilon_p\right) t}-1 }{\Xi_{i\nu}-\epsilon_p} + \frac{e^{i\Delta t} - e^{-i\left(\Xi_{i\nu} - \epsilon_p\right) t}}{\Xi_{i\nu} - \epsilon_q}$ and $g(\Delta  ) = \Delta$, then $\lim_{q \to p} T_{i\nu}(\Delta ) = \lim_{q \to p}\frac{f(\Delta )}{g(\Delta )} = \frac{0}{0}$. So we can use L'Hôpital's rule to get $\lim_{q \to p} T_{i\nu}(\Delta ) = \lim_{q \to p}\frac{f'(\Delta)}{g'(\Delta)} = \lim_{\Delta \to 0} \frac{f'(\Delta)}{1} = \lim_{\Delta \to 0} \frac{\left(i t e^{i \Delta t}\right)\left(\Xi_{i\nu} - \epsilon_p+\Delta\right)-\left(e^{i \Delta t}-e^{-i (\Xi_{i\nu} - \epsilon_p) t}\right)}{\left(\Xi_{i\nu} - \epsilon_p+\Delta\right)^2} = \frac{i \left(\Xi_{i\nu} - \epsilon_p\right) t -1 +e^{-i (\Xi_{i\nu} - \epsilon_p) t }}{\left(\Xi_{i\nu} - \epsilon_p\right)^2} $, which is indeed the diagonal result.
\end{tcolorbox}
% Then
% $$
% N^{\prime}(\Delta)=\frac{\left(i t e^{i \Delta t}\right)(A+\Delta)-\left(e^{i \Delta t}-e^{-i A t}\right)}{(A+\Delta)^2} .
% $$

% Evaluate at $\boldsymbol{\Delta}=\mathbf{0}$ :
% $$
% N^{\prime}(0)=\frac{i t A-\left(1-e^{-i A t}\right)}{A^2} .
% $$
\begin{align}
C_{pq}(t) &=  i^2\left[ \sum_{i\nu} W_{p i \nu} W_{q i \nu} \left(\underbrace{\frac{1}{\Delta}\left[\frac{e^{-i\left(\Xi_{i\nu} - \epsilon_p\right) t}-1 }{\Xi_{i\nu}-\epsilon_p} + \frac{e^{i\Delta t} - e^{-i\left(\Xi_{i\nu} - \epsilon_p\right) t}}{\Xi_{i\nu} - \epsilon_q}\right] }_{T_{i\nu}(t)}\right)\right.\\
& \left. + \sum_{a\nu} W_{p a \nu} W_{q a \nu} 
\left(\underbrace{\frac{1}{\Delta}\left[{\frac{e^{-i\left(\Xi_{a\nu} - \epsilon_p\right) t}-1 }{\Xi_{a\nu}-\epsilon_p} + \frac{e^{i\Delta t} - e^{-i\left(\Xi_{a\nu} - \epsilon_p\right) t}}{\Xi_{a\nu} - \epsilon_q}}\right]}_{T_{a\nu}(t)}\right) \right] \notag \\
\end{align}
Now we just work with
\begin{align}
    T_{i\nu}(t) &= \frac{ \left( \Xi_{i\nu} - \epsilon_q \right) \left[ e^{-i(\Xi_{i\nu} - \epsilon_p)t} - 1  \right] + \left( \Xi_{i\nu} - \epsilon_p \right) \left[ e^{i\left(\epsilon_p - \epsilon_q\right)t} - e^{-i(\Xi_{i\nu} - \epsilon_p)t} \right] }{ (\epsilon_p - \epsilon_q)(\Xi_{i\nu} - \epsilon_p)(\Xi_{i\nu} - \epsilon_q) } \\
&= \frac{- \left( \Xi_{i\nu} - \epsilon_q \right) + \left( \Xi_{i\nu} - \epsilon_p \right) e^{i\left(\epsilon_p - \epsilon_q\right)t} + \left( \epsilon_p - \epsilon_q - 2\Xi_{i\nu} \right) e^{-i(\Xi_{i\nu} - \epsilon_p)t} }{ (\epsilon_p - \epsilon_q)(\Xi_{i\nu} - \epsilon_p)(\Xi_{i\nu} - \epsilon_q) } \\
&= -\frac{1}{\left(\epsilon_p-\epsilon_q\right)\left(\Xi_{i\nu}-\epsilon_p\right)}+\frac{e^{i\left(\epsilon_p-\epsilon_q\right) t}}{\left(\epsilon_p-\epsilon_q\right)\left(\Xi_{i\nu}-\epsilon_q\right)}+\frac{e^{-i\left(\Xi_{i\nu}-\epsilon_p\right) t}}{\left(\Xi_{i\nu}-\epsilon_p\right)\left(\Xi_{i\nu}-\epsilon_q\right)}-\frac{2 \Xi_{i\nu} e^{-i\left(\Xi_{i\nu}-\epsilon_p\right) t}}{\left(\epsilon_p-\epsilon_q\right)\left(\Xi_{i\nu}-\epsilon_p\right)(\Xi_{i\nu}-\epsilon_q)} .
\end{align}
and we can do the analogous computation to get $T_{a\nu}(t)$.
Now we can plug into the expression for the retarded Green's function to get
\begin{align}
    G_{pq}(t) & = G_{pq}^{HF}(t) e^{C_{pq}(t)} \\
& = -i \Theta(t) e^{-i \epsilon_p t + C_{pq}(t)} \\
& = -i \Theta(t) \exp\left[-i \epsilon_p t - \sum_{i\nu} W_{p i \nu} W_{q i \nu} T_{i\nu}(t) - \sum_{a\nu} W_{p a \nu} W_{q a \nu} T_{a\nu}(t) \right] \\
\end{align}
The Fourier transform to the frequency domain is given by
\begin{align}
    G_{pq}(\omega) &= \int dt e^{i\omega t} G_{pq}(t) \\
&= -i \int_0^\infty dt e^{i(\omega - \epsilon_p) t} \exp{-\sum_{i\nu} W_{p i \nu} W_{q i \nu} T_{i\nu}(t) - \sum_{a\nu} W_{p a \nu} W_{q a \nu} T_{a\nu}(t)} \\
&\approx -i \int_0^\infty dt e^{i(\omega - \epsilon_p) t} \left[1- \sum_{i\nu} W_{p i \nu} W_{q i \nu} T_{i\nu}(t) - \sum_{a\nu} W_{p a \nu} W_{q a \nu} T_{a\nu}(t) \right] \\
\end{align}
where in the last step we have made a Taylor expansion of the exponential, keeping just the 0th and 1st order terms. This is the right thing to do because in one of our first steps we chose to make the expression exact up to the first order in the screened Coulomb interaction $W$. If we were to expand beyond the first order, we would be including terms beyond first order in $W$.
So the form for the off-diagonal GW+C spectral function would be
\begin{align}
    A_{pq}(\omega) &= -\frac{1}{\pi} \text{Im} G_{pq}(\omega) \\
\end{align}
so we would need to determine
\begin{align}
   \operatorname{Im}G_{pq}(\omega) &\approx - \operatorname{Re}\int_0^\infty dt \left[e^{i(\omega - \epsilon_p) t} \left(1 - \sum_{i\nu} W_{p i \nu} W_{q i \nu} T_{i\nu}(t) - \sum_{a\nu} W_{p a \nu} W_{q a \nu} T_{a\nu}(t) \right) \right] \\
&= - \pi \delta(\omega - \epsilon_p) + \sum_{i\nu} W_{p i \nu} W_{q i \nu} \operatorname{Re}\int_0^\infty dt e^{i(\omega - \epsilon_p) t} T_{i\nu}(t) + \sum_{a\nu} W_{p a \nu} W_{q a \nu} \operatorname{Re}\int_0^\infty dt e^{i(\omega - \epsilon_p) t} T_{a\nu}(t)
\end{align}
which leaves us with the task of computing
\begin{align}
    \operatorname{Re}\int_0^\infty dt e^{i(\omega - \epsilon_p) t} T_{i\nu}(t) &= -\frac{\pi\delta(\omega - \epsilon_p)}{\left(\epsilon_p-\epsilon_q\right)\left(\Xi_{i\nu}-\epsilon_p\right)} + \frac{\pi\delta(\omega - \epsilon_q)}{\left(\epsilon_p-\epsilon_q\right)\left(\Xi_{i\nu}-\epsilon_q\right)}\\
&  + \frac{\pi\delta(\omega - \Xi_{i\nu})}{\left(\Xi_{i\nu}-\epsilon_p\right)\left(\Xi_{i\nu}-\epsilon_q\right)}-\frac{\pi2\Xi_{i\nu}\delta(\omega - \Xi_{i\nu})}{\left(\epsilon_p-\epsilon_q\right)\left(\Xi_{i\nu}-\epsilon_p\right)\left(\Xi_{i\nu}-\epsilon_q\right)} \notag \\
\end{align}
and similarly for $T_{a\nu}(t)$. So
\begin{align}
    A_{pq}(\omega) &\approx \delta(\omega - \epsilon_p) - \sum_{i\nu} W_{p i \nu} W_{q i \nu} \left[-\frac{\delta(\omega - \epsilon_p)}{\left(\epsilon_p-\epsilon_q\right)\left(\Xi_{i\nu}-\epsilon_p\right)} + \frac{\delta(\omega - \epsilon_q)}{\left(\epsilon_p-\epsilon_q\right)\left(\Xi_{i\nu}-\epsilon_q\right)} + \frac{\delta(\omega - \Xi_{i\nu})}{\left(\Xi_{i\nu}-\epsilon_p\right)\left(\Xi_{i\nu}-\epsilon_q\right)} \right. \\
&\left. - \frac{2\Xi_{i\nu}\delta(\omega - \Xi_{i\nu})}{\left(\epsilon_p-\epsilon_q\right)\left(\Xi_{i\nu}-\epsilon_p\right)\left(\Xi_{i\nu}-\epsilon_q\right)}\right] - \sum_{a\nu} W_{p a \nu} W_{q a \nu} \left[-\frac{\delta(\omega - \epsilon_p)}{\left(\epsilon_p-\epsilon_q\right)\left({\Xi}_{a\nu}-\epsilon_p\right)} + \frac{\delta(\omega - \epsilon_q)}{\left(\epsilon_p-\epsilon_q\right)\left({\Xi}_{a\nu}-\epsilon_q\right)} \right. \\
& \left.+ \frac{\delta(\omega - {\Xi}_{a\nu})}{\left({\Xi}_{a\nu}-\epsilon_p\right)\left({\Xi}_{a\nu}-\epsilon_q\right)}  - \frac{2{\Xi}_{a\nu}\delta(\omega - {\Xi}_{a\nu})}{\left(\epsilon_p-\epsilon_q\right)\left({\Xi}_{a\nu}-\epsilon_p\right)\left({\Xi}_{a\nu}-\epsilon_q\right)}\right] \notag
\end{align}
We can replace all of the delta functions with Lorentzians in practice.
\subsection{Assorted}
\subsubsection{MKCT}
\begin{equation}
\Omega_n=\frac{\left((\mathrm{i} \mathcal{L})^n \hat{A}, \hat{A}\right)}{(\hat{A}, \hat{A})},
\end{equation}
with the corresponding auxiliary kernels
\begin{equation}
K_n(t)=\frac{\left((\mathrm{i} \mathcal{L})^n \hat{f}(t), \hat{A}\right)}{(\hat{A}, \hat{A})}
\end{equation}
So we can show the main result of their first paper that the higher-order kernels satisfy the following coupled ordinary differential equation (ODE):
\begin{align}
\dot{K}_n(t) & =\frac{\left((\mathrm{i} \mathcal{L})^n \dot{\tilde{f}}(t), \hat{A}\right)}{(\hat{A}, \hat{A})} \\
& =\frac{\left((\mathrm{i} \mathcal{L})^n \mathrm{i} \mathcal{Q} \mathcal{L} \hat{f}(t), \hat{A}\right)}{(\hat{A}, \hat{A})} \\
& =\frac{\left((\mathrm{i} \mathcal{L})^{n+1} \hat{f}(t), \hat{A}\right)}{(\hat{A}, \hat{A})}-\frac{\left((\mathrm{i} \mathcal{L})^n \mathrm{i} \mathcal{P} \mathcal{L} \hat{f}(t), \hat{A}\right)}{(\hat{A}, \hat{A})} \\
& =K_{n+1}(t)-\frac{(\mathrm{i} \mathcal{L} \hat{f}(t), \hat{A})}{(\hat{A}, \hat{A})} \times \frac{\left((\mathrm{i} \mathcal{L})^n \hat{A}, \hat{A}\right)}{(\hat{A}, \hat{A})} \\
& =K_{n+1}(t)-K_1(t) \Omega_n,
\end{align}
where we have used the fact that the random fluctuation operator is $\hat{f}(t)=e^{\mathrm{i} t}{ }^{\mathrm{Q}}{ }^{\mathrm{C}} \mathcal{Q} \mathrm{i} \mathcal{L} \hat{A} \Longrightarrow \dot{f}(t)=\mathrm{i} \mathcal{Q} \mathcal{L} \hat{f}(t)$ and we can deduce the initial conditions in a similar fashion with $\hat{f}(0)=\mathcal{Q} i \mathcal{L} \hat{A}$, so
\begin{align}
K_n(0) & =\frac{\left((\mathrm{i} \mathcal{L})^n \mathrm{Qi} \mathcal{L} \hat{A}, \hat{A}\right)}{(\hat{A}, \hat{A})} \\
& =\frac{\left((\mathrm{i} \mathcal{L})^n(\mathrm{i} \mathcal{L} \hat{A}-\mathrm{P} \mathrm{i} \mathcal{L} \hat{A}), \hat{A}\right)}{(\hat{A}, \hat{A})} \\
& =\frac{\left((\mathrm{i} \mathcal{L})^{n+1} \hat{A}, \hat{A}\right)}{(\hat{A}, \hat{A})}-\frac{\left((\mathrm{i} \mathcal{L})^n \mathrm{P} \mathrm{i} \mathcal{L} \hat{A}, \hat{A}\right)}{(\hat{A}, \hat{A})} \\
& =\Omega_{n+1}-\Omega_1 \Omega_n,
\end{align}
suggesting that the central quantity to compute is the $\left\{\Omega_n\right\}$. But an issue with this ODE is that it extends to infinite order, and hard truncation to finite order can lead to numerical instabilities. To this end, they introduce a truncation scheme with Padé approximants. First, they noticed that the $m$-th derivative of kernel $K_n(t)$ evaluated at $t=0$ is
\begin{align}
K_n^{(m)} &= \frac{\left((\mathrm{i} \mathcal{L})^n(\mathcal{Q} \mathrm{i} \mathcal{L})^{m+1} \hat{A}, \hat{A}\right)}{(\hat{A}, \hat{A})} \\
&= \frac{\left((\mathrm{i} \mathcal{L})^n \mathcal{Q} \mathrm{i} \mathcal{L} (\mathcal{Q} \mathrm{i} \mathcal{L})^{m}\hat{A}, \hat{A}\right)}{(\hat{A}, \hat{A})} \\
&= \frac{\left((\mathrm{i} \mathcal{L})^n \left(1-\mathcal{P}\right) \mathrm{i} \mathcal{L} (\mathcal{Q} \mathrm{i} \mathcal{L})^{m} \hat{A}, \hat{A}\right)}{(\hat{A}, \hat{A})} \\
&= \frac{\left((\mathrm{i} \mathcal{L})^{n+1} (\mathcal{Q} \mathrm{i} \mathcal{L})^{m} \hat{A}, \hat{A}\right)}{(\hat{A}, \hat{A})} - \frac{\left((\mathrm{i} \mathcal{L})^n \mathcal{P} \mathrm{i} \mathcal{L} (\mathcal{Q} \mathrm{i} \mathcal{L})^{m} \hat{A}, \hat{A}\right)}{(\hat{A}, \hat{A})} \\
&= K_{n+1}^{(m-1)} - \frac{\left(\mathrm{i} \mathcal{L}(\mathcal{Q} \mathrm{i} \mathcal{L})^m \hat{A}, \hat{A}\right)}{(\hat{A}, \hat{A})} \times \frac{\left((\mathrm{i} \mathcal{L})^n \hat{A}, \hat{A}\right)}{(\hat{A}, \hat{A})} \\
&= K_{n+1}^{(m-1)} - \bar{\Omega}_m \Omega_n 
\end{align}
where we introduced the auxiliary moment $\bar{\Omega}_m=\frac{\left(\mathrm{i} \mathcal{L}(\mathcal{Q} \mathrm{i} \mathcal{L})^m \hat{A}, \hat{A}\right)}{(\hat{A}, \hat{A})}$.
So recursively applying this expression leads to the relation: 
\begin{align}
K_n^{(m)} &= K_{n+2}^{(m-2)} - \bar{\Omega}_{m-1} \Omega_{n+1} - \bar{\Omega}_m \Omega_{n} \\
&= K_{n+m}^{(0)} - \sum_{j=0}^{m-1} \bar{\Omega}_{m-j} \Omega_{n+j} \\
&= \Omega_{n+m+1} - \Omega_{1} \Omega_{n+m} - \sum_{j=0}^{m-1} \bar{\Omega}_{m-j} \Omega_{n+j} \\
&= \Omega_{n+m+1} - \bar{\Omega}_{0} \Omega_{n+m} - \sum_{j=0}^{m-1} \bar{\Omega}_{m-j} \Omega_{n+j} \\
&= \Omega_{n+m+1} - \sum_{j=0}^{m} \bar{\Omega}_{m-j} \Omega_{n+j} \\
\end{align}
that $K_{n}^{(m)}$ can be expressed with moments and auxiliary moments. Similarly, the auxiliary moments themselves have recursions
\begin{equation}    
\begin{aligned}
    \tilde{\Omega}_m &= \frac{\left((\mathrm{i} \mathcal{L} \mathcal{Q})^{m} \mathrm{i} \mathcal{L} \hat{A}, \hat{A}\right)}{(\hat{A}, \hat{A})} - \tilde{\Omega}_{m-1} \Omega_1\\
% (\im\Lv (\Qroj \im \Lv)^{m-1} \im \Lv \hat{A}, \hat{A}) (\hat{A}, \hat{A})^{-1} - \tilde{\Omega}_{m-1} \Omega_1, \\
    &= \Omega_{m+1} - \sum_{j=0}^{m-1} \tilde{\Omega}_j \Omega_{m-j} \\ 
\end{aligned}
\end{equation}
which means the auxiliary moments $\{\tilde{\Omega}_m\}$ can be obtained by the moments $\{\Omega_n\}$; it becomes a problem of efficiently computing the moments. The series expansion for the $n$-th order auxiliary kernel can then be given by Pad\'{e} approximant. Initially, $K_n(t)$ is expressed as a truncated Taylor series:
\begin{equation}
    K_n(t) \approx \sum_{j=0}^{M} \frac{K_n^{(j)}(0)}{j!} t^j,
\end{equation}
which is a good local approximation but lacks accuracy over a broader range of $t$. A more reliable approximation can be achieved using the Pad\'{e} approximant: 
\begin{equation}\label{eqn:pade}
    K_n(t) \approx \frac{p_{M_1}(t)}{q_{M_2}(t)} = \frac{a_0 + a_1 t + \dots + a_{M_1} t^{M_1}}{1 + b_1 t + \dots + b_{M_2} t^{M_2}},
\end{equation}
where $ p_{M_1}(t)$ and $q_{M_2}(t)$ are polynomials of degrees $M_1$ and $M_2$, respectively. The coefficients $\{a_i\}$ and $\{b_i\}$ are computed using the python library SciPy, which implements the standard Pad\'{e} approximant procedure as described in Ref.~\cite{Baker1996pade}. Overall, Eq.~\ref{eqn:pade} provides a numerically stable truncation for the MKCT Eq.~\ref{eqn:kn_ode}, where \emph{all} coefficients can be evaluated with higher-order moments $\{ \Omega_n \}$.
\subsubsection{Relation of improper self energy to cumulant}
The Dyson equation for the one particle greens function can be written using an improper $\Sigma^I$ self energy as
\begin{align}
    \label{dyson}
    & G_k(t,t') =G_k^0(t,t') + G_k^0(t,t')\Sigma_k^I(t,t')G_k^0(t,t'),\\
\end{align}
instead of using the proper self energy $\Sigma^*$, which satisfies $G_k(t,t') = G_k^0(t,t') + G_k^0(t,t')\Sigma_k^*(t,t')G_k(t,t').$ 
% We can rewrite
% \begin{align}
% & G\left(t, t^{\prime}\right)-G^0\left(t, t^{\prime}\right)=\iint d t_1 d t_2 G^0\left(t, t_1\right) \Sigma^I\left(t_1, t_2\right) G^0\left(t_2, t^{\prime}\right)\\
% & \implies \frac{G\left(t, t^{\prime}\right)}{G^0\left(t, t^{\prime}\right)}-1=\left[G^0\left(t, t^{\prime}\right)\right]^{-1} \iint d t_1 d t_2 G^0\left(t, t_1\right) \Sigma^I\left(t_1, t_2\right) G^0\left(t_2, t^{\prime}\right)\\
%     % &= G_k^0(t,t')[1 + \Sigma_k^{I}(t,t')G_k^0(t,t')] \label{dysoni}\\
%     % &= {G_k^0(t,t') \over 1 - \Sigma_k^*(t,t')G_k^0(t,t')} \label{dysonp},
% \end{align} 
Now, the retarded cumulant ansatz can be written as
\begin{align}
&G_k^R(t,t') = G_k^{0,R}(t,t')e^{C_k^R(t,t')}\\
& \implies {G_k^R(t,t') \over G_k^{0,R}(t,t')} = e^{C_k^R(t,t')}\\
\end{align}
%  This leads to the following form for the cumulant 
% \begin{eqnarray}
% C_k(t,t') &=& \ln\left(1 + \left[G_k^{R,0}(t,t')\right]^{-1}\iint dt_1dt_2 \textcolor{white}{\int}G_k^{R,0}(t,t_1)\Sigma_k^{R,I}(t_1,t_2)G_k^{R,0}(t_2,t')\right).
% \label{cum_time}
% \end{eqnarray}
% So
% \begin{align}
% &e^{C_k\left(t, t^{\prime}\right)}-1=\left[G_k^{0}\left(t, t^{\prime}\right)\right]^{-1} \iint d t_1 d t_2 G_k^{0}\left(t, t_1\right) \Sigma_k^{I}\left(t_1, t_2\right) G_k^{0}\left(t_2, t^{\prime}\right) \\
% & \implies C_k\left(t, t^{\prime}\right)=\ln \left(1+\left[G_k^{0}\left(t, t^{\prime}\right)\right]^{-1} \iint d t_1 d t_2 G_k^{0}\left(t, t_1\right) \Sigma_k^{I}\left(t_1, t_2\right) G_k^{0}\left(t_2, t^{\prime}\right)\right)
% \end{align}
 Let's deal directly with this ratio instead of needing to introduce a self energy.
The retarded Green's function can be written as
\begin{align}
G_k^R(t)&=-i \theta(t)\left\langle\left\{c_k(t), c_k^{\dagger}(0)\right\}\right\rangle \\
&=-i \theta(t)\left\langle\left\{e^{i H t} c_k e^{-i H t}, c_k^{\dagger}\right\}\right\rangle \\
&=-i \theta(t)\left\langle\left\{\sum_{n=0}^{\infty} \frac{(-i t)^n}{n!} \mathcal{L}^n c_k, c_k^{\dagger}\right\}\right\rangle \\
% &=-i \theta(t)\left( \left\langle\left\{c_k, c_k^{\dagger}\right\}\right\rangle+i t\left\langle\left\{\left[H, c_k\right], c_k^{\dagger}\right\}\right\rangle+\frac{(i t)^2}{2}\left\langle\left\{\left[H,\left[H, c_k\right]\right], c_k^{\dagger}\right\}\right\rangle+\cdots \right)\\
&=-i \theta(t)\left( \mu_{0,k}- t\mu_{1,k}+\frac{t^2}{2}\mu_{2,k}+O(t^3) \right)
\end{align}
where we have defined the moments $\mu_{n,k}=\left\langle\left\{\left(i \mathcal{L} \right)^n c_k, c_k^{\dagger}\right\}\right\rangle$ and the Liouvillian superoperator $\mathcal{L} O=[H,O]$. The non-interacting retarded Green's function is given by
\begin{align}
G_k^{R,0}(t)&=-i \theta(t)\left( \mu_{0,k}^0- t\mu_{1,k}^0+\frac{t^2}{2}\mu_{2,k}^0+O(t^3) \right)
\end{align}
where the non-interacting moments are $\mu_{n,k}^0=\left\langle\left\{\left( i \mathcal{L}_0\right)^n c_k, c_k^{\dagger}\right\}\right\rangle$ and $\mathcal{L}_0 O=[H_0,O]$. Defining $A(t)= 1-t\mu_{1,k}+\frac{t^2}{2}\mu_{2,k}+O(t^3)$ and $B(t)= 1- t\mu_{1,k}^0+\frac{t^2}{2}\mu_{2,k}^0+O(t^3)$ we can use a geometric series to write
\begin{align}
\frac{1}{B(t)}&=1+ t \mu_{1, k}^0+t^2\left[\left(\mu_{1, k}^0\right)^2-\frac{1}{2}\mu_{2, k}^0\right]+O\left(t^3\right)
\end{align}
Note that the inverse power series coefficients can be computed by Wronski's formula (if one is interested in still higher orders of $t$).
So, we can write the ratio and then group terms by order in $t$, to get
\begin{align}
R_k(t)=\frac{G_k^R(t)}{G_k^{R,0}(t)}
&=\frac{-i\theta (t)}{-i\theta (t)}\times\frac{A(t)}{B(t)} \\
&= \left( 1- t\mu_{1,k}+\frac{t^2}{2}\mu_{2,k}+O(t^3) \right) \left( 1+ t \mu_{1, k}^0+t^2\left[\left(\mu_{1, k}^0\right)^2-\frac{1}{2}\mu_{2, k}^0\right]+O\left(t^3\right) \right)\\
&= \left( 1+t\underbrace{\left[-\left(\mu_{1,k}-\mu_{1,k}^0\right)\right]}_{\alpha _{1,k}}+t^2\underbrace{\left[\frac{1}{2}\left(\mu_{2,k}-\mu_{2,k}^0\right)-\mu_{1,k}\mu_{1,k}^0+\left(\mu_{1,k}^0\right)^2\right]}_{\alpha _{2,k}}+O(t^3) \right)
\end{align}
Given that the cumulant can be expanded as $C_k^R(t) = \kappa_{1,k} t + \kappa_{2,k} t^2 + O(t^3)$, we can Taylor expand the exponential $e^{C_k^R(t)}$ and then match coefficients with the above expression for $R_k(t)$ to plug into the usual relations between the cumulants $\kappa _{n,k}$ and moments $\alpha_{n,k}$, to get
\begin{align}
    \kappa_{1,k} = \alpha_{1,k} &= -\left(\mu_{1,k}-\mu_{1,k}^0\right)\\
    \kappa_{2,k} = \alpha_{2,k} - \frac{\alpha_{1,k}^2}{2} &= \frac{1}{2}\left(\mu_{2,k}-\mu_{2,k}^0\right)-\mu_{1,k}\mu_{1,k}^0+\left(\mu_{1,k}^0\right)^2 -\frac{1}{2}{\left[-\left(\mu_{1,k}-\mu_{1,k}^0\right)\right]^2}\\
&= \frac{1}{2}\left(\mu_{2,k}-\mu_{2,k}^0\right)-\mu_{1,k}\mu_{1,k}^0+\left(\mu_{1,k}^0\right)^2 -\frac{1}{2}\left(\mu_{1,k}-\mu_{1,k}^0\right)^2\\
&= \frac{1}{2}\left[\left( \mu_{2,k} - \mu_{2,k}^0 \right) +\left( \mu_{1,k} \right)^2 + \left( \mu_{1,k}^0 \right)^2 \right]
\end{align}
So now we need to determine how to compute the moments for this cumulant approach and potentially for the MKCT approach. But let's begin by discerning the form for the 2/2 Padé approximant to the cumulant for the memory kernel at n=1:
\begin{align}
    K_1(t) &\approx \frac{a_0 + a_1 t+ a_2 t^2}{1 + b_1 t + b_2 t^2}\\
& \approx \left(a_0 + a_1 t+ a_2 t^2\right)\left(1 - b_1 t + \left(b_1^2 - b_2\right) t^2 + O(t^3)\right) \\
& \approx a_0 + t\left(a_1 - a_0 b_1\right) + t^2\left(a_2 - a_1 b_1 + a_0\left(b_1^2 - b_2\right)\right) + O(t^3)
\end{align}
with the derivatives given by $c_0 = K_1(0)$, $c_1 = K_1^{(1)}(0)$, and $c_2 = \frac{K_1^{(2)}(0)}{2}$. Matching coefficients of $t^0$, $t^1$, and $t^2$ gives us the following system of equations:
\begin{align}
    c_0 &= a_0 \\
    c_1 &= a_1 - a_0 b_1 \implies a_1 = c_1 + a_0 b_1\\
    c_2 &= a_2 - a_1 b_1 + a_0\left(b_1^2 - b_2\right) \implies a_2 = c_2 + a_1 b_1 - a_0\left(b_1^2 - b_2\right)
\end{align}
We can find that $a_0 = K_1(0)$, $a_1 = K_1^{(1)}(0) - \frac{K_1(0) K_1^{(2)}(0)}{2 K_1^{(1)}(0)}$, and $b_1 = -\frac{K_1^{(2)}(0) / 2}{K_1^{(1)}(0)}$. We start by computing
\begin{align}
    K_1(0) &= \Omega_2 - \Omega_1^2 = \mu_{2,k} - \mu_{1,k}^2\\
    K_1^{(1)}(0) &= K_2(0) - \bar{\Omega}_1 \Omega_1 = \Omega_3 - \Omega_1 \Omega_2 - \bar{\Omega}_1 \Omega_1 = \mu_{3,k} - \mu_{1,k} \mu_{2,k} - \bar{\Omega}_1 \mu_{1,k}\\
\end{align}
% So:
% $$
% b_1=-\frac{K_n^{(2)}(0) / 2}{K_n^{(1)}(0)}, \quad a_1=K_n^{(1)}(0)-\frac{K_n(0) K_n^{(2)}(0)}{2 K_n^{(1)}(0)}
% $$

% Final [1/1] Padé:
% $$
% K_n^{[1 / 1]}(t)=\frac{K_n(0)+\left(K_n^{(1)}(0)-\frac{K_n(0) K_n^{(2)}(0)}{2 K_n^{(1)}(0)}\right) t}{1-\frac{K_n^{(2)}(0)}{2 K_n^{(1)}(0)} t} .
% $$
% \begin{align}
%     \mu_{2,k} &= \left\langle\left\{\mathcal{L}^2 c_k, c_k^{\dagger}\right\}\right\rangle = \left\langle\left\{[H,[H,c_k]], c_k^{\dagger}\right\}\right\rangle \\
%     &= \left\langle\left\{[H_0 + V,[H_0 + V,c_k]], c_k^{\dagger}\right\}\right\rangle \\4
% \end{align}
\subsubsection{Fundamental starting point}
We start with the definition of the retarded Green's function in the time domain:
\begin{equation}
G_k^R(t) \equiv -i \theta(t)\left\langle\left\{c_k(t), c_k^{\dagger}(0)\right\}\right\rangle .
\end{equation}
Immediately, we can Fourier transform to the frequency domain:
\begin{equation}
G_k^R(\omega)=\int_{-\infty}^{\infty} d t e^{i \omega t} G_k^R(t) .
\end{equation}
But because $G_k^R(t)$ is already zero for $t<0$ (by the $\theta(t)$ ), this integral reduces to
\begin{align}
G_k^R(\omega)&=-i \int_0^{\infty} d t e^{i \omega t}\left\langle\left\{c_k(t), c_k^{\dagger}(0)\right\}\right\rangle \\
&=-i \int_0^{\infty} d t e^{i \omega t} \sum_{n=0}^{\infty} \frac{(i t)^n}{n!}\mu_{n,k} \\
&=-i \sum_{n=0}^{\infty} \frac{i^n}{n!}\mu_{n,k} \int_0^{\infty} d t e^{i \omega t}t^n \\
&=-i \sum_{n=0}^{\infty} \frac{i^n}{n!}\mu_{n,k} \frac{n!}{(i \omega)^{n+1}} \\
&=-\sum_{n=0}^{\infty} \frac{\mu_{n,k}}{\omega^{n+1}} ,
\end{align}
where we have defined the moments $\mu_{n,k}=\left\langle\left\{\mathcal{L}^n c_k, c_k^{\dagger}\right\}\right\rangle$. This works because $c_k(t)=e^{i H t} c_k e^{-i H t} = e^{i \mathcal{L} t} c_k$, where $\mathcal{L}$ is the Liouvillian superoperator, so the time evolution operator can be expanded in a Taylor series, as $e^{i \mathcal{L} t}=\sum_{n=0}^{\infty} \frac{(i t)^n}{n!} \mathcal{L}^n \implies \left\langle\left\{c_k(t), c_k^{\dagger}(0)\right\}\right\rangle  = \sum_{n=0}^{\infty} \frac{(i t)^n}{n!}\mu_{n,k}$. The complementary projection operator is $\mathcal{Q} = 1 - \mathcal{P}$. The Liouvillian superoperator acts on an arbitrary operator $X$ as $\mathcal{L} X = [H,X]$.
Since we know that Mori's is formally a projector method with $\mathcal{P}X = \frac{(X,c_k)}{(c_k,c_k)} c_k$, where $(A,B) = \langle \{A, B^\dagger\} \rangle$ is the fermionic Mori inner product, so $(c_k, c_k) = \langle \{c_k, c_k^\dagger\} \rangle = 1$. Lanczos is also formally a projection onto the Krylov subspace, so we can try to exploit this by devising a Lanczos procedure. Then we can initiate a Lanczos sequence with $|f_0\rangle = c_k$ and then for $n=0$, we have
\begin{equation}
|f_1\rangle = \mathcal{L} |f_0\rangle - a_0 |f_0\rangle ,
\end{equation}
where $a_0 = \frac{(\mathcal{L} f_0, f_0)}{(f_0, f_0)} = \mu_{1,k}$. Then for 
$n \geq 1$ define
\begin{equation}
|f_{n+1}\rangle = \mathcal{L} |f_n\rangle - a_n |f_n\rangle - b_n^2 |f_{n-1}\rangle ,
\end{equation}
where the coefficients are given by $a_n = \frac{(\mathcal{L} f_n, f_n)}{(f_n, f_n)}$ and $b_{n+1}^2 = \frac{(f_{n+1}, f_{n+1})}{(f_n, f_n)}$. Let us try to determine the action of the Liouvillian onto a given state $|f_n\rangle$. We can write
\begin{align}
\mathcal{L} |f_n\rangle &= [H_0 + V, |f_n\rangle] \\
&= \mathcal{L}_0 |f_n\rangle + [V, |f_n\rangle] \\
\end{align}
where $H_0$ is the non-interacting Hamiltonian and $V$ is the interaction. We can also define a non-interacting Liouvillian $\mathcal{L}_0$ such that $\mathcal{L}_0 X = [H_0, X]$. So we can write
