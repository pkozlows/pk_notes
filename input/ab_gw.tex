

\section{Constructing the Hamiltonian}
The idea of this method is not to work in the MO basis, but rather to work in a basis of particle-hole excitations, which are approximated as bosons. So $\hat{a}_a^\dagger \hat{a}_i \approx \hat{b}_\nu^\dag$ and $\hat{a}_i^\dagger \hat{a}_a \approx \hat{b}_\nu$, where in second quantization $\hat{a}$ are fermionic and $\hat{b}$ are bosonic operators, respectively. The drawback is that these bosonic operators no longer obey the Pauli exclusion principle; what was done above is known as the quasi-boson approximation. Then we define the electron-boson Hamiltonian as
\begin{equation}
\hat{H}^{\mathrm{eB}}=\hat{H}^{\mathrm{e}}+\hat{H}^{\mathrm{B}}+\hat{V}^{\mathrm{eB}}
\end{equation}
where $\hat{H}^{\mathrm{e}}$ is the electronic Hamiltonian, $\hat{H}^{\mathrm{B}}$ is the bosonic Hamiltonian, and $\hat{V}^{\mathrm{eB}}$ is the electron-boson coupling term, given as
\begin{align}
\hat{H}^e&=\sum_{p q} f_{p q}\left\{\hat{a}_p^{\dagger} \hat{a}_q\right\} \\
\hat{H}^{B}&=\sum_{\nu \mu} A_{\nu \mu} \hat{b}_\nu^{\dagger} \hat{b}_\mu+\frac{1}{2} \sum_{\nu \mu} B_{\nu \mu}\left(\hat{b}_\nu^{\dagger} \hat{b}_\mu^{\dagger}+\hat{b}_\nu \hat{b}_\mu\right)
\label{b} \\
\hat{V}^{eB}&=\sum_{p q, \nu} V_{p q \nu}\left\{\hat{a}_p^{\dagger} \hat{a}_q\right\}\left(\hat{b}_\nu^{\dagger}+\hat{b}_\nu\right)
\label{eb}
\end{align}
 $A_{\nu \mu}$ and $B_{\nu \mu}$ denote the dRPA matrices, as
\begin{equation}
\begin{split}
A_{\nu \mu}&=A_{i a, j b}=\delta_{i j} \delta_{a b}\left(\epsilon _a-\epsilon _i\right)+(ia|bj) \\
B_{v \mu}&=B_{i a, j b}=(i a | j b)
\end{split}
\end{equation}
and the electron-boson coupling term is defined as
\begin{equation}
V_{p q \nu}=V_{p q, i a}=(p q| i a)
\end{equation}
% \subsection{Showing TDA case}
% We seek to prove $H_{m,p \nu}$.
% \begin{align}
% H_{m,p \nu} &= \left\langle 0_{\mathrm{F}} 0_{\mathrm{B}}\right|\left[a_m^\dag, \left[\hat{H}^{\mathrm{eB}}, a_p b_\nu^{\dagger} \right]\right]\left|0_{\mathrm{F}} 0_{\mathrm{B}}\right\rangle = \left\langle 0_{\mathrm{F}} 0_{\mathrm{B}}\right|\left[a_m^\dag, \left[\hat{V}^{\mathrm{eB}}, a_p b_\nu^{\dagger} \right]\right]\left|0_{\mathrm{F}} 0_{\mathrm{B}}\right\rangle
% % H_{p,i\nu} &= \left\langle 0_{\mathrm{F}} 0_{\mathrm{B}}\right|\left[a_p^\dag, \left[\hat{H}^{\mathrm{eB}}, a_i b_\nu^{\dagger} \right]\right]\left|0_{\mathrm{F}} 0_{\mathrm{B}}\right\rangle = \left\langle 0_{\mathrm{F}} 0_{\mathrm{B}}\right|\left[a_p^\dag, \left[\hat{V}^{\mathrm{eB}}, a_i b_\nu^{\dagger} \right]\right]\left|0_{\mathrm{F}} 0_{\mathrm{B}}\right\rangle
% \end{align}
% Note that we have neglected the electronic and bosonic Hamiltonians, because using them in this arrangement will not conserve the number of bosons. So we have
% \begin{align}
% [\hat V^{eB},\; a_p b_\nu^\dagger ] &= \sum_{rs,\mu} V_{rs, \mu} [ a_r^{\dagger} a_s ( b_{\mu} + b_{\mu}^{\dagger} ), a_p b_\nu^\dagger ] = \sum_{rs,\mu} V_{rs, \mu} [ a_r^{\dagger} a_s, a_p ][ b_{\mu}, b_\nu^\dagger ] = \sum_{rs} V_{rs, \nu} [ a_r^{\dagger} a_s a_p - a_p a_r^{\dagger} a_s ]\\
% [a_m^\dagger,\;[\hat V^{eB},\; a_p b_\nu^\dagger]] &= \sum_{rs} V_{rs, \nu} [ a_m^\dagger a_r^{\dagger} a_s a_p - a_r^{\dagger} a_s a_p a_m^\dagger - (a_m^\dag a_p a_r^{\dagger}a_s - a_p a_r^{\dagger}a_s a_m^\dag)] \\
% a_m^\dagger a_r^{\dagger} a_s a_p &= \wick{\c1 a_m^\dagger \c2 a_r^{\dagger} \c1 a_s \c2 a_p} + \wick{\c2 a_m^\dagger \c1 a_r^{\dagger} \c1 a_s \c2 a_p} = -\underbrace{\delta_{ms} \delta_{pr}}_{m,p \in O} + \underbrace{\delta_{mp} \delta_{rs}}_{m,r \in O} \\
% a_r^{\dagger} a_s a_p a_m^\dagger &= 0 \\
% a_m^\dag a_p a_r^{\dagger}a_s &= \wick{\c1 a_m^\dag \c1 a_p \c1 a_r^{\dagger} \c1 a_s} + \wick{\c2 a_m^\dag \c1 a_p \c1 a_r^{\dagger} \c2 a_s} = \underbrace{\delta_{mp} \delta_{rs}}_{m,r \in O} + \underbrace{\delta_{ms} \delta_{pr}}_{m \in O, p \in V} \\
% a_p a_r^{\dagger}a_s a_m^\dag &= 0 \\
% % [ V^{eB},\; a_i b_\nu^\dagger ] &= \sum_{rs,\nu'} W_{rs, \nu'} [ a_r^{\dagger} a_s ( b_{\nu'} + b_{\nu'}^{\dagger} ), a_i b_\nu^\dagger ] = \sum_{rs,\nu'} W_{rs, \nu'} [ a_r^{\dagger} a_s, a_i ][ b_{\nu'}, b_\nu^\dagger ] \\
% % &= \sum_{rs} W_{rs, \nu} \left( a_r^{\dagger} a_s a_i -  a_i a_r^{\dagger} a_s \right) \\
% % [a_p^\dagger,\;[\hat V^{eB},\;a_i b_\nu^\dagger]] &= \sum_{rs} W_{rs, \nu} \left[ a_p^\dagger \hat{a}_r^{\dagger} \hat{a}_s a_i  - \hat{a}_r^{\dagger} \hat{a}_s a_i a_p^\dagger  - \left( a_p^\dag a_i a_r^{\dagger}a_s   - a_i a_r^{\dagger}a_s a_p^\dagger \right) \right] \\
% % a_p^\dagger \hat{a}_r^{\dagger} \hat{a}_s a_i  &= \wick{\c1 a_p^\dagger \c2 a_r^{\dagger} \c1 a_s \c2 a_i} + \wick{\c2 a_p^\dagger \c1 a_r^{\dagger} \c1 a_s \c2 a_i} = -\delta_{ps} \delta_{ri} + \delta_{pi} \delta_{rs} \\
% % \hat{a}_r^{\dagger} \hat{a}_s a_i a_p^\dagger &= \wick{\c1 a_r^{\dagger} \c2 a_s \c1 a_i \c2 a_p^\dagger} + \wick{\c1 a_r^{\dagger} \c1 a_s \c1 a_i \c1 a_p^\dagger} = -\delta_{ri} \delta_{ps} + 0 \\
% % a_p^\dag a_i a_r^{\dagger}a_s &= \wick{\c1 a_p^\dag \c1 a_i \c1 a_r^{\dagger} \c1 a_s} + \wick{\c2 a_p^\dag \c1 a_i \c1 a_r^{\dagger} \c2 a_s} = \delta_{pi} \delta_{rs} + 0 \\
% % a_i a_r^{\dagger}a_s a_p^\dagger &= 0 \\
% \end{align}
% So we can write $H_{m,p \nu} = -V_{ik,\nu}+V_{ak,\nu}$.
As we will see shortly, this formalism has the ability to introduce the desired RPA screening. But the connection to Booth's ED is already clear; the physical space is represented by the electronic Hamiltonian, the auxiliary space by the bosonic Hamiltonian, and the coupling between them by the electron-boson coupling term. Right now the bosonic Hamiltonian (specifically its second term) does not conserve the boson number. To remedy this, we perform a unitary transformation
\begin{equation}
\hat{U}^{\dagger} \hat{H}^{\mathrm{eB}} \hat{U} \rightarrow \tilde{H}^{\mathrm{eB}}.
\end{equation}
\subsection{Nature of the unitary transformation}
First consider what the bosonic Hamiltonian looks like when expressed in the bosonic basis $\bm{{b}} = \left( \hat{b}_1, \hat{b}_2, \ldots \right)$ as
\begin{align}
\hat{H}^{\mathrm{B}}\left(\hat{b}, \hat{b}^{\dagger}\right)&=-\frac{1}{2} \operatorname{tr} \mathbf{A}+\frac{1}{2}\left(\begin{array}{ll}
\mathbf{b}^{\dagger} & \mathbf{b}
\end{array}\right)\left(\begin{array}{ll}
\mathbf{A} & \mathbf{B} \\
\mathbf{B} & \mathbf{A}
\end{array}\right)\binom{\mathbf{b}}{\mathbf{b}^{\dagger}}
\label{eq:rpa_rec}
 \\
&=-\frac{1}{2} \operatorname{tr} \mathbf{A}+\frac{1}{2}\begin{pmatrix}\bm{b}^{\dagger}& \bm{b}\end{pmatrix}\begin{pmatrix}
    \bm{A}\bm{b} + \bm{B}\bm{b}^{\dagger} \\
    \bm{B}\bm{b} + \bm{A}\bm{b}^{\dagger}
\end{pmatrix} \\
&=-\frac{1}{2} \operatorname{tr} \mathbf{A}+\frac{1}{2}\left[\bm{b}^{\dagger} \bm{A} \bm{b} + \bm{b}^{\dagger} \bm{B} \bm{b}^{\dagger} + \bm{b} \bm{B} \bm{b} + \bm{b} \bm{A} \bm{b}^{\dagger}\right]
\label{eq:normal}
 \\
&=\bm{b}^{\dagger} \bm{A} \bm{b} + \frac{1}{2}\left[\bm{b}^{\dagger} \bm{B} \bm{b}^{\dagger} + \bm{b} \bm{B} \bm{b}\right] + \bm{0}
\label{string}\\
&=\sum_{\nu \mu} A_{\nu \mu} \hat{b}_\nu^{\dagger} \hat{b}_\mu+\frac{1}{2} \sum_{\nu \mu} B_{\nu \mu}\left(\hat{b}_\nu^{\dagger} \hat{b}_\mu^{\dagger}+\hat{b}_\nu \hat{b}_\mu\right)
\end{align}
Going from \ref{eq:normal} to \ref{string}, we used the fact that the final term needs to be put into normal order so we can do $\bm{b} \bm{A} \bm{b}^{\dagger}=\sum_{\mu\nu}A_{\mu\nu}b_\mu b_\nu^{\dagger}= \sum_{\mu\nu}A_{\mu\nu}\left(b_\nu^\dagger b_\mu + \delta_{\mu\nu}\right) = \bm{b}^\dagger \bm{A} \bm{b} + \operatorname{Tr}\left(\bm{A}\right)$. In the above we have showed equivalence to the previously defined form in \ref{b}.  Within this representation of the bosonic Hamiltonian in the bosonic basis, in \ref{eq:rpa_rec}, we recognize the appearance of the RPA matrix. From \ref{eq:rpa_rec}, we can obtain a diagonalized form 
\begin{equation}
\hat{H}^{\mathrm{B}}\left(\bar{b}, \bar{b}^{\dagger}\right)=-\frac{1}{2} \operatorname{tr} \mathbf{A}+\frac{1}{2}\left(\overline{\mathbf{b}}^{\dagger} \overline{\mathbf{b}}\right)\left(\begin{array}{cc}
\Omega \mathbf{1} & 0 \\
0 & \Omega \mathbf{1}
\end{array}\right)\binom{\overline{\mathbf{b}}}{\overline{\mathbf{b}}^{\dagger}}
\end{equation}
through a redefinition of the bosonic operators as
\begin{align}
\binom{\overline{\mathbf{b}}}{\overline{\mathbf{b}}^{\dagger}}=\left(\begin{array}{cc}
\mathbf{X} & -\mathbf{Y} \\
-\mathbf{Y} & \mathbf{X}
\end{array}\right)^T\binom{\mathbf{b}}{\mathbf{b}^{\dagger}} \quad \text{and} \quad
\binom{\mathbf{b}}{\mathbf{b}^{\dagger}}=\left(\begin{array}{ll}
\mathbf{X} & \mathbf{Y} \\
\mathbf{Y} & \mathbf{X}
\end{array}\right)\binom{\overline{\mathbf{b}}}{\overline{\mathbf{b}}^{\dagger}} .
\end{align}
\subsubsection{Effect on the bosonic Hamiltonian}
Now by expanding, we see
\begin{align}
\hat{H}^{\mathrm{B}}\left(\overline{\mathbf{b}},
\overline{\mathbf{b}}^{\dagger}\right)&=-\frac{1}{2} \operatorname{tr} \mathbf{A}+\frac{1}{2}\left(\overline{\mathbf{b}}^{\dagger} \overline{\mathbf{b}}\right)\left(\begin{array}{cc}\Omega \mathbf{1} & 0 \\
0 & \Omega \mathbf{1}
\end{array}\right)\binom{\overline{\mathbf{b}}}{\overline{\mathbf{b}}^{\dagger}} \\
&=-\frac{1}{2} \operatorname{tr} \mathbf{A}+ \frac{1}{2}\begin{pmatrix}
\overline{\mathbf{b}}^{\dagger} & \overline{\mathbf{b}}\end{pmatrix}
\begin{pmatrix}
\bm{\Omega} \overline{\mathbf{b}} \\
\bm{\Omega} \overline{\mathbf{b}}^{\dagger}
\end{pmatrix}
\\
&=-\frac{1}{2} \operatorname{tr} \mathbf{A}+\frac{1}{2}\left[\overline{\mathbf{b}}^{\dagger} \bm{\Omega} \overline{\mathbf{b}} + \underbrace{\overline{\mathbf{b}} \bm{\Omega} \overline{\mathbf{b}}^{\dagger}}_{\overline{b}^\dagger \bm{\Omega} \overline{b} + \operatorname{Tr}\left(\bm{\Omega}\right)}\right] \\
&=\overline{\mathbf{b}}^{\dagger} \bm{\Omega} \overline{\mathbf{b}} + \frac{1}{2}\operatorname{Tr}\left(\bm{\Omega} - \mathbf{A}\right) \\
&=\sum_{\nu} \Omega_\nu \overline{b}_\nu^{\dagger} \overline{b}_\nu + E_{\mathrm{RPA}}^c
\end{align}
so we removed the non-boson conserving quality of the bosonic Hamiltonian (compare to \ref{b}).
\subsubsection{Effect on the electron-boson coupling term}
Originally, the electron-boson coupling term is given as
\begin{equation}
\hat{V}^{\mathrm{eB}}=\sum_{p q, \nu} V_{p q, \nu}\left\{\hat{a}_p^{\dagger} \hat{a}_q\right\}\left(\hat{b}_\nu^{\dagger}+\hat{b}_\nu\right)
\end{equation}
Here, we will use that the redefinition of the bosonic operators such that $\hat{b_\nu} = \sum_\mu \left(\mathbf{X}_{\mu}^{\nu} \hat{\overline{b}}_\nu + \mathbf{Y}_{\mu}^{\nu} \hat{\overline{b}}_\nu^{\dagger}\right)$ and $\hat{b_\nu}^\dagger = \sum_\mu \left(\mathbf{X}_{\mu}^{\nu} \hat{\overline{b}}_\nu^\dagger + \mathbf{Y}_{\mu}^{\nu} \hat{\overline{b}}_\nu\right)$, which gives $\hat{b}_\nu + \hat{b}_{\nu}^\dagger = \sum_\mu \left(\mathbf{X}_{\mu}^{\nu} + \mathbf{Y}_{\mu}^{\nu}\right) \left(\hat{\overline{b}}_\nu + \hat{\overline{b}}_\nu^\dagger\right)$, so after we plug in
\begin{align}
\hat{V}^{\mathrm{eB}}&=\sum_{p q, \nu} V_{p q, \nu}\left\{\hat{a}_p^{\dagger} \hat{a}_q\right\}\left(\sum_\mu \left(\mathbf{X}_{\mu}^{\nu}  + \mathbf{Y}_{\mu}^{\nu} \right) \right)\left(\hat{\overline{b}}_\nu + \hat{\overline{b}}_\nu^\dagger\right) \\
&= \sum_{p q, \nu} W_{p q, \nu} \left\{ \hat{a}_p^{\dagger} \hat{a}_q \right\}\left(\bar{b}_\nu+\bar{b}_\nu^{\dagger}\right)
\end{align}
where now we had identified the RPA screened Coulomb interaction $W_{p q, \nu} = V_{p q, \nu} \sum_\mu\left(\mathbf{X}_{\mu}^{\nu} + \mathbf{Y}_{\mu}^{\nu}\right)$. 

\section{Connection to Booth supermatrix}
We then build the supermatrices $\mathbf{H}$ and $\mathbf{S}$ with matrix elements,
$$
\begin{gathered}
H_{I J}=\left\langle 0_{\mathrm{F}} 0_{\mathrm{B}}\right|\left[C_I,\left[\tilde{H}^{\mathrm{eB}}, C_J^{\dagger}\right]\right]\left|0_{\mathrm{F}} 0_{\mathrm{B}}\right\rangle \\
S_{I J}=\left\langle 0_{\mathrm{F}} 0_{\mathrm{B}}\right|\left[C_I, C_J^{\dagger}\right]\left|0_{\mathrm{F}} 0_{\mathrm{B}}\right\rangle
\end{gathered}
$$
where $\left\{C_I^{\dagger}\right\}=\left\{\underbrace{a_i}_{1h}, \underbrace{a_a}_{1p}, \underbrace{a_i b_\nu^{\dagger}}_{2h1p}, \underbrace{a_a b_\nu}_{1p2p}\right\}$ and $|0\rangle_{\mathrm{F}}$ and $|0\rangle_{\mathrm{B}}$ are the Fermi and boson vacuums. Then constructing $-\bm{S}^{-1}\bm{H}$ yields Booth's ED, which is
\begin{equation}
    \bm{H}^{G_0 W_0} = \begin{pmatrix} \bm{F} & \bm{W}^< & \bm{W}^> \\ \bm{W}^{\dagger<} & \bm{d}^< & \bm{0} \\ \bm{W}^{\dagger>} & \bm{0} & \bm{d}^> \end{pmatrix}
\end{equation}
where $\bm{F}$ is the Fock matrix, $\bm{W}^<$ and $\bm{W}^>$ are the lesser and greater components of the RPA screened Coulomb interaction, defined as
\begin{equation}
\begin{split}
    W_{pk\nu}^{<} = \sum_{ia} (pk|ia) \left( X_{ia}^{\nu} + Y_{ia}^{\nu} \right) \quad \text{and} \quad W_{pc\nu}^{>} = \sum_{ia} (pc|ia) \left( X_{ia}^{\nu} + Y_{ia}^{\nu} \right)
\end{split}
\end{equation}
and the auxiliary blocks $\bm{d}^<$ and $\bm{d}^>$ are defined as
\begin{equation}
\begin{split}
    d_{k\nu,l\nu'}^{<} = \left(\epsilon_k - \Omega_\nu\right) \delta_{k,l} \delta_{\nu,\nu'}\quad \text{and} \quad
    d_{c\nu,d\nu'}^{>} = \left(\epsilon_c + \Omega_\nu\right) \delta_{c,d} \delta_{\nu,\nu'}\\
\end{split}
\end{equation}
\subsection{Derivation of the supermatrices for the 2h1p sector}
\subsubsection{Overlap}
Computing the matrix elements of the $\bm{S}$ gives:
\begin{align}
\bm{S} &= \begin{pmatrix}
    \delta_{ij} & 0 & 0 & 0 \\
    0 & -\delta_{ab} & 0 & 0 \\
    0 & 0 & \delta_{ij}\delta_{\nu\nu'} & 0 \\
    0 & 0 & 0 & -\delta_{ab}\delta_{\nu\nu'}
\end{pmatrix}
\end{align}
\newpage
$\bm{H}$ takes more care. 
\subsubsection{Physical} 
\begin{equation}
\begin{split}
    H_{ij} &= \left\langle 0_{\mathrm{F}} 0_{\mathrm{B}}\right|\left[a_i^\dag, \left[\hat{H}^{\mathrm{eB}}, a_j\right]\right]\left|0_{\mathrm{F}} 0_{\mathrm{B}}\right\rangle = \left\langle 0_{\mathrm{F}} 0_{\mathrm{B}}\right|\left[a_i^\dag, \left[\hat{H}^{\mathrm{e}}, a_j\right]\right]\left|0_{\mathrm{F}} 0_{\mathrm{B}}\right\rangle \\
\end{split}
\end{equation}
We can make this simplification because the electronic operators commute with all bosonic operators. Now
\begin{align}
[\hat H^e,\;a_j]
&= \sum_{p q}f_{pq}\,\bigl[a_p^\dagger a_q,\;a_j\bigr]
= \sum_{p q}f_{pq}\left(a_p^\dagger\,[\,a_q,a_j\,]+ [a_p^\dagger, a_j] a_q\right)  \\
&= \sum_{p q}f_{pq}\left(a_p^\dagger a_q a_j - a_p^\dagger a_j a_q + a_p^\dagger a_j a_q -  a_j a_p^\dagger a_q \right) \\
[a_i^\dagger,\;[\hat H^e,\;a_j]] &= \sum_{p q}f_{pq}\,\bigl[a_i^\dagger,\left( a_p^\dagger a_q a_j  -  a_j a_p^\dagger a_q  \right)\bigr] \\
&= \sum_{p q}f_{pq}\,\bigl([a_i^\dagger, a_p^\dagger a_q a_j]  - [a_i^\dagger, a_j a_p^\dagger a_q]\bigr) \\
&= \sum_{p q}f_{pq}\,\bigl(  a_i^\dagger a_p^\dagger a_q a_j - a_p^\dagger a_q a_j a_i^\dagger -a_i^\dag a_j  a_p^\dagger a_q + a_j  a_p^\dagger a_q a_i^\dag \bigr) \\
    a_i^\dagger a_p^\dagger a_q a_j &= \wick{\c1 a_i^\dagger \c2 a_p^\dagger \c1 a_q \c2 a_j} + \wick{\c2 a_i^\dagger \c1 a_p^\dagger \c1 a_q \c2 a_j} = -\delta_{iq} \delta_{jp}+\cancel{\delta_{ij} \delta_{pq}} \\
a_p^\dagger a_q a_j a_i^\dagger &= 0 \\
a_i^\dag a_j a_p^\dagger a_q  &= \wick{\c2 a_i^\dag \c1 a_j \c1 a_p^\dagger \c2 a_q} + \wick{\c1 a_i^\dag \c1 a_j \c1 a_p^\dagger \c1 a_q} = 0 + \cancel{\delta_{ij} \delta_{pq}} \\
a_j a_p^\dagger a_q  a_i^\dagger &= 0 \\
&= - \sum_{p q}f_{pq} \delta_{iq} \delta_{jp} = - f_{ji} 
\end{align}
So ${H_{ij} = -f_{ji}}$ and then $\boxed{(-{S^{-1}H})_{ij} = +\delta_{ij}f_{ji}=f_{ii} = \epsilon_i}$.
\subsubsection{Auxiliary}
\begin{align}
H_{i \nu j \nu'} &= \left\langle 0_{\mathrm{F}} 0_{\mathrm{B}}\right|\left[a_i^{\dagger} b_\nu, \left[\hat{H}^{\mathrm{eB}}, a_j b_{\nu'}^{\dagger}\right]\right]\left|0_{\mathrm{F}} 0_{\mathrm{B}}\right\rangle = \left\langle 0_{\mathrm{F}} 0_{\mathrm{B}}\right|\left[a_i^{\dagger} b_\nu, \left[\hat{H}^{\mathrm{e}} + \hat{H}^{\mathrm{B}} + \hat{V}^{\mathrm{eB}}, a_j b_{\nu'}^{\dagger}\right]\right]\left|0_{\mathrm{F}} 0_{\mathrm{B}}\right\rangle \\
\end{align}
First
\begin{align}
[\hat H^e,\;a_j b_{\mu}^{\dagger}] &= \sum_{p q}f_{pq}\,[a_p^\dagger a_q, a_j b_{\mu}^{\dagger}] = \sum_{pq}f_{pq} [a_p^\dagger a_q a_j - a_j a_p^\dagger a_q] b_{\mu}^{\dagger} \\
[a_i^\dagger b_\nu,\;[\hat H^e,\;a_j b_{\mu}^{\dagger}]] &= [a_i^\dagger b_\nu,\; \sum_{pq}f_{pq} [a_p^\dagger a_q a_j - a_j a_p^\dagger a_q] b_{\mu}^{\dagger}] = \sum_{pq}f_{pq} [a_i^\dagger , a_p^\dagger a_q a_j - a_j a_p^\dagger a_q] [b_\nu, b_{\mu}^{\dagger}] \\
&= \delta_{\nu \mu} \sum_{pq}f_{pq} \left( a_i^\dagger a_p^\dagger a_q a_j - a_p^\dagger a_q a_j a_i^\dagger - a_i^\dagger a_j a_p^\dagger a_q + a_j a_p^\dagger a_q a_i^\dagger \right) \\
a_i^\dagger a_p^\dagger a_q a_j &= \wick{\c1 a_i^\dagger \c2 a_p^\dagger \c1 a_q \c2 a_j} + \wick{\c2 a_i^\dagger \c1 a_p^\dagger \c1 a_q \c2 a_j} = -\underbrace{\delta_{iq} \delta_{jp}}_{q,p \in O}+\cancel{\delta_{ij} \delta_{pq}} \\
a_p^\dagger a_q a_j a_i^\dagger &= 0 \\
a_i^\dagger a_j a_p^\dagger a_q &= \wick{\c2 a_i^\dagger \c1 a_j \c1 a_p^\dagger \c2 a_q} + \wick{\c1 a_i^\dagger \c1 a_j \c1 a_p^\dagger \c1 a_q} = 0 + \cancel{\delta_{ij} \delta_{pq}} \\
a_p^\dagger a_j a_q a_i^\dagger &= 0 \\
\end{align}
So $[a_i^\dagger b_\nu,\;[\hat H^e,\;a_j b_{\mu}^{\dagger}]] = - \delta_{\nu \mu} f_{ji}$.
Next
\begin{align}
[\hat H^B,\;a_j b_{\mu}^\dagger] &= [\sum_{\nu} \Omega_\nu {b}_\nu^{\dagger} {b}_\nu + E_{\mathrm{RPA}}^c, a_j b_{\mu}^\dag] = \sum_{\nu} \Omega_\nu a_j[ {b}_\nu^{\dagger} {b}_\nu, b_{\mu}^\dag]=\sum_{\nu} \Omega_\nu a_j {b}_\nu^{\dagger} [{b}_\nu, b_{\mu}^\dag]=\Omega_\mu a_j {b}_\mu^{\dagger} \\
[a_i^\dagger b_\nu,\;[\hat H^B,\;a_j b_{\mu}^\dagger]] &= \Omega_\mu [a_i^\dagger b_\nu, a_j {b}_\mu^{\dagger}] = \Omega_{\mu} [a_i^\dagger, a_j] [b_\nu, b_{\mu}^\dag] = \delta_{ij} \delta_{\nu \mu} \Omega_{\mu}\\
\end{align}
So $[a_i^\dagger b_\nu,\;[\hat H^B,\;a_j b_{\mu}^\dagger]] = \delta_{ij} \delta_{\nu \mu} \Omega_{\mu}$.
% [a_i b_\nu^\dagger,\;[\hat H^B,\;a_j^\dagger b_{\mu}]] &=  \sum_{\lambda} \Omega_{\lambda}[a_i b_\nu^\dagger, a_j^{\dagger}[ {b}_{\lambda}^{\dagger} {b}_{\lambda}, b_{\mu}] ] = -\Omega_{\mu} [a_i,\;a_j^\dagger] [b_\nu^\dagger,[ {b}_{\lambda}^{\dagger} {b}_{\lambda}, b_{\mu}]] \\
last
\begin{align}
[a_i^\dagger b_\nu,\;[\hat V^{eB},\;a_j b_{\nu'}^\dagger]] & = 0
\end{align}
because we notice that in the coupling term, the number of bosons is not conserved.
So ${H_{i\nu j \nu'} = \delta_{\nu \nu'} \left(\delta_{ij}\Omega_{\nu'}- f_{ji}\right)}$ and then $\boxed{(-{S^{-1}H})_{i\nu j \nu'} = -\delta_{ij}\left( \delta_{\nu \nu'} \left(\delta_{ij}\Omega_{\nu}- f_{ji}\right)\right)= \delta_{ij}\delta_{\nu \nu'}\left(\epsilon_i - \Omega_{\nu}\right)}$.
\subsubsection{Coupling}
\begin{align}
H_{i,p\nu} &= \left\langle 0_{\mathrm{F}} 0_{\mathrm{B}}\right|\left[a_i^\dag, \left[\hat{H}^{\mathrm{eB}}, a_p b_\nu^{\dagger} \right]\right]\left|0_{\mathrm{F}} 0_{\mathrm{B}}\right\rangle = \left\langle 0_{\mathrm{F}} 0_{\mathrm{B}}\right|\left[a_i^\dag, \left[\hat{V}^{\mathrm{eB}}, a_p b_\nu^{\dagger} \right]\right]\left|0_{\mathrm{F}} 0_{\mathrm{B}}\right\rangle
\end{align}\
Note that we have neglected the electronic and bosonic Hamiltonians, because using them in this arrangement will not conserve the number of bosons. So we have
\begin{align}
[ V^{eB},\; a_p b_\nu^\dagger ] &= \sum_{rs,\nu'} W_{rs, \nu'} [ a_r^{\dagger} a_s ( b_{\nu'} + b_{\nu'}^{\dagger} ), a_p b_\nu^\dagger ] = \sum_{rs,\nu'} W_{rs, \nu'} [ a_r^{\dagger} a_s, a_p ][ b_{\nu'}, b_\nu^\dagger ] \\
&= \sum_{rs} W_{rs, \nu} \left( a_r^{\dagger} a_s a_p -  a_p a_r^{\dagger} a_s \right) \\
[a_i^\dagger,\;[\hat V^{eB},\;a_p b_\nu^\dagger]] &= \sum_{rs} W_{rs, \nu} \left[ a_i^\dagger \hat{a}_r^{\dagger} \hat{a}_s a_p  - \hat{a}_r^{\dagger} \hat{a}_s a_p a_i^\dagger  - \left( a_i^\dag a_p a_r^{\dagger}a_s   - a_p a_r^{\dagger}a_s a_i^\dagger \right) \right] \\
a_i^\dagger \hat{a}_r^{\dagger} \hat{a}_s a_p  &= \wick{\c1 a_i^\dagger \c2 a_r^{\dagger} \c1 a_s \c2 a_p} + \wick{\c2 a_i^\dagger \c1 a_r^{\dagger} \c1 a_s \c2 a_p} = -\underbrace{\delta_{is} \delta_{rp}}_{s \in O} + \cancel{\delta_{ip} \delta_{rs}} \\
\hat{a}_r^{\dagger} \hat{a}_s a_p a_i^\dagger &= \wick{\c1 a_r^{\dagger} \c2 a_s \c1 a_p \c2 a_i^\dagger} + \wick{\c1 a_r^{\dagger} \c1 a_s \c1 a_p \c1 a_i^\dagger} = 0 \\
a_i^\dag a_p a_r^{\dagger}a_s &= \wick{\c1 a_i^\dag \c1 a_p \c1 a_r^{\dagger} \c1 a_s} + \wick{\c2 a_i^\dag \c1 a_p \c1 a_r^{\dagger} \c2 a_s} = \cancel{\delta_{ip} \delta_{rs}} + 0 \\
a_p a_r^{\dagger}a_s a_i^\dagger &= 0 \\
% a_i^\dagger b_\nu ] &= \sum_{rs,\nu'} W_{rs, \nu'} [ a_r^{\dagger} a_s ( b_{\nu'} + b_{\nu'}^{\dagger} ), a_i^{\dagger} b_\nu ] = \sum_{rs,\nu'} W_{rs, \nu'} \left[ a_r^{\dagger} a_s a_i^{\dagger} [ b_{\nu'} + b_{\nu'}^{\dagger}, b_\nu ] + [ a_r^{\dagger} a_s, a_i^{\dagger} ] b_{\nu} ( b_{\nu'} + b_{\nu'}^{\dagger} ) \right] \\
% [a_p,\;[\hat V^{eB},\;a_i^\dagger b_\nu]] &= \sum_{rs,\nu'} W_{rs, \nu'} \left[ a_p, -\hat{a}_r^{\dagger} \hat{a}_s a_i^{\dagger} \delta_{\nu' \nu} + \hat{a}_r^{\dagger} \delta_{si} b_{\nu}(\hat{b}_{\nu'} + \hat{b}_{\nu'}^{\dagger}) \right] = \sum_{rs} W_{rs, \nu} \left[ a_p, -\hat{a}_r^{\dagger} \hat{a}_s a_i^{\dagger}  \right] \\
% &= \sum_{rs} W_{rs, \nu} \left( -a_p \hat{a}_r^{\dagger} \hat{a}_s a_i^{\dagger} + \hat{a}_r^{\dagger} \hat{a}_s a_i^{\dagger} a_p\right) = \sum_{rs} W_{rs, \nu} \left( \hat{a}_r^{\dagger} \hat{a}_s a_i^{\dagger} a_p\right) 
\end{align}
So we can write ${H_{i,p \nu} = -W_{pi,\nu}=-W_{ip,\nu}}$ because of the permutational symmetry of the ERIs. Then $\boxed{(-{S^{-1}H})_{p,i \nu} =+ W_{pi,\nu}}$.
% \begin{equation}
% \begin{split}
% \hat{a}_p^{\dagger} \hat{a}_q a_i^{\dagger} [\hat{b}_{\nu'} + \hat{b}_{\nu'}^{\dagger}, b_\nu] = -\hat{a}_p^{\dagger} \hat{a}_q a_i^{\dagger} \delta_{\nu' \nu}\\
% [\hat{a}_p^{\dagger} \hat{a}_q, a_i^{\dagger}]b_{\nu}(\hat{b}_{\nu'} + \hat{b}_{\nu'}^{\dagger}) = \hat{a}_p^{\dagger} \delta_{qi} b_{\nu}(\hat{b}_{\nu'} + \hat{b}_{\nu'}^{\dagger})
% \end{split}
% \end{equation}
\section{Implementation for periodic systems}
\subsection{RI}
We know the periodic integrals can be expressed as:
\begin{equation}
    \left(i \mathbf{k}_1, a \mathbf{k}_2 \mid j \mathbf{k}_3, b \mathbf{k}_4\right)=\delta_{\mathbf{k}_1+\mathbf{k}_2, \mathbf{k}_3+\mathbf{k}_4+\mathbf{G}}(i a \mid j b)_{\mathbf{q}'}
\end{equation}
where $\mathbf{q}'=\mathbf{k}_1-\mathbf{k}_3$ and $\mathbf{G}$ is a reciprocal lattice vector. The RI coefficients are then given by
\begin{equation}
    R_{jb}^L(\mathbf{q})=\sum_Q\left[\mathbf{V}^{-1 / 2}(\mathbf{q})\right]_{LQ} \left( Q \mathbf{q} \mid j \mathbf{k}_j, b \mathbf{k}_b\right)
    % \left(i \mathbf{k}_i, a \mathbf{k}_a \mid Q \mathbf{q}\right)
\end{equation}
It is understood that we must have $\bm{q} = \mathbf{k}_j - \mathbf{k}_b$ to ensure momentum conservation and $\mathbf{V}(\mathbf{q})$ is the Coulomb metric in the auxiliary basis for that $\mathbf{q}$:
\begin{equation}
    V_{P Q}(\mathbf{q})=(P \mathbf{q} \mid Q \mathbf{q})=\sum_{\mathbf{R}} e^{-i \mathbf{q} \cdot \mathbf{R}} \int d \mathbf{r} d \mathbf{r}^{\prime} \chi_P(\mathbf{r}) \frac{1}{\left|\mathbf{r}-\mathbf{r}^{\prime}\right|} \chi_Q\left(\mathbf{r}^{\prime}-\mathbf{R}\right)
\label{coul}
\end{equation}
So we can recover the ERIs as
\begin{align}
    (i a \mid j b)_{\mathbf{q}} &= \sum_L \left( R_{i a}^L(\mathbf{q}) \right) ^\dagger R_{j b}^L(\mathbf{q}) \\
    &= \sum_L \left( \sum_Q (i \mathbf{k}_i, a \mathbf{k}_a \mid Q \mathbf{q}) [\mathbf{V}^{-1/2}(\mathbf{q})]_{Q L} \right)
       \left( \sum_R [\mathbf{V}^{-1/2}(\mathbf{q})]_{LR}(R \mathbf{q} \mid j \mathbf{k}_j, b \mathbf{k}_b)\right) \\
    &= \sum_{Q R} (i \mathbf{k}_i, a \mathbf{k}_a \mid Q \mathbf{q}) \left( \sum_L [\mathbf{V}^{-1/2}(\mathbf{q})]_{Q L} [\mathbf{V}^{-1/2}(\mathbf{q})]_{LR} \right)
       (R \mathbf{q} \mid j \mathbf{k}_j, b \mathbf{k}_b)\\
    &= \sum_{Q R} (i \mathbf{k}_i, a \mathbf{k}_a \mid Q \mathbf{q}) [\mathbf{V}^{-1}(\mathbf{q})]_{Q R}
       (R \mathbf{q} \mid j \mathbf{k}_j, b \mathbf{k}_b) \\
    &= (i \mathbf{k}_i, a \mathbf{k}_a \mid j \mathbf{k}_j, b \mathbf{k}_b)_{\mathbf{q}}.
\end{align}
where in the last step we used the definition of \ref{coul} to make the appropriate cancellation. Then, we can define a basis of auxiliary bosons as
\begin{equation}
    \hat{b}_\nu(\mathbf{q}) = \sum_Q^{N_{AB}} C_\nu^Q(\mathbf{q}) \hat{b}_Q(\mathbf{q})
\end{equation}
where we define the expansion coefficients as
\begin{equation}
    C_\nu^Q(\mathbf{q}) = \sum_L R_\nu^L(\mathbf{q})\left[\bm{S}^{-1 / 2}(\mathbf{q})\right]_{L M} P_M^Q
\end{equation}
where $S_{L M}(\mathbf{q}) = \sum_\nu R_\nu^L(\mathbf{q}) R_\nu^M(\mathbf{q})= \sum_Q P_L^Q (\mathbf{q}) E_Q(\mathbf{q}) P_M^Q(\mathbf{q})$ can be thought of as the overlap matrix in this AB basis where the final expression employs its eigendecomposition. Even though the axillary bases already is small, we can obtain a further truncated basis by only choosing the eigenvalues over a certain threshhold $E_Q(\mathbf{q}) > \epsilon_{AB}$. This is a similar idea to the one that was used to identify rank deficiency in Lanczos. 
\subsubsection{Scaling comments}
Getting the RI coefficients into the MO basis scales as $O(N_{\text{orb}}^3N_{\text{aux}}N_{\mathbf{k}})$, the determination of the overlap matrix scales as $O(N_oN_vN_{\text{aux}}^2N_{\mathbf{k}}^2)$, and then computing the AB basis will cost $O(N_{\text{AB}}N_{\text{aux}}^2N_{\mathbf{k}})$. In order to construct the transformed electron boson Hamiltonian in the AB basis we first need to solve the symmetrized dRPA eigenvalue problem 

%  $\hat{b}_\nu(\mathbf{q}) \approx \sum_{L} R_\nu^L(\mathbf{q}) \hat{b}_L(\mathbf{q})$ where $\hat{b}_L(\mathbf{q})$ are the auxiliary bosons in the $\mathbf{q}$-dependent basis. The orthogonalization and AB construction also becomes $\mathbf{q}$-dependent:
% The orthogonalization and AB construction also becomes $\mathbf{q}$-dependent:
% \begin{align}
% S_{L M}(\mathbf{q}) &= \sum_\nu R_\nu^L(\mathbf{q}) R_\nu^M(\mathbf{q}) \\
% C_\nu^Q(\mathbf{q}) &= \sum_{L M} R_\nu^L(\mathbf{q})\left[S^{-1 / 2}(\mathbf{q})\right]_{L M} P_M^Q
% \end{align} 
% % In a periodic setting, the Rl expansion coefficients become $\mathbf{k}$-dependent:



