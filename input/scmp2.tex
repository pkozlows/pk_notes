For self-consistent MP2, do we mean 
\begin{enumerate}
    \item Start with HF reference orbitals and energies.
    \item Solve the QP equation for each orbital using the MP2 self-energy, as given in equation \ref{eq:mp2_se}.
    \item Update all the orbital energies in \ref{eq:mp2_se}, keeping the numerator fixed.
    \item Repeat steps 2-3 until convergence in the orbital energies of interest.
\end{enumerate}
\begin{align}
    \Sigma^{(2)}_{pp}(\omega) 
    & = \sum_{iab} \frac{(pa|ib)\Big(2(qa|ib) - (qb|ia)\Big)}{\omega - \epsilon_a - \epsilon_b + \epsilon_i + i \eta} + \sum_{ija} \frac{ (pi|aj)\Big(2(qi|aj) - (qj|ai)\Big)}{\omega - \epsilon_i - \epsilon_j + \epsilon_a + i \eta} 
\label{eq:mp2_se}
\end{align}
Note that scGF2, by virtue of operating on the imaginary frequency axis, can also update the orbitals, but in this real time method we cannot.
\section{Why it is bad for the UEG}
The VASP paper by Kresse and Grunes proves that scMP2 consists of terminating the Dyson equation for the interacting polarizability to lowest order, so setting $\chi(\omega ) \approx \chi_0(\omega )$. This implies that the MP2 self-energy is only reliable if the polarizability of the system is small, so the neglect of higher order terms does not matter so much. I believe this is why the MP2 pole at $r_s=4$ is bad. This can be proven analytically be examining the magnitude of the Linhard function.
% \subsection{Suggestions for improvement}
% Inclusion of Thomas-Fermi screening, as proposed by Kresse
\section{Notes}
David Tew's recent paper investigates the performance of this quasiparticle MP2 for finite systems, but they only do small molecules and the Hubbard model.
\section{Decisions to make}
\subsection{Basis choice}
I think def2-TZVPP-RIFIT would be good and then all-electron?
