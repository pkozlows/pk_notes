\section{Self-consistent schemes for QPMP2}
I can envision the following self-consistency schemes for this QPMP2 using
\begin{align}
    \Sigma^{(2)}_{pp}(\omega) 
    & = \sum_{iab} \frac{(pa|ib)\Big(2(pa|ib) - (pb|ia)\Big)}{\omega - \epsilon_a - \epsilon_b + \epsilon_i + i \eta} + \sum_{ija} \frac{ (pi|aj)\Big(2(pi|aj) - (pj|ai)\Big)}{\omega - \epsilon_i - \epsilon_j + \epsilon_a + i \eta} 
\label{eq:mp2_se}
\end{align}
\subsection{No self consistency (1shotQPMP2)}
This just comprises of solving the QP equation using the self energy defined in \ref{eq:mp2_se} with the HF orbitals and MO energies in a one-shot fashion. So for each orbital $p$ we solve
\begin{align}
    \epsilon_p^{\text{QP}} = \epsilon_p^{\text{HF}} + Z_p \Sigma^{(2)}_{pp}(\epsilon_p^{\text{HF}})
\label{eq:qp_eq}
\end{align}
where $Z_p = \Big(1 - \frac{\partial \Sigma^{(2)}_{pp}(\omega)}{\partial \omega}\big|_{\omega = \epsilon_p^{\text{HF}}}\Big)^{-1}$ is the renormalization factor. 
\subsection{Eigenvalue self-consistency (evQPMP2)}
\label{sec:evqpmp2}
\begin{enumerate}
    \item Start with HF reference orbitals and energies.
    \item Solve the QP equation \ref{eq:qp_eq} for each orbital using the MP2 self-energy, as given in equation \ref{eq:mp2_se}.
    \item Update all the orbital energies in \ref{eq:mp2_se}, keeping the numerator fixed.
    \item Repeat steps 2-3 until convergence in the orbital energies of interest.
\end{enumerate}
\section{Exploring the similarity renormalization group approach to quasi-particle self-consistency}
The unrestricted form for the MP2 self-energy is
\begin{align}
    \Sigma_{pq}^{(2)}(\omega) =& \frac{1}{2} \sum_{iab} \frac{\langle pi || ab \rangle \langle ab || qi \rangle}{\omega + \epsilon_i - \epsilon_a - \epsilon_b} + \frac{1}{2} \sum_{ija} \frac{\langle pa || ij \rangle \langle ij || qa \rangle}{\omega + \epsilon_a - \epsilon_i - \epsilon_j} \\
\end{align}
To derive the restricted form, consider just the 2p1h sector and expand the antisymmetrized integrals. We can associate spin variables $\sigma_p, \sigma_q, \sigma_i, \sigma_a, \sigma_b$ to each of the orbitals $p,q,i,a,b$. Then, with $\Delta_{iab} \equiv \epsilon_a + \epsilon_b - \epsilon_i$, we have
\begin{align}
    \Sigma_{pq}^{(2p1h)}(\omega)
&= \frac{1}{2} \sum_{iab} \frac{\left( \langle pi | ab \rangle - \langle pi | ba \rangle \right) \left( \langle ab | qi \rangle - \langle ab | iq \rangle \right)}{\omega - \Delta_{iab}} \\
\implies \Sigma_{p \sigma_p, q \sigma_q}^{(2p1h)}(\omega)
&= \frac{1}{2} \sum_{\sigma_i, \sigma_a, \sigma_b} \sum_{iab} \frac{\left( (p\sigma_p a\sigma_a | i\sigma_i b\sigma_b) - (p\sigma_p b\sigma_b | i\sigma_i a\sigma_a) \right) \left( (a\sigma_a b\sigma_b | q\sigma_q i\sigma_i) - (a\sigma_a b\sigma_b | i\sigma_i q\sigma_q) \right)}{\omega - \Delta_{iab}} \\
\end{align}
% and the corresponding coupling blocks read [see eq 5]
% $$
% W_{p, i \nu}^{2 \mathrm{hlp}}=W_{p i}^\nu, \quad W_{p, a \nu}^{2 \mathrm{pp}}=W_{p a}^\nu
% $$

% The usual $G W$ nonlinear equation can be obtained by applying the Löwdin partitioning technique ${ }^{192}$ to eq 21 yielding ${ }^{190}$
% $$
% \begin{aligned}
% \boldsymbol{\Sigma}(\omega)= & \boldsymbol{W}^{2 \mathrm{hlp}}\left(\omega \mathbf{1}-\boldsymbol{C}^{2 \mathrm{hlp}}\right)^{-1}\left(\boldsymbol{W}^{2 \mathrm{hlp}}\right)^{\dagger} \\
% & +\boldsymbol{W}^{2 \mathrm{plh}}\left(\omega \mathbf{1}-\boldsymbol{C}^{2 \mathrm{plh}}\right)^{-1}\left(\boldsymbol{W}^{2 \mathrm{plh}}\right)^{\dagger}
% \end{aligned}
% $$
% which can be further developed to recover exactly eq us
.
\subsection{Downfolding the QPMP2 Hamiltonian}
If we use that Lowdin partitioning we are able to recast the QPMP2 equation into a frequency independent Hamiltonian of the form
\begin{align}
    H^{\text{QPMP2}}= \begin{pmatrix}
        \bm{F} & \bm{W}^{2h1p} & \bm{W}^{2p1h} \\
        (\bm{W}^{2h1p})^\dagger & \bm{C}^{2h1p} & 0 \\
        (\bm{W}^{2p1h})^\dagger & 0 & \bm{C}^{2p1h}
    \end{pmatrix}
\end{align}
where now the 2h1p and 2p1h blocks are defined as
\begin{align}
    C^{2h1p}_{ija, klb} & = (\epsilon_k - \epsilon_l + \epsilon_b) \delta_{ik} \delta_{jl} \delta_{ab} \\
    C^{2p1h}_{abi, cdj} & = (\epsilon_c + \epsilon_d - \epsilon_j) \delta_{ac} \delta_{bd} \delta_{ij}
\end{align}
and the coupling blocks are
\begin{align}
    W^{2h1p}_{p, ija} & = \frac{\sqrt{2}}{2} \langle pa || ij \rangle \\
    W^{2p1h}_{p, abi} & = \frac{\sqrt{2}}{2} \langle pi || ab \rangle
\end{align}
Now, we can multiply this matrix by a trial vector $\Psi = (\bm{X}^{1h1p}, \bm{X}^{2h1p}, \bm{X}^{2p1h})^T$ to obtain the following set of coupled equations
\begin{align}
    \bm{F} \bm{X}^{1h1p} + \bm{W}^{2h1p} \bm{X}^{2h1p} + \bm{W}^{2p1h} \bm{X}^{2p1h} & = E \bm{X}^{1h1p} \\
    (\bm{W}^{2h1p})^\dagger \bm{X}^{1h1p} + \bm{C}^{2h1p} \bm{X}^{2h1p} & = E \bm{X}^{2h1p} \\
    (\bm{W}^{2p1h})^\dagger \bm{X}^{1h1p} + \bm{C}^{2p1h} \bm{X}^{2p1h} & = E \bm{X}^{2p1h}
\end{align}
From the last two equations we can solve for $\bm{X}^{2h1p}$ and $\bm{X}^{2p1h}$ as
\begin{align}
    \bm{X}^{2h1p} & = (E \bm{1} - \bm{C}^{2h1p})^{-1} (\bm{W}^{2h1p})^\dagger \bm{X}^{1h1p} \\
    \bm{X}^{2p1h} & = (E \bm{1} - \bm{C}^{2p1h})^{-1} (\bm{W}^{2p1h})^\dagger \bm{X}^{1h1p}
\end{align}
Substituting these expressions back into the first equation we obtain an effective eigenvalue equation for the 1h1p space
\begin{align}
    \Bigg[ \bm{F} + \bm{W}^{2h1p} (E \bm{1} - \bm{C}^{2h1p})^{-1} (\bm{W}^{2h1p})^\dagger + \bm{W}^{2p1h} (E \bm{1} - \bm{C}^{2p1h})^{-1} (\bm{W}^{2p1h})^\dagger \Bigg] \bm{X}^{1h1p} = E \bm{X}^{1h1p}
\end{align}
where we recognize the terms in square brackets as a sum of the Fock matrix and the MP2 self-energy evaluated at energy $E$.
\subsection{SRG approach}
 Now, we define, within the SRG formalism, the diagonal and off-diagonal parts of the $\bm{H}^{\text{QPMP2}}$ as
\begin{align}
\bm{H}^{\text{d}}(s) &= 
\begin{pmatrix}
\bm{F}	&	\bm{0}	                	&     \bm{0}		\\
\bm{0}	&	\bm{C}^{\text{2h1p}}		&	\bm{0}			\\
\bm{0}    &	\bm{0}				&	\bm{C}^{\text{2p1h}}	\\
\end{pmatrix} \equiv \begin{pmatrix}
\bm{F}	&	\bm{0}\\
\bm{0}	&	\bm{C}    \\
\end{pmatrix},
\\  \bm{H}^{\text{od}}(s)  &= 
\begin{pmatrix}
\bm{0}			                &     \bm{W}^{\text{2h1p}} &     \bm{W}^{\text{2p1h}} \\
(\bm{W}^{\text{2h1p}})^\dag	&	\bm{0}	   	        &	\bm{0}                    	\\
(\bm{W}^{\text{2p1h}})^\dag	&	\bm{0}		        &	\bm{0}	                \\
\end{pmatrix} \equiv \begin{pmatrix}
\bm{0}	&	\bm{W}	\\
\bm{W}^\dag	&	\bm{0}    \\
\end{pmatrix}
\end{align}
where we omit the $s$ dependence of the matrices for the sake of brevity.
Then, our aim is to solve, order by order, the flow equation 
\begin{equation}
    \dv{\bm{H}(s)}{s} = [\boldsymbol{\eta}(s), \bm{H}(s)]
\label{eq:flowEquation}
\end{equation}
% \begin{equation}
%   \label{eq:flowEquation}
%   \dv{\bH(s)}{s} = \comm{\boldsymbol{\eta}(s)}{\bH(s)},
% \end{equation}
 knowing that the initial conditions are
\begin{align}
    \bm{H}_d^{(0)}(0)  = \begin{pmatrix}
        \bm{F} & \bm{0} \\
        \bm{0} & \bm{C} \\
    \end{pmatrix},& \bm{H}_{od}^{(0)}(0) = \bm{0}, \\
    \bm{H}_d^{(1)}(0) = \bm{0},& \bm{H}_{od}^{(1)}(0) = \begin{pmatrix}
        \bm{0} & \bm{W} \\
        \bm{W}^\dag & \bm{0} \\
    \end{pmatrix}
\end{align}
Once the closed-form expressions of the low-order perturbative expansions are known, they can be inserted into the downfolded self-energy to define a renormalized version of the quasiparticle equation. 
\subsection{Order by order expansion}
Now, Wegner's generator is
\begin{align}
    \boldsymbol{\eta}(s) &= [\bm{H}^d(s), \bm{H}^{od}(s)] \\
    & =[ \bm{H}^{d(0)}(s) + \lambda \bm{0}+ \lambda^2 \bm{H}^{d(2)}(s) + \order*{\lambda^3}, \bm{0} + \lambda \bm{H}^{od(1)}(s) + \lambda^2 \bm{H}^{od(2)}(s) + \order*{\lambda^3}] \\
    & = \underbrace{\bm{H}_d^{(0)}(s)}_{\eta^{(0)}(s)} + \lambda \underbrace{[\bm{H}_d^{(0)}(s), \bm{H}_{od}^{(1)}(s)]}_{\eta^{(1)}(s)} + \lambda^2 \underbrace{\Big( [\bm{H}_d^{(0)}(s), \bm{H}_{od}^{(2)}(s)] \Big)}_{\eta^{(2)}(s)} + \order*{\lambda^3}
\end{align}
and the flow equation \ref{eq:flowEquation} can also be expanded order by order in $\lambda$ as
\begin{align}
    \dv{\bm{H}(s)}{s} & = [\boldsymbol{\eta}(s), \bm{H}(s)] \\
\text{LHS} & = \dv{\Big( \bm{H}^{(0)}(s) + \lambda \bm{H}^{(1)}(s) + \lambda^2 \bm{H}^{(2)}(s) + \order*{\lambda^3} \Big)}{s}\\
\text{RHS} & = \lambda [\boldsymbol{\eta}^{(1)}(s), \bm{H}^{(0)}(s)] + \lambda^2 \Big( [\boldsymbol{\eta}^{(1)}(s), \bm{H}^{(1)}(s)] + [\boldsymbol{\eta}^{(2)}(s), \bm{H}^{(0)}(s)] \Big) + \order*{\lambda^3}
\end{align}
\subsubsection{Equating to zeroth order}
From the expansion of the flow equation, we get
\begin{equation}
    \dv{\bm{H}^{(0)}(s)}{s} = \bm{0}
\end{equation}
and then using our initial condition gives
\begin{equation}
%   \bH^{(0)}(s) = \bH^{(0)}(0).
\bm{H}^{(0)}(s) = \bm{H}^{(0)}(0) = \begin{pmatrix}
        \bm{F} & \bm{0} \\
        \bm{0} & \bm{C} \\
    \end{pmatrix}
\end{equation}
\subsubsection{Equating to first order}
At first order we have
\begin{align}
    &\dv{\bm{H}^{(1)}(s)}{s} = [\boldsymbol{\eta}^{(1)}(s), \bm{H}^{(0)}(s)] \\
\text{LHS} &= \begin{pmatrix}
    \dv{\bm{F}^{(1)}(s)}{s}	&	\dv{\bm{W}^{(1)}(s)}{s}	\\
    \dv{(\bm{W}^{(1)}(s))^\dag}{s}	& \dv{\bm{C}^{(1)}(s)}{s}  \\
\end{pmatrix} \\
\text{RHS} &= [ [\bm{H}_d^{(0)}(s), \bm{H}_{od}^{(1)}(s)], \bm{H}_d^{(0)}(s)] \\
&= [\bm{H}_d^{(0)}(s) \bm{H}_{od}^{(1)}(s) - \bm{H}_{od}^{(1)}(s) \bm{H}_d^{(0)}(s), \bm{H}_d^{(0)}(s)] \\
&= \bm{H}_d^{(0)}(s) \bm{H}_{od}^{(1)}(s) \bm{H}_d^{(0)}(s) - \bm{H}_{od}^{(1)}(s) \bm{H}_d^{(0)}(s) \bm{H}_d^{(0)}(s) - \bm{H}_d^{(0)}(s) \bm{H}_d^{(0)}(s) \bm{H}_{od}^{(1)}(s) + \bm{H}_d^{(0)}(s) \bm{H}_{od}^{(1)}(s) \bm{H}_d^{(0)}(s)\\
&= \begin{pmatrix}
    \bm{F} & \bm{0} \\
    \bm{0} & \bm{C}
\end{pmatrix}
\begin{pmatrix}
    \bm{0} & \bm{W} \\
    \bm{W}^\dag & \bm{0}
\end{pmatrix}
\begin{pmatrix}
    \bm{F} & \bm{0} \\
    \bm{0} & \bm{C}
\end{pmatrix} 
 - \begin{pmatrix}
    \bm{0} & \bm{W} \\
    \bm{W}^\dag & \bm{0}
\end{pmatrix}
\begin{pmatrix}
    \bm{F} & \bm{0} \\
    \bm{0} & \bm{C}
\end{pmatrix}
\begin{pmatrix}
    \bm{F} & \bm{0} \\
    \bm{0} & \bm{C}
\end{pmatrix} \\
& - \begin{pmatrix}
    \bm{F} & \bm{0} \\
    \bm{0} & \bm{C}
\end{pmatrix}
\begin{pmatrix}
    \bm{F} & \bm{0} \\
    \bm{0} & \bm{C}
\end{pmatrix}
\begin{pmatrix}
    \bm{0} & \bm{W} \\
    \bm{W}^\dag & \bm{0}
\end{pmatrix} + \begin{pmatrix}
    \bm{F} & \bm{0} \\
    \bm{0} & \bm{C}
\end{pmatrix}
\begin{pmatrix}
    \bm{0} & \bm{W} \\
    \bm{W}^\dag & \bm{0}
\end{pmatrix}
\begin{pmatrix}
    \bm{F} & \bm{0} \\
    \bm{0} & \bm{C}
\end{pmatrix} \\
&= \begin{pmatrix}
    \bm{0} & 2 \bm{F} \bm{W} \bm{C} - \bm{F}^2 \bm{W} - \bm{W} \bm{C}^2 \\
    2 \bm{C} \bm{W}^\dag \bm{F} - \bm{C}^2 \bm{W}^\dag - \bm{W}^\dag \bm{F}^2 & \bm{0}
\end{pmatrix}
\end{align}
which implies that the first-order flow equations are
\begin{align}
    \dv{\bm{F}^{(0)}(s)}{s} = \bm{0}, & \dv{\bm{C}^{(0)}(s)}{s} = \bm{0} \\
    \implies \bm{F}^{(0)}(s) = \bm{F}^{(0)}(0), & \bm{C}^{(0)}(s) = \bm{C}^{(0)}(0) \\
\end{align}
Now, we know that in the canonical HF basis both $\bm{F}$ and $\bm{C}$ are diagonal, so really we have found that
\begin{align}
    \dv{W_{pq,ia}^{(1)}(s)}{s} & = 2 F_{pp} W_{pq,ia}^{(1)}(s) C_{ia,ia} - F_{pp}^2 W_{pq,ia}^{(1)}(s) - W_{pq,ia}^{(1)}(s) C_{ia,ia}^2 \\
    &= \left[ 2 F_{pp} C_{ia,ia} - F_{pp}^2 - C_{ia,ia}^2 \right] W_{pq,ia}^{(1)}(s) \\
&= \left[ 2 \epsilon_p (\epsilon_i - \epsilon_a) - \epsilon_p^2 - (\epsilon_i - \epsilon_a)^2 \right] W_{pq,ia}^{(1)}(s) \\
&= -(\epsilon_p - \epsilon_i + \epsilon_a)^2 W_{pq,ia}^{(1)}(s) \\
\implies
    W_{pq,ia}^{(1)}(s) & = W_{pq,ia}^{(1)}(0) e^{-(\epsilon_p - \epsilon_i + \epsilon_a)^2 s}
\end{align}
so we get that $W_{pq,ia}^{(1)}(0) = W_{pq,ia}$ and $\lim_{s \to \infty} W_{pq,ia}^{(1)}(s) = 0$.
\subsubsection{Equating to second order}
At second order we have
\begin{align}
    &\dv{\bm{H}^{(2)}(s)}{s} = [\boldsymbol{\eta}^{(1)}(s), \bm{H}^{(1)}(s)] + [\boldsymbol{\eta}^{(2)}(s), \bm{H}^{(0)}(s)] = [\boldsymbol{\eta}^{(1)}(s), \bm{H}^{(1)}(s)] \\
\text{LHS} & = \begin{pmatrix}
    \dv{\bm{F}^{(2)}(s)}{s}	&    \dv{\bm{W}^{(2)}(s)}{s}	\\
    \dv{(\bm{W}^{(2)}(s))^\dag}{s}	& \dv{\bm{C}^{(2)}(s)}{s}  \\
\end{pmatrix} \\
\text{RHS} & = [ [\bm{H}_d^{(0)}(s), \bm{H}_{od}^{(1)}(s)], \bm{H}^{(1)}(s)] \\
&= [\bm{H}_d^{(0)}(s) \bm{H}_{od}^{(1)}(s) - \bm{H}_{od}^{(1)}(s) \bm{H}_d^{(0)}(s), \bm{H}^{(1)}_{od}(s)] \\
&= \bm{H}_d^{(0)}(s) \bm{H}_{od}^{(1)}(s) \bm{H}_{od}^{(1)}(s) - \bm{H}_{od}^{(1)}(s) \bm{H}_d^{(0)}(s) \bm{H}_{od}^{(1)}(s) - \bm{H}_{od}^{(1)}(s) \bm{H}_{d}^{(0)}(s) \bm{H}_{od}^{(1)}(s) + \bm{H}_{od}^{(1)}(s) \bm{H}_{od}^{(1)}(s) \bm{H}_d^{(0)}(s)\\
&= \begin{pmatrix}
    \bm{F} & \bm{0} \\
    \bm{0} & \bm{C}
\end{pmatrix}
\begin{pmatrix}
    \bm{0} & \bm{W}\\
    \bm{W}^\dag & \bm{0}
\end{pmatrix}
\begin{pmatrix}
    \bm{0} & \bm{W}\\
    \bm{W}^\dag & \bm{0}
\end{pmatrix} 
 - \begin{pmatrix}
    \bm{0} & \bm{W}\\
    \bm{W}^\dag & \bm{0}
\end{pmatrix}
\begin{pmatrix}
    \bm{F} & \bm{0} \\
    \bm{0} & \bm{C}
\end{pmatrix}
\begin{pmatrix}
    \bm{0} & \bm{W}\\
    \bm{W}^\dag & \bm{0}
\end{pmatrix} \\
& - \begin{pmatrix}
    \bm{0} & \bm{W}\\
    \bm{W}^\dag & \bm{0}
\end{pmatrix}
\begin{pmatrix}
    \bm{F} & \bm{0} \\
    \bm{0} & \bm{C}
\end{pmatrix}
\begin{pmatrix}
    \bm{0} & \bm{W}\\
    \bm{W}^\dag & \bm{0}
\end{pmatrix} + \begin{pmatrix}
    \bm{0} & \bm{W}\\
    \bm{W}^\dag & \bm{0}
\end{pmatrix}
\begin{pmatrix}
    \bm{0} & \bm{W}\\
    \bm{W}^\dag & \bm{0}
\end{pmatrix}
\begin{pmatrix}
    \bm{F} & \bm{0} \\
    \bm{0} & \bm{C}
\end{pmatrix} \\
&= \begin{pmatrix}
    \bm{F} \bm{W} \bm{W}^\dag -2 \bm{W} \bm{C} \bm{W}^\dag + \bm{W} \bm{W}^\dag \bm{F} & \bm{0} \\
    \bm{0} & \bm{C} \bm{W}^\dag \bm{W} - 2 \bm{W}^\dag \bm{F} \bm{W} + \bm{W}^\dag \bm{W} \bm{C}
\end{pmatrix}
\end{align}
which implies that the second-order flow equations are
\begin{align}
    \dv{\bm{F}^{(2)}(s)}{s} & = \bm{F} \bm{W} \bm{W}^\dag -2 \bm{W} \bm{C} \bm{W}^\dag + \bm{W} \bm{W}^\dag \bm{F} \\
    \dv{\bm{C}^{(2)}(s)}{s} & = \bm{C} \bm{W}^\dag \bm{W} - 2 \bm{W}^\dag \bm{F} \bm{W} + \bm{W}^\dag \bm{W} \bm{C}
\end{align}
Continuing with this approach yields the desired result.

% %-----------------------------------------------
% \paragraph{Zeroth-order blocks.}
% %-----------------------------------------------

% Because the diagonal blocks of Wegner’s generator vanish,
% \begin{equation}
%   \eta^{(1)}_{11} = 0, \qquad \eta^{(1)}_{22} = 0,
% \end{equation}
% the diagonal blocks of the flow equation satisfy
% \begin{equation}
%   \dv{\bF^{(1)}}{s} = \bO,
%   \qquad
%   \dv{\bC^{(1)}}{s} = \bO.
% \end{equation}
% Using the initial conditions $\bF^{(1)}(0)=\bO$ and $\bC^{(1)}(0)=\bO$, we obtain
% \begin{subequations}
% \begin{align}
%   \bF^{(1)}(s) &= \bO, \\
%   \bC^{(1)}(s) &= \bO.
% \end{align}
% \end{subequations}

% %-----------------------------------------------
% \paragraph{First-order flow for the coupling block.}
% %-----------------------------------------------

% The first-order generator has the off-diagonal blocks
% \begin{equation}
%   \eta_{12}^{(1)} = \bF^{(0)}\bW^{(1)} - \bW^{(1)}\bC^{(0)},
%   \qquad
%   \eta_{21}^{(1)} = \bC^{(0)}\bW^{(1)\dagger} - \bW^{(1)\dagger}\bF^{(0)}.
% \end{equation}

% The $(1,2)$ block of the first-order flow equation becomes
% \begin{align}
%   \dv{\bW^{(1)}}{s}
%   &= \eta^{(1)}_{12}\,\bC^{(0)} - \bF^{(0)} \eta^{(1)}_{12} \\[4pt]
%   &= \left(\bF^{(0)}\bW^{(1)} - \bW^{(1)}\bC^{(0)}\right)\bC^{(0)}
%    - \bF^{(0)}\left(\bF^{(0)}\bW^{(1)} - \bW^{(1)}\bC^{(0)}\right)
%    \\[6pt]
%   &= 2 \bF^{(0)}\bW^{(1)}\bC^{(0)}
%    - (\bF^{(0)})^2 \bW^{(1)}
%    - \bW^{(1)} (\bC^{(0)})^2.
% \end{align}

% This is precisely the differential equation quoted in the reference:
% \begin{equation}
%   \label{eq:W1}
%   \dv{\bW^{(1)}}{s}
%    = 2 \bF^{(0)}\bW^{(1)}\bC^{(0)}
%     - (\bF^{(0)})^2 \bW^{(1)}
%     - \bW^{(1)}(\bC^{(0)})^2.
% \end{equation}

% %-----------------------------------------------
% \paragraph{Solution using diagonal F and C.}
% %-----------------------------------------------

% In the Hartree--Fock orbital basis, both $\bF^{(0)}$ and $\bC^{(0)}$ are diagonal:
% \begin{equation}
%   F^{(0)}_{pp'} = \epsilon_p \delta_{pp'}, \qquad
%   C^{(0)}_{\nu\nu'} = E_\nu \delta_{\nu\nu'}.
% \end{equation}

% Therefore each matrix element $W^{(1)}_{pq,\nu}(s)$ obeys
% \begin{equation}
%   \dv{}{s} W^{(1)}_{pq,\nu}(s)
%   = -(\epsilon_{pq} - E_\nu)^2 \, W^{(1)}_{pq,\nu}(s),
% \end{equation}
% where we define the energy difference
% \begin{equation}
%   \Delta_{pq}^{\nu} = \epsilon_{pq} - E_\nu.
% \end{equation}

% The solution is
% \begin{equation}
%   W_{pq}^{\nu(1)}(s)
%   = W_{pq}^{\nu(1)}(0)\,
%     e^{-(\Delta_{pq}^{\nu})^2 s}.
% \end{equation}

% Identifying $W_{pq}^{\nu(1)}(0) = W_{pq}^{\nu}$ (the screened integral at $s=0$), we obtain the final compact expression
% \begin{equation}
%   W_{pq}^{\nu(1)}(s) = W_{pq}^{\nu} \, e^{-(\Delta_{pq}^{\nu})^2 s},
% \end{equation}
% with the limits
% \begin{equation}
%   W_{pq}^{\nu(1)}(0) = W_{pq}^{\nu},
%   \qquad
%   \lim_{s\to\infty} W_{pq}^{\nu(1)}(s) = 0.
% \end{equation}

\subsection{Quasiparticle self-consistency (qpQPMP2)}
We can follow what is done in the GW literature to define a quasiparticle self-consistent scheme for QPMP2. In GW, it involves implementing an off-diagonal self-energy 
\begin{align}
    \Sigma^{qsGW}_{pq}(\omega) &= \frac{1}{2} \Big( \Sigma_{pq}(\epsilon_p) + \Sigma_{pq}(\epsilon_q) \Big) \\
&= \frac{1}{2} \sum_\nu \Bigg[ \sum_{i} \left[ \frac{\Delta_{p i}^\nu}{(\Delta_{p i}^\nu)^2 + \eta^2} + \frac{\Delta_{q i}^\nu}{(\Delta_{q i}^\nu)^2 + \eta^2} \right] W_{p i}^\nu W_{q i}^\nu\\ & + \sum_{a} \left[ \frac{\Delta_{p a}^\nu}{(\Delta_{p a}^\nu)^2 + \eta^2} + \frac{\Delta_{q a}^\nu}{(\Delta_{q a}^\nu)^2 + \eta^2} \right] W_{p a}^\nu W_{q a}^\nu \Bigg] \\
% &= \frac{1}{2} \sum_\nu \Bigg[ \sum_{i} \frac{\Delta_{p i}^\nu}{(\Delta_{p i}^\nu)^2 + \eta^2} W_{p i}^\nu W_{q i}^\nu + \sum_{a} \frac{\Delta_{p a}^\nu}{(\Delta_{p a}^\nu)^2 + \eta^2} W_{p a}^\nu W_{q a}^\nu \Bigg] \\
\end{align}
with $\Delta_{p i}^\nu = \epsilon_p - \epsilon_i - \Omega_\nu$ and $\Delta_{p a}^\nu = \epsilon_p - \epsilon_a + \Omega_\nu$. In QPMP2, it will be analogous, and the expression will be


\section{Why it is bad for the UEG}
The VASP paper by Kresse and Grunes proves that scMP2 consists of terminating the Dyson equation for the interacting polarizability to lowest order, so setting $\chi(\omega ) \approx \chi_0(\omega )$. This implies that the MP2 self-energy is only reliable if the polarizability of the system is small, so the neglect of higher order terms does not matter so much. I believe this is why the MP2 pole at $r_s=4$ is bad. This can be proven analytically be examining the magnitude of the Linhard function.
% \subsection{Suggestions for improvement}
% Inclusion of Thomas-Fermi screening, as proposed by Kresse
\section{Notes}
David Tew's recent paper investigates the performance of this quasiparticle MP2 for finite systems, but they only do small molecules and the Hubbard model.
\section{Decisions to make}
\subsection{Basis choice}
I think def2-TZVPP-RIFIT would be good and then all-electron?
