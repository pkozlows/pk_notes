\documentclass{beamer}
\usetheme{Madrid}
\usepackage{amsmath}
\usepackage{amssymb}
\usepackage{graphicx}
\usepackage{physics}
\usepackage{tikz}
\usepackage{simpler-wick}
\usepackage{cancel}
\usepackage{geometry}
\usepackage{algorithm}
\usepackage{algorithmic}
\usepackage[numbers]{natbib}
\bibliographystyle{plainnat}  % or another style like 'apsrev4-2'
\usepackage{hyperref}
\usepackage{bm}
\usepackage{xcolor}
\usepackage{listings}
\title{Slides for Patryk's Notes}
\author{Patryk Kozlowski}
\date{\today}

\hypersetup{
    colorlinks=true,    % false: boxed links; true: colored links
    linkcolor=blue,     % color of internal links
    urlcolor=cyan       % color of external links
}
\setbeamertemplate{section in toc}[sections numbered]

\begin{document}

\begin{frame}
    \titlepage
\end{frame}
\begin{frame}
  \frametitle{Outline}
  \tableofcontents
\end{frame}
\section{GW Supermatrices}
\begin{frame}
    \frametitle{Booth GW Supermatrix}

\begin{equation}
    \bm{H}^{G_0 W_0} = \begin{pmatrix} \bm{F} & \bm{W}^< & \bm{W}^> \\ \bm{W}^{\dagger<} & \bm{d}^< & \bm{0} \\ \bm{W}^{\dagger>} & \bm{0} & \bm{d}^> \end{pmatrix}
\end{equation}
where $\bm{F}$ is the Fock matrix, $\bm{W}^<$ and $\bm{W}^>$ are the lesser and greater components of the RPA screened Coulomb interaction, defined as
\begin{equation}
\begin{split}
    W_{pk\nu}^{<} = \sum_{ia} (pk|ia) \left( X_{ia}^{\nu} + Y_{ia}^{\nu} \right) \quad \text{and} \quad W_{pc\nu}^{>} = \sum_{ia} (pc|ia) \left( X_{ia}^{\nu} + Y_{ia}^{\nu} \right)
\end{split}
\end{equation}
and the auxiliary blocks $\bm{d}^<$ and $\bm{d}^>$ are defined as
\begin{equation}
\begin{split}
    d_{k\nu,l\nu'}^{<} = \left(\epsilon_k - \Omega_\nu\right) \delta_{k,l} \delta_{\nu,\nu'}\quad \text{and} \quad
    d_{c\nu,d\nu'}^{>} = \left(\epsilon_c + \Omega_\nu\right) \delta_{c,d} \delta_{\nu,\nu'}\\
\end{split}
\end{equation}

\end{frame}

\begin{frame}
    \frametitle{Garnet GW via auxiliary bosons}
They used a basis of particle-hole excitations, approximated as bosons. So $\hat{a}_a^\dagger \hat{a}_i \approx \hat{b}_\nu^\dag$ and $\hat{a}_i^\dagger \hat{a}_a \approx \hat{b}_\nu$
Define \begin{equation}
\hat{H}^{\mathrm{eB}}=\hat{H}^{\mathrm{e}}+\hat{H}^{\mathrm{B}}+\hat{V}^{\mathrm{eB}}
\end{equation}
where $\hat{H}^{\mathrm{e}}$ is the electronic Hamiltonian, $\hat{H}^{\mathrm{B}}$ is the bosonic Hamiltonian, and $\hat{V}^{\mathrm{eB}}$ is the electron-boson coupling term, given as
\begin{align}
\hat{H}^e&=\sum_{p q} f_{p q}\left\{\hat{a}_p^{\dagger} \hat{a}_q\right\} \\
\hat{H}^{B}&=\sum_{\nu \mu} A_{\nu \mu} \hat{b}_\nu^{\dagger} \hat{b}_\mu+\frac{1}{2} \sum_{\nu \mu} B_{\nu \mu}\left(\hat{b}_\nu^{\dagger} \hat{b}_\mu^{\dagger}+\hat{b}_\nu \hat{b}_\mu\right)
\label{b} \\
\hat{V}^{eB}&=\sum_{p q, \nu} V_{p q \nu}\left\{\hat{a}_p^{\dagger} \hat{a}_q\right\}\left(\hat{b}_\nu^{\dagger}+\hat{b}_\nu\right)
\label{eb}
\end{align}
\end{frame}

\begin{frame}
    \frametitle{Form in bosonic basis}
Originally $\hat{H}^{\mathrm{B}}$ in the bosonic basis was
\begin{align}
\hat{H}^{\mathrm{B}}\left(\hat{b}, \hat{b}^{\dagger}\right)&=-\frac{1}{2} \operatorname{tr} \mathbf{A}+\frac{1}{2}\left(\begin{array}{ll}
\mathbf{b}^{\dagger} & \mathbf{b}
\end{array}\right)\left(\begin{array}{ll}
\mathbf{A} & \mathbf{B} \\
\mathbf{B} & \mathbf{A}
\end{array}\right)\binom{\mathbf{b}}{\mathbf{b}^{\dagger}}
\label{eq:rpa_rec}
 \\
&=\sum_{\nu \mu} A_{\nu \mu} \hat{b}_\nu^{\dagger} \hat{b}_\mu+\frac{1}{2} \sum_{\nu \mu} B_{\nu \mu}\left(\hat{b}_\nu^{\dagger} \hat{b}_\mu^{\dagger}+\hat{b}_\nu \hat{b}_\mu\right)
\end{align}
But if we redefine the basis via a Bogoliubov transformation
\begin{align}
\binom{\overline{\mathbf{b}}}{\overline{\mathbf{b}}^{\dagger}}=\left(\begin{array}{cc}
\mathbf{X} & -\mathbf{Y} \\
-\mathbf{Y} & \mathbf{X}
\end{array}\right)^T\binom{\mathbf{b}}{\mathbf{b}^{\dagger}}
\end{align}

\end{frame}

\begin{frame}
    \frametitle{Effect of transformation on Hamiltonians}
Then
\begin{align}
    \hat{H}^{\mathrm{B}}\left(\overline{\mathbf{b}},
\overline{\mathbf{b}}^{\dagger}\right)&=-\frac{1}{2} \operatorname{tr} \mathbf{A}+\frac{1}{2}\left(\overline{\mathbf{b}}^{\dagger} \overline{\mathbf{b}}\right)\left(\begin{array}{cc}\Omega \mathbf{1} & 0 \\
0 & \Omega \mathbf{1}
\end{array}\right)\binom{\overline{\mathbf{b}}}{\overline{\mathbf{b}}^{\dagger}} \\
&=\sum_{\nu} \Omega_\nu \overline{b}_\nu^{\dagger} \overline{b}_\nu + E_{\mathrm{RPA}}^c \\
\end{align}
Because $\hat{b}_\nu + \hat{b}_{\nu}^\dagger = \sum_\mu \left(\mathbf{X}_{\mu}^{\nu} + \mathbf{Y}_{\mu}^{\nu}\right) \left(\hat{\overline{b}}_\nu + \hat{\overline{b}}_\nu^\dagger\right)$, we also get
\begin{align}
\hat{V}^{\mathrm{eB}}\left(\overline{\mathbf{b}},
\overline{\mathbf{b}}^{\dagger}\right)
&= \sum_{p q, \nu} W_{p q, \nu} \left\{ \hat{a}_p^{\dagger} \hat{a}_q \right\}\left(\bar{b}_\nu+\bar{b}_\nu^{\dagger}\right)
\end{align}
\end{frame}
\begin{frame}
    \frametitle{Supermatrix construction}
We then build the supermatrices $\mathbf{H}$ and $\mathbf{S}$ with matrix elements,
$$
\begin{gathered}
H_{I J}=\left\langle 0_{\mathrm{F}} 0_{\mathrm{B}}\right|\left[C_I,\left[\tilde{H}^{\mathrm{eB}}, C_J^{\dagger}\right]\right]\left|0_{\mathrm{F}} 0_{\mathrm{B}}\right\rangle \\
S_{I J}=\left\langle 0_{\mathrm{F}} 0_{\mathrm{B}}\right|\left[C_I, C_J^{\dagger}\right]\left|0_{\mathrm{F}} 0_{\mathrm{B}}\right\rangle
\end{gathered}
$$
where $\left\{C_I^{\dagger}\right\}=\left\{\underbrace{a_i}_{1h}, \underbrace{a_a}_{1p}, \underbrace{a_i b_\nu^{\dagger}}_{2h1p}, \underbrace{a_a b_\nu}_{1p2p}\right\}$ and $|0\rangle_{\mathrm{F}}$ and $|0\rangle_{\mathrm{B}}$ are the Fermi and boson vacuums. Then constructing $-\bm{S}^{-1}\bm{H}$ yields Booth's ED. Derivation is in my notes. Nothing too complicated, but long due to many Wick contractions.
\end{frame}

\begin{frame}
    \frametitle{Realization of the idea}
Describe the bosons via an auxiliary basis, scaling linearly with system size.
\begin{equation}
\begin{aligned}
& \hat{b}_\nu^{\dagger} \approx \sum_Q^{N_{\mathrm{AB}}} C_\nu^Q \hat{b}_Q^{\dagger}, \quad \hat{b}_\nu \approx \sum_Q^{N_{\mathrm{AB}}} C_\nu^Q \hat{b}_Q
\end{aligned}
\end{equation}
Use RI technique to get the $C_\nu^Q$ coefficients. Define 
\begin{equation}
(i a \mid j b) \approx \sum_L R_{i a}^L R_{j b}^L
\end{equation}
Then 
$C_\nu^Q=\sum_{L M} R_\nu^L\left[\mathbf{S}^{-1 / 2}\right]_{L M} P_M^Q$ with $S_{L M}=\sum_\nu R_\nu^L R_\nu^M=\sum_Q P_L^Q E_Q P_M^Q$

\end{frame}

\begin{frame}
    \frametitle{Realization of the idea continued}
\begin{itemize}
    \item 1. Get the excitation energies $\bm{\Omega}$ and vectors $\bm{X}+\bm{Y}$ by solving the symmetrized Casida eigenproblem in $O(N_{AB}^3)$ time 
    \item Recall last week we identified using $\textbf{T} = \boldsymbol{\Omega }^{\frac{1}{2}} \left(\textbf{A}-\textbf{B}\right)^{-\frac{1}{2}}\left(\textbf{X} + \textbf{Y}\right)$ to get excitation vectors as problematic; but that was in a different context and now we have explicit access to $\bm{\Omega }$ and $\bm{A}-\bm{B}$, so we can do this
    \item 2. Transform the excitation vectors into a screened Coulomb interaction in $O(N_{\text{orb}}^2 N_{\text{AB}}^2)$ time, where $N_{\text{orb}}=O+V$
    \item 3. Diagonalize the Hamiltonian with a Davidson procedure in $\mathcal{O}\left(N_{\text {orb }}^2 N_{\mathrm{AB}}\right) / \mathcal{O}\left(N_{\text {orb }} N_{\mathrm{AB}}^2\right)$ time for each root
\end{itemize}
Interestingly, their highest scaling step is 2. 
\end{frame}
\section{RPA via Equation of Motion}
\begin{frame}
    \frametitle{Equation of motion formalism}
Define an oscillator that satisfies 
\begin{align}
    [H, O^\dag] = \omega O^\dag , \quad  \quad [H, O] = -\omega O , \quad \quad [O, O^\dag] = 1
\end{align}
With the arbitrary operator $R$ we have
\begin{align}
\langle \phi | [R,[H, O^\dagger]] | \phi \rangle &= \omega \langle \phi | [R, O^\dagger] | \phi \rangle\\
\langle \phi | [R,[H, O]] | \phi \rangle 
    &= -\omega \langle \phi | [R, O] | \phi \rangle\\
    \implies\langle\phi|\left[R, H, O^{\dagger}\right]|\phi\rangle&=\omega\langle\phi|\left[R, O^{\dagger}\right]|\phi\rangle
\label{duble_commutator}
\end{align}
where we have defined the double commutator as
\begin{equation}
    2\left[R, H, O^{\dagger}\right]=\left[R,\left[H, O^{\dagger}\right]\right]+\left[[R, H], O^{\dagger}\right] 
\end{equation}
This approach can save because we exploit Hermiticity and the commutator is of lower-rank than the product, so we don't need to know much about the wavefunction to get good matrix elements.
\end{frame}


\begin{frame}
    \frametitle{The particle hole approximation leads to RPA}
Define the excitation operator $
\hat O^\dagger
=\sum_{a i}\bigl(Y_{a i}\,a_a^\dagger a_i - Z_{i a}\,a_i^\dagger a_a\bigr).$ Then, 
\begin{align}
A_{ai,bj}
&=\langle\phi|\bigl[a_i^\dagger a_a,\,H,\,a_b^\dagger a_j\bigr]|\phi\rangle\\
B_{ai,bj}
&=-\,\langle\phi|\bigl[a_i^\dagger a_a,\,H,\,a_j^\dagger a_b\bigr]|\phi\rangle \\
U_{ai,bj}
&=\langle\phi|\bigl[a_i^\dagger a_a,\,a_b^\dagger a_j\bigr]|\phi\rangle
\end{align}
or in matrix form
\begin{equation}
\begin{pmatrix}
A & B \\[6pt]
B^\dagger & A^*
\end{pmatrix}
\begin{pmatrix}
Y \\ Z
\end{pmatrix}
\;=\;
\omega
\begin{pmatrix}
U & 0 \\[3pt]
0 & -\,U^*
\end{pmatrix}
\begin{pmatrix}
Y \\ Z
\end{pmatrix}.
\label{eq:block_matrix}
\end{equation}
Then if we choose the basis that diagonalizes the single-particle Hamiltonian, we get the RPA equations
\begin{align}
    A_{a i b j} &= \langle 0_{\mathrm{F}} | a_a^\dagger \left[H, a_b^\dagger a_i\right] | 0_{\mathrm{F}} \rangle = \delta_{a b} \delta_{i j}\left(\varepsilon_i-\varepsilon_a\right) + V_{a j i b} \\
    B_{a i b j} &= \langle 0_{\mathrm{F}} | a_a^\dagger \left[H, a_b a_i^\dagger\right] | 0_{\mathrm{F}} \rangle = V_{a b i j} \\
    U_{a i b j} &= \langle 0_{\mathrm{F}} | a_a^\dagger \left[H, a_b a_i\right] | 0_{\mathrm{F}} \rangle = \delta_{a b} \delta_{i j} .
\end{align}

\end{frame}
\section{BSE}
\begin{frame}
    \frametitle{The BSE problem}
We want to solve the problem
\begin{align}
&\bm{L}^{-1} = \bm{L}_0^{-1} - \bm{\Xi}^{\mathrm{eh}} \\
        &\implies \begin{pmatrix}
        \mathcal{\bm{A}}(\omega ) & \mathcal{\bm{B}}(\omega ) \\
        \mathcal{\bm{B}}(\omega ) & \mathcal{\bm{A}}(\omega )
    \end{pmatrix}
\begin{pmatrix}
    \bm{X}^m \\
\bm{Y}^m
\end{pmatrix}
=
\Omega ^m
\begin{pmatrix}
    \bm{X}^m \\
\bm{Y}^m
\end{pmatrix}
\end{align}
with
\begin{align}
\label{a}
\mathcal{A}_{\mu \nu} \equiv \mathcal{A}_{ai,bj} &= \underbrace{\left( \epsilon_a^{QP} - \epsilon_i^{QP} \right) \delta_{ab} \delta_{ij} + (ai|jb)}_{\tilde{A}_{ai,bj}} - {\Xi}_{ab,ji}(\omega) \\
\mathcal{B}_{\mu \nu} \equiv \mathcal{B}_{ai,bj} &= (ai|bj) - {\Xi}_{bi|aj}(\omega)
\end{align}
 BSE@GW approximates the kernel as the screened Coulomb interaction
\begin{equation}
    \bm{\Xi}(\omega ) \approx \bm{\Xi}_{GW}(\omega ) = W(\omega )
\end{equation}
Common to do $\bm{\Xi}_{GW}(\omega ) \approx W(\omega=0)$, which introduces errors
\end{frame}

\begin{frame}
\frametitle{Tim's full frequency and frequency free BSE@TDA}
In TDA, the upfolded 2p Hamiltonian is given by
\begin{equation}
\mathcal{H}=
\begin{pmatrix}
\mathbf{\tilde{A}} & -\mathbf{V}^{\mathrm{e}} & -\mathbf{V}^{\mathrm{h}} \\
\left(\mathbf{V}^{\mathrm{h}}\right)^{\dagger} & \mathbf{D} & \mathbf{0} \\
\left(\mathbf{V}^{\mathrm{e}}\right)^{\dagger} & \mathbf{0} & \mathbf{D}
\end{pmatrix}
\end{equation}
The single excitation block $\tilde{A}$ was defined last slide; the rest is:
\begin{align}
\mathbf{D}_{iajb,iajb} &= \left[-\boldsymbol{E}_{\mathrm{occ}}\right] \oplus_{\text{kron}} \boldsymbol{E}_{\mathrm{vir}} \oplus_{\text{kron}} \mathbf{S} \\
V_{ia,ldkc}^{\mathrm{h}} &= \sqrt{2}\,(il|kc)\,\delta_{ad} \\
V_{ia,ldkc}^{\mathrm{e}} &= \sqrt{2}\,(kc|ad)\,\delta_{il}
\end{align}
Here, $\mathbf{S}$ is the direct RPA matrix in the TDA. Claim: this downfolds to \ref{a}, thus preserving full frequency dependence; I have not been able to prove this yet.
\end{frame}

\begin{frame}
\frametitle{Where I am stuck in the derivation}
\begin{align}
\mathcal{A}(\omega)
&= \tilde{\mathbf{A}} - \mathbf{V}^{\mathrm{e}}(\omega \mathbf{I} - \mathbf{D})^{-1} (\mathbf{V}^{\mathrm{h}})^{\dagger} - \mathbf{V}^{\mathrm{h}}(\omega \mathbf{I} - \mathbf{D})^{-1} (\mathbf{V}^{\mathrm{e}})^{\dagger} \\
\end{align}
This implies the kernel shoulld be
\begin{align}
K_{abij}^{(\mathrm{p})}(\omega) &= \mathbf{V}^{\mathrm{e}}(\omega \mathbf{I} - \mathbf{D})^{-1} (\mathbf{V}^{\mathrm{h}})^{\dagger} + \mathbf{V}^{\mathrm{h}}(\omega \mathbf{I} - \mathbf{D})^{-1} (\mathbf{V}^{\mathrm{e}})^{\dagger} \\
&= \frac{\mathbf{V}^{\mathrm{e}} \bm{\tilde{X}}  (\bm{V}^{\mathrm{h}}\bm{\tilde{X}})^{\dagger}}{\omega \mathbf{I} - \left(-\bm{E}_O \oplus \bm{E}_V \oplus \bm{\Omega }_{OV}\right)} + \frac{\bm{V}^{\mathrm{h}} \bm{\tilde{X}}  (\bm{V}^{\mathrm{e}}\bm{\tilde{X}})^{\dagger}}{\omega \mathbf{I} - \left(-\bm{E}_O \oplus \bm{E}_V \oplus \bm{\Omega }_{OV}\right)} \\
\end{align}
I should be getting
\begin{align}
K_{abij}^{(\mathrm{p})}(\omega) &= 2 \sum_m^{\Omega_m>0}\left(i j| \rho_m\right)\left(a b|\rho_m\right)\left[\frac{1}{\omega-\left(E_b-E_i\right)-\Omega_m}+\frac{1}{\omega-\left(E_a-E_j\right)-\Omega_m}\right]
\end{align}
where $\left(p q | \rho_m\right)=\sum_{i a} X_{i a}^m(p q | i a)$.

\end{frame}
\section{Ideas of what to look at next}
\begin{frame}
\frametitle{Starting from QRPA}
If ground state $|\phi\rangle$ is the quasiparticle vacuum 
\begin{equation}
|\tilde{\phi}\rangle=\prod_{\nu>0}\left(U_\nu+V_\nu a_\nu^{\dagger} a_{\overline{\nu}}^{\dagger}\right)|-\rangle
\end{equation}
with quasiparticles (satisfying $U_\nu^2+V_\nu^2=1$) defined by: 
\begin{align}
    \alpha_\nu^{\dagger}=U_\nu a_\nu^{\dagger}-V_\nu a_{\bar{\nu}} \\
    \alpha_{\bar{\nu}}^{\dagger}=U_\nu a_{\bar{\nu}}^{\dagger}+V_\nu a_\nu
\end{align}
Then $\alpha_\nu |\tilde{\phi}\rangle=0$ 

\end{frame}

\begin{frame}
\frametitle{Starting from QRPA continued}
Define excitation vector as
\begin{equation}
    O^{\dagger}=\sum_{\mu \nu}\left(Y_{\mu \nu} \alpha_\mu^{\dagger} \alpha_\nu^{\dagger}+Z_{\mu \nu} \alpha_\mu \alpha_\nu\right)
\end{equation}
Then
\begin{align}
    A_{\mu \nu \mu^{\prime} \nu^{\prime}} &= \langle\phi|\left[\alpha_\nu \alpha_\mu, H, \alpha_{\mu^{\prime}}^{\dagger} \alpha_{\nu^{\prime}}^{\dagger}\right]|\phi\rangle, \\
    B_{\mu \nu \mu^{\prime} \nu^{\prime}} &= \langle\phi|\left[\alpha_\nu \alpha_\mu, H, \alpha_{\mu^{\prime}} \alpha_{\nu^{\prime}}\right]|\phi\rangle, \\
    U_{\mu \nu \mu^{\prime} \nu^{\prime}} &= \langle\phi|\left[\alpha_\nu \alpha_\mu, \alpha_{\mu^{\prime}}^{\dagger} \alpha_{\nu^{\prime}}^{\dagger}\right]|\phi\rangle .
\end{align}
\end{frame}

\begin{frame}
\frametitle{Form of $H^{eB}$}
Define \begin{equation}
\hat{H}^{\mathrm{eB}}=\hat{H}^{\mathrm{e}}+\hat{H}^{\mathrm{B}}+\hat{V}^{\mathrm{eB}}
\end{equation}
where $\hat{H}^{\mathrm{e}}$ is the electronic Hamiltonian, $\hat{H}^{\mathrm{B}}$ is the bosonic Hamiltonian, and $\hat{V}^{\mathrm{eB}}$ is the electron-boson coupling term, given as
\begin{align}
\hat{H}^e&=\sum_{p q} f_{p q}\left\{\hat{a}_p^{\dagger} \hat{a}_q\right\} \\
\hat{H}^{B}&=\sum_{\nu \mu} A_{\nu \mu} \hat{b}_\nu^{\dagger} \hat{b}_\mu+\frac{1}{2} \sum_{\nu \mu} B_{\nu \mu}\left(\hat{b}_\nu^{\dagger} \hat{b}_\mu^{\dagger}+\hat{b}_\nu \hat{b}_\mu\right)
\label{b} \\
\hat{V}^{eB}&=\sum_{p q, \nu} V_{p q \nu}\left\{\hat{a}_p^{\dagger} \hat{a}_q\right\}\left(\hat{b}_\nu^{\dagger}+\hat{b}_\nu\right)
\label{eb}
\end{align}
\end{frame}



\end{document}

