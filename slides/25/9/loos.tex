
\section{Diagonal GW+C}


\begin{frame}
    \frametitle{Cumulant}

This is the simplest approach to deriving the cumulant expansion for the retarded Green function. Equating the Dyson equation and the Taylor expansion of the exponential to first order gives
\begin{align}
    \bm{G}_{pp}^{0}(t) \bm{C}_{pq}(t) &= {\iint \dd t_1 \dd t_2 \bm{G}_{pp}^{0}(t-t_1) \bm{\Sigma}_{pq}^c(t_1 - t_2) \bm{G}_{qq}^{0}(t_2)} \\
\implies C_{pp}(t) &= i \int \frac{d\omega}{2\pi} \frac{ \Sigma_{pp}^c\left(\omega+\epsilon_p^{HF}\right)}{(\omega + i \eta)^2} e^{-i \omega t}
\end{align}
Loos plugs in the $GW$ self-energy here to get
\begin{equation}
    C_{pp}(t) = \sum_{i\nu} \zeta_{pi\nu} \left[ e^{-i\Delta_{pi\nu} t} - 1 + i\Delta_{pi\nu} t \right] + \sum_{a\nu} \zeta_{pa\nu} \left[ e^{-i\Delta_{pa\nu} t} - 1 + i\Delta_{pa\nu} t \right]
\end{equation}
\color{orange}
This first order expansion is exact up to the first order in the screened Coulomb interaction $W$. An open research question would be to insert the BWs2 self-energy here instead.

\end{frame}

\begin{frame}
    \frametitle{Green's function}
Plugging back this form gives
\begin{equation}
    G_{pp}(t) = -i \Theta(t) Z_p^{QP} e^{-i \epsilon_p^{QP} t} e^{\sum_{i\nu} \zeta_{pi\nu} e^{-i\Delta_{pi\nu} t} + \sum_{a\nu} \zeta_{pa\nu} e^{-i\Delta_{pa\nu} t}}
\end{equation}
The weight of the quasiparticle peak is
\begin{equation}
    Z_p^{QP} = e^{-\sum_{i\nu} \zeta_{pi\nu} - \sum_{a\nu} \zeta_{pa\nu}} = \exp\left(\left[\frac{\partial \Sigma_{pp}^c(\omega)}{\partial \omega}\right]_{\omega = \epsilon_p^{HF}}\right)
\end{equation}
with the quasiparticle energy
\begin{equation}
    \epsilon_p^{QP} = \epsilon_p^{HF} - \left(\sum_{i\nu} \zeta_{pi\nu} \Delta_{pi\nu} + \sum_{a\nu} \zeta_{pa\nu} \Delta_{pa\nu}\right) = \epsilon_p^{HF} + \Sigma_{pp}^c\left(\epsilon_p^{HF}\right)
\end{equation}
With the Fourier transform and a first order expansion of the exponential:
\begin{equation}
    G_{pp}(\omega) = \frac{Z_p^{QP}}{\omega - \epsilon_p^{QP} + i\eta} + \sum_{i\nu} \frac{Z_p^{QP} \zeta_{pi\nu}}{\omega - \epsilon_p^{QP} - \Delta_{pi\nu} + i\eta} + \sum_{a\nu}  \ldots
\end{equation}



\end{frame}

\begin{frame}
    \frametitle{Spectral function}
\begin{align}
    A_{p}(\omega) &= -\frac{1}{\pi} \operatorname{Im} G_{pp}(\omega) \\
&=-\frac{1}{\pi}\left[\frac{\left(\operatorname{Re} Z_p^{QP}\right)\left(\operatorname{Im} \epsilon_p^{QP}\right) + \left(\operatorname{Im} Z_p^{QP}\right)\left(\omega - \operatorname{Re} \epsilon_p^{QP}\right)}{\left(\omega - \operatorname{Re} \epsilon_p^{QP}\right)^2 + \left(\operatorname{Im} \epsilon_p^{QP}\right)^2}\right.\\
& \left. + \sum_{i\nu} \frac{\left(\operatorname{Re} Z_{pi\nu}^{sat}\right)\left(\operatorname{Im} \epsilon_{pi\nu}^{sat}\right) + \left(\operatorname{Im} Z_{pi\nu}^{sat}\right)\left(\omega - \operatorname{Re} \epsilon_{pi\nu}^{sat}\right)}{\left(\omega - \operatorname{Re} \epsilon_{pi\nu}^{sat}\right)^2 + \left(\operatorname{Im} \epsilon_{pi\nu}^{sat}\right)^2} + \sum_{a\nu}\ldots \right]
\end{align}
\color{orange}
It would not be useful to include higher order terms because this would be including higher powers of $W$, but we already made the truncation with the hope of being exact up to first order in $W$.
\end{frame}
\section{Off-diagonal GW+C}



\begin{frame}
    \frametitle{C to G}
\color{orange}
We can instead consider (with $\Xi_{i\nu} \equiv \epsilon_i - \Omega_\nu$, $\Xi_{a\nu} \equiv \epsilon_a + \Omega_\nu$, and $\Delta \equiv \epsilon_q - \epsilon_p$)
\begin{align}
    C_{pq}(t) &= i \int \frac{d\omega}{2\pi} e^{-i(\omega-\epsilon_p^{HF})t} \frac{\Sigma_{pq}^c(\omega)}{(\omega - \epsilon_p^{HF} + i\eta)(\omega - \epsilon_q^{HF} + i\eta)} \\
&=  i^2\left[ \sum_{i\nu} W_{p i \nu} W_{q i \nu} \frac{1}{\Delta}\left[\frac{e^{-i\left(\Xi_{i\nu} - \epsilon_p\right) t}-1 }{\Xi_{i\nu}-\epsilon_p} + \frac{e^{i\Delta t} - e^{-i\left(\Xi_{i\nu} - \epsilon_p\right) t}}{\Xi_{i\nu} - \epsilon_q}\right] \right.\\
& \left. + \sum_{a\nu} W_{p a \nu} W_{q a \nu} 
\frac{1}{\Delta}\left[{\frac{e^{-i\left(\Xi_{a\nu} - \epsilon_p\right) t}-1 }{\Xi_{a\nu}-\epsilon_p} + \frac{e^{i\Delta t} - e^{-i\left(\Xi_{a\nu} - \epsilon_p\right) t}}{\Xi_{a\nu} - \epsilon_q}}\right] \right] \notag \\
\implies G_{pq}(t) &= -i \int_0^\infty dt e^{i(\omega - \epsilon_p) t} \left[1- \sum_{i\nu} W_{p i \nu} W_{q i \nu} T_{i\nu}(t) - \sum_{a\nu} W_{p a \nu} W_{q a \nu} T_{a\nu}(t) \right] 
\end{align}

\end{frame}
\begin{frame}
    \frametitle{G to A}
\color{orange}
\begin{align}
    &A_{pq}(\omega) = \\ & \delta(\omega - \epsilon_p) - \sum_{i\nu} W_{p i \nu} W_{q i \nu} \left[-\frac{\delta(\omega - \epsilon_p)}{\left(\epsilon_p-\epsilon_q\right)\left(\Xi_{i\nu}-\epsilon_p\right)} + \frac{\delta(\omega - \epsilon_q)}{\left(\epsilon_p-\epsilon_q\right)\left(\Xi_{i\nu}-\epsilon_q\right)} + \right. \\
 & \left. + \frac{\delta(\omega - \Xi_{i\nu})}{\left(\Xi_{i\nu}-\epsilon_p\right)\left(\Xi_{i\nu}-\epsilon_q\right)} - \frac{2\Xi_{i\nu}\delta(\omega - \Xi_{i\nu})}{\left(\epsilon_p-\epsilon_q\right)\left(\Xi_{i\nu}-\epsilon_p\right)\left(\Xi_{i\nu}-\epsilon_q\right)}\right] - \sum_{a\nu} \ldots \notag
\end{align}
We can replace all of the delta functions with Lorentzians in practice.

\end{frame}
